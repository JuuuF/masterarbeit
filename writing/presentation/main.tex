\documentclass[10pt, aspectratio=169]{beamer}

\usetheme[
	titleformat=regular,		% regular*,smallcaps,allsmallcaps,allcaps
	sectionpage=progressbar,	% none,simple,progressbar*
	subsectionpage=none,		% none*,simple,progressbar
	numbering=counter,			% none,counter*,fraction
	progressbar=frametitle,		% none*,head,frametitle,foot
	block=fill,					% transparent*,fill
	background=light			% dark,light*
]{metropolis}

\usepackage[ngerman]{babel}
\usepackage{graphicx}
\usepackage{booktabs}
\usepackage{tikz}
\usepackage{pgfplots}
% \usepackage{microtype}
\usepackage{amsmath, amssymb, amsfonts}
\usepackage{setspace} % set spacing

\usepackage{caption}
\captionsetup{font=small}

% \usepackage{enumitem}
% \newcommand\labelitemi{--}

\definecolor{cau-purple}{HTML}{98007D}
\setbeamercolor{progress bar}{fg=cau-purple}
\setbeamercolor{alerted text}{fg=cau-purple}
\setbeamercolor{background canvas}{bg=white}

\setbeamertemplate{frame footer}{CAU Kiel}

% =================================================================================================
    
\title{Nutzung von realitätsnahen, synthetisch erzeugten Daten zur Verbesserung des KI-gestützten Scorings von Steeldarts in einem Single-Camera-System}
\subtitle{Präsentation zur Masterarbeit}
\author{Justin Fürstenwerth}
\institute[CAU Kiel]{
    Christian-Albrechts-Universität zu Kiel\\
    Arbeitsgruppe Intelligente Systeme\\
    Betreuer: Simon Reichhuber
}
\date{15. Mai 2025}

% =================================================================================================

\begin{document}

\begin{frame}
    \titlepage
\end{frame}

\begin{frame}{Aufbau der Präsentation}
    \tableofcontents
\end{frame}

\section{Projektübersicht}

\begin{frame}{Thema der Masterarbeit}
    \begin{columns}
        \begin{column}{0.4\linewidth}

            \begin{figure}
                \centering
                \includegraphics[height=0.7\textheight]{imgs/dartboard.png}
                \caption{Bild einer Dartscheibe}
            \end{figure}

        \end{column}
        \begin{column}{0.6\linewidth}

            \begin{itemize}
                \item automatisches Dart-Scoring von Steeldarts-Runden
                \item Eingabe: Einzelnes Bild der Dartscheibe
                \item Ausgabe: getroffene Felder + erzielte Punktzahl
            \end{itemize}

        \end{column}
    \end{columns}
\end{frame}

\begin{frame}{Motivation}

    \begin{columns}
        \begin{column}{0.6\linewidth}

            \textbf{DeepDarts}: System für automatisches Dart-Scoring
            \begin{itemize}
                \item verspricht gute Ergebnisse ($>80\%$ Korrektheit)
                \item Einsatz neuronaler Netze zur Erkennung von Keypoints (Dartscheibe + Dartpfeile)
            \end{itemize}

        \end{column}
        \begin{column}{0.4\linewidth}

            \begin{figure}
                \centering
                \includegraphics[width=0.8\linewidth]{imgs/dd_keypoints.pdf}
                \vspace*{-0.2cm}
                \caption{Keypoint-Detection von DeepDarts}
            \end{figure}

        \end{column}
    \end{columns}

    \vspace*{-0.2cm}

    \visible<2->{
        \begin{block}{Problem}
            Einseitige Datenlage + massives Overfitting $\rightarrow$ keine Generalisierbarkeit
        \end{block}
    }

\end{frame}

\begin{frame}{Projektübersicht}
    \begin{figure}
        \centering
        \includegraphics[width=\linewidth]{imgs/ma_project_structure.pdf}
        \caption{Visualisierung der Projektstruktur}
    \end{figure}
\end{frame}

\begin{frame}{Forschungsfragen}
    \begin{itemize}
        \item<2-> In welcher Qualität lassen sich automatisch Daten erstellen?
        \item<3-> Wie zuverlässig kann eine algorithmische Normalisierung der Dartscheiben erarbeitet werden?
        \item<4-> Wie zuverlässig ist ein auf den synthetischen Daten trainiertes neuronales Netz?
        \item<5-> Kann durch die erarbeiteten Systeme eine Verbesserung gegenüber DeepDarts erreicht werden?
    \end{itemize}
\end{frame}
\section{Thema I: Synthetische Datenerstellung durch 3D-Rendering}

\begin{frame}{Datengrundlage}
    \begin{columns}
        \begin{column}{0.6\linewidth}

            \begin{block}{Ziel}
                Vielzahl korrekt annotierter Daten für das Training eines neuronalen Netzes
            \end{block}

            \begin{block}{Herangehensweise}
                Synthetische Datenerstellung durch 3D-Rendering
            \end{block}

        \end{column}
        \begin{column}{0.4\linewidth}

            \begin{figure}
                \centering
                \includegraphics[height=0.65\textheight]{imgs/datagen.png}
                \caption{Exemplarisches Resultat der Datenerstellung}
            \end{figure}

        \end{column}
    \end{columns}
\end{frame}


\begin{frame}{Datenerstellung}
    \begin{figure}
        \centering
        \includegraphics[height=0.6\textheight]{imgs/blender.png}
        \caption{Blender-Szene}
    \end{figure}
\end{frame}

\begin{frame}{Zusammensetzung der Szene}
    Weitreichende Randomisierung der Szene
    \begin{itemize}\footnotesize
        \setlength\itemsep{0em}
        \item Dartscheibe
              \vspace*{-0.1cm}
              \begin{itemize}\footnotesize
                  \item Prozedurale Materialien
                  \item Simulation von Alter und Abnutzung
              \end{itemize}
              \vspace*{-0.15cm}
        \item Dartpfeile
              \vspace*{-0.1cm}
              \begin{itemize}\footnotesize
                  \item Zufällige Zusammensetzung aus unterschiedlichen Bestandteilen
                  \item Positionierung anhand von Heatmaps
              \end{itemize}
              \vspace*{-0.15cm}
        \item Beleuchtung:
              \vspace*{-0.1cm}
              \begin{itemize}\footnotesize
                  \item $>200$ Environment Maps
                  \item Kamerablitz, Deckenbeleuchtung, Spotlight, Ringlicht
              \end{itemize}
              \vspace*{-0.15cm}
        \item Kamera
              \vspace*{-0.1cm}
              \begin{itemize}\footnotesize
                  \item Zufällige Position in gegebenem Raum
                  \item Randomisierung von Brennweite, Fokuspunkt, Auflösung, ISO etc.
              \end{itemize}
    \end{itemize}

\end{frame}

\begin{frame}{Datenerstellung}
    \begin{columns}
        \begin{column}{0.33\linewidth}
            \begin{figure}
                \centering
                \includegraphics[width=\textwidth]{imgs/dartscheibe_a=0.png}
                \caption{Dartscheibe mit geringem Alter}
            \end{figure}
        \end{column}
        \begin{column}{0.33\linewidth}
            \begin{figure}
                \centering
                \includegraphics[width=\textwidth]{imgs/dartscheibe_a=1.png}
                \caption{Dartscheibe mit hohem Alter}
            \end{figure}
        \end{column}
        \begin{column}{0.33\linewidth}
            \begin{figure}
                \centering
                \includegraphics[width=\textwidth]{imgs/darts_examples.png}
                \caption{Erstellte Dartpfeile}
            \end{figure}
        \end{column}
    \end{columns}
\end{frame}

\begin{frame}{Ergebnisse}
    
\end{frame}

\section{Thema II: Normalisierung durch herkömmliche Computer Vision}

\begin{frame}{Test}
    test
\end{frame}

\begin{frame}{Ergebnisse}
    
\end{frame}

\section{Thema III: Lokalisierung durch neuronale Netze}

\begin{frame}{Test}
    test
\end{frame}

\begin{frame}{Ergebnisse}
    \begin{figure}
        \centering
        \includegraphics[height=0.55\textheight]{imgs/nn_results.png}
        \caption{Visualisierung der Netzwerkvorhersagen}
    \end{figure}
\end{frame}

\begin{frame}{Ergebnisse}
    \begin{figure}
        \centering
        \includegraphics[height=0.55\textheight]{imgs/nn_results_2.png}
        \caption{Visualisierung korrekter Netzwerkvorhersagen}
    \end{figure}
\end{frame}

\input{chapters/ergebnisse}
% !TEX root = ../main.tex

\section{Diskussion}

\begin{frame}{Diskussion: Datenerstellung}
    \begin{columns}
        \begin{column}{0.7\linewidth}

            Datenumfang:\\
            \begin{itemize}
                \item Wenig Lichtobjekte
                \item Wenig vorgefertigte Bestandteile für Dartpfeile
            \end{itemize}

            Realismus:\\
            \begin{itemize}
                \item Kein Fotorealismus
                \item Dartscheibe fliegt im Raum
            \end{itemize}

            Effizienz:\\
            \begin{itemize}
                \item System nicht $100\,\%$ ausgelastet $\rightarrow$ Zeitverlust
            \end{itemize}

        \end{column}
        \begin{column}{0.3\linewidth}

            \begin{figure}
                \centering
                \includegraphics[height=0.74\textheight]{imgs/disc_example_render.png}
                \caption{Render-Ergebnis}
            \end{figure}

        \end{column}
    \end{columns}
\end{frame}

\begin{frame}{Diskussion: Dartscheiben-Normalisierung}
    \begin{columns}
        \begin{column}{0.7\linewidth}

            Geschwindigkeit:\\
            \begin{itemize}
                \item Algorithmisch statt KI
                \item Implementierung in Python
            \end{itemize}

            Zuverlässigkeit:\\
            \begin{itemize}
                \item Skalierung der Bilder durch Vorverarbeitung $\rightarrow$ Informationsverlust
                \item Verwendung von RANSAC $\rightarrow$ nicht deterministisch
            \end{itemize}

        \end{column}
        \begin{column}{0.3\linewidth}

            \begin{figure}
                \centering
                \includegraphics[width=\linewidth]{imgs/disc_example_cv.png}
                \caption{Normalisierungs-Ergebnis}
            \end{figure}

        \end{column}
    \end{columns}
\end{frame}

\begin{frame}{Diskussion: Dartpfeil-Lokalisierung}
    \begin{columns}
        \begin{column}{0.7\linewidth}

            Training:\\
            \begin{itemize}
                \item Synthetische Daten $\neq$ echte Daten
                \item Manuelles Intervenieren
            \end{itemize}

            Netzwerk:\\
            \begin{itemize}
                \item Eigene Implementierung des Netzwerks
                \item Kein Transfer-Learning
            \end{itemize}

            Vergleich mit DeepDarts:\\
            \begin{itemize}
                \item Unterschiedliche Netzwerkgrößen
            \end{itemize}

        \end{column}
        \begin{column}{0.3\linewidth}

            \begin{figure}
                \centering
                \includegraphics[width=\linewidth]{imgs/disc_example_nn.png}
                \caption{Lokalisierungs-Ergebnis}
            \end{figure}

        \end{column}
    \end{columns}
\end{frame}

% !TEX root = ../main.tex

\section{Fazit}

\newlength{\blocklabelwidth}
\settowidth{\blocklabelwidth}{\underline{IV}.}

\begin{frame}{Forschungsfragen}
    \visible<1->{
        \begin{block}{\makebox[\blocklabelwidth][c]{\underline{I}.} Qualität der synthetischen Daten}
            \hphantom{~~~~~~} Annähernd realistisch
        \end{block}
    }

    \visible<2->{
        \begin{block}{\makebox[\blocklabelwidth][c]{\underline{II}.} Zuverlässigkeit der algorithmischen Normalisierung}
            \hphantom{~~~~~~} $>97\,\%$ Erfolg\\
            \hphantom{~~~~~~} $<35\,\text{px}$ mittlere Verschiebung ($\approxeq 5\,\%$)
        \end{block}
    }

    \visible<3->{
        \begin{block}{\makebox[\blocklabelwidth][c]{\underline{III}.} Generalisierbarkeit nach OOD-Training}
            \hphantom{~~~~~~} Lediglich geringer Unterschied in PCS
        \end{block}
    }

    \visible<4->{
        \begin{block}{\makebox[\blocklabelwidth][c]{\underline{IV}.} Verbesserung von DeepDarts}
            \hphantom{~~~~~~} - Langsamere Ausführung\\
            \hphantom{~~~~~~} + Kein Overfitting\\
            \hphantom{~~~~~~} + Zuverlässigere Vorhersagen
        \end{block}
    }
\end{frame}


\begin{frame}[standout]
    Fragen?
\end{frame}

\end{document}
