\section{Thema I: Synthetische Datenerstellung durch 3D-Rendering}

\begin{frame}{Datengrundlage}
    \begin{columns}
        \begin{column}{0.6\linewidth}

            \begin{block}{Ziel}
                Automatische Datenerstellung mit Annotationen
            \end{block}

            \begin{block}{Herangehensweise}
                Synthetische Datenerstellung durch 3D-Rendering
            \end{block}

        \end{column}
        \begin{column}{0.4\linewidth}

            \begin{figure}
                \centering
                \includegraphics[height=0.65\textheight]{imgs/datagen.png}
                \caption{Exemplarisches Resultat der Datenerstellung}
            \end{figure}

        \end{column}
    \end{columns}
\end{frame}

% \begin{frame}{Datenerstellung}
%     \begin{figure}
%         \centering
%         \includegraphics[height=0.7\textheight]{imgs/blender.png}
%         \caption{Blender-Szene}
%     \end{figure}
% \end{frame}

\begin{frame}{Zusammensetzung der Szene}
    \begin{columns}
        \begin{column}{0.6\linewidth}

            Objekte in der 3D-Szene:
            \begin{itemize}\footnotesize
                \setlength\itemsep{0em}
                \item<2-> Dartscheibe
                      \vspace*{-0.1cm}
                      \begin{itemize}\footnotesize
                          \item Prozedurale Materialien
                          \item Simulation von Alter und Abnutzung
                      \end{itemize}
                      \vspace*{-0.15cm}
                \item<3-> Dartpfeile
                      \vspace*{-0.1cm}
                      \begin{itemize}\footnotesize
                          \item Zufällige Zusammensetzung aus unterschiedlichen Bestandteilen
                          \item Positionierung anhand von Heatmaps
                      \end{itemize}
                      \vspace*{-0.15cm}
                \item<4-> Kamera
                      \vspace*{-0.1cm}
                      \begin{itemize}\footnotesize
                          \item Zufällige Position in gegebenem Raum
                          \item Randomisierung von Brennweite, Fokuspunkt, Auflösung, ISO etc.
                      \end{itemize}
                \item<5-> Beleuchtung:
                      \vspace*{-0.1cm}
                      \begin{itemize}\footnotesize
                          \item $>200$ Environment Maps
                          \item Kamerablitz, Deckenbeleuchtung, Spotlight, Ringlicht
                      \end{itemize}
                      \vspace*{-0.15cm}
            \end{itemize}

        \end{column}
        \begin{column}{0.4\linewidth}

            \begin{figure}
                \centering
                \includegraphics[height=0.75\textheight]{imgs/blender.png}
                \caption{3D-Szene}
            \end{figure}

        \end{column}
    \end{columns}

\end{frame}

% \begin{frame}{Datenerstellung}
%     \begin{columns}
%         \begin{column}{0.33\linewidth}
%             \begin{figure}
%                 \centering
%                 \includegraphics[width=\textwidth]{imgs/dartscheibe_a=0.png}
%                 \caption{Dartscheibe mit geringem Alter}
%             \end{figure}
%         \end{column}
%         \begin{column}{0.33\linewidth}
%             \begin{figure}
%                 \centering
%                 \includegraphics[width=\textwidth]{imgs/dartscheibe_a=1.png}
%                 \caption{Dartscheibe mit hohem Alter}
%             \end{figure}
%         \end{column}
%         \begin{column}{0.33\linewidth}
%             \begin{figure}
%                 \centering
%                 \includegraphics[width=\textwidth]{imgs/darts_examples.png}
%                 \caption{Erstellte Dartpfeile}
%             \end{figure}
%         \end{column}
%     \end{columns}
% \end{frame}

\begin{frame}{Ergebnisse: Beispiele}
    \begin{figure}
        \centering
        \includegraphics[height=0.75\textheight]{imgs/example_renders.pdf}
        \caption{Exemplarische Render-Ergebnisse}
    \end{figure}
\end{frame}

\begin{frame}{Ergebnisse: Auswertung}
    \begin{columns}
        \begin{column}{0.7\linewidth}

            Quantitative Auswertung nicht aussagekräftig möglich

            \vspace*{1em}

            \visible<2->{
                Qualitative Auswertung:
                \begin{itemize}
                    \item Realistische Simulation von Dartscheiben + Dartpfeilen
                    \item Variable Darstellungen
                    \item Annähernd realistisches Aussehen
                    \item Nicht fotorealistisch
                \end{itemize}
            }

            \vspace*{1em}

            \visible<3->{
                Korrektheit:
                \begin{itemize}
                    \item Korrekte Normalisierung der Bilder
                    \item Korrekte Annotationen der Dartpfeile
                    \item Metainformationen über Positionen + Parameter
                \end{itemize}
            }

        \end{column}
        \begin{column}{0.3\linewidth}

            \begin{figure}
                \centering
                \includegraphics[height=0.75\textheight]{imgs/exmaple_renders_2.pdf}
                \caption{Auswahl gerenderter Bilder}
            \end{figure}

        \end{column}
    \end{columns}
\end{frame}
