\section{Thema I: Synthetische Datenerstellung durch 3D-Rendering}

\begin{frame}{Datengrundlage}
    \begin{columns}
        \begin{column}{0.6\linewidth}

            \begin{block}{Ziel}
                Vielzahl korrekt annotierter Daten für das Training eines neuronalen Netzes
            \end{block}

            \begin{block}{Herangehensweise}
                Synthetische Datenerstellung durch 3D-Rendering
            \end{block}

        \end{column}
        \begin{column}{0.4\linewidth}

            \begin{figure}
                \centering
                \includegraphics[height=0.65\textheight]{imgs/datagen.png}
                \caption{Exemplarisches Resultat der Datenerstellung}
            \end{figure}

        \end{column}
    \end{columns}
\end{frame}


\begin{frame}{Datenerstellung}
    \begin{figure}
        \centering
        \includegraphics[height=0.6\textheight]{imgs/blender.png}
        \caption{Blender-Szene}
    \end{figure}
\end{frame}

\begin{frame}{Zusammensetzung der Szene}
    Weitreichende Randomisierung der Szene
    \begin{itemize}\footnotesize
        \setlength\itemsep{0em}
        \item Dartscheibe
              \vspace*{-0.1cm}
              \begin{itemize}\footnotesize
                  \item Prozedurale Materialien
                  \item Simulation von Alter und Abnutzung
              \end{itemize}
              \vspace*{-0.15cm}
        \item Dartpfeile
              \vspace*{-0.1cm}
              \begin{itemize}\footnotesize
                  \item Zufällige Zusammensetzung aus unterschiedlichen Bestandteilen
                  \item Positionierung anhand von Heatmaps
              \end{itemize}
              \vspace*{-0.15cm}
        \item Beleuchtung:
              \vspace*{-0.1cm}
              \begin{itemize}\footnotesize
                  \item $>200$ Environment Maps
                  \item Kamerablitz, Deckenbeleuchtung, Spotlight, Ringlicht
              \end{itemize}
              \vspace*{-0.15cm}
        \item Kamera
              \vspace*{-0.1cm}
              \begin{itemize}\footnotesize
                  \item Zufällige Position in gegebenem Raum
                  \item Randomisierung von Brennweite, Fokuspunkt, Auflösung, ISO etc.
              \end{itemize}
    \end{itemize}

\end{frame}

\begin{frame}{Datenerstellung}
    \begin{columns}
        \begin{column}{0.33\linewidth}
            \begin{figure}
                \centering
                \includegraphics[width=\textwidth]{imgs/dartscheibe_a=0.png}
                \caption{Dartscheibe mit geringem Alter}
            \end{figure}
        \end{column}
        \begin{column}{0.33\linewidth}
            \begin{figure}
                \centering
                \includegraphics[width=\textwidth]{imgs/dartscheibe_a=1.png}
                \caption{Dartscheibe mit hohem Alter}
            \end{figure}
        \end{column}
        \begin{column}{0.33\linewidth}
            \begin{figure}
                \centering
                \includegraphics[width=\textwidth]{imgs/darts_examples.png}
                \caption{Erstellte Dartpfeile}
            \end{figure}
        \end{column}
    \end{columns}
\end{frame}

\begin{frame}{Ergebnisse}
    
\end{frame}
