\section{Projektübersicht}

\begin{frame}{Thema der Masterarbeit}
    \begin{columns}
        \begin{column}{0.4\linewidth}

            \begin{figure}
                \centering
                \includegraphics[height=0.7\textheight]{imgs/dartboard.png}
                \caption{Bild einer Dartscheibe}
            \end{figure}

        \end{column}
        \begin{column}{0.6\linewidth}

            \begin{itemize}
                \item automatisches Dart-Scoring von Steeldarts-Runden
                \item Eingabe: Einzelnes Bild der Dartscheibe
                \item Ausgabe: Getroffene Felder + erzielte Punktzahl
            \end{itemize}

        \end{column}
    \end{columns}
\end{frame}

\begin{frame}{Motivation}

    \begin{columns}
        \begin{column}{0.6\linewidth}
            
            \textbf{DeepDarts}: System für automatisches Dartscoring
            \begin{itemize}
                \item verspricht gute Ergebnisse ($\ge 80\%$ Korrektheit der Vorhersagen)
                \item Einsatz neuronaler Netze zur Erkennung von Keypoints (Dartscheibe + Dartpfeile)
            \end{itemize}

        \end{column}
        \begin{column}{0.4\linewidth}
            
            \begin{figure}
                \centering
                \includegraphics[height=0.45\textheight]{imgs/dd_keypoints.pdf}
                \vspace*{-0.2cm}
                \caption{Keypoint-Detection von DeepDarts}
            \end{figure}

        \end{column}
    \end{columns}

    \vspace*{-0.2cm}

    \visible<2->{
        \begin{block}{Ursprünglicher Anstoß}
            Verbesserung der Erkennung von Edge-Cases
        \end{block}
    }

    \visible<3->{
        \begin{block}{Realität}
            Neuinterpretation des gesamten Systems aufgrund von Overfitting
        \end{block}
    }
\end{frame}

\begin{frame}{Projektübersicht}
    \begin{figure}
        \centering
        \includegraphics[height=0.8\textheight]{imgs/ma_project_structure.pdf}
        \caption{Visualisierung der Projektstruktur}
    \end{figure}
\end{frame}

\begin{frame}{Forschungsfragen}
    \begin{itemize}
        \item<1-> In welcher Qualität lassen sich automatisch Daten erstellen?
        \item<2-> Wie zuverlässig kann eine algorithmische Normalisierung der Dartscheiben erarbeitet werden?
        \item<3-> Wie zuverlässig ist ein auf den synthetischen Daten trainiertes neuronales Netz?
        \item<4-> Kann durch die erarbeiteten Systeme eine Verbesserung gegenüber DeepDarts erreicht werden?
    \end{itemize}
\end{frame}