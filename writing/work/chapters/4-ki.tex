% !TeX root = ../main.tex

\chapter{Identifizierung von Dartpfeilen mit neuronalen Netzen}
\label{cha:ki}

An diesem Punkt in der Arbeit sind die Datenerstellung, bei der synthetische Bilddaten generiert werden, sowie die Vorverarbeitung der Bilder, durch die normalisierte Aufnahmen entstehen, erklärt und ausgewertet worden. Der dritte und letzte inhaltliche Themenbereich dieser Thesis umfasst die Lokalisierung der Dartpfeile in normalisierten Aufnahmen von Dartscheiben und das daraus resultierende Scoring der gespielten Runde. Diese Aufgabe wird durch die Verwendung neuronaler Netze angegangen, die auf Grundlage der in \autoref{cha:daten} synthetisch erstellten Daten trainiert werden.

Ziel des Trainings ist es, ein System zu konzipieren, welches in der Lage ist, Dartpfeilspitzen exakt in Bildern zu lokalisieren und die getroffene Feldfarbe auszugeben. Anhand dieser Informationen können die getroffenen Felder und damit einhergehend die erzielte Punktzahl robust abgeleitet werden. In DeepDarts, dem Referenzsystem dieser Thesis, werden die getroffenen Felder ausschließlich anhand der relativen Positionen der Dartpfeile zu den Orientierungspunkten ermittelt. Diese Herangehensweise beruht darauf, dass Orientierungspunkte exakt lokalisiert werden können. Ist das nicht der Fall, werden Dartpfeile möglicherweise fehlerhaft identifiziert. Durch Ermittlung der getroffenen Feldfarbe ist ein gewisser Grad der Robustheit gegen Ungenauigkeiten in der Identifizierung der Orientierungspunkte bei der Normalisierung gegeben.

Die Unterteilung dieses Kapitels folgt dem bereits in \autoref{cha:daten} und \autoref{cha:cv} verwendeten Schema. Begonnen wird mit einer Erklärung von Grundlagen für das Verständnis der weiteren Unterabschnitte. Darauf folgt die Vorstellung der verwendeten Methodik hinsichtlich der Auswahl des Netzwerks und des Trainingsprozesses. Details zur Implementierung des neuronalen Netzes und des Trainings werden im Anschluss daran erläutert. Abschließend werden die Ergebnisse des Trainings und des resultierenden Gesamtsystems anhand unterschiedlicher Metriken ausgewertet.

% !TEX root = ../main.tex

\section{Grundlagen}
\label{sec:ki:grundlagen}

Zum Verständnis dieses Kapitels wird analog zu \autoref{cha:cv} mit Grundlagen zu Konzepten und Begrifflichkeiten begonnen. Diese ermöglichen ein grundlegendes Verständnis, um die in diesem Kapitel eingesetzten Techniken zu verstehen und den weiteren Unterkapiteln folgen zu können.

Begonnen wird mit der Klärung, was neuronale Netze sind und wie sie technisch funktionieren in \autoref{sec:was_nn}. Dabei wird spezifisch auf bestimmte Arten neuronaler Netze und Arten von Vorhersagen, die in dieser Arbeit genutzt werden, eingegangen. Danach folgt ein Überblick über das Training neuronaler Netze in \autoref{sec:was_nn_training} und grundlegende Terminologie in \autoref{sec:nn_terminologie}. Im Anschluss darauf wird de Begriff der Augmentierung in \autoref{sec:was_augmentierung} erklärt und es wird eine für diese Arbeit relevante Netzwerkarchitektur erläutert \autoref{sec:was_yolov8}.

% -------------------------------------------------------------------------------------------------

\subsection{Was sind neuronale Netze?}
\label{sec:was_nn}

Neuronale Netze sind Kern eines spezifischen Bereichs des Machine Learnings, in dem sich auf das Erlernen von Eigenschaften auf Grundlage von Daten fokussiert wird. Durch sie wird die Tür zur Approximation beliebiger Funktionen geöffnet, indem Resultate der zu erlernenden Funktionen gegeben werden. Die Komplexität der Funktionen ist dabei beliebig, sodass die Spanne möglicher Einsatzbereiche von Sinuswellenapproximation bis zur Generierung natürlicher Sprache und Interaktion mit Menschen im Einsatzbereich von neuronalen Netzen liegt.

Hauptsächlich ausschlaggebend für den Erfolg des neuronalen Netzes ist seine Architektur, ihr innerer Aufbau. Herkömmliche neuronale Netze werden aus subsequenten Schichten aufgebaut, die miteinander interagieren und eingehende Daten transformieren. Die Art der Schicht gibt die Spezifikation der Transformationen an, sodass unterschiedliche Schichten die Daten unterschiedlich verarbeiten. Innerhalb dieser Schichten existieren Parameter, die die Arbeitsweise der Transformation steuern.

Die Arbeitsweise eines neuronalen Netzes gleicht der eines Fließbands, dessen Eingabe Rohmaterial ist, welches durch unterschiedliche Verarbeitungsschritte zu einem Endprodukt geformt wird. Die Einstellungen von Stellschrauben der Verarbeitungsschritte bestimmen die Feinabstimmung der Verarbeitungsschritte und agieren analog zu den Parametern einer Netzwerkschicht. Durch wiederholtes Verarbeiten von Eingaben und Adjustierung der Stellschrauben wird ein gewünschtes Endergebnis erzielt. Im Kontext neuronaler Netze wird dieses Identifizieren gewünschter Parameter als Training bezeichnet.

Der Name der neuronalen Netze leitet sich von in Gehirnen vorzufindenden Neuronen ab, die für den Gedankenfluss verantwortlich sind. Das Erlernen von Parametern zur Steuerung von Ausgaben ist der Arbeitsweise von Neuronen nachempfunden. Der ursprüngliche Kern neuronaler Netze war die Nachahmung von Sachverhalten in Gehirnen, jedoch ist es mittlerweile darüber hinaus gewachsen, sodass Rückschlüsse auf neuronale Verhaltensweisen unterschiedlicher Netzwerkschichten nicht zwingend möglich sind.

\subsubsection{Convolutional Neural Networks (CNNs)}
\label{sec:cnns}

Die Art der verarbeiteten Daten in einem neuronalen Netz kann viele Formen annehmen. Insbesondere die Verarbeitung von Bilddaten ist ein großer Themenbereich neuronaler Netze und essenziell für diese Thesis. Bereits in \autoref{sec:was_filterung} wurde die Faltung auf Bilddaten eingeführt, die auf Grundlage von Kerneln funktioniert. Auf dieser Arbeitsweise fußen die Convolutional-Schichten. Die Parameter dieser Schichten bestimmen die Ausprägung eines Kerbels, der auf eingehende Bilddaten angewandt wird. Durch vielfache Hintereinanderreihung von Convolutional-Schichten können inkrementell komplexere Strukturen in Bildern identifiziert und abstrakt festgehalten werden \cite{alexnet}. Neuronale Netze, die auf Convolutional-Schichten aufbauen, werden als Convolutional Neural Networks, kurz CNNs, bezeichnet.

\subsubsection{Klassifizierung und Regression}
\label{sec:klassifizierung_regression}

Ebenso komplex wie Eingabedaten neuronaler Netze können ihre Ausgaben sein. Allgemein lassen sich Ausgaben von neuronalen Netzen in zwei Kategorien einteilen: Klassifikation und Regression\footnote{Neben Klassifikation und Regression sind weitere Arten von Ausgaben möglich, beispielsweise Embeddings von Autoencodern. Für den Kontext dieser Arbeit sind diese Arten der Ausgaben jedoch nicht relevant, weshalb sich auf die gängigen Ausgaben klassischer neuronaler Netze beschränkt wird.} \cite{nn_terminology}.

Bei der Klassifikation werden Datenpunkte in Form von Klassen vorhergesagt. Bei der Klassifizierung von Bildern sind Netzwerkausgaben der Klassen unterschiedlicher Objekte, Lebewesen oder Eigenschaften möglich (Beispielsweise die Beantwortung der Frage \quotes{\textit{Welches} Tier ist in diesem Bild zu sehen?}). Ebenso ist eine binäre Klassifikation hinsichtlich der Existenz bestimmter Sachverhalte üblich (Beispielsweise die Beantwortung der Frage \quotes{\textit{Existiert} eine Katze in diesem Bild?}). Die Ausgabe von Netzwerken geschieht für diese Arten der Fragen typischerweise in Form von Vektoren, die diese Kategorien durch One-Hot-Encoding (oder 1-of-n-Encoding) darstellen \cite{one_hot_encoding}. Dabei ist jeder Kategorie ein Eintrag im Vektor zugeordnet; die Größe der Zahlenwerte geben die Ausgaben des Netzes für die jeweiligen Kategorien an.

Konträr zur Klassifikation diskreter Gegebenheiten ist die Vorhersage kontinuierlicher Werte als Ausgabe eines neuronalen Netzes möglich, die Regression. Beispiele für Ausgaben einer Regression beinhalten Funktionswerte oder Koordinaten. Ziel einer Regression ist es, konkrete Zahlenwerte vorherzusagen. Sofern eine Begrenzung der ausgegebenen Werte möglich ist, ist die Normalisierung von Daten in die Intervalle $[0, 1]$ oder $[-1, 1]$ üblich. Der Hintergrund dieser Normalisierung liegt in der Arbeitsweise der Netzwerkschichten und wird üblicherweise in einem Nachverarbeitungsschritt nach der Vorhersage des neuronalen Netzes wieder umgekehrt.

In dieser Thesis werden sowohl binäre als auch klassenbezogene Klassifikation sowie Regression verwendet. Die binäre Klassifikation wird zur Identifizierung von Dartpfeilen genutzt, klassenbezogene Klassifikation zur Identifizierung von Feldfarben unter Dartpfeilen und Regression wird genutzt, um die exakten Positionen der Dartpfeile auf der Dartscheibe darzustellen.

% -------------------------------------------------------------------------------------------------

\subsection{Training Neuronaler Netze}
\label{sec:was_nn_training}

Das Training neuronaler Netze kann abhängig von seinen Ausgaben auf unterschiedliche Arten verlaufen. In dieser Arbeit wird Supervised Learning verwendet, bei welchem Eingabedaten mit ihren zugehörigen Ausgaben gegeben sind. Das Netzwerk erlernt auf der Grundlage dieser Daten PArameter, durch die die Ausgaben zu den jeweiligen Eingaben ableiten. Weitere Methoden zum Training neuronaler Netze sind Unsupervised Learning und Reinforcement Learning. Bei Unsupervised Learning liegen lediglich Eingabedaten vor und die Ausgaben werden von dem Netzwerk identifiziert. Diese Art des Lernens wird beispielsweise bei Clustering von Datenpunkten oder Findung von Wort-Embeddings verwendet. Reinforcement Learning wird genutzt, um einem System das Agieren in einer Umgebung zu ermöglichen und basiert auf Belohnung gewünschter Ereignisse und Bestrafung nicht gewünschter Ereignisse \cite{nn_terminology}. Im Kontext dieser Arbeit lediglich Supervised Learning angewendet, um das neuronale Netz zu trainieren.

Supervised Training basiert auf der Korrektur von Fehlern getätigter Vorhersagen des neuronalen Netzes. Als Forward Pass eines neuronalen Netzes wird die Vorhersage von Daten bezeichnet, durch die eine Ausgabe des Netzes erzeugt wird. Der Fehler von Vorhersagen wird durch eine Metrik gemessen, die hinsichtlich der Parameter des Netzwerks differenzierbar ist. Diese Metrik wird als Loss-Funktion\footnote{Alternativ wird diese Funktion auch als Cost- oder Error-Funktion bezeichnet. In dieser Arbeit wird die Terminologie der Loss-Funktion verwendet.} bezeichnet, der Fehler des Netzwerks als Loss. Durch die Differenzierbarkeit ist ihr Gradient bekannt und kann genutzt werden, um lokale Minima zu identifizieren. Je geringer der Fehler ist, desto korrekter sind die Vorhersagen des Systems. Das Erreichen eines Minimums der Loss-Funktion ist das Ziel des Trainings. Die Identifizierung der Parameterangleichungen zur Annäherung an ein Minimum der Loss-Funktion geschieht durch einen Prozess, der als Backpropagation bezeichnet wird. Während der Backpropagation werden iterativ Parameter der Netzwerkschichten angepasst, um die Vorhersage für die gegebenen Daten derart zu korrigieren, dass ein folgender Forward Pass auf den selben Daten einen geringeren Loss mit sich zieht.

Die Implementierung der Backpropagation und die Umsetzung der Parameteranpassung geschieht durch Optimierungsalgorithmen. Die Arbeitsweise dieser Algorithmen ist grundlegend ähnlich hinsichtlich der Eingaben des Problems, in den konkreten Arbeitsweisen unterscheiden sie sich jedoch. Die Auswahl eines passenden Optimierungsalgorithmus ist abhängig von der zugrundeliegenden Aufgabe und der Netzwerkarchitektur.

\todo{noch einmal gegenlesen und Quellen suchen}

% -------------------------------------------------------------------------------------------------

\subsection{Terminologie}
\label{sec:nn_terminologie}

Der Themenbereich der neuronalen Netze umfasst eine Vielzahl von Konzepten und Begriffen. Dieses Unterkapitel gibt einen Überblick über die zentralen Begriffe, die in dieser Arbeit von Bedeutung sind \cite{nn_terminology}.

\paragraph{Trainingsdaten}

Dreh- und Angelpunkt des Trainings neuronaler Netze sind die Trainingsdaten. Sie werden genutzt, um die Parameter des zu trainierenden neuronalen Netzes anzupassen, indem getroffene Vorhersagen bewertet werden. Mit dieser Fehlerbewertung durch die Loss-Funktion werden die Parameter während der Backpropagation angepasst. Trainingsdaten haben üblicherweise die größte Kardinalität aller für das Training und die Evaluation verwendeten Datensätze.

Um ein effektives Training eines neuronalen Netzes zu gewährleiten, ist die Wahl der Trainingsdaten essenziell. Da der Trainingserfolg eines neuronalen Netzes abhängig von den Trainingsdaten ist und die durch die Parameter erlernten Strukturen in den Daten nicht bekannt sind, ist eine möglichst uniforme Abdeckung der zugrundeliegenden Daten wichtig. Jegliche Verzerrungen der Datenlage wird potenziell von dem neuronalen Netz erlernt und kann zu einer fehlerhaften Inferenz auf neuen Daten führen.

\paragraph{Validierungsdaten}

Validierungsdaten werden ebenso wie die Trainingsdaten während des Trainings eines neuronalen Netzes genutzt, konträr zu Trainingsdaten haben diese jedoch keinen Einfluss auf den Trainingserfolg. In regelmäßigen Intervallen wird der aktuelle Stand der Netzwerkparameter auf den Validierungsdaten ausgewertet, um Einblicke in die Performance des Netzwerks auf Daten zu gewinnen, die nicht für das Training verwendet wurden. Durch diese Daten können Rückschlüsse auf die Fähigkeit des neuronalen Netzes gezogen werden, das Gelernte auf neue Daten zu übertragen und die zugrundeliegenden Strukturen der Daten zu generalisieren. Die strikte Separierung von Trainings- und Validierungsdaten ist dabei obligatorisch, um eine Verzerrung der Generalisierbarkeit zu vermeiden \cite{nn_terminology}.

Die Wahl der Validierungsdaten unterliegt den gleichen Voraussetzungen wie den Trainingsdaten, um Verzerrungen in die Einblicke der Netzwerk-Performance zu vermeiden. Darüber hinaus sollten Validierungsdaten jedoch auf eine Art und Weise gewählt werden, die nicht zu große Ähnlichkeiten zu den Trainingsdaten aufweist, da diese Nähe der Daten ebenfalls eine Verzerrung der Datenlage mit sich ziehen kann, sofern keine uniforme Verteilung der Trainingsdaten vorliegt.

\paragraph{Testdaten}

Nach dem Training eines neuronalen Netzes werden Testdaten genutzt, um die Netzwerk-Performance zu evaluieren. Während Trainings- und Validierungsdaten Einfluss auf den Verlauf des Trainings nehmen, werden Testdaten genutzt, um einen unabhängigen Einblick in die Netzwerkperformance nach Beendigung des Trainings zu gewinnen \cite{nn_terminology}. Durch Testdaten wird die Inferenz des trainierten Netzes auf unbekannten Daten simuliert, wodurch eine objektive Abschätzung der Generalisierbarkeit ermöglicht wird.

Für die Wahl der Testdaten sind die selben Voraussetzungen zu beachten, die für Trainings- und Validierungsdaten gelten, um einer Verzerrung der Datenlage vorzubeugen. Die Wahl von Trainings-, Validierungs- und Testdaten spielt für die Auswertung des neuronalen Netzes dieser Thesis eine wichtige Rolle.

\paragraph{\Acl{ood}-Training}

Dass die für das Trainings verwendeten Daten einen universellen Überblick über die gesamte Datenlage geben, ist häufig nicht möglich. Verzerrungen der Datenlage sind -- gewollt oder ungewollt -- in den meisten Fällen nicht zu umgehen. Weichen die Trainingsdaten jedoch bewusst von den Validierungs- und Testdaten ab, spricht man von \Acf{ood}-Training. Das neuronale Netz wird auf Daten trainiert, die daher einer anderen Verteilung entsprechen als der zu erwartenden Daten für die Inferenz des Netzes.

\paragraph{Under- und Overfitting}

Die Validierungsdaten eines Trainings werden verwendet, um den Erfolg eines Trainings zu beurteilen. Während Optimierungsalgorithmen darauf ausgelegt sind, den Trainings-Loss zu minimieren, ist es möglich, dass der Validierungs-Loss von diesem abweicht. Befindet sich der Wert des Validierungs-Loss signifikant über dem Trainings-Loss, spricht man von Underfitting \cite{nn_terminology}. In dieser Situation ist das neuronale Netz nicht in der Lage, das gelernte auf neue Daten anzuwenden, da es noch nicht ausreichend trainiert wurde. Fallen Trainings- und Validierungs-Loss zeitweise gleichermaßen, gefolgt von einem Anstieg des Validierungs-Losses, wird dies als Overfitting bezeichnet. In dieser Situation werden Vorhersagen auf den Trainingsdaten besser, jedoch verliert das neuronale Netz die Fähigkeit der Generalisierbarkeit gelernter Strukturen auf neue Daten. Dies ist analog zum Auswendiglernen der Trainingsdaten und dem Verlernen der zugrundeliegenden Strukturen, die die Daten ausmachen.

\todo{Gegenlesen und ggf. Quellen finden}

% -------------------------------------------------------------------------------------------------

\subsection{Was ist Augmentierung?}
\label{sec:was_augmentierung}

Für ein robustes Training eines neuronalen Netzes ist eine große und möglichst umfangreiche Datenlage notwendig. Zur Vergrößerung der vorhandenen Datenlage kann Augmentierung genutzt werden. Als Augmentierung wird die Abänderung der Daten beschrieben, die die Integrität der Daten aufrechterhält. So hat das Hinzufügen von Rauschen auf ein Bild geringfügig bis keinen Einfluss auf die Semantik des Bildes während die zugrundeliegenden Daten stark stark beeinflusst werden können. Ein robust trainiertes neuronales Netz ist in der Lage, sowohl die Bedeutung des unveränderten als auch die des augmentierten Bildes korrekt zu bestimmen. Durch Variation der Eingabedaten kann die Quantität der Datenlage künstlich vervielfacht werden, ohne Einbußen in ihrer Qualität zu erfahren.

\todo{Gegenlesen, Quellen finden}

% -------------------------------------------------------------------------------------------------

\subsection{Die YOLOv8-Architektur}
\label{sec:was_yolov8}

- Anwendungsbereich
- Multi-Scale-Output
- Bounding Boxes
- Non-Maximum-Suppression

\todo{YOLOv8-Architektur beschreiben}

% -------------------------------------------------------------------------------------------------

% !TEX root = ../main.tex

\section{Methodik}
\label{sec:ki:methodik}

Methodik hier.

% -------------------------------------------------------------------------------------------------
\subsection{Wahl des Netzwerks}
\label{sec:netzwerk_wahl}

\todo{Netzwerkwahl beschreiben}

\subsubsection{Warum YOLOv8?}
\label{sec:warum_yolov8}

\todo{Warum denn nun?}

\subsubsection{Adaption der Netzwerkarchitektur}
\label{sec:yolo_adaption}

\paragraph{Multi-Scale-Output}

\paragraph{Bounding Boxes}

\paragraph{Dreiteilung des Outputs}

\begin{itemize}
    \item Existenz
    \item Position
    \item Klasse
    \item alles 3x pro Zelle
\end{itemize}

\todo{Adaption erklären}
% !TeX root = ../main.tex

\section{Implementierung}
\label{sec:ki:implementierung}

Nach der Darstellung der Methodik wird in diesem Abschnitt auf Details zur Implementierung eingegangen. Dieser Abschnitt ist unterteilt in zwei Unterabschnitte: Im ersten Unterabschnitt werden konkrete Details zur Implementierung der YOLOv8*-Architektur thematisiert, der zweite Unterabschnitt befasst sich mit den spezifischen Aspekten des Trainings.

% -------------------------------------------------------------------------------------------------

\subsection{Implementierung der YOLOv8*-Architektur}
\label{sec:yolov8_implementierung}

Vortrainierte neuronale Netze der YOLO-Familie werden als Modelle veröffentlicht, die mit dem Framework PyTorch erstellt sind. Für diese Thesis wird TensorFlow als Framework für die Erstellung und das Training neuronaler Netze verwendet, welches nicht reibungslos mit PyTorch vereinbar ist. Die Frameworks arbeiten auf unterschiedliche Arten, wodurch eine Übersetzung der verwendeten Schichten und der vortrainierten Gewichte zwischen diesen lediglich bedingt möglich ist. Durch die Abwandlung von YOLOv8* zu YOLOv8 ist eine eigene Implementierung notwendig und eine Einbettung vortrainierter Gewichte in diese Architektur ist nicht trivial möglich. Auf Grundlage der offiziellen Dokumentation sowie des Quelltexts und Konfigurationsdateien wurde die YOLOv8-Architektur in TensorFlow übersetzt und implementiert. Atomare Bestandteile wie Convolution, Batch-Normalisierung, Pooling-Operationen und Aktivierungsschichten können analog von PyTorch zu TensorFlow übertragen werden. Schichten, die nicht in TensorFlow verfügbar sind -- beispielsweise die Split-Operation -- wurden durch Einbindung eigener Schichten implementiert.

Nachdem alle grundlegenden Bestandteile verfügbar waren, wurden aus diesen kombinierte Netzwerkbestandteile zusammengesetzt, dargestellt in \autoref{img:yolov8_parts}. Die Dimensionierungen der jeweiligen Schichten sowie die verwendeten Parameter konnten aus der Dokumentation der Architektur entnommen und analog übertragen werden. Nachdem alle Bestandteile der Architektur implementiert waren, wurde eine generische Implementierung der Architektur vorgenommen, in der die bereitgestellten Parameter für unterschiedliche Größenvariationen von YOLOv8 mit einbezogen wurden. Durch den Vergleich der Anzahl der Netzwerkparameter unterschiedlicher Größenkonfigurationen sowie der Analyse des Netzwerkaufbaus durch Verbindungen der Schichten konnte sichergestellt werden, dass die TensorFlow-Implementierung von YOLOv8 der vorgestellten Architektur entsprach.

Nachdem die grundlegende Architektur in TensorFlow verfügbar war, wurde sie durch eigene Adaptionen erweitert. Eine grundlegende Änderung ist das Hinzufügen von Dropout-Schichten, welche nicht in der Dokumentation erwähnt sind, jedoch die Stabilität des Trainings erhöhen und die Wahrscheinlichkeit des Overfittings senken. Eine weitere Adaption ist das Hinzufügen von Transition-Blocks, welche eine residuale Dense-Schicht umfassen. In einem Transition-Block wird der Eingabetensor durch einen Tensor moduliert, der den globalen Kontext des Eingabetensors durch eine Dense-Schicht einfängt.

% -------------------------------------------------------------------------------------------------

\subsection{Training von YOLOv8* zur Identifizierung von Dartpfeilen}
\label{sec:training}

In diesem Unterabschnitt wird das Training des neuronalen Netzes thematisiert. Es wird begonnen mit der verwendeten Infrastruktur und Rahmenbedingungen des Trainings. Danach folgt eine detaillierte Betrachtung der verwendeten Augmentierungsparameter. Zuletzt wird der Verlauf des Trainings erläutert.

\subsubsection{Infrastruktur und Rahmenbedingungen}

Das Training von YOLOv8* wurde auf einer NVIDIA GeForce RTX 4090 aus einem TensorFlow-Docker-Container ausgeführt. Es wurde eine Batch-Size von $32$ verwendet mit dem AdamW-Optimizer und einer dynamischen Learning Rate, wie in \autoref{sec:dynamisches_training} erläutert. Für das Training wurden $24.960$ Trainings- und $672$ Validierungsdaten verwendet. Die Trainingsdaten setzen sich zusammen aus $24.576$ generierten Daten ($20.480$ Daten auf Grundlage regulärer Heatmaps und $4.096$ Daten auf Grundlage von Multiplier-Heatmaps), $256$ Daten des DeepDarts-$d_1$-Trainingssatzes und $128$ manuell aufgenommenen Daten. Die Validierungsdaten sind zusammengesetzt aus $256$ synthetischen Daten, $256$ Daten der DeepDarts-$d_2$-Trainingsdaten sowie $160$ manuell aufgenommen Daten. Die Trainings- und Validierungsdaten unterliegen einer strikten thematischen Trennung, sodass die realen Datensätze in unterschiedlichen Umgebungen aufgenommen wurden. Auf diese Weise wird Darstellung der Generalisierbarkeit durch den Validation-Loss nicht durch eine vorteilhafte Datenlage verzerrt.

\subsubsection{Augmentierungsparameter}

Die Daten werden vektorisiert als TensorFlow Datasets eingelesen, wodurch ein effizientes und parallelisiertes Laden der Daten ermöglicht wird. Ein Schritt des Einlesens der Daten ist die Augmentierung, welche dynamisch auf jedes Bild der Trainingsdaten mit zufälligen Parametern angewendet wird. Die Farbkanäle für rot, grün und blau werden unabhängig voneinander mit einem zufälligen, uniform gewählten Gewicht $A_{\text{cha}, \{r,g,b\}} \in [0,\!5\,..\,1]$ moduliert. Die Helligkeit des Bildes wird durch den Parameter $A_b$ bestimmt, der die mittlere Helligkeit des Bildes zufällig uniform verteilt in dem Intervall $[\text{min}(0.1, b_\text{img}-b_\text{adj}), b_\text{img} + b_\text{adj}]$ setzt, wobei $b_\text{img}$ die mittlere Helligkeit des Bildes ist und $b_\text{adj} = 0,\!03$ die maximale Änderung der Helligkeit vorgibt. Der Kontrast des Bildes wird durch den zufällig uniform gewählten Parameter $A_\text{cont} \in [0,\!7\,..\,1,\!3]$ bestimmt und die Farbsättigung wird um einen ebenfalls uniform verteilten Faktor $A_\text{sat} \in [0,\!8\,..\, 1,\!2]$ moduliert. Das Hinzufügen von normalverteiltem Rauschen auf jeden Pixel jedes Farbkanals findet mit einer Normalverteilung $\sigma_\text{noise} = 0,\!15$ statt. Hinsichtlich der transformativen Augmentierungsparameter wird das Bild horizontal und vertikal mit einer Wahrscheinlichkeit von je $50\,\%$ gespiegelt. Eine Rotation um den Mittelpunkt des Bildes erfolgt zufällig um einen Winkel aus der Menge $A_\text{rot} \in \{i \in \mathbb{N} ~\vert~ i = n \cdot 18\degree, n \in \mathbb{N}_{<20}\}$. Zuletzt erfolgt eine normalverteilte Translation des Bildes $A_\text{trans} \in \mathbb{R}^2$ mit einer Standardabweichung von $\sigma_\text{trans} = 5\,\text{px}$.
\nomenclature{$A_{\text{cha}, \{r,g,b\}}$}{Augmentierungsgewichte für rot-, grün- unb blau-Kanäle.}
\nomenclature{$A_\text{cont}$}{Augmentierungsgewicht der Kontraständerung.}
\nomenclature{$A_\text{sat}$}{Augmentierungsgewicht der Sättigungsänderung.}
\nomenclature{$A_\text{rot}$}{Augmentierungsgewicht der Rotation.}
\nomenclature{$b_\text{img}$}{Mittlere Helligkeit eines Bildes.}
\nomenclature{$b_\text{adj}$}{Maximale Helligkeitsänderung der Augmentierung.}
\nomenclature{$\sigma_\text{noise}$}{Standardabweichung der Rausch-Augmentierung.}
\nomenclature{$\sigma_\text{trans}$}{Standardabweichung der Translations-Augmentierung.}

% !TeX root = ../main.tex

\section{Ergebnisse}
\label{sec:ki:ergebnisse}

Im Zentrum dieses Abschnitts steht die Auswertung des in den vorherigen Abschnitten erläuterten neuronalen Netzes und seines Trainings. Es werden sowohl absolute Ergebnisse betrachtet sowie in Relation gesetzte Ergebnisse hinsichtlich der Referenzsysteme DeepDarts-$d_1$ und DeepDarts-$d_2$. Die Auswertung geschieht auf Grundlage unterschiedlicher Metriken, die teilweise spezifisch für diese Auswertung konzipiert sind, sowie den \ac{pcs}, welcher für die Auswertung von DeepDarts entwickelt wurde. Durch erneutes Aufgreifen des \ac{pcs} ist einerseits die Verifizierung der korrekten Verwendung der DeepDarts-Systeme überprüfbar, andererseits kann einer Verzerrung der Darstellung von Ergebnissen durch eine voreingenommene Wahl von Metriken entgegengewirkt werden.

Dieser Abschnitt ist unterteilt in mehrere Bereiche. Zuerst werden die Metriken und die zur Auswertung verwendeten Datensätze erläutert. Anschließend werden die Ergebnisse der Auswertung dieser Metriken auf den beschriebenen Daten dargestellt. Abschließend werden die Ergebnisse des \ac{pcs} dargestellt, anhand derer Rückschlüsse auf die Auswertung von DeepDarts geschlossen werden können.

% -------------------------------------------------------------------------------------------------

\subsection{Metriken}
\label{sec:ki_metriken}

Für die Auswertung der Genauigkeit der jeweiligen Systeme werden mehrere Metriken verwendet. Zur Auswertung von DeepDarts wurde der in \autoref{sec:deepdarts} beschriebene \ac{pcs} verwendet, um die relative Anzahl korrekt vorhergesagter Daten zu bestimmen. Diese Metrik ist jedoch dahingehend fehleranfällig, dass False Positives zustande kommen können. \ac{pcs} misst die Fähigkeit, die korrekte Punktzahl vorherzusagen, statt der Fähigkeit, die Dartpfeile korrekt zu ermitteln. Um einen Einblick in die Fähigkeiten der Systeme zu gewinnen, werden in dieser Arbeit drei weitere Metriken verwendet: Existenz-Metrik $\mu_\text{xst}$, Klassen-Metrik $\mu_\text{cls}$ und Positions-Metrik $\mu_\text{pos}$.
\nomenclature{$\mu_\text{xst}$}{Existenz-Metrik zur Bestimmung korrekter Anzahl der Dartpfeile je Bild.}
\nomenclature{$\mu_\text{cls}$}{Klassen-Metrik zur Bestimmung korrekter Klassen der Dartpfeile je Bild.}
\nomenclature{$\mu_\text{pos}$}{Positions-Metrik zur Bestimmung der Abweichungen der Dartpfeilpositionen.}

\subsubsection{Existenz-Metrik $\mu_\text{xst}$}

Mit dieser Metrik wird bestimmt, ob die korrekte Anzahl der Dartpfeile bestimmt wird. $\mu_\text{xst}$ ist definiert als:
\begin{equation*}
    \mu_\text{xst} = \frac{1}{N} \sum_{i=1}^{N}1 - \vert \frac{1}{3} \cdot ( N_\text{Dart, i} - \widehat{N}_\text{Dart, i} ) \vert
\end{equation*}
\nomenclature{$N_\text{Dart, i} \in \mathbb{N}$}{Anzahl vorhandener Dartpfeile in dem Bild mit Index $i$.}
\nomenclature{$\widehat{N}_\text{Dart, i} \in \mathbb{N}$}{Anzahl vorhergesagter Dartpfeile in dem Bild mit Index $i$.}
In dieser Formel stehen $N_\text{Dart, i} \in \mathbb{N}$ und $\widehat{N}_\text{Dart, i} \in \mathbb{N}$ für die Anzahl vorhandener und vorhergesagter Dartpfeile je Bild mit Index $i$. Anhand des Werts von $\mu_\text{xst}$ wird ermittelt, wie die Anzahl der zu ermittelnden Dartpfeile zu der Vorhersage der Dartpfeile vergleichbar ist. Ohne weiteren Kontext gibt diese Metrik keinen Aufschluss über die Korrektheit der Vorhersagen der Dartpfeile aus. Eine Korrelation zwischen Existenz und Position von Dartpfeilen wird in dieser Metrik nicht festgehalten.

\subsubsection{Klassen-Metrik $\mu_\text{cls}$}

Die Klassen-Metrik $\mu_\text{cls}$ betrachtet die Korrektheit der vorhergesagten Klassen der Dartpfeile. Für diese Metrik wird ein Matching vorgenommen, anhand dessen die Klassen vorhergesagter Dartpfeile mit den Klassen existierender Dartpfeile verglichen werden:
\begin{equation*}
    \mu_\text{cls} = \frac{1}{N}\sum_{i=1}^{N} \frac{1}{3} \cdot N_{K, \text{correct}, i}
\end{equation*}
\nomenclature{$N_{K, \text{correct}, i} \in \mathbb{N}$}{Anzahl korrekt vorhergesagter Klassen in dem Bild mit Index $i$.}
$N_{K, \text{correct}, i} \in \mathbb{N}$ beschreibt die Anzahl korrekt vorhergesagter Klassen in dem Bild mit Index $i$. Das Matching der Klassen wird mit einem Greedy-Algorithmus durchgeführt, in welchem zusätzlich erkannte Klassen verworfen werden. Diese Ungenauigkeit der Metrik wird durch die Kombination mit der Metrik $\mu_\text{xst}$ ausgeglichen.

\subsubsection{Positions-Metrik $\mu_\text{pos}$}

Ziel dieser Metrik ist es, die durchschnittlichen Abweichungen der Dartpfeilspitzen einzufangen. Die Dartpfeilspitzen werden analog zu $\mu_\text{cls}$ durch ein Greedy-Matching korreliert, indem die vorhandenen und vorhergesagten Dartpfeilpositionen mit den je geringsten Abständen zueinander gepaart werden, sofern sie noch nicht gepaart wurden. Diese Metrik gibt einen Einblick in die Präzision, mit welcher Dartpfeilspitzen erkannt werden. Der Wert von $\mu_\text{pos}$ ist definiert durch:
\begin{equation*}
    \mu_\text{pos} = \frac{1}{N} \sum_{i=1}^{N} \sum_{d=1}^{3} \left\Vert P_{i, d} - \widehat{P}_{i, d} \right\Vert _2
\end{equation*}
\nomenclature{$P_{i, d} \in \mathbb{R}^2$}{Position des Dartpfeils mit Index $d$ in Bild $i$.}
\nomenclature{$\widehat{P}_{i, d} \in \mathbb{R}^2$}{Vorhergesagte Position des Dartpfeils mit Index $d$ in Bild $i$.}
$P_{i, d} \in \mathbb{R}^2$ und $\widehat{P}_{i, d} \in \mathbb{R}^2$ sind die annotierten und vorhergesagten Positionen der Dartpfeile mit dem Index $d$ in dem Bild mit dem Index $i$. Vorhergesagte Positionen für nicht vorhandene Dartpfeile haben keinen Einfluss auf diese Metrik. Dieser Aspekt wird analog zu $\mu_\text{cls}$ durch die Auswertung von $\mu_\text{xst}$ abgebildet.

% -------------------------------------------------------------------------------------------------

\newpage
\subsection{Datenquellen und Herangehensweise}
\label{sec:nn_datenquellen}

\begin{table}
    \centering
    \small
    \begin{tabular}{r||c|cc|cc|cc}
        \multirow{2}{*}{Datenquelle} & \multirow{2}{*}{\begin{tabular}[c]{@{}c@{}}Gerenderte\\ Bilder\end{tabular}} & \multicolumn{2}{c|}{Reale Bilder} & \multicolumn{2}{c|}{DeepDarts-$d_1$} & \multicolumn{2}{c}{DeepDarts-$d_2$}                             \\
                                     &                                                                              & Validierung                       & Test                                 & Validierung                         & Test & Validierung & Test \\ \hline
        Anzahl Bilder                & 2048                                                                         & 125                               & 55                                   & 1000                                & 2000 & 70          & 150
    \end{tabular}
    \caption{Datenquellen für die Auswertung der Dartscheibenentzerrungen.}
    \label{tab:datenquellen_nn}
\end{table}

Für die Auswertung der Performance des neuronalen Netzes werden Daten unterschiedlicher Quellen verwendet, aufgelistet in \autoref{tab:datenquellen_nn}. Analog zur Auswertung der algorithmischen Normalisierung der Bilder in \autoref{sec:cv:ergebnisse} werden dieselben gerenderten Bilder sowie die Validierungs- und Testdaten von DeepDarts einbezogen. Zusätzlich wurden Bilder von Darts-Runden manuell aufgenommen und händisch annotiert, um weitere unabhängige Daten einzubinden. Diese sind aufgeteilt in Daten, die zur Validierung während des Trainings verwendet wurden, und Daten, die ausschließlich zum Testen verwendet werden.

Die in den folgenden Unterabschnitten dargestellten Auswertungen stellen die Ergebnisse des gesamten Systems unter Einbezug der Normalisierung dar. Hintergrund dieses Zusammenschlusses der Verarbeitungsschritte ist die Vergleichbarkeit mit DeepDarts, in welchem die Verarbeitungsschritte miteinander verschmolzen und ebenfalls als Gesamtsystem evaluiert sind. Es werden das für diese Arbeit trainierte System sowie die für DeepDarts trainierten Systeme ausgewertet, um einen objektiven Vergleich der Performances darzustellen und einen Vergleich der Systeme zu ermöglichen.

% -------------------------------------------------------------------------------------------------

\subsection{Auswertung der Existenz-Metrik \texorpdfstring{$\mu_\text{xst}$}{µ\_xst}}
\label{sec:auswertung_xst}

\pgfplotstableread[col sep=comma]{
    system,         Render-Daten,   Testdaten, Val-Daten,  d1-val,     d1-test,    d2-val,     d2-test
    Thesis,         88.31,          78.79,      81.87,      65.24,      66.46,      74.76,      87.33
    DeepDarts-d1,   11.08,          21.21,      18.13,      65.23,      66.45,      25.24,      12.0
    DeepDarts-d2,   11.85,          28.48,      52.8,       46.63,      63.57,      74.76,      77.0
}\NNXst

\begin{figure}
    \centering
    \begin{tikzpicture}
        \begin{axis}[
                width=0.8\textwidth,
                height=6cm,
                ybar,
                ymin=0,
                ymax=100,
                bar width=0.3cm,
                enlarge x limits=0.25,
                ylabel={$\mu_\text{xst}$ [\%]},
                symbolic x coords={Thesis, DeepDarts-d1, DeepDarts-d2},
                xtick={Thesis,DeepDarts-d1,DeepDarts-d2},
                legend style={at={(1.02,1.00)}, anchor=north west},
                every axis plot/.append style={
                        single ybar legend,
                    },
            ]
            \addplot+[draw=black, fill=bar_1]    table[x=system,y=Render-Daten]  {\NNXst};
            \addplot+[draw=black, fill=bar_2]    table[x=system,y=Val-Daten]     {\NNXst};
            \addplot+[draw=black, fill=bar_2!60] table[x=system,y=Testdaten]    {\NNXst};
            \addplot+[draw=black, fill=bar_3]    table[x=system,y=d1-val]        {\NNXst};
            \addplot+[draw=black, fill=bar_3!60] table[x=system,y=d1-test]       {\NNXst};
            \addplot+[draw=black, fill=bar_4]    table[x=system,y=d2-val]        {\NNXst};
            \addplot+[draw=black, fill=bar_4!60] table[x=system,y=d2-test]       {\NNXst};
            \legend{Render-Daten, Val-Daten, Testdaten, $d_1$-val, $d_1$-test, $d_2$-val, $d_2$-test}
        \end{axis}
    \end{tikzpicture}
    \caption{Auswertung von $\mu_\text{xst}$ der Systeme auf unterschiedlichen Datenquellen.}
    \label{fig:nn_xst}
\end{figure}

Die Auswertungen der Existenz-Metrik $\mu_\text{xst}$ sind in \autoref{fig:nn_xst} dargestellt; die Abbildung zeigt die Auswertungen der Metrik der Systeme auf den jeweiligen Datenquellen. Das neuronale Netz dieser Thesis konnte Werte zwischen $65\,\%$ und $88\,\%$ erzielt werden. Die Verteilung der Auswertungsdifferenzen ist nahezu gleichverteilt über die unterschiedlichen Datenquellen. Die größte Genauigkeit wird auf den gerenderten Daten erzielt, während die geringsten Metrik-Werte auf den Validierungsdaten von DeepDarts-$d_1$ erzielt werden.

Die Auswertung von DeepDarts-$d_1$ zeigt eine signifikante Korrelation zwischen Datenquelle und Metrik-Auswertung. Die Auswertung auf den dem System zugeordneten Daten fällt mit durchschnittlich $66\,\%$ weitaus besser aus als auf unabhängigen Quellen, die nicht Teil des Trainings oder der Auswertung des Systems waren. Auf diesen Daten wird eine durchschnittliche Auswertung von $17\,\%$ erzielt.

DeepDarts-$d_2$ zeigt eine weitaus bessere Auswertung als DeepDarts-$d_1$, sodass auf DeepDarts-$d_1$-Daten durchschnittlich $56\,\%$ der Existenzen identifiziert werden und bei DeepDarts-$d_2$-Daten durchschnittlich $76\,\%$. Die Ergebnisse auf den für diese Auswertung aufgenommenen reale Daten liegen bei $20\,\%$ und bei den gerenderten Daten sind es lediglich $11\,\%$.

% -------------------------------------------------------------------------------------------------

\subsection{Auswertung der Klassen-Metrik \texorpdfstring{$\mu_\text{cls}$}{µ\_cls}}
\label{sec:auswertung_cls}

\pgfplotstableread[col sep=comma]{
    system,         Render-Daten,   Testdaten, Val-Daten,  d1-val,     d1-test,    d2-val,     d2-test
    Thesis,         77.07,          66.55,      69.01,      56.54,      56.39,      63.81,      71.55
    DeepDarts-d1,   11.08,          21.21,      18.13,      61.83,      64.38,      25.24,      12
    DeepDarts-d2,   11.13,          20.61,      10.13,      29.87,      29.25,      71.43,      82.67
}\NNCls

\begin{figure}
    \centering
    \begin{tikzpicture}
        \begin{axis}[
                width=0.8\textwidth,
                height=6cm,
                ybar,
                ymin=0,
                ymax=100,
                bar width=0.3cm,
                enlarge x limits=0.25,
                ylabel={$\mu_\text{cls}$ [\%]},
                symbolic x coords={Thesis, DeepDarts-d1, DeepDarts-d2},
                xtick={Thesis,DeepDarts-d1,DeepDarts-d2},
                legend style={at={(1.02,1.00)}, anchor=north west},
                every axis plot/.append style={
                        single ybar legend,
                    },
            ]
            \addplot+[draw=black, fill=bar_1]     table[x=system,y=Render-Daten]  {\NNCls};
            \addplot+[draw=black, fill=bar_2]     table[x=system,y=Val-Daten]     {\NNCls};
            \addplot+[draw=black, fill=bar_2!60]  table[x=system,y=Testdaten]    {\NNCls};
            \addplot+[draw=black, fill=bar_3]     table[x=system,y=d1-val]        {\NNCls};
            \addplot+[draw=black, fill=bar_3!60]  table[x=system,y=d1-test]       {\NNCls};
            \addplot+[draw=black, fill=bar_4]     table[x=system,y=d2-val]        {\NNCls};
            \addplot+[draw=black, fill=bar_4!60]  table[x=system,y=d2-test]       {\NNCls};
            \legend{Render-Daten, Val-Daten, Testdaten, $d_1$-val, $d_1$-test, $d_2$-val, $d_2$-test}
        \end{axis}
    \end{tikzpicture}
    \caption{Auswertung von $\mu_\text{cls}$ der Systeme auf unterschiedlichen Datenquellen.}
    \label{fig:nn_cls}
\end{figure}

Die Auswertung der unterschiedlichen Systeme hinsichtlich der Metrik $\mu_\text{cls}$ ist in \autoref{fig:nn_cls} dargestellt. Die Resultate spiegeln die Auswertung von $\mu_\text{xst}$ wider indes die relativen Verteilungen einer ähnlichen Struktur folgen. Auswertungen des Ansatzes dieser Thesis zeigen Werte im Bereich um $70\,\%$ für die Daten, die für diese Arbeit erstellt wurden mit einer messbar besseren Auswertung auf gerenderten Daten im Vergleich zu realen Daten. Auf den DeepDarts-$d_1$-Datensätzen konnten mit $56\,\%$ die geringsten Resultate erzielt werden, während auf den DeepDarts-$d_2$-Daten im Mittel eine Auswertung von $68\,\%$ erzielt wird. Mit diesen Resultaten liegen die Genauigkeiten der Findung von Feldfarben unter den Genauigkeiten der Fähigkeit, die Dartpfeile zu identifizieren.

DeepDarts-$d_1$ zeigt hinsichtlich $\mu_\text{cls}$ ähnliche Auswertungen zu $\mu_\text{xst}$: Es werden einzig gute Ergebnisse mit Werten um $63\,\%$ auf den diesem System zugeschriebenen Datensätzen erzielt. Auf Datensätzen, die nicht zum Training oder zur Auswertung des Systems verwendet werden, lag im Durchschnitt lediglich eine Klassen-Genauigkeit von $17\,\%$ vor.

Ähnliche Züge der Evaluation hinsichtlich $\mu_\text{cls}$ sind für DeepDarts-$d_2$ zu verzeichnen: Die $d_2$-Datensätze werden mit einer hohen Genauigkeit von durchschnittlich $77\,\%$ erkannt, während weitere Datensätze mit durchschnittlich $20\,\%$ Genauigkeit erkannt werden. Die Fähigkeit, Feldfarben korrekt zu identifizieren, liegt für DeepDarts-$d_2$ deutlich unter der Fähigkeit, Dartpfeile zu identifizieren.

% -------------------------------------------------------------------------------------------------

\vspace*{3em}
\subsection{Auswertung der Positions-Metrik \texorpdfstring{$\mu_\text{pos}$}{µ\_pos}}
\label{sec:auswertung_pos}

\pgfplotstableread[col sep=comma]{
    system,         Render-Daten,   Testdaten, Val-Daten,  d1-val,     d1-test,    d2-val,     d2-test
    Thesis,         47.01,          52.35,      37.83,      22.74,      28.62,      14.54,      33.79
    DeepDarts-d1,   0,              0,          0,          22.49,      12.08,      0,          0
    DeepDarts-d2,   8.1,            63.72,      443.29,     137.23,     260.59,     19,         22.24
}\NNPos

\begin{figure}
    \centering
    \begin{tikzpicture}
        \begin{axis}[
                width=0.8\textwidth,
                height=6cm,
                ybar,
                ymode=log,
                log origin=infty,
                ymin=1,
                ymax=550,
                bar width=0.3cm,
                enlarge x limits=0.25,
                ylabel={$\mu_\text{pos}$ [px]},
                symbolic x coords={Thesis, DeepDarts-d1, DeepDarts-d2},
                xtick={Thesis,DeepDarts-d1,DeepDarts-d2},
                legend style={at={(1.02, 1.00)}, anchor=north west},
                every axis plot/.append style={
                        single ybar legend,
                    },
            ]
            \addplot+[draw=black, fill=bar_1]     table[x=system,y=Render-Daten]  {\NNPos};
            \addplot+[draw=black, fill=bar_2]     table[x=system,y=Val-Daten]     {\NNPos};
            \addplot+[draw=black, fill=bar_2!60]  table[x=system,y=Testdaten]    {\NNPos};
            \addplot+[draw=black, fill=bar_3]     table[x=system,y=d1-val]        {\NNPos};
            \addplot+[draw=black, fill=bar_3!60]  table[x=system,y=d1-test]       {\NNPos};
            \addplot+[draw=black, fill=bar_4]     table[x=system,y=d2-val]        {\NNPos};
            \addplot+[draw=black, fill=bar_4!60]  table[x=system,y=d2-test]       {\NNPos};
            \legend{Render-Daten, Val-Daten, Testdaten, $d_1$-val, $d_1$-test, $d_2$-val, $d_2$-test}
        \end{axis}
    \end{tikzpicture}
    \caption{Auswertung von $\mu_\text{pos}$ der Systeme auf unterschiedlichen Datenquellen. Je geringer die Werte, desto besser die Auswertung.}
    \label{fig:nn_pos}
\end{figure}

Mit der Metrik $\mu_\text{pos}$ wird die Genauigkeit der Lokalisierung der Dartpfeilspitzen untersucht. Je geringer die Wertigkeit, desto besser ist die Vorhersage. Die erzielten Ergebnisse der unterschiedlichen Systeme sind in \autoref{fig:nn_pos} dargestellt. Es ist für das in dieser Thesis trainierte neuronale Netz zu erkennen, dass die Vorhersagen realer und synthetischer Daten nicht dem Bild der Auswertungen von $\mu_\text{xst}$ und $\mu_\text{cls}$ folgen, indem die Ergebnisse der synthetischen Daten nicht signifikant von den Ergebnissen realer Daten abweichen und die Ergebnisse auf realen Daten tendenziell besser sind als auf synthetischen Daten. Die mittlere Genauigkeit aller Datensätze liegt bei einer Verschiebung von etwa $33\,\text{px}$.

DeepDarts-$d_1$ hingegen zeigt ähnliche Auswertungen in dieser Metrik wie in den zuvor ausgewerteten Metriken. Auf den Validierungs- und Testdaten von DeepDarts-$d_1$ konnte eine mittlere Abweichung von $17\,\text{px}$ zu den annotierten Dartpfeilspitzen festgestellt werden.

Die Auswertungen von $\mu_\text{pos}$ für DeepDarts-$d_2$ folgen ebenso wie die Auswertung des Systems dieser Thesis nicht der Struktur der Auswertungen von $\mu_\text{xst}$ und $\mu_\text{cls}$. Die geringsten Werte werden mit einer mittleren Verschiebung von $8\,\text{px}$ auf den synthetischen Daten erzielt, während auf den eigenen Daten im Durchschnitt $21\,\text{px}$ Abweichung und auf allen weiteren Daten $226\,\text{px}$ vorliegen.

% -------------------------------------------------------------------------------------------------

\subsection{Auswertung der \acs{pcs}-Metrik}
\label{sec:auswertung_pcs}

\pgfplotstableread[col sep=comma]{
    system,         Render-Daten,   Testdaten, Val-Daten,  d1-val,     d1-test,    d2-val,     d2-test
    Thesis,         66.53,          57.82,      62.08,      61.48,      56.15,      64.57,      57.47
    DeepDarts-d1,   2.78,           5.45,       0,          90,         93.3,       1.43,       0.67
    DeepDarts-d2,   2.78,           5.45,       0,          9.7,        24.45,      90,         84.67
}\NNPCS

\begin{figure}
    \centering
    \begin{tikzpicture}
        \begin{axis}[
                width=0.8\textwidth,
                height=6cm,
                ybar,
                ymin=0,
                ymax=100,
                bar width=0.3cm,
                enlarge x limits=0.25,
                ylabel={PCS [\%]},
                symbolic x coords={Thesis, DeepDarts-d1, DeepDarts-d2},
                xtick={Thesis,DeepDarts-d1,DeepDarts-d2},
                legend style={at={(1.02, 1.00)}, anchor=north west},
                every axis plot/.append style={
                        single ybar legend,
                    },
            ]
            \addplot+[draw=black, fill=bar_1]     table[x=system,y=Render-Daten]  {\NNPCS};
            \addplot+[draw=black, fill=bar_2]     table[x=system,y=Val-Daten]     {\NNPCS};
            \addplot+[draw=black, fill=bar_2!60]  table[x=system,y=Testdaten]    {\NNPCS};
            \addplot+[draw=black, fill=bar_3]     table[x=system,y=d1-val]        {\NNPCS};
            \addplot+[draw=black, fill=bar_3!60]  table[x=system,y=d1-test]       {\NNPCS};
            \addplot+[draw=black, fill=bar_4]     table[x=system,y=d2-val]        {\NNPCS};
            \addplot+[draw=black, fill=bar_4!60]  table[x=system,y=d2-test]       {\NNPCS};
            \legend{Render-Daten, Val-Daten, Testdaten, $d_1$-val, $d_1$-test, $d_2$-val, $d_2$-test}
        \end{axis}
    \end{tikzpicture}
    \caption{Auswertung von PCS der Systeme auf unterschiedlichen Datenquellen.}
    \label{fig:nn_pcs}
\end{figure}

Für die Auswertung des \ac{pcs} zeichnet sich ein ähnliches Bild, wie es bereits in \autoref{fig:cv_genauigkeit} ermittelt wurde, ab, indem signifikant bessere Vorhersagen von DeepDarts auf den Systemen zugeordneten Daten erzielt wird als auf den Systemen unbekannten Daten. Mit dem System dieser Thesis wird eine mittlere Korrektheit der Vorhersagen von etwa $61\,\%$ erzielt. Zu erkennen ist eine geringfügig bessere Auswertung auf synthetischen Daten, jedoch beläuft sich der Unterschied auf wenige Prozentpunkte; die Auswertungen befinden sich innerhalb einer Spanne von $9\,\%$.

Die Spanne der durch DeepDarts-$d_1$ erzielten Werte des \ac{pcs} ist hingegen weitaus größer. Auf den $d_1$-Daten konnte eine mittlere Korrektheit von $92\,\%$ erzielt werden, während auf restlichen Daten lediglich durchschnittlich $2\,\%$ der Daten korrekt vorhergesagt werden konnten. Vor den Hintergrund der \ac{cv}-Auswertung in \autoref{sec:findung_normalisierung} kann abgeleitet werden, dass diese Genauigkeiten auf die Tatsache der Standard-Antwort von 0 Punkten zurückzuführen ist, die im Fehlerfall ausgegeben wird.

Hinsichtlich DeepDarts-$d_2$ ist ebenfalls eine starke Präferenz eigener Daten aus der Auswertung des \ac{pcs} zu erkennen. Die Korrektheit der Vorhersagen auf $d_2$-Daten beträgt $90\,\%$ auf Validierungs- und $85\,\%$ auf Testdaten. Diese Auswertung deckt sich mit der für DeepDarts angegebenen Korrektheit von $84\,\%$ im DeepDarts-Paper \cite{deepdarts}. Hinsichtlich Daten, die nicht zum Training von DeepDarts-$d_2$ verwendet wurden, werden lediglich \ac{pcs} zwischen $0\,\%$ und $5\,\%$ auf Daten dieser Thesis erzielt und $10\,\%$ und $24\,\%$ auf den $d_1$-Daten.

