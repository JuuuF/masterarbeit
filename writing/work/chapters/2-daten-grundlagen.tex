% !TEX root = ../main.tex

\section{Grundlagen}
\label{sec:daten:grundlagen}

Grundlagen hier.

% -------------------------------------------------------------------------------------------------
\subsection{Rendering}
\label{sec:rendering}

.

\subsubsection{Ray-Tracing}
\label{sec:ray_tracing}

.

\subsubsection{GPU-Nutzung}
\label{sec:gpu_nutzung}

.

\todo{Rendering erklären.}

% -------------------------------------------------------------------------------------------------
\subsection{Binärbilder und Masken}
\label{sec:masken}

\todo{Masken und Binärbilder beschreiben}

% -------------------------------------------------------------------------------------------------
\subsection{Kameramodelle}
\label{sec:kameras}

\paragraph{Brennweite}

.

\paragraph{Öffnungswinkel}

.

\paragraph{ISO}

.

\paragraph{Motion Blur}

.

\paragraph{Film Grain}

.

\paragraph{Lens Distortion}

.


\todo{Kameramodelle beschreiben}

% -------------------------------------------------------------------------------------------------
\subsection{Noise-Texturen}
\label{sec:noise}

\subsection{White Noise}

\cite{white_noise}

\subsection{Perlin Noise}


\cite{perlin_noise_original,perlin_noise_extension}



\begin{figure}
    \centering
    \includegraphics[width=0.8\textwidth]{imgs/rendering/perlin_noise.png}
    \caption{Generierung von Perlin Noise \cite{perlin_noise_img}.}
    \label{img:perlin_noise_generation}
\end{figure}

\subsection{Seeding}

.

\subsection{Thresholding bei Noise}

.


\todo{Noise beschreiben}

% -------------------------------------------------------------------------------------------------
\subsection{Texturen, Material und Shader}
\label{sec:material_texturen}

.

\subsubsection{Normal Maps}

.

\subsubsection{Lichtreaktionen}

\paragraph{Emission}

.

\paragraph{Reflexion}

.

\paragraph{Diffuse Lichtbrechung}

.

\paragraph{Transmission (?)}

.

\todo{Material und Texturen beschreiben}

% -------------------------------------------------------------------------------------------------
\subsection{Dart-Terminologie}
\label{sec:dart_terminologie}

\paragraph{Tip, Shaft, Barrel, Flight}

.

\paragraph{Spinne}

.

\paragraph{Double/Triple Ring}

Oder doch \quotes{Treble}?

\todo{Dart-Terminologie beschreiben}

% -------------------------------------------------------------------------------------------------
\subsection{Dartscheiben-Geometrie}
\label{sec:dartscheiben_geometrie}

\todo{Dartscheiben-Geometrie beschreiben}
