% !TEX root = ../main.tex

\section{Grundlagen}
\label{sec:daten:grundlagen}

Grundlagen hier.

% -------------------------------------------------------------------------------------------------

\subsection{Prozedurale Datenerstellung}  % DATENERSTELLUNG =======================================
\label{sec:prozedurale_datenerstellung}

Was ist es? Wozu ist es gut?

\todo{}

% -------------------------------------------------------------------------------------------------

\subsection{3D-Rendering}
\label{sec:3d_rendering}

Kurz Ray-Tracing anreißen.

\todo{}

% -------------------------------------------------------------------------------------------------

\subsection{Kameraparameter}
\label{sec:kameras}

Bei der Datenerstellung mittels 3D-Software und Ray-Tracing ist die Präsenz einer Kamera unabdingbar. Ursprüngliche Kameramodelle begonnen mit einer Pinhole-Kamera, die als Projektion eines 3D-Raumes in einen 2D-Raum vornimmt \cite{pinhole_camera}. Auf diesem Modell aufbauend wurden weitere Kameraparameter modelliert, bis die Simulation echter Kameras ermöglicht wurde. In aktuellen 3D-Softwares zum Rendern von Szenen sind eine Vielzahl an Kameraparametern implementiert und modifizierbar, sodass fotorealistische Aufnahmen simuliert werden können. Die Unterschiede zwischen der Verwendung einer simulierten und einer Echten Kamera sind daher für das ungeschulte Auge verschwindend gering und für das Erstellen von Trainingsdaten für ein neuronales Netz ideal geeignet. Die wichtigsten Parameter einer Kameraaufnahme werden in diesem Unterabschnitt grob erläutert, um ein oberflächliches Verständnis der Arbeitsweise einer Kamera zu erlangen.

\paragraph{Brennweite}

Die Brennweite einer Kamera -- bzw. eines Objektives -- bestimmt die Lichtbrechung bei der Aufnahme eines Bildes. Diese Lichtbrechung resultiert in einem unterschiedlich großen Bereich, der von der Kamera eingefangen wird. Optisch ist die Brennweite für den Zoom des Bildes zuständig. Eine Brennweite von 50mm ist eine typische Brennweite, die dem menschlichen Blickwinkel nahe kommt. Geringere Brennweiten sorgen für ein größeres Sichtfeld während größere Brennweiten mit einem größeren Zoom einhergehen. \cite{focal_lentgh}.

\paragraph{Öffnungswinkel}

\todo{}

\paragraph{Belichtungsdauer und Bewegungsunschärfe}

\todo{}

\paragraph{ISO und Rauschen}

Bei der Aufnahme von Bildern wird zwischen zwei Arten von Rauschen unterschieden: temporales und fixiertes Rauschen. Wohingegen fixiertes Rauschen zwischen Aufnahmen gleich bleibt, ändert sich temporales Rauschen zwischen Aufnahmen nichtdeterministisch. Die Ursprünge dieses Rauschens sind weitreichend von Imperfektionen in Objektiven zu physikalischen Gegebenheiten durch die Diskretisierung einer Szene auf dem Kamerasensor. Ein wesentlicher Grund für die Existenz von Rauschen ist die ISO. Der ISO-Wert gibt die Empfindlichkeit des Kamerasensors und damit die Lichtmenge an, die bei der Aufnahme mit einer Kamera auf den Sensor gelangt. Je höher der ISO-Wert ist, desto sensitiver ist der Kamerasensor für eintreffende Lichtstrahlen, wodurch hohe ISO-Werte übermäßiges Rauschen mit sich ziehen können \cite{camera_everything}. Dieses Rauschen wird insbesondere bei automatischer Einstellung der Kameraparameter in dunklen Umgebungen deutlich, wodurch dunkle Aufnahmen mit starkem Rauschen einhergehen. Diese Art von Bildern ist insbesondere bei Aufnahmen in Mobiltelefonen vermehrt zu finden, weshalb es im Kontext dieser Thesis besonders relevant ist.

\paragraph{Farbsäume}

Kameralinsen bestehen aus Glas mit einem Refraktionsindex, der von der Wellenlänge des eintreffenden Lichtes abhängt. Daraus resultiert eine unterschiedliche Lichtbrechung der jeweiligen Lichtwellen und es entstehen Farbsäume in dem aufgenommenen Bild. \cite{lens_distortion, camera_everything}. Bei Farbsäumen handelt es sich um die prismatische Auftrennung der Farbinformationen, die besonders an Kanten von Objekten und am Rand des aufgenommenen Bildes verstärkt auftreten. Dieser Effekt kann in einer Software auf zwei unterschiedliche Arten umgesetzt werden: Simulation oder Komposition. Bei der Simulation wird das eingefangene Licht durch rekonstruierte Linsen gebrochen und die Farbsäume werden direkt durch die Kamera aufgenommen. Dieser Schritt ist rechnerisch aufwendig und wird daher im Vergleich zur Komposition selten eingesetzt. Bei der Komposition wird die Aufnahme in der Nachverarbeitung derart abgeändert, dass der Effekt der Farbsäume nachgestellt wird. Da der Effekt in herkömmlichen Aufnahmen für das ungeschulte Auge schwer erkennbar ist, ist der Unterschied dieser Methoden für diese Anwendung verschwindend gering.

% -------------------------------------------------------------------------------------------------

\subsection{Binärbilder und Masken}
\label{sec:masken}

Was bringt die Erstellung von Masken? -> Informationsgewinn / -erhaltung / -extraktion -> mehr Informationen als in Render-Bild

\todo{}

% -------------------------------------------------------------------------------------------------

\subsection{Dartscheiben-Geometrie}  % DARTS ======================================================
\label{sec:dartscheiben_geometrie}

Die Geometrie und der Aufbaue einer Dartscheibe ist für die Erstellung der Daten zentral. Eine schematische Darstellung einer Dartscheibe ist in \autoref{img:dart_board} gegeben.

\paragraph{Die Dartscheibe}

Die Dartscheibe besteht grundlegend aus einer Scheibe mit 451mm Durchmesser. Sie besteht aus 20 einheitlich großen, radial angeordneten Feldern mit Zahlenwerden von 1 bis 20. Jedes Feld besitzt einen Double- und einen Triple-Ring mit einem Durchmesser von 8-10mm. Der Triple-Ring ist etwa 10cm vom Mittelpunkt entfernt; der Double-Ring etwa 16cm. Insgesamt ergibt sich ein Durchmesser von 34cm punkteerzielender Felder, der Bereich jenseits der 17cm Abstand des Mittelpunktes gibt keine Punkte \cite{wdf-rules}.

In der Mitte der Dartscheibe befinden sich die Felder Bull und Double Bull (oder Bull's Eye). Sie haben Durchmesser jeweils ca. 32mm und 12,7mm. Das Bull gibt 25 Punkte, das Double Bull 50.

Die Felder mit einfachen Punktzahlen sind abwechselnd schwarz und weiß gefärbt, die Mehrfach-Felder der schwarzen Felder sind rot und die der weißen Felder grün gefärbt. Das Bull ist grün und das Double Bull rot.

\paragraph{Rund um die Dartscheibe}

Die Punktzahlen der Dartfelder werden durch aus Drähten gefertigten Zahlen angezeigt. Diese befinden sich an einem Ring am Äußeren Ende der Dartscheibe und sind radial um die Dartscheibe angeordnet. Dieser Zahlenring ist nicht fest an der Dartscheibe montiert, sodass er nach Belieben rotiert werden kann, um eine ungleichmäßige Abnutzung der Dartscheibe auszugleichen.

\paragraph{Material}

Die Felder der Steeldarts-Dartscheibe werden aus Fasern hergestellt, typisch sind dabei Sisal-Fasern. Diese besitzen die Eigenschaft, dass ein eintreffender Dartpfeil nicht in sie eindringt, sondern die lediglich verdrängt, wodurch die bleibenden Schäden der Einstichlöcher gering gehalten werden.

\paragraph{Double und Triple}

Mit Double und Triple (auch Treble) werden die Ringe an Feldern bezeichnet, die Vielfache der Feldgrundzahl als Score erteilen. Der Triple-Ring befindet sich zwischen den Einzelfeldern der Dartscheibe und gewährt das Dreifache der Grundpunktzahl des Feldes. Der Double-Ring befindet sich auf der Außenseite der Dartscheibe und gewährt seinem Namen entsprechend das Doppelte der Feldzahl. Das Feld mit der größten Punktzahl ist die Triple-20 mit $3 \times 20 = 60$ Punkten.

\begin{figure}
    \centering
    \includegraphics[width=\textwidth]{imgs/rendering/dart_board.pdf}
    \caption{Dartscheibe}
    \label{img:dart_board}
\end{figure}


% -------------------------------------------------------------------------------------------------

\subsection{Dart-Terminologie}
\label{sec:dart_terminologie}

Im Darts gibt es eine Vielzahl an Begrifflichkeiten, von denen einige auch in dieser Thesis genutzt werden. Die grundlegenden Begriffe und ihre Bedeutungen werden in diesem Unterkapitel erläutert.

\paragraph{Tip, Barrel, Shaft, Flight}

Ein Dartpfeil besteht aus unterschiedlichen Bestandteilen. Die vier wesentlichen Bestandteile sind Tip, Barrel, Shaft und Flight\footnote{Abweichungen dieser Bezeichnungen sind möglich, sodass der Shaft auch häufig als Stem bezeichnet wird. Zur Vereinheitlichung wurden die genannten Begriffe verwendet.}, aufgezählt von der Vorderseite nach hinten \cite{wdf-rules,pdc_rules}. Die Tip ist die Spitze des Dartpfeils, die in die Dartscheibe eintrifft. Die Barrel ist der Teil des Dartpfeils, an dem er gegriffen wird und liegt direkt hinter der Tip. Auf die Barrel folgt der Shaft, der die Brücke zum Flight, dem Flügelende des Dartpfeils, schließt. Der Flight besteht aus vier Einzelflügeln, die in Abständen von $90\degree$ zueinander stehen und orthogonal zum Shaft verlaufen.

\paragraph{Spinne}

Als Spinne wird der Metallrahmen der Dartscheibe bezeichnet, der die Feldsegmente voneinander trennt. Die Ausprägung der Spinne ist unterschiedlich, jedoch ist ein tendenzieller Trend zu erkennen, dass neue Dartscheiben dünnere und unauffälligere Spinnen besitzen als alte Dartscheiben. Je unauffälliger die Spinne ist, desto geringer ist die Wahrscheinlichkeit eines Bounce-Outs, bei dem der Dartpfeil auf die Spinne trifft und nicht auf der Dartscheibe landet.

% -------------------------------------------------------------------------------------------------

\subsection{Steeldarts}
\label{sec:steeldarts}

\todo{Was Steeldarts?}

% -------------------------------------------------------------------------------------------------

\subsection{Material und Texturen}  % TEXTURIERUNG ================================================

Licht-Eigenschaften / Normal Maps / Shaders

\todo{}

% -------------------------------------------------------------------------------------------------

\subsection{Noise-Texturen}
\label{sec:noise}

Essenziell für prozedurale Datenerstellung ist die Verwendung von Noise-Texturen. Diese ermöglichen es, Kontrolle über die Variation der Daten zu behalten während der Zufall der Datenerstellung erhalten bleibt. Es gibt unterschiedliche Arten von Noise-Texturen, die für die Generierung zufälliger Texturen verwendet werden \cite{noise_generation}. In dieser Arbeit wurden vorgehend White Noise und Perlin Noise in der Datengenerierung genutzt, daher werden diese in den folgenden Unterabschnitten genauer erläutert.

\subsubsection{White Noise}

White Noise -- zu deutsch: weißes Rauschen -- ist eine Zufallsverteilung von Zahlenwerten, die nicht vorhersehbar ist und doch einem Muster folgt. Sie ist dadurch charakterisiert, dass jeder Ausgangswert mit der selben Wahrscheinlichkeit versehen ist \cite{white_noise}. Im 1-dimensionalen Beispiel von Audio entspricht weißes Rauschen der Charakteristik, dass jede Frequenz in einem Signal gleichermaßen vertreten ist. Hinsichtlich einer diskreten 2-dimensionalen Textur ist die Wahl jedes Intensitätswerts eines Pixels unabhängig voneinander gleichverteilt über das Intervall $[0, 1]$.

\subsubsection{Perlin Noise}

Perlin Noise bildet die Basis für gezielte Generierung von Strukturen prozeduraler Natur. 1982 von Ken Perlin für den Film Tron entwickelt, zielte Perlin Noise in Vergleich zu White Noise auf die Generierung von zusammenhängendem Rauschen ab, das zur Generierung natürlicher Strukturen verwendet werden kann \cite{perlin_noise_original,perlin_noise_extension}.

Zur Berechnung einer 2D-Textur mit Perlin Noise wird ein Bild in Regionen unterteilt. Jeder Ecke dieser Regionen wird ein Vektor zugeordnet, der in eine zufällige Richtung zeigt, die uniform aus dem Intervall $[0, 2\pi]$ gewählt wird, wie in \autoref{img:perlin_noise_generation} dargestellt. Für jeden Pixel $(x, y)$ im Bild wird die korrespondierende Region bestimmt und die Punktprodukte zwischen den Vektoren der Ecken der Region und den Verbindungsvektoren zu dem Punkt werden gebildet. Die Intensität des Pixels wird durch bilineare Interpolation dieser Werte auf Grundlage seiner Position in der Region bestimmt. Auf diese Weise entsteht ein zusammenhängendes Muster, das als Perlin Noise bezeichnet wird \cite{perlin_noise_original}.

Der Detailgrad von Perlin Noise wird durch Überlagerung mehrerer Schichten erzielt, die sich in der Größe ihrer Regionen unterscheiden; je kleiner die Regionen, desto größer der Detailgrad. Durch Variation der Gitterauflösung und Überlagerung mehrerer Schichten kann ein beliebiger Detailgrad erzielt werden, der in einem natürlichen Rauschen resultiert.

\begin{figure}
    \centering
    \includegraphics[width=0.8\textwidth]{imgs/rendering/grundlagen/perlin_noise.png}
    \caption{Generierung von Perlin Noise \cite{perlin_noise_img}. Zufällige Rotationsvektoren werden uniform über ein Bild verteilt, durch sie wird die Texturierung bestimmt.}
    \label{img:perlin_noise_generation}
\end{figure}

\subsection{Seeding}

Die algorithmische Generierung von Zufallszahlen ist nicht ohne Verzerrung möglich. Daher werden von Computern generierte Zufallszahlen auch als pseudo-zufällig bezeichnet. Die Generierung von Zufallszahlen beruht auf deterministischen Algorithmen, die üblicherweise einen Seed verwenden, um Zahlen zu generieren. Durch die Verwendung des selben Seeds ist das Ziehen von Zufallszahlen deterministisch wiederholbar, wodurch die Generierung von Daten reproduzierbar und nachvollziehbar ist.

\subsection{Thresholding und Maskierung}

Warum braucht man das? Wozu macht man das? -> Thresholding zur Kombination von Texturen miteinander / Layering mehrerer Texturen

In der Datengenerierung dieser Thesis wird Thresholding und Maskierung verwendet, um Texturen gezielt miteinander zu kombinieren. Im Bezug auf die Verwendung in der Datengenerierung wird  für Thresholding ein Schwellwert verwendet, der die Farbwerte einer Textur bestimmen. Ist die Pixelintensität oberhalb des Schwellwerts, wird dieser entweder verwendet, andernfalls nicht; das selbe ist umgekehrt möglich. Typische Einsätze von Thresholding in dieser Arbeit ist die Eingrenzung von Noise-Texturen auf die größten Pixelintensitäten oberhalb eines Schwellwerts.

Wird dieses globale Thresholding aller Pixel lokalisiert, sodass jedem Pixel ein dedizierter Threshold zugeordnet wird, spricht man von Maskierung. Eine Maske bestimmt die Grenzwerte, die zur Einblendung der modulierten Textur notwendig sind. Neben binärer Ein- und Ausblendung von Pixeln ermöglichen Masken durch Verwendung von Pixelwerten im Intervall $[0, 1]$ anteiliges Anzeigen der Quelltextur. Diese Arbeitsweise wird in dieser Thesis zur Erstellung einer realistischen Dartscheibe durch Überlagerung unterschiedlicher maskierter Texturen verwendet.

\todo{Dieser Text ist nicht gut, gerne überarbeiten. Mehr Zahlen und Definitionen. Und eine Quelle wäre wohl nicht verkehrt.}

% -------------------------------------------------------------------------------------------------

\subsection{Prozedurale Texturen}
\label{sec:was_prozedurale_texturen}

Prozedurale Texturen sind der Kern der Datenerstellung in dieser Thesis. Sie dienen als Blaupause zur Generierung zufälliger Texturen, indem sie ein generelles Erscheinungsbild einer Textur definieren. Dieses Erscheinungsbild ist aufgebaut aus Bestandteilen, die Seeding integrieren, um ihr Aussehen zu bestimmen. Die Änderung des verwendeten Seeds führt folglich zu einer Änderung der Textur. Der Grad der Änderung ist vorhersehbar und kann durch Grenzwerte begrenzt sein, die konkrete Änderung unterliegt jedoch weiterhin der Pseudo-Zufälligkeit. Unter Einbindung dieser Technik kann eine beliebige Anzahl unterschiedlicher Texturen generiert werden.
