% !TEX root = ../main.tex

\chapter{Ausblick}
\label{cha:ausblick}

Hier steht der Ausblick.

% -------------------------------------------------------------------------------------------------

\section{Ausblick der Datenerstellung}
\label{sec:ausblick_data}

- Implementierung von PBR in Datenerstellung
- Verbesserung der Datengenerierung, um realistischer zu werden und mehr Umgebungsbedingungen zu simulieren
- Datenerstellung auf weitere Farben und Formen der Dartscheibe erweitern
- - z.B. blau-rote Felder
- - ist aber meist nicht in Steeldarts gegeben, sondern eher in elektronischen Dartscheiben
- - und bei elektronischen Dartscheiben ist dieses System ohnehin überflüssig

\todo{Ausblick Daten}

% -------------------------------------------------------------------------------------------------

\section{Ausblick der Normalisierung}
\label{sec:ausblick_cv}

- Verbesserung der Erkennung / Klassifizierung von Orientierungspunkten
- - aktuell: pessimistische Klassifizierung, um Outlier zu minimieren, siehe \autoref{img:orientierung} (3)
- - Problem: Einige Punkte werden nicht korrekt erkannt
- - Lösung: Durch Überarbeitung der Klassifizierung genauer klassifizieren (ggf. mit kleinem CNN, das Surroundings / Farben klassifiziert)

- Kompilierung der CV-Pipeline
- - entweder Cython \cite{cython} / Numba \cite{numba} oder Implementierung in kompilierter Sprache (C/C++/Rust)

- Ellipsen-Erkennung in CV einbauen (\cite{ellipse_detection_algorithm})
- - Möglichkeit zur Identifizierung von Ellipsen im Bild

- Vergleich mit DD auf unterschiedlichen Plattformen
- - aktuell: eine Plattform
- - möglich: Ausführung beides Systeme auf einem Mobiltelefon
- - Hintergrund: System ist darauf ausgelegt, mobil genutzt zu werden

\todo{Ausblick CV}

% -------------------------------------------------------------------------------------------------

\section{Ausblick der Dartpfeilerkennung}
\label{sec:ausblick_ki}

- neues Trainieren des Systems auf mehr echten Daten
- - Arbeit spezifisch hinsichtlich Aufnahme und Annotation neuer Daten

- Quantisierung der Netzwerke
- - Verbesserung der Geschwindigkeit
- - Verringerung der Ressourcennutzung (insbesondere hinsichtlich mobilem Deployment)

- KI-Prediction auf Grundlage einer leeren Dartscheibe
- - Kalibrierungs-Bild schließen und als Referenz nutzen
- - Wenn bekannt ist, dass keine Dartpfeile auf Kalibrierungs-Bild vorhanden sind, ist die Wahrscheinlichkeit von Fehlklassifikationen des Hintergrundes geringer

- neue Architekturen austesten
- - Arbeit mit Fokus spezifisch auf Auswahl des Netzwerks

- Warm Restarts der Learning Rate: \cite{lr_warm_restart}

\todo{Ausblick NN}
