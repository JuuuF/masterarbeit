% !TEX root = ../main.tex

\chapter{Ausblick}
\label{cha:ausblick}

Diese Arbeit stellt ein System vor, mit welchem ein automatisches Dart-Scoring durch die Verwendung herkömmlicher Computer Vision sowie ein aktuelles neuronales Netz, welches durch automatisch und synthetisch erstellte Daten trainiert wurde. Das Potenzial dieser Systeme ise nicht vollkommen ausgeschöpft und kann auf unterschiedliche Arten weitergeführt werden. Dieses Kapitel gibt einen Überblick über mögliche Ausgestaltungen und Verbesserungen dieser Systeme. Es wird begonnen mit dem Ausblick der Datenerstellung in \autoref{sec:ausblick_data}, gefolgt von dem Ausblick der Pipeline zur Normalisierung in \autoref{sec:ausblick_cv}. Abschließend wird in \autoref{sec:ausblick_ki} ein Überblick über weitere mögliche Arbeit an der Dartpfeilerkennung gegeben.

% -------------------------------------------------------------------------------------------------

\section{Ausblick der Datenerstellung}
\label{sec:ausblick_data}

\paragraph{Integration von \ac{pbr}}

Ein wesentlicher Bestandteil der Datenerstellung ist die prozedurale Erstellung von Materialien. Diese ist aktuell nicht standardisiert und bedarf der Integration von \ac{pbr}. In diesem werden unterschiedliche Texturen zur Steuerung verschiedener Parameter wie Glanz und Oberflächenbeschaffenheit strikt getrennt voneinander definiert, um realistische Texturen zu erstellen.

% - Implementierung von PBR in Datenerstellung

\paragraph{Anzahl der Objekte in der Szene}

Die Szene umfasst aktuell fünf Objekte, die dynamisch ein- und ausgeblendet werden können, lediglich zwei dieser Objekte -- Lichtring und Dartschrank -- sind in den Aufnahmen zu sehen; die restlichen Objekte sind Lichtquellen. Durch Hinzufügen weiterer Objekte in die Szene, beispielsweise Wände und Dekoration, können mehr Bedingungen und Störfaktoren in die Daten aufgenommen werden, die in echten Aufnahmen zu erwarten sind. Die Aufnahmen können durch das gezielte Hinzufügen weiterer Objekte realistischer werden.

% - Verbesserung der Datengenerierung, um realistischer zu werden und mehr Umgebungsbedingungen zu simulieren

\paragraph{Aussehen der Dartscheibe}

Das Aussehen der Dartscheibe ist konform der Regelwerke der \ac{wdf} und der Professional Darts Corporation gestaltet \cite{wdf-rules,pdc_rules}. Diese gibt Form und Farben der Dartscheibe mit etwaigen Toleranzen weitestgehend vor. Für die aktuelle Datenerstellung wurden lediglich diese und ähnliche Dartscheiben als Vorlagen in Betracht gezogen, wodurch Möglichkeiten weiterer Dartscheiben ausgeschlossen wurden. Durch Hinzufügen von Parametern zur Steuerung der Art der Dartscheibe können ebenfalls Dartscheiben mit anderen Farbgebungen und Geometrien in die Datenerstellung eingebunden werden, beispielsweise Dartscheiben mit blauen und roten Feldern statt schwarzer und weißer Felder.

Ebenfalls ist eine Adaption der unmittelbaren Ausgestaltung der Dartscheibe durch Zahlenring und Texte denkbar. In der hier vorgestellten Dartscheibengenerierung werden Beschriftungen und Logos um die Dartfelder herum durch Textbausteine abstrahiert. Bei der Identifizierung von Dartpfeilen auf echten Dartscheiben hat sich jedoch gezeigt, dass diese Abstraktion von Logos durch Texte nicht ausreichend ist, um diese im nicht als Dartpfeile klassifizieren. Eine weitergehende Ausgestaltung dieser Bereiche der Dartscheibe ist daher notwendig.

% - Datenerstellung auf weitere Farben und Formen der Dartscheibe erweitern
% - - z.B. blau-rote Felder
% - - ist aber meist nicht in Steeldarts gegeben, sondern eher in elektronischen Dartscheiben
% - - und bei elektronischen Dartscheiben ist dieses System ohnehin überflüssig

% -------------------------------------------------------------------------------------------------

\section{Ausblick der Normalisierung}
\label{sec:ausblick_cv}

Der in dieser Thesis vorgestellte Algorithmus zur Lokalisierung und Normalisierung von Dartscheiben in beliebigen Bildern basiert ausschließlich auf Techniken herkömmlicher Computer Vision. Dadurch ist die Adaption des Algorithmus weitreichend möglich, sodass spezifische Aspekte gezielt angegangen und verbessert werden können. Im Verlauf der Entwicklung sowie der Auswertung sind verbesserungswürdige Bereiche des Algorithmus identifiziert worden, die in diesem Abschnitt unter die Lupe genommen werden. Sie bieten eine Grundlage zur gezielten Erweiterung und Verbesserung der bestehenden Methodiken.

\paragraph{Verbesserung der Erkennung und Klassifizierung von Orientierungspunkten}

Die Erkennung von Orientierungspunkten geschieht in einem zweischrittigen Verfahren: im ersten Schritt werden mögliche Kandidaten von Orientierungspunkten lokalisiert, welche in einem zweiten Schritt klassifiziert werden. Diese Klassifizierungen basieren auf der Klassifikation von Farbwerten, die durch die Analyse aller identifizierter Kandidaten abgeleitet werden. Die Klassifizierung ist dadurch tendenziell fehleranfällig, sodass eine pessimistische Klassifizierung zur Minimierung von Fehlklassifizierungen stattfindet. Unter den Kandidaten der Orientierungspunkte befindet sich eine ausreichende Anzahl korrekter Orientierungspunkte, welche großzügiges Aussortieren von Kandidaten ermöglichen.

Trotz erfolgreicher Normalisierungen geschieht an diesem Punkt ein Verlust relevanter Daten, die potenziell für robustere Ergebnisse des Algorithmus relevant sind. Durch eine Überarbeitung der Methodik zur Klassifizierung von Orientierungspunkten kann dieser Bereich des Algorithmus weiter ausgebaut werden. Zur Bewerkstelligung dieser Optimierung ist der Einsatz von CNNs zur denkbar. Die CNNs könnten die Aufgabe der Einordnung von Surroundings übernehmen, indem sie auf einer korrekt erkannten Surroundings trainiert werden. Die Aufgabe dieser CNNs befindet sich in einem Bereich der Komplexität, in welchem ein Netzwerk mit einer geringen Parameterzahl eingesetzt werden kann.

% - Verbesserung der Erkennung / Klassifizierung von Orientierungspunkten
% - - aktuell: pessimistische Klassifizierung, um Outlier zu minimieren, siehe \autoref{img:orientierung} (3)
% - - Problem: Einige Punkte werden nicht korrekt erkannt
% - - Lösung: Durch Überarbeitung der Klassifizierung genauer klassifizieren (ggf. mit kleinem CNN, das Surroundings / Farben klassifiziert)

\paragraph{Kompilierung des Algorithmus}

Zur Implementierung des Algorithmus, wie er in dieser Thesis vorgestellt und ausgewertet wurde, wurde Python als Programmiersprache verwendet. Trotz der Verwendung der Bibliotheken NumPy und OpenCV, welche zu Teilen kompilierte Funktionen einsetzen, geschieht ein großer Anteil der Berechnungen durch interpretierten Quelltext. Interpretation von Quelltext ist weitaus ineffizienter als die Ausführung kompilierten Quelltexts.

Die Kompilierung des Algorithmus kann auf unterschiedliche Weisen umgesetzt werden. Bibliotheken wie Cython und Numba ermöglichen die Ausführung von Python-basiertem Quelltext bzw. Kompilierung von Teilen des Python-Quelltexts zur Optimierung von Ausführungszeiten \cite{cython,numba}. Durch diese Bibliotheken kann potenziell eine Optimierung der Laufzeit mit geringer Abänderung des vorhandenen Quelltexts erzielt werden.

- Kompilierung der CV-Pipeline
- - entweder Cython \cite{cython} / Numba \cite{numba} oder Implementierung in kompilierter Sprache (C/C++/Rust)

- Ellipsen-Erkennung in CV einbauen (\cite{ellipse_detection_algorithm})
- - Möglichkeit zur Identifizierung von Ellipsen im Bild

- Vergleich mit DD auf unterschiedlichen Plattformen
- - aktuell: eine Plattform
- - möglich: Ausführung beides Systeme auf einem Mobiltelefon
- - Hintergrund: System ist darauf ausgelegt, mobil genutzt zu werden

\todo{Ausblick CV}

% -------------------------------------------------------------------------------------------------

\section{Ausblick der Dartpfeilerkennung}
\label{sec:ausblick_ki}

- neues Trainieren des Systems auf mehr echten Daten
- - Arbeit spezifisch hinsichtlich Aufnahme und Annotation neuer Daten

- Quantisierung der Netzwerke
- - Verbesserung der Geschwindigkeit
- - Verringerung der Ressourcennutzung (insbesondere hinsichtlich mobilem Deployment)

- KI-Prediction auf Grundlage einer leeren Dartscheibe
- - Kalibrierungs-Bild schließen und als Referenz nutzen
- - Wenn bekannt ist, dass keine Dartpfeile auf Kalibrierungs-Bild vorhanden sind, ist die Wahrscheinlichkeit von Fehlklassifikationen des Hintergrundes geringer

- neue Architekturen austesten
- - Arbeit mit Fokus spezifisch auf Auswahl des Netzwerks

- Warm Restarts der Learning Rate: \cite{lr_warm_restart}

\todo{Ausblick NN}
