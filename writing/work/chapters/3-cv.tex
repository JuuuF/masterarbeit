% !TeX root = ../main.tex

\chapter{Vorverarbeitung von Bildern durch klassische \acl{cv}}
\label{cha:cv}

Sowohl für das Training als auch für die Inferenz werden normalisierte Bilder von dem neuronalen Netz erwartet. Diese Normalisierung kann auf unterschiedliche Arten geschehen. Für DeepDarts wurde sich entschieden, die Normalisierung und Vorhersage der Dartpfeilpositionen in einem Schritt von dem neuronalen Netz vorhersagen zu lassen. Dieser Ansatz geht jedoch mit einigen Schwachpunkten einher. Wesentliche Nachteile werden offensichtlich in Hinsicht auf die Identifizierung spezifischer Fixpunkte, die möglicherweise verdeckt sein könnten, und Variabilität der Trainingsdaten. Das Angehen einer Aufgabe durch ein neuronales Netz stützt sich auf die Korrektheit und einen angemessenen Umfang der für das Training genutzten Daten. Durch gewollte oder ungewollte Einbindung einer verzerrten Datenlage erlernt ein neuronales Netz eben diese Eigenheiten und es besteht die Gefahr des Overfittings, sodass eine Verallgemeinerung der System-Performance auf unbekannte Daten nicht oder nur bedingt möglich ist.

Eine algorithmische Herangehensweise ist im Gegensatz zu einem neuronalen Netz keine Blackbox und die innere Arbeitsweise ist bekannt und klar definiert. Es kann strikt nachvollzogen werden, an welcher Stelle und aus welchem Grund eine fehlerhafte Vorhersage auftritt, und Problemquellen können gezielt angegangen werden. Darüber hinaus kann eine algorithmische Normalisierung von Daten ohne den Mehraufwand eines neuen Trainings eines neuronalen Netzes auf neue Daten erweitert werden. Zur Anpassung des Systems an neue Gegebenheiten sind lediglich wenige exemplarische Bilddaten notwendig, anhand derer charakteristische Eigenschaften abzuleiten sind und durch welche die Arbeitsweise des Algorithmus angepasst werden kann.

Zusätzlich ist die zu lösende Aufgabe der Identifizierung von Dartscheiben durch ihre Geometrie und ihren Aufbau mit kontrastreichen und farblich markanten Feldern prädestiniert für eine algorithmische Verarbeitung. Diese Charakteristiken werden von Algorithmen und Techniken der herkömmlichen \ac{cv} genutzt, um relevante Informationen zu extrahieren.

Aus diesen Gründen wird sich in dieser Thesis für eine algorithmische Normalisierung der Daten entschieden, die nach der Datenerstellung den zweiten Themenbereich ausmacht. In diesem Kapitel werden in einem ersten Schritt die für das Verständnis des Algorithmus notwendigen Grundlagen in \autoref{sec:cv:grundlagen} erläutert. Darauf folgend wird in \autoref{sec:cv:methodik} auf die Methodik der Normalisierung eingegangen und anschließend wird auf einige relevante Themen der Implementierungen eingegangen; \autoref{sec:cv:implementierung}. Zuletzt werden die Ergebnisse dieser Normalisierung anhand idealer Entzerrungen ausgewertet und mit dem Ansatz des DeepDarts-Systems verglichen.

% !TeX root = ../main.tex

\newpage
\section{Grundlagen}
\label{sec:cv:grundlagen}

Bevor in die Thematik der Normalisierung eingegangen wird, ist die Klärung von Grundbegriffen und -konzepten relevant. Diese legen den Grundbaustein für die Nachvollziehbarkeit der Algorithmen und Techniken, die für die Verarbeitung der Bilddaten relevant sind. Diese Grundlagen setzen mathematische Kenntnisse voraus, wie sie in einem Studium der Informatik erlernt und verstanden werden. Die Grundlagen werden in einem Detailgrad erklärt, der ein Verständnis der eingesetzten Techniken ermöglicht.

% -------------------------------------------------------------------------------------------------

\subsection{Polarlinien}
\label{sec:polarlinien}

Die polare Darstellung von Linien ist für die Identifizierung der Dartscheibe dahingehend relevant, dass sie es ermöglicht, einer Linie einen Winkel zuzuordnen, in dem diese verläuft. Eine Polarlinie ist definiert als ein Tupel $(\rho, \theta)$, in dem $\rho$ der minimale Abstand der Linie zum Koordinatenursprung ist und $\theta$ der Winkel der Liniennormalen zur x-Achse. Die Charakterisierung einer Linie durch einen Winkel in \autoref{sec:linien} (\nameref{sec:linien}) verwendet.

Zur Umrechnung einer Linie, gegeben durch zwei Punkte $P_1 = (x_1, y_1)$ und $P_2 = (x_2, y_2)$, in Polarform werden folgende Gleichungen genutzt \cite{polar_linien}:
\[ \rho = \sqrt{(x_2 - x_1)^2 + (y_2 - y_1)^2} \]
\[ \theta = \atan \frac{y_2 - y_1}{x_2 - x_1} \]
Festzuhalten ist, dass Start- und Endpunkte der Linie durch diese Art der polaren Beschreibung nicht berücksichtigt werden. Dies beruht auf der Gegebenheit, dass mathematische Beschreibungen von Linien infinite Längen haben. Diese Eigenschaft in dieser Darstellung von Polarlinien wird in der Methodik dieser Arbeit zur Identifizierung von Charakteristiken von Vorteil genutzt und wird gezielt in \autoref{sec:mittelpunktextraktion} eingesetzt.

% -------------------------------------------------------------------------------------------------

\subsection{Thresholding}
\label{sec:thresholding}

Als Thresholding wird in der Datenverarbeitung die Einteilung von Werten eines Definitionsbereichs $D$ in zwei Kategorien verstanden. Dabei wird ein Grenzwert $T \in D$, der Threshold, definiert, anhand dessen die Kategorie eines Wertes festgelegt wird. Üblich sind in der Bildverarbeitung Definitionsbereiche $D \in \{\mathbb{R}, \mathbb{N}\}$ und die Einteilung $v \leq T$ bzw. $v > T$ mit $v \in D$.
\nomenclature{$D$}{Generischer Definitionsbereich}
\nomenclature{$T$}{Generischer Threshold}

Thresholding findet seine Verwendung im Kontext dieser Masterarbeit in der Extraktion relevanter Informationen auf Bildern durch die Betrachtung von Pixeln innerhalb bestimmter Grenzwerte in Bildern. Diese Technik ermöglicht unter anderem das Identifizieren von Merkmalen anhand von Farben, die in bestimmten Farbräumen besonders hervorgehoben sind und durch Thresholding eindeutig identifiziert werden können. Durch Kombination mehrerer Thresholds können komplexe Sachverhalte und Charakteristiken aus Bildern extrahiert werden.

In dieser Thesis wird Thresholding in vielerlei Hinsicht verwendet, unter anderem in \autoref{sec:filterung} zur Differenzierung zwischen Kanten und Hintergrund oder in \autoref{sec:linienfilterung} zur Filterung von Linien anhand einer Abstandsmetrik.

% -------------------------------------------------------------------------------------------------

\subsection{Binning}
\label{sec:binning}

Binning ist als Erweiterung von Thresholding zu verstehen. Es bezeichnet die Diskretisierung von Werten in definierte Intervalle, sogenannte Bins oder Buckets. Durch Binning wird es ermöglicht, Spannen von Werten in definierte Bereiche zu unterteilen und somit in diskrete Kategorien einzuordnen. Dabei wird zwischen Hard-Binning und Soft-Binning unterschieden.

Beim Hard-Binning werden Intervallgrenzen $I = \{i_0, i_1, ..., i_n\} \subseteq D$ definiert, die einen Definitionsbereich $D$ von Daten in halboffene Intervalle $I_{k \in [0, n-1]} = [i_k, i_{k+1})$ unterteilen. Diese Intervalle korrespondieren mit Bins $B_{i \in [0, n]}$, in welche diejenigen Werte akkumuliert werden, die in das dementsprechende Intervall fallen. Es existieren harte Grenzen der Bins, die unter anderem dafür sorgen, dass nahe beieinander liegende Werte $v_0 \in D$ und $v_1 = v_0 - \epsilon_{>0}$ in unterschiedlichen Bins zugeordnet werden, sofern $x_0 \in I$. Um dieses Artefakt zu umgehen, gibt es das Soft-Binning, in dem Werte anteilig in Bins eingeordnet werden, sodass Werte im Umkreis um Intervallgrenzen in beide Bins einsortiert werden, gewichtet mit der Distanz zu der Intervallgrenze.

Binning wird in dieser Thesis unter anderem in \autoref{sec:mittelpunktextraktion} (\nameref{sec:mittelpunktextraktion}) genutzt, um Polarlinien anhand ihrer Winkel zu kategorisieren.

% -------------------------------------------------------------------------------------------------

\subsection{Faltung}
\label{sec:was_filterung}

Die Faltung, auch Convolution genannt, ist eine mathematische Operation, die ihren Ursprung in der Signalverarbeitung hat. Bei der Faltung werden Funktionen $f$ und $g$ kombiniert, um eine resultierende Funktion $h = (f * g)$ zu errechnen. Die Definition einer Faltung lautet \cite{convolution,cv_general}:
\[ (f*g)(x) = \int_{-\infty}^{-\infty} f(t) g(x-t) dt\]
Hinsichtlich der Bildverarbeitung wird eine diskrete 2D-Faltung genutzt, die die Extraktion von Merkmalen aus Bildern ermöglicht. Die diskrete 2D-Faltung funktioniert, indem ein Kernel auf jede Position eines Bildes angewandt wird. Mathematisch ist sie wie folgt beschrieben \cite{discrete_convolution}:
\[ I[x, y] = \sum_{i=0}^{x_k} \sum_{j=0}^{y_k} I[x - \lfloor\frac{x_k}{2}\rfloor + i - 1, y - \lfloor \frac{y_k}{2} \rfloor + j - 1] \times k[i, j] \]
\nomenclature{$I \in \mathbb{N}^2$}{Bild mit einem Kanal}
\nomenclature{$k \in \mathbb{N}^2$}{Faltungs-Kernel}
\nomenclature{$k_x, k_y \in \mathbb{N}$}{$x$- und $y$-Dimension eines Kernels}
\nomenclature{$c_0 \in \mathbb{N}$}{Anzahl der Eingabekanäle in einen Kernel}
\nomenclature{$c_1 \in \mathbb{N}$}{Anzahl der Ausgabekanäle eines Kernels}
\nomenclature{$x, y \in \mathbb{N}$}{Variable Koordinaten in einem kartesischen Koordinatensystem}
Dabei ist $I \in \mathbb{N}^2$ ein Eingabebild mit einem Farbkanal, $x$ und $y$ sind Positionen im Bild in $k \in \mathbb{N}^2$ ein Kernel der Größe $x_k \times y_k$.

Diese allgemeine Formel lässt sich auf die Verwendung mehrerer Kanäle erweitern. Für eine Faltung eines Bildes mit $c_0$ Kanälen, einer Kernel-Größe von $k_x \times k_y$ und einer Ausgabe von $c_1$ Kanälen wird ein Kernel der Größe $ c_0 \times k_x \times k_y \times c_1$ erstellt. Dabei werden für jeden Pixel in jedem Kanal alle Eingabekanäle betrachtet und gefiltert.

% -------------------------------------------------------------------------------------------------

\subsection{Kantenerkennung}
\label{sec:kantenerkennung}

Nachdem die Prinzipien der Faltung bekannt sind, wird in diesem Abschnitt auf die konkrete Anwendung der Faltung in der Bildverarbeitung eingegangen.

Anhängig von den Werten eines Kernels (auch Filter genannt) werden unterschiedliche Charakteristiken eines Bildes hervorgehoben. So können mit einem Kantenerkennungsfilter hochfrequente Bestandteile in einem Bild hervorgehoben werden, während ein Bild mit einem Glättungsfilter weichgezeichnet wird und hochfrequente Anteile herausgefiltert werden. Der Sobel-Filter ist ein in der \ac{cv} etablierter Filter zur robusten Kantenerkennung \cite{sobel,cv_general}. Er kombiniert die Prinzipien der Glättung und Kantenerkennung, indem das Bild durch diesen in einer Richtung geglättet wird und in der anderen eine Kantenerkennung durchgeführt wird. Die Kantenerkennung ist durch diesen Filter robust hinsichtlich Rauschen im Bild. Exemplarische Filter für Kantenerkennung und Weichzeichnung sowie ein Sobel-Filter sind in \autoref{fig:filter} dargestellt.

\begin{figure}
    \centering
    \begin{subfigure}{0.3\textwidth}
        \centering
        \begin{tikzpicture}
            \matrix[every node/.style={draw, minimum size=0.75cm, anchor=center}, draw, matrix of nodes, nodes in empty cells]{
                1 & 0 & -1 \\
                1 & 0 & -1 \\
                1 & 0 & -1 \\
            };
        \end{tikzpicture}
        \caption{Horizontaler Kantenfilter}
        \label{fig:kantenfilter}
    \end{subfigure}
    \hfill
    \begin{subfigure}{0.3\textwidth}
        \centering
        \begin{tikzpicture}
            \matrix[every node/.style={draw, minimum size=0.75cm, anchor=center}, draw, matrix of nodes, nodes in empty cells]{
                1 & 2 & 1 \\
                2 & 4 & 2 \\
                1 & 2 & 1 \\
            };
        \end{tikzpicture}
        \caption{Gaußscher Glättungsfilter}
        \label{fig:glättungsfilter}
    \end{subfigure}
    \hfill
    \begin{subfigure}{0.3\textwidth}
        \centering
        \begin{tikzpicture}
            \matrix[every node/.style={draw, minimum size=0.75cm, anchor=center}, draw, matrix of nodes, nodes in empty cells]{
                1 & 0 & -1 \\
                2 & 0 & -2 \\
                1 & 0 & -1 \\
            };
        \end{tikzpicture}
        \caption{Horizontaler Sobel-Filter}
        \label{fig:sobel}
    \end{subfigure}
    \caption{Unterschiedliche 2D-Kernel der Größe $3 \times 3$.}
    \label{fig:filter}
\end{figure}

Filterung und Kantenerkennung werden in dieser Thesis ausgiebig genutzt. So wird in \autoref{sec:kanten} Kantenerkennung als erster Schritt der Normalisierung angewandt und Filterung an sich ist in \autoref{cha:ki} ein wichtiger Bestandteil von Schichten neuronaler Netze.

% -------------------------------------------------------------------------------------------------

\subsection{Harris Corner Detection}
\label{sec:harris_corners}

Die Harris Corner Detection ist ein etablierter Algorithmus in der \ac{cv}, der 1988 von C. Harris und M. Stephens vorgestellt wurde \cite{harris_corners,cv_general}. Ziel des Algorithmus ist es, Ecken in einem Eingabebild zu identifizieren. Der Kerngedanke hinter der Harris Corner Detection basiert auf der Beobachtung, dass eine Ecke dadurch spezifiziert ist, dass sie der Endpunkt zweier aufeinandertreffender Kanten ist. Durch Kombination von Kanteninformationen und Thresholding werden Ecken in einem Bild identifiziert.

Als erster Schritt der Harris Corner Detection wird das Bild mit Kantenerkennungsfiltern verarbeitet, die, wie in \autoref{sec:kantenerkennung} beschrieben, Kanten im Bild hervorheben. Diese Kantenerkennung erfolgt in horizontaler und vertikaler Richtung, woraus zwei Kantenreaktionen hervorgehen. Kantenreaktionen benachbarter Pixelregionen werden je in einem Koordinatensystem dargestellt, die die Magnitude der Kantenreaktionen von Pixeln in horizontaler und vertikaler Richtung darstellen. In diesen Koordinatensystemen werden umliegende Ellipsen um die resultierenden Cluster gelegt, wie in \autoref{fig:harris} visualisiert.

\begin{figure}
    \centering
    \begin{subfigure}{0.3\textwidth}
        \centering
        \includegraphics[width=0.8\textwidth]{imgs/cv/grundlagen/harris_flat.pdf}
        \caption{Reaktion auf einen Bildausschnitt ohne Gradienten.}
        \label{fig:harris_flat}
    \end{subfigure}
    \hfill
    \begin{subfigure}{0.3\textwidth}
        \centering
        \includegraphics[width=0.8\textwidth]{imgs/cv/grundlagen/harris_edge.pdf}
        \caption{Reaktion auf einen Bildausschnitt mit einer Kante.}
        \label{fig:harris_edge}
    \end{subfigure}
    \hfill
    \begin{subfigure}{0.3\textwidth}
        \centering
        \includegraphics[width=0.8\textwidth]{imgs/cv/grundlagen/harris_corner.pdf}
        \caption{Reaktion auf einen Bildausschnitt mit einer Ecke.}
        \label{fig:harris_corner}
    \end{subfigure}
    \caption{Visualisierung der Reaktionen von Kantenfiltern auf unterschiedliche Bildstrukturen und die Repräsentation ihrer Gradienten \cite{harris_visualization}. Die Haupt- und Nebenachsen sind in \autoref{fig:harris_flat} jeweils kurz, in \autoref{fig:harris_edge} ist die Hauptachse lang, während die Nebenachse kurz ist, und in \autoref{fig:harris_edge} sind sowohl Haupt- als auch Nebenachse lang.}
    \label{fig:harris}
\end{figure}

Anhand der Exzentrizität und Größe einer Ellipse wird beurteilt, ob es sich bei einem Cluster von Pixeln um eine Fläche, Kante oder Ecke handelt. Benachbarte Pixel um einen Pixel auf einer Ecke werden dabei ebenfalls als Ecken identifiziert. Um diese zu unterdrücken, wird \ac{nms} verwendet, bei der lediglich diejenigen Pixel mit der stärksten Antwort auf die Eckenerkennung hervorgehoben werden. Alle benachbarten Pixel in einem vordefinierten Fenster um diesen maximalen Pixel werden entweder in ihrer Intensität gedämpft oder unterdrückt und auf null gesetzt.

% -------------------------------------------------------------------------------------------------

\subsection{Hough-Transformation}
\label{sec:hough_transformation}

Die Hough-Transformation, veröffentlicht 1962 von Paul Hough, ist ein Algorithmus zur Identifizierung von simplen Strukturen in Kantenbildern \cite{hough_transform,hough_transform}. In diesem Abschnitt wird die Erkennung von Linien mit der Hough-Transformation beschrieben, wie sie in \autoref{sec:linienerkennung} (\nameref{sec:linienerkennung}) verwendet wird.

Hauptaspekt des Algorithmus ist die Transformation von Punkten im Image-Space zu Linien im Parameter-Space. Der Image-Space ist das Koordinatensystem des Bildes mit den Achsen $x$ und $y$; der Parameter-Space ist ein Koordinatensystem mit den Achsen $\rho$ und $\theta$. Linien im Image-Space sind mathematisch beschrieben als Geraden der polaren Form:
\[ 0 = x \sin{\theta} - y \cos{\theta} + \rho \]
\nomenclature{$(\rho, \theta) \in \mathbb{R}^2$}{Koordinaten (Abstand und Winkel) in einem polaren Koordinatensystem}
Dabei sind $\rho$ und $\theta$ -- wie in \autoref{sec:polarlinien} eingeführt -- Winkel und Abstand einer Linie zum Koordinatenursprung, welcher in Bildern in der oberen linken Ecke liegt. Ein Punkt $(x, y)$ im Image-Space ist im Parameter-Space als Sinuswelle dargestellt, welche alle Linien beschreibt, die durch den Punkt $(x, y)$ im Image-Space verlaufen. Ein Punkt $(\rho, \theta)$ im Parameter-Space beschreibt eine Linie in Image-Space.

Eingabedaten der Hough-Transformation sind durch Thresholding vorverarbeitete Bilder, in denen typischerweise Kanten oder Ecken extrahiert sind. Die extrahierten Pixel werden im Parameter-Space akkumuliert, in dem sie sich in Form von Sinuswellen überschneiden. Liegen demnach Punkten im Image-Space in einer Linie, so überschneiden sich ihre korrespondieren Sinus-Darstellungen im Parameter-Space in einem Punkt. Dieser Punkt liegt an den Koordinaten $(\rho, \theta)$ und beschreibt die Parameter der Geraden, die durch die Punkte verläuft. Das Identifizieren von Peaks im Parameter-Space führt analog zur Identifizierung von Linien im Image-Space.

% -------------------------------------------------------------------------------------------------

\subsection{Transformationsmatrizen}
\label{sec:transformations_matrizen}

Transformationen von Bildern in der \ac{cv} basieren auf der Grundlage von Transformationsmatrizen \cite{transformationen_1,transformationen_2,cv_general}. Die unterschiedlichen Arten von Transformationsmatrizen und ihre Arbeitsweisen werden in den folgenden Unterabschnitten erläutert. Es wird dabei begonnen mit Grundlagen homogener Koordinaten, gefolgt von unterschiedlichen Arten von Transformationsmatrizen.

\subsubsection{Homogene Koordinaten}
\label{sec:homogene_koordinaten}

Punkte im 2D-Raum besitzen eine $x$- und eine $y$-Koordinate, sodass ein Punkt definiert ist durch $P = (x, y)$. Eine Erweiterung dieser Koordinatendarstellung sind homogene Koordinaten. Um einen 2D-Punkt $P$ in einen homogenen Punkt $\widetilde{P} = (\widetilde{x}, \widetilde{y}, \widetilde{z})$ umzuwandeln, wird eine Koordinate $\widetilde{z} \neq 0$ hinzugefügt, die zur Normalisierung der Koordinaten genutzt wird \cite{cv_general}. Zur Umwandlung von $P$ in homogene Koordinaten wird $z=1$ gesetzt; die Rücktransformation geschieht durch $P = (\frac{\widetilde{x}}{\widetilde{z}}, \frac{\widetilde{y}}{\widetilde{z}})$. Homogene Koordinaten sind eine dreidimensionale Einbettung des zweidimensionalen Raumes und ermöglichen Transformationen von Koordinaten und als Erweiterung dessen auch die Transformation von Bildern durch Transformationsmatrizen $M$. Diese sind $3 \times 3$ Matrizen, die durch Multiplikation mit homogenen Punkten Transformationen auf diese anwenden:
\[ \widetilde{P}' = M \times \widetilde{P} \]
\nomenclature{$P \in \mathbb{R}^2$}{Position in kartesischem Koordinatensystem}
\nomenclature{$\widetilde{P} \in \mathbb{R}^3$}{Homogene Position $(\widetilde{x}, \widetilde{y}, \widetilde{z})$ in kartesischem Koordinatensystem}
\nomenclature{$\widetilde{P}' \in \mathbb{R}^3$}{Transformierte homogene Position in kartesischem Koordinatensystem}
\nomenclature{$M_{i \in \mathbb{N}} \in \mathbb{R}^{3 \times 3}$}{Transformationsmatrix}
Die unterschiedlichen Einträge der Transformationsmatrizen bestimmen die Art der Transformation. Durch Verkettung von Transformationsmatrizen können mehrere Transformationen in einer einzelnen Transformation zusammengefasst werden. Dabei ist die Rechtsassoziativität der Matrixmultiplikation zu beachten, durch die eine Anwendung der Matrizen von rechts nach links geschieht. Darüber hinaus ist die Kommutativität bei Matrixmultiplikationen nicht gegeben, sodass die Reihenfolge der Anwendungen relevant ist:
{\setlength{\belowdisplayskip}{0.5ex}
\begin{align*}
    \widetilde{P}' & = M_n \times \dots \times M_2 \times M_1 \times \widetilde{P}   \\
                   & = (M_n \times \dots \times M_2 \times M_1) \times \widetilde{P} \\
                   & = M_{n, \dots, 2, 1} \times \widetilde{P}
\end{align*}}
Die Anwendung der Matrix $M_{n, \dots, 2, 1}$ auf ein Bild hat denselben Effekt wie die aufeinanderfolgende Anwendung der Matrizen $M_1$ bis $M_n$.

\newpage
\subsubsection{Affine Transformationsmatrizen}
\label{sec:affine_transformations_matrizen}

Affine Transformationen zeichnen sich durch die Aufrechterhaltung von Punkten, geraden Linien und Flächen aus. Nach einer affinen Transformation verbleiben parallele Linien weiterhin parallel, Winkel zwischen Geraden können sich jedoch ändern \cite{cv_general}. Es gibt unterschiedliche Arten affiner Transformationen, die in diesem Unterabschnitt vorgestellt werden.

\paragraph{Translationsmatrix}
\label{par:translation}

Die Translationsmatrix verschiebt Punkte um gegebene Distanzen. $x$- und $y$-Verschiebungen werden mit $t_x$ und $t_y$ beschrieben und sind wie folgt aufgebaut:

{\setlength{\belowdisplayskip}{0.5ex}
\begin{align*}
    M_\text{trans} \times \threedvec{x}{y}{1}
     & =
    \left[
        \begin{array}{ccc}
            1 & 0 & t_x \\
            0 & 1 & t_y \\
            0 & 0 & 1   \\
        \end{array}
        \right]
    \threedvec{x}{y}{1} \\
     & =
    \threedvec{x + t_x}{y + t_y}{1}
\end{align*}}
\nomenclature{$M_\text{trans} \in \mathbb{R}^{3 \times 3}$}{Translationsmatrix}

\paragraph{Skalierungsmatrix}
\label{par:skalierung}

Die Skalierungsmatrix skaliert ein Bild, unterteilt in eine horizontale Skalierung $s_x$ und eine vertikale Skalierung $s_y$. Skalierungen $<1$ resultieren in einer Stauchung, Skalierungen $>1$ in Streckungen. Skalierungsmatrizen sind wie folgt aufgebaut:

{\setlength{\belowdisplayskip}{0.5ex}
\begin{align*}
    M_\text{scl} \times \threedvec{x}{y}{1}
     & =
    \left[
        \begin{array}{ccc}
            s_x & 0   & 0 \\
            0   & s_y & 0 \\
            0   & 0   & 1 \\
        \end{array}
        \right]
    \threedvec{x}{y}{1} \\
     & =
    \threedvec{s_x \cdot x}{s_y \cdot y}{1}
\end{align*}}
\nomenclature{$M_\text{scl} \in \mathbb{R}^{3 \times 3}$}{Skalierungsmatrix}

\paragraph{Scherungsmatrix}
\label{par:scherung}

Bei der Scherung werden Punkte parallel zur $x$-Achse mit $a_x$ und parallel zur $y$-Achse mit $a_y$ geschert; d neutrale Wert einer Scherung beträgt null. Scherungsmatrizen haben die Form:

{\setlength{\belowdisplayskip}{0.5ex}
\begin{align*}
    M_\text{shr} \times \threedvec{x}{y}{1}
     & =
    \left[
        \begin{array}{ccc}
            1   & a_x & 0 \\
            a_y & 1   & 0 \\
            0   & 0   & 1 \\
        \end{array}
        \right]
    \threedvec{x}{y}{1} \\
     & =
    \threedvec{x + a_x \cdot y}{a_y \cdot x + y}{1}
\end{align*}}
\nomenclature{$M_\text{shr} \in \mathbb{R}^{3 \times 3}$}{Scherungsmatrix}

\paragraph{Rotation}
\label{par:rotation}

Die Rotation erfolgt anhand eines Winkels $\alpha$, der Punkte in mathematisch positiver Richtung um den Koordinatenursprung rotiert. Ihre Zusammensetzung kann aufgeteilt werden in einen Skalierungs- und einen Scherungsanteil. Die Skalierung erfolgt uniform in horizontaler und vertikaler Richtung um den Faktor $s_{x, y} = \cos(\alpha)$; die Anteile der Scherung sind gegeben mit $a_x = -\sin(\alpha)$ und $a_y = \sin(\alpha)$. Aus diesen Voraussetzungen ergibt sich die folgende Definition der Rotationsmatrix:

{\setlength{\belowdisplayskip}{0.5ex}
\begin{align*}
    M_\text{rot} \times \threedvec{x}{y}{1}
     & =
    \left[
        \begin{array}{ccc}
            \cos(\alpha) & -\sin(\alpha) & 0 \\
            \sin(\alpha) & \cos(\alpha)  & 0 \\
            0            & 0             & 1 \\
        \end{array}
        \right]
    \threedvec{x}{y}{1} \\
     & =
    \threedvec{\cos(\alpha) \cdot x - \sin(\alpha) \cdot y}{\sin(\alpha) \cdot x + \cos(\alpha) \cdot y}{1}
\end{align*}}
\nomenclature{$M_\text{rot} \in \mathbb{R}^{3 \times 3}$}{Rotationsmatrix}

\newpage
\subsubsection{Homographien}
\label{sec:homographien}

Im Gegensatz zu affinen Transformationen können Matrixeinträge in Homographien beliebig sein, sodass eine allgemeine Homographie die folgende Form besitzt:

\begin{align*}
    H & =
    \left[
        \begin{array}{ccc}
            h_{0, 0} & h_{0, 1} & h_{0, 2} \\
            h_{1, 0} & h_{1, 1} & h_{1, 2} \\
            h_{2, 0} & h_{2, 1} & h_{2, 2} \\
        \end{array}
        \right]
\end{align*}
\nomenclature{$H \in \mathbb{R}^{3 \times 3}$}{Homographie}
\nomenclature{$h_{i, j \in [0, 2]} \in \mathbb{R}$}{Parameter einer Homographie}

Durch Fixierung der Skalierung mit $\sqrt{\sum_{ij} h_{i, j}^2} = 1$ sind in einer allgemeinen Homographie $H$ acht freie Parameter vorhanden. Diese können z.\,B. durch vier Punktverschiebungen mit je zwei Koordinaten gegeben sein. Dadurch ist eine beliebige Transformation eines Bildes ermöglicht, in dem vier Punkte eines Quellbildes auf vier Punkte eines Zielbildes transformiert werden. Diese Eigenschaft wird bei der Normalisierung der Dartscheibe in \autoref{sec:entzerrung} (\nameref{sec:entzerrung}) genutzt, um Orientierungspunkte der Eingabebilder auf bekannte Positionen in normalisierten Bildern zu transformieren.

% -------------------------------------------------------------------------------------------------

\subsection{Log-polare Entzerrung}
\label{sec:logpolare_entzerrung}

Die log-polare Darstellung eines Bildes wird durch eine Transformation des Koordinaten-systems erlangt. Während Koordinaten in einem kartesischen Koordinatensystem durch $x$- und $y$-Koordinaten angegeben sind, werden Punkte im log-polaren Koordinatensystem durch ein Tupel $\left(\rho, \theta\right)$ beschrieben. Sie sind definiert als logarithmischer Abstand und Winkel zu einem spezifischen Punkt $\left(c_x, c_y\right)$ \cite{logpolar}. Die Umwandlung der Koordinaten ist definiert durch:
\nomenclature{$(c_x, c_y) \in \mathbb{R}^2$}{Position eines Mittelpunktes in einem Bild}
\begin{align*}
    \rho(x, y)   & = \ln \sqrt{(x - c_x)^ 2 + (y - c_y)^2}      \\
    \theta(x, y) & = \arctan2 \left((y - c_y), (x - c_x)\right)
\end{align*}

In dieser Thesis wird die log-polare Darstellung eines Bildes in \autoref{sec:orientierungspunkte_finden} genutzt, um Dartscheiben um ihren Mittelpunkt abzuwickeln. Der Effekt dieser Entzerrung ist die Transformation der Dartfelder von Kreisabschnitten zu Rechtecken.

% -------------------------------------------------------------------------------------------------

\subsection{Farbräume}
\label{sec:farbräume}

Farbinformationen in Bildern werden auf unterschiedliche Arten gespeichert. Diese unterschiedlichen Arten der Darstellungen von Farben werden als Farbräume bezeichnet \cite{color_space,cv_general}. Zu den am weitesten verbreiteten Farbräumen zählen unter anderem der RGB- und der HSV-Farbraum. Im RGB-Farbraum werden Farbinformationen nach dem Vorbild des menschlichen Auges in designierten Kanälen für rote, grüne und blaue Bestandteile des Bildes unterteilt; im HSV-Farbraum stehen die Kanäle H, S und V für Färbung (\quotes{hue}), Sättigung (\quotes{saturation}) und Helligkeit (\quotes{value}). Visuell nähere Verbundenheit zwischen der menschlichen Farbwahrnehmung wird durch die Farbräume YCbCr und Lab erzielt. Diese Farbräume nutzen ebenfalls drei Farbkanäle, die jedoch abstrakter gestaltet sind und Farbinformationen auf eine Art encodieren, die eine Interpolation von Farben zueinander mit visuellem Einklang ermöglichen, der in RGB und HSV nicht trivial erzielbar ist.

Die Verwendung unterschiedlicher Farbräume steht mit verschiedenen Vor- und Nachteilen in Verbindung. In dieser Arbeit werden die aufgezählten Farbräume zur Identifizierung von Farben und Hervorhebung von Merkmalen in Bildern genutzt. Die Verwendung von Farbraumtransformationen in dieser Thesis geschieht insbesondere in \autoref{sec:orientierungspunkte_klassifizieren} und \autoref{sec:farbidentifizierung_impl}.

% -------------------------------------------------------------------------------------------------

\newpage
\subsection{\acl{ssim}}
\label{sec:ssim}

Für die Aufgabe der Ähnlichkeitsbestimmung von Bildern wurde 2004 von \citeauthor{ssim} die \acf{ssim} entwickelt, die über das triviale Vergleichen von Farbinformationen hinausgeht \cite{ssim,cv_general}. Das Grundprinzip der \ac{ssim}-Metrik ist das Einfangen und Vergleichen visuell auffälliger Änderungen im Bild und Unterdrückung von Effekten wie Weichzeichnung und lokaler Verschiebungen von Pixeln. Es werden Informationen zu Luminanz, Kontrast und Struktur auf Eingabebildern identifiziert und jeweils miteinander verglichen. Ein gewichtetes Produkt der jeweiligen Ähnlichkeitsfaktoren bestimmt schließlich den \ac{ssim}-Wert.

Diese Handhabung der Informationsverarbeitung sorgt für Robustheit gegenüber geringen augenscheinlichen Änderungen, die jedoch starke Auswirkungen auf die zugrundeliegenden Daten besitzen. So ist das Weichzeichnen eines Bildes oder das Hinzufügen von Rauschen eine solche Änderung, bei der die Pixelwerte der Bilder stark verändert werden, die Aussage hinter den Daten jedoch nicht.

% -------------------------------------------------------------------------------------------------

\subsection{RANSAC}
\label{sec:ransac}

Bei der Verarbeitung von Daten ist nicht in jedem Fall davon auszugehen, dass alle vorhandenen Datenpunkte Teil relevante Informationen halten. Outlier sind Datenpunkte, die nicht der zu erwartenden oder gewünschten Aussage der Daten folgen, sondern als fehlerhafte Aufnahmen in den Daten vorhanden sind. Triviale Least-Squares-Approximationen unter Einbezug aller Datenpunkte ist für diese Anomalien in Daten anfällig, wodurch die Ergebnisse dieser Methoden beeinflusst werden. Um mit dieser Art der Anomalien umzugehen, stellten \citeauthor{ransac} 1981 den RANSAC-Algorithmus vor \cite{ransac,cv_general}.

RANSAC steht für \quotes{\underline{Ran}dom \underline{Sa}mple \underline{C}onsensus} und beruht auf dem zufälligen Auswählen eines Subsets von Datenpunkten zur Approximation einer akkuraten Beschreibung der Daten. Nach dem Auswählen zufälliger Punkte und Approximieren einer Zielverteilung wird jeder Punkt anhand von Grenzbedingungen als Inlier oder Outlier klassifiziert. Das Verhältnis von Inliern zu Outliern wird als Metrik zur Bewertung der Approximation genutzt. Durch wiederholtes Auswählen zufälliger Datenpunkte und Klassifizierung ist eine Identifizierung einer Approximation möglich, auf die Outlier in den Datenpunkten wenig Einfluss nehmen.

In dieser Arbeit wird dieser Ansatz in \autoref{sec:entzerrung} (\nameref{sec:entzerrung}) genutzt, um eine Normalisierung der Dartscheibe auf Grundlage von Orientierungspunkten trotz möglicher Outlier robust zu identifizieren.

% !TEX root = ../main.tex

\section{Methodik}
\label{sec:cv:methodik}

Methodik hier.

% -------------------------------------------------------------------------------------------------
\subsection{Warum Computer Vision?}
\label{sec:warum_cv}

Normalisierung von Daten für ein neuronales Netz kann auf unterschiedliche Arten umgesetzt werden. Bei der Herangehensweise von \citeauthor{deepdarts} für DeepDarts werden Normalisierung von Dartscheibe und Lokalisierung von Dartpfeilen in einem Durchlauf von einem neuronalen Netz ausgeführt. Dazu werden Orientierungspunkte auf der Dartscheibe identifiziert, deren Positionen in der entzerrten Darstellung der Dartscheibe bekannt sind. Auf diese Weise konnte eine Homographie zur Normalisierung abgeleitet werden. In dieser Arbeit wird auf herkömmliche Techniken der Computer Vision zurückgegriffen, um etwaige Nachteile eines neuronalen Netzes gezielt anzugehen. Der wichtigste Aspekt der algorithmischen Normalisierung ist die Nachvollziehbarkeit der Arbeitsweise und Wartung bzw. Anpassung des Systems an neue Gegebenheiten. Wohingegen ein neuronales Netz eine Black-Box ist, deren Arbeitsweise nicht bekannt ist, und die lediglich durch aufwändiges Training neu eingestellt werden kann, kann bei einem Algorithmus nachvollzogen werden, wo er scheitert und er kann gezielt erweitert oder adaptiert werden.

Ebenfalls war die Verwendung von Computer Vision aufgrund der auffälligen Geometrie einer Dartscheibe naheliegend, da sie Ähnlichkeiten mit Schachbrettern aufweist, welche in der Computer Vision zur Identifizierung von Kameraparametern verwendet werden. Da die Nutzung bekannter Geometrien eine zentrale Arbeitsweise der Computer Vision darstellt, war die Intuition gegeben, dass auch eine Erkennung eines ähnlich markanten Objektes in einem Bild möglich ist. Aus dieser Intuition heraus wurde der in diesem Abschnitt beschriebene Algorithmus entwickelt.

% -------------------------------------------------------------------------------------------------
\subsection{Vorverarbeitung}
\label{sec:vorverarbeitung}

Die Algorithmen der Computer Vision arbeiten auf Bildern beliebiger Größe. Da die Dauer der Verarbeitung mit der Größe der Eingabebilder skaliert, ist eine angemessene Skalierung der Eingaben ein relevanter Bestandteil der Laufzeitoptimierung. Damit einher geht jedoch der Verlust von Informationen im Bild, was für eine Abwägung zwischen Geschwindigkeit und Genauigkeit sorgt. In dieser Arbeit wurde sich für eine schrittweise Verkleinerung der Eingabebilder mit Abmessungen $(w, h)$, entsprechend Breite und Höhe, entschieden, bis $\max (w, h) < d_{max} = 1600\text{px}$. Dabei werden Eingabebilder jeweils um den Faktor 2 verkleinert, um Artefakte durch Interpolierung zu minimieren. Der Wert von $d_{max}$ wurde heuristisch ermittelt als geeignetes Mittel zwischen Geschwindigkeit und Genauigkeit.

Der Schritt der Vorverarbeitung kann übersprungen werden, indem $d_{max} = \infty$ gesetzt wird. Die Laufzeit der Normalisierung kann dadurch jedoch stark beeinträchtigt werden, da die Anzahl der Pixel quadratisch mit der Größe des Bildes skaliert.

% -------------------------------------------------------------------------------------------------
\subsection{Kantenverarbeitung}
\label{sec:kanten}

Nachdem die Eingabebilder vorverarbeitet sind, werden die wichtigen Kanten im Bild extrahiert. Eingabebilder enthalten neben den für die Normalisierung relevanten Informationen der Dartscheibe sehr viel Rauschen, das nicht für die Normalisierung benötigt wird. Mit der Kantenverarbeitung wird der Umfang an Informationen stark reduziert auf die wichtigen Charakteristiken des Bildes.

\subsubsection{Filterung}
\label{sec:filterung}

Für eine universelle Extraktion von Kanten in Bildern existieren Algorithmen und Filter, wie sie bereits in \autoref{sec:kantenerkennung} beschrieben wurden. Diese Filter sind für allgemeine Fälle geeignet, in denen das Ziel eine generelle Kantenerkennung ist oder wenig Annahmen über die Kanteninformationen in Eingabebildern getroffen werden können. In dem hier betrachteten Fall liegt der Fokus der Kantenerkennung nicht auf generischen Kanten im Bild, sondern spezifisch auf den Kanten zwischen den Flächen der Dartscheibe. Diese sind charakteristisch für die Dartscheibe und durch ihr festgelegtes Design vorgegeben. Durch die Erkennung dieser Kanten wird darauf abgezielt, den Mittelpunkt und die grobe Orientierung der Dartscheibe zu ermitteln.

Geometrie und Farbgebung der Felder einer Dartscheibe sorgen für starke Gradienten der Pixelintensitäten entlang der Kanten zwischen benachbarten Feldern. Zudem ist bekannt, dass diese Kanten geradlinig verlaufen und weitgehend uniforme Bereiche im Bild voneinander trennen, in denen zudem wenig Kanten erwartet werden. Auf Grundlage dieser Beobachtungen wurde sich für einen untypisch großen Sobel-Kernel mit einer Größe von $15 \times 15$ Pixeln entschieden, dargestellt in \autoref{img:kernel}. Dieser Kernel sorgt für eine gezielte Erkennung der geschriebenen Eigenschaften in Bildern.

\begin{figure}
    \centering
    \includegraphics[width=0.3\textwidth]{imgs/cv/methodik/edges_kernel.png}
    \caption{Vertikaler Sobel-Kernel der Größe $15\times15$ zur Identifizierung großer und uniformer Kanten in einem Bild. Helle Pixel stehen für positive, dunkle Pixel für negative Werte.}
    \label{img:kernel}
\end{figure}

Um die gewünschten Charakteristiken hervorzuheben, wird das Eingabebild vor der Kantenerkennung in Graustufen umgewandelt und der Kontrast wird erhöht, um den Unterschied zwischen hellen und dunklen Bereichen zu betonen. Um Rauschen vor der Filterung zu entfernen, wird das Bild weichgezeichnet. Hochfrequente Informationen werden dadurch verworfen und etwaige Unterbrechungen oder Störungen der Kanten zwischen den Feldern verringert. Auf dieses Bild wird der beschriebe Sobel-Kernel in vertikaler und horizontaler Richtung angewendet, um Filterreaktionen von Intensitätsänderungen entlang beider Richtungen zu erlangen. Diese werden miteinander kombiniert und durch Thresholding binarisiert. Die Ausgabe ist eine binäre Maske, in denen Pixel des Wertes 1 Kanten im Eingabebild darstellen.

\subsubsection{Skelettierung}
\label{sec:skelettierung}

Das gefilterte Kantenbild der Dartscheibe enthält aufgrund der Verwendung eines großen Kernels redundante Kanteninformationen durch mehrere Pixel breite Kanten. Diese breiten Kanten werden mittels Skelettierung auf ihre zentrale Kante reduziert \cite{skeletonization}. Bei der Skelettierung werden die existierenden Kanten iterativ verringert, bis eine zentrale Kante erzielt wurde. Dazu wird das Konzept der Erosion verwendet, bei Cluster von Pixeln in Binärbildern entlang ihrer Kontur verkleinert werden. Bildlich lässt sich das veranschaulichen mit einer in Wasser liegenden Insel, die durch Erosion an Höhe über dem Meeresspiegel verliert und sich dadurch von außen nach innen verkleinert. Nach der Skelettierung des Kantenbildes verbleibt eine minimale Darstellung der extrahierten Kanten, in der diese auf ihre wesentlichen Züge heruntergebrochen wurden. Der verbliebene Informationsgehalt des Bildes wurde dadurch auf das für die kommenden Schritte wesentliche reduziert.

Der Prozess der Kantenerkennung ist in \autoref{img:kantenerkennung} dargestellt. Das verwendete Bild stammt aus dem für DeepDarts verwendeten Datensatz und wurde ebenfalls im Paper des Systems zur Veranschaulichung von dessen Arbeitsweise genutzt. Im Sinne der Vergleichbarkeit der Systeme wurde sich daher dazu entschieden, die Arbeitsweise dieses Algorithmus anhand des selben Bildes zu veranschaulichen.

\begin{figure}
    \centering
    \includegraphics[width=0.8\textwidth]{imgs/cv/methodik/edges.pdf}
    \caption{Schritte der Kantenverarbeitung. (1) Input-Bild aus dem DeepDarts-Datensatz \cite{deepdarts-data}. (2) Umwandlung des Bildes in Graustufen. (3) Kontrasterhöhung des Bildes zur Hervorhebung der Unterschiede schwarzer und weißer Felder. (4) Weichzeichnung zur Verminderung von Störungen. (5) Filterung durch Sobel-Filter, gefolgt von Thresholding. (6) Skelettiertes Kantenbild.}
    \label{img:kantenerkennung}
\end{figure}

% -------------------------------------------------------------------------------------------------
\subsection{Linienverarbeitung}
\label{sec:linien}

An diesem Punkt in der CV-Pipeline sind relevante Kanteninformationen aus dem Bild extrahiert und als minimale binäre Maske vorhanden. Der nächste Schritt zur Normalisierung der Dartscheibe ist das Identifizieren von Linien in der Kantenmaske. Ziel der Linienverarbeitung ist es, eine mathematische Darstellung der radial angeordneten Kanten zu erlangen, die die Felder der Dartscheibe voneinander trennen. Über diese Darstellung wird mittels Transformationen eine erste Stufe der Entzerrung vorgenommen, indem die Winkel dieser Linien aneinander angeglichen werden.

Die Schritte der Linienverarbeitung sind in \autoref{img:linienverarbeitung} dargestellt und auf die jeweiligen Schritte wird in den folgenden Unterabschnitten genauer eingegangen.

\begin{figure}
    \centering
    \includegraphics[width=0.8\textwidth]{imgs/cv/methodik/lines.pdf}
    \caption{Veranschaulichung der Schritte der Linienverarbeitung. (1) Identifizierung von Linien im Kantenbild. Jede Linie ist zur Visualisierung in einer zufälligen Farbe dargestellt. (2) Extraktion des Mittelpunktes anhand unterschiedlicher Linienwinkel. Jede Klasse von Winkeln ist in einer zufälligen Farbe dargestellt. (3) Filterung der Linien anhand des Mittelpunktes. Verbleibende Linien sind grün hervorgehoben; Feldlinienwinkel $\phi_i$ in blau. (4) Akkumulation der Winkel von Pixeln in gefilterten Linien. Braune Balken sind ungefilterte, blaue Balken gefilterte Werte. (5) Entzerrte Feldlinien. Alle Winkel $\phi_i=18\degree$ sind weiß hervorgehoben.}
    \label{img:linienverarbeitung}
\end{figure}

\subsubsection{Linienerkennung}
\label{sec:linienerkennung}

Um die Dartscheibe anhand von Linien zu Entzerren, müssen im ersten Schritt Linien identifiziert werden. Für diesen Prozess wird die Hough-Transformation genutzt. Diese ermöglicht die Identifizierung von Liniensegmenten in Bildern und gibt diese als Liste von Start- und Endpunkten zurück: $L_\text{points} = \{(p_{i, 0}, p_{i, 1})\ |\ i \in [0, n]\}$, wobei $n$ die Anzahl der gefundenen Liniensegmente ist. In \autoref{img:linienverarbeitung} (1) werden erkannte Linien anhand eines Beispielbildes dargestellt. Jeder Linie wurde zur Visualisierung eine zufällige Farbe zugeordnet. Zu erkennen ist, dass neben den zu erwartenden langen Linien auch viele sehr kurze Linien erkannt werden. Der Grund für eine Häufung vieler kurzer Linien liegt in der diskretisierten Darstellung von Pixeln und Ungenauigkeiten durch Verwackelungen, ungerade Feldgrenzen oder Verzerrungen der Kameralinse. Bei dem Prozess der Linienerkennung kann jedenfalls nicht davon ausgegangen werden, dass Linien exakt erkannt werden. Trotz dessen tragen zu kurze Linien mit hoher Wahrscheinlichkeit wenig relevante Informationen, sodass Linien, die kürzer als 5 Pixel sind, herausgefiltert werden.

Aus den Start- und Endpunkten der Liniensegmente lassen sich unter Verwendung der in \autoref{sec:polarlinien} eingeführten Gleichungen die polaren Darstellungen $L_\text{polar} = \{(\rho_i, \theta_i)\ |\ i \in [0, n]\}$ errechnen mit $\rho_i \in [0,\ \text{diag}(w, h)]$ und $\theta \in [0\degree, 180\degree]$. Wie bereits bei der Einführung der Gleichung erwähnt, sind in dieser Darstellungsform keine Informationen zu der Länge der Linie enthalten. Dieser Aspekt wird in dem kommenden Unterabschnitt zum Vorteil genutzt.

\subsubsection{Mittelpunktextraktion}
\label{sec:mittelpunktextraktion}

Anhand der polaren Gleichungen $L_\text{polar}$ wird in diesem Schritt der Mittelpunkt der Dartscheibe ermittelt. Der Mittelpunkt zeichnet sich dadurch aus, dass alle Linien, die zwischen Dartfeldern liegen, folgend als Feldlinien bezeichnet, auf diesen gerichtet sind. Unter der Annahme, dass alle Feldlinien in $L_\text{polar}$ vorhanden sind, überschneiden sich eine Vielzahl dieser Linien im Mittelpunkt der Dartscheibe. Insbesondere ist bekannt, dass diese Linien in jeweils unterschiedlichen Winkeln auftreten, deren grobe Werte bekannt sind.

Unter Berücksichtigung dieser Beobachtung geschieht ein Billing der Linien $L_\text{polar}$ anhand ihrer Winkel $\theta_i$ in $b=10$ uniforme Bins $B$ der Größe $\frac{180\degree}{b}=18\degree$ mit den Intervallen $ B_i = [i \times \frac{180\degree}{b}, (i+1) \times \frac{180\degree}{b})$. Für jeden dieser Bins wird eine binäre Maske erstellt, auf der die jeweiligen Polarlinien mit einheitlicher Intensität gezeichnet werden. Diese Masken werden anschließend überlagert und weichgezeichnet, um den Einfluss von Ungenauigkeiten zu minimieren. In dem resultierenden Bild zeichnet sich der Punkt $P_\text{max} = (x_\text{max}, y_\text{max})$ mit dem höchsten Wert dadurch aus, dass durch ihn die meisten Linien verschiedener Richtungen verlaufen. Diese Eigenschaft ist durch die Art der Filterung dadurch verfeinert, dass statt beliebiger Kanten gezielt Kanten mit bestimmten Eigenschaften als Grundlage für die Linien dienen. Durch diese Wahl an Eigenschaften ist mit hoher Wahrscheinlichkeit davon auszugehen, dass mit dem Punkt $P_\text{max}$ der Mittelpunkt der Dartscheibe $M=(m_x, m_y)$ identifiziert wurde.

Hinsichtlich der Robustheit dieses Algorithmus ist der Fall hervorzuheben, dass Feldlinien durch u.\,a. perspektivische Verzerrungen oder fehlerhafte Kanten- und Linienerkennung möglicherweise von den zu erwartenden Winkelintervallen abweichen können und nicht in den ihnen zugewiesenen Bins eingeordnet werden. Es kann dadurch zur Einordnung mehrerer Feldlinien in gleiche Bins und folglich dem Auslassen von Bins führen. Da bei der Ermittlung des Mittelpunktes jedoch nach einem globalen Maximum statt einem bestimmten Zahlenwert gesucht wird, ist ein gewisser Grad an Robustheit gegen nicht oder nicht korrekt gefüllte Bins gegeben.

Visualisiert ist die Extraktion des Mittelpunkts in \autoref{img:linienverarbeitung} (2). Linien gleicher Bins wurden in der Visualisierung mit gleichen Farben dargestellt. Zu erkennen ist ein Highlight im Bulls Eye der Dartscheibe, in der sich die Linien der unterschiedlichen Bins überschneiden. Dieses Highlight ist der Mittelpunkt der Dartscheibe.

Eine Untersuchung zur Änderung der Robustheit durch Variation der Anzahl an Bins bleibt aus, da die Identifizierung des Mittelpunktes mit 10 Bins in den durchgeführten Tests auf Bildern unterschiedlicher Quellen zuverlässig und erfolgreich war.
\todo{Den Teil evtl. in Diskussion}

\subsubsection{Linienfilterung}
\label{sec:linienfilterung}

Die Mengen der Linien $L_\text{points}$ und $L_\text{polar}$ umfassen neben den für die Entzerrung relevanten Feldlinien weitere Linien, die nicht relevant für die Geometrie der Dartscheibe in dem Bild sind. Diese werden in diesem Schritt unter Verwendung des Mittelpunktes der Dartscheibe herausgefiltert. Zur Differenzierung zwischen möglichen Feldlinien und Linien, die mit Sicherheit keine Feldlinien sind, wird die Lotfuß-Distanz der Polarlinien zum Mittelpunkt genutzt. Ist eine Linie nicht auf den Mittelpunkt gerichtet, ist sie mit Sicherheit keine Feldlinie.

Die minimale Lotfuß-Distanz zwischen einem Punkt $(\hat{x}, \hat{y})$ und einer Linie in impliziter Form ist definiert durch \cite{point_line_distance}:
\begin{align*}
    \text{dist}(ax + by + c = 0, (\hat{x}, \hat{y})) & = \frac{| a \hat{x} + b \hat{y} + c|}{\sqrt{a^2+c^2}}
\end{align*}

Die implizite Form der Geraden lässt sich mit folgenden Gleichungen aus der Polarform berechnen:
\begin{align*}
    \rho          & = x \cos{\theta} + y \sin{\theta}        \\
    \iff 0        & = x \cos{\theta} + y \sin{\theta} - \rho \\
    \Rightarrow a & = \cos{\theta}                           \\
    \Rightarrow b & = \sin{\theta}                           \\
    \Rightarrow c & = -\rho
\end{align*}

Durch Einsetzen dieser ermittelten Variablen in die Distanzberechnung folgt:
\begin{align*}
    \text{dist}(ax + by + c = 0, (\hat{x}, \hat{y})) & = \frac{| a \hat{x} + b \hat{y} + c|}{\sqrt{a^2+c^2}}                                                   \\
                                                     & = \frac{| \cos{\theta} \hat{x} + \sin{\theta} \hat{y} - \rho |}{\sqrt{\cos^2{\theta} + \sin^2{\theta}}} \\
                                                     & = | \cos{\theta} \hat{x} + \sin{\theta} \hat{y} - \rho |
\end{align*}

Mit dieser Gleichung lässt sich für jede ermittelte Polarlinie $(\rho_i, \theta_i) \in L_\text{polar}$ der Abstand zum Mittelpunkt der Dartscheibe $M$ ermitteln. Anhand dieses Abstands werden die Linien gefiltert, sodass Linien, die mehr als 10 Pixel von dem Mittelpunkt entfernt verlaufen, herausgefiltert werden.

Auf diese Weise werden diejenigen Linien $\widetilde{L}_\text{polar}$ und $\widetilde{L}_\text{points}$ ermittelt, die auf den Mittelpunkt der Dartscheibe gerichtet sind und voraussichtlich Teile der Feldlinien sind. Es kann an diesem Punkt jedoch nicht sicher ausgeschlossen werden, dass sich keine Outlier unter den gefilterten Linien befinden. Zu erkennen ist die Existenz von Outliern in den gefilterten Linien in \autoref{img:linienverarbeitung} (3). In dem Beispiel liegen Liniensegmente in den Schriftzügen auf der Dartscheibe, die auf den Mittelpunkt gerichtet sind und kein Teil von Feldlinien sind.

\subsubsection{Feldlinien-Brechnung}
\label{sec:feldlinien_berechnung}

Zur Identifizierung der Winkel $\phi_i$ der Feldlinien wird eine adaptierte Hough-Transformation auf die gefilterten Linien $\widetilde{L}_\text{polar}$ und $\widetilde{L}_\text{points}$ verwendet. In dieser wird für jeden Pixel $p$ aller Linien je Winkel $\theta_{i, p}$ und Abstand $d_{i, p}$ zum Mittelpunkt ermittelt. In einem Akkumulator-Array $A^{360}$ werden die Winkel in 360 Bins mit einer Granularität von $0.5\degree$ aufsummiert, gewichtet invers proportional zu $d_{i, p}$. Dadurch wird Pixeln, die weit von dem Mittelpunkt entfernt liegen, ein geringes Gewicht zugeordnet, da diese einer größeren Wahrscheinlichkeit unterliegen, kein Bestandteil einer Feldlinie zu sein. Ziel der Verwendung von $A^{360}$ ist das Identifizieren von Clustern der Winkel.

Zur Minderung von Outliern und zur Festigung der mittleren Winkel wird $A^{360}$ zweifach radial mit einem Fenster von $5\degree$ -- entsprechend 10 Bins -- geglättet. In dem resultierenden Akkumulator werden die 10 größten Peaks $\phi_i$ durch Non-Maximum-Suppression identifiziert; diese Peaks sind die häufigsten Winkel von Liniensegmenten zum Mittelpunkt der Dartscheibe. Durch die getroffenen Annahmen ist davon auszugehen, dass diese Werte die Winkel der Feldlinien angeben.

Eine Darstellung eines Akkumulators $A^{360}$ ist mit \autoref{img:linienverarbeitung} (4) gegeben.

\subsubsection{Winkelentzerrung}
\label{sec:winkelentzerrung}

An dieser Stelle sind Mittelpunkt und Winkel der Feldlinien der Dartscheibe bekannt. Ziel dieses Schrittes ist es, die Winkel der Feldlinien zu normalisieren, sodass die Lage der Feldlinien bekannt und entzerrt ist.

Um diese Entzerrung vorzunehmen, wird eine Minimierung vorgenommen, in der eine affine Transformation gesucht wird, die diese Winkel bestmöglich aneinander anpasst und auf einen Winkelabstand von $18\degree$ angleicht. Diese Optimierung beginnt bei einer Startlinie und entzerrt alle restlichen Linien iterativ und wird für jede der 10 Linien als Startlinie ausgeführt. Als finale Transformation wird der Mittelwert aller optimierten Transformationen verwendet. Im folgenden wird die allgemeine Transformationssequenz angegeben.

Die erste Teiltransformation ist die Translation des Mittelpunktes $M$ in den Koordinatenursprung $O =(0, 0)$. Dieser Schritt ist relevant, da atomare affine Transformationen um $O$ zentriert sind. Darauf folgt die vertikale Ausrichtung der Startlinie $L_s$ durch eine Rotation um $-\phi_s$, sodass $\phi'_s = 0\degree$ erzielt wird. Gefolgt wird diese Rotation von der horizontalen Ausrichtung der Orthogonalen $L_o = L_{(i+5) \mod 10}$ durch eine Scherung entlang der Vertikalen. Wichtig bei diesem Schritt ist, dass die vertikale Scherung den Winkel von $\phi'_s$ nicht beeinflusst während eine Ausrichtung $\phi'_o = 90\degree$ erreicht wird. An diesem Punkt sind 2/10 Winkel entzerrt; die restlichen Winkel werden mit einer vertikalen Skalierung derart ausgerichtet, dass ein minimaler Abstand zwischen Zielwinkeln und Feldlinienwinkeln resultiert. Sind alle Feldlinienwinkel perfekt erkannt, ist eine optimale Skalierung möglich, sodass dieser Fehler gleich Null ist. Jedoch ist dies durch u.\ a. Diskretisierung und Artefakte in Linienerkennungen nicht gegeben und ein mittleres Minimum aller Winkeldifferenzen zu ihren Zielpositionen muss gebildet werden. Im Anschluss wird die vertikale Ausrichtung der Startlinie $L_s$ rückgängig gemacht, sodass die $\phi'_s$ seinen Zielwinkel besitzt. Zuletzt wird eine Translation des Koordinatenursprungs auf $M$ durchgeführt und die Transformationssequenz ist abgeschlossen.

Diese Schritte werden für alle Startindizes $s \in [0, 9]$ ausgeführt und die finale Transformation wird durch Mittelwertbildung errechnet. Dadurch wird eine optimale Entzerrung aller Winkel $\phi_i$ erlangt, die nicht durch die Wahl der Startlinie beeinflusst ist. In \autoref{img:linienverarbeitung} (5) ist eine Dartscheibe nach der Entzerrung der Feldlinienwinkel dargestellt. Zu erkennen ist dabei, dass trotz Angleichung der Winkel $\phi_i$ keine Normalisierung der Dartscheibe erreicht ist. Um die Normalisierung zu vollenden, muss die Dartscheibe von einer elliptischen in eine runde Form gebracht und korrekt skaliert werden. Diese Schritte geschehen in dem Verarbeitungsabschnitt der Orientierung.

% -------------------------------------------------------------------------------------------------
\subsection{Orientierung}
\label{sec:orientierung}

.

\subsubsection{Identifizierung von Orientierungspunkten}
\label{sec:orientierungspunkte_finden}

.

\subsubsection{Klassifikation von Orientierungspunkten}
\label{sec:orientierungspunkte_klassifizieren}

.

\subsubsection{Homographiebildung}
\label{sec:homographie}

.

\subsubsection{Entzerrung}
\label{sec:entzerrung}

.

% !TEX root = ../main.tex

\section{Implementierung}
\label{sec:cv:implementierung}

\todo{CV-Implementierung beschreiben.}

% -------------------------------------------------------------------------------------------------

\subsection{Winkelfindung aus gefilterten Linien}
\label{sec:winkelfindung_impl}

Die Aufgabe der Winkelfindung gefilterter Linien ist die Identifizierung von Winkel-Clustern. Die Winkel für diese Berechnung stammen aus Linien, deren unendliche Verlängerung nahe dem Dartscheibenmittelpunkt verläuft.

Für die Bewältigung dieser Aufgabe wird eine Adaption der Hough-Transformation implementiert. Eingabe in diesen Teilalgorithmus ist ein binäres Bild, auf dem die gefilterten Linien eingezeichnet sind. Für jegliche Pixel, die in diesem Bild eingezeichnet sind, wird der Winkel $\varphi_i$ zum Mittelpunkt bestimmt:
\[\varphi_i = \arctan2\left( y_i - m_\text{Dart, y}, x_i - m_\text{Dart, x} \right)\]
\nomenclature{$\varphi_i$}{Winkel eines Pixels zum Mittelpunkt der Dartscheibe.}
Für die Implementierung dieser Berechnung wurde die von NumPy zur Verfügung gestellte vektorisierte Funktion \textit{np.arctan2} verwendet. Diese Funktion, wie auch weitere Funktionen der Bibliothek, zeichnet sich durch eine effiziente und vektorisierte Berechnung von einer Eingabeliste aus. Unter der Verwendung dieser vektorisierter und zusätzlich kompilierter Funktionen ist eine schnelle Ausführung aufwändiger Berechnungen trotz der Ausführung mit Python möglich.

% - Codebeispiel: get-rough-line-angles

% -------------------------------------------------------------------------------------------------

\subsection{Farben-Identifizierung}
\label{sec:farbidentifizierung_impl}

Für die Identifizierung von Orientierungspunkten werden die Farben der Umgebungen der Kandidaten der Orientierungspunkte klassifiziert. Der Kontext, in welchem diese Verarbeitung stattfindet, ist in \autoref{img:orientierung} (3) dargestellt: Die Dartscheibe ist log.polar um ihren Mittelpunkts abgerollt und die Kandidaten sind identifiziert. Durch diese Form der Darstellung sind korrekte Orientierungspunkte derart positioniert, dass ihre Surrounding die vier unterschiedlichen Farben der Dartscheibenfelder in je einer Ecke enthält. Zur Klassifizierung werden Funktionen verwendet, die die mittleren Farben dieser Eckbereiche der Surroundings in schwarz, weiß und farbig klassifizieren.

\paragraph{CrYV-Farbraum}

Die Farben der Surroundings werden zur Einordnung in den CrYV-Farbraum umgewandelt. Der CrYV-Farbraum ist derart gestaltet, dass eine gezielte Isolation für die Unterscheidung relevanter Farbcharakteristiken durch gezieltes Thresholding durchgeführt werden kann. Die Farbkanäle der CrYV-Bilder werden separatem Thresholding unterzogen, um Farbeinflüsse spezifisch zu untersuchen und auszumachen, ob eine gewisse Farbgebung vorhanden ist.

\paragraph{Klassifizierung schwarzer und weißer Bereiche}

Die durchschnittliche Farben schwarzer und weißer Felder der Dartscheibe sind durch bereits vollzogene Vorverarbeitungsschritte bekannt. Zur Einordnung, ob ein Feld schwarz oder weiß ist, wird der zu überprüfende Bereich der Surrounding (Patch) mit der schwarzen bzw. weißen Farbe analysiert. In einem ersten Schritt wird die mittlere Farbe des Patches bestimmt. Die absoluten Differenzen der jeweiligen Farbkanäle des PAtches sowie der Ziel-Farbe werden berechnet und die Summe dieser Kanäle wird berechnet. Anhand eines empirisch ermittelten Schwellwerts wird ein Thresholding durchgeführt, durch welches eine Klassifizierung in \quotes{schwarz} oder \quotes{nicht schwarz} (und analog für weiß) geschieht:
\begin{equation*}
    \text{is\_bw}(C_p, C_r, T_C) =
    \begin{cases}
        1, & \text{wenn} ~\sum_{i=0}^{2} \vert~ C_r[i] - C_p[i] ~\vert < T_C \\
        0  & \text{sonst}
    \end{cases}
\end{equation*}
\nomenclature{$C_p \in \mathbb{R}^3$}{CrYV-Farbe eines Patches.}
\nomenclature{$C_r \in \mathbb{R}^3$}{CrYV-Referenzfarbe.}
\nomenclature{$T_C \in \mathbb{R}$}{Threshold zur Klassifizierung von CrYV-Farbdifferenzen.}
In dieser Berechnung stehen $C_p \in \mathbb{R}^3$ und $C_r \in \mathbb{R}^3$ für 3-Kanal CrYV-Farben von Patch und Referenzfarbe. $T_C \in \mathbb{R}$ ist der Farb-Threshold, der unterschritten werden muss, um als die Referenzfarbe klassifiziert zu werden.

\paragraph{Klassifizierung roter und grüner Bereiche}

Im Gegensatz zur Einordnung schwarzer und weißer Farben stehen für die Einordnung roter und grüner Farben aus technischen Gründen keine Referenzfarben zur Verfügung. Vor der Hintergrund dieser Herausforderung ist der CrYV-Farbraum derart konzipiert, dass rote und grüne Farben entsprechend markant sind und durch Thresholding gezielt identifiziert werden können. Die Farbinformationen $C_p$ eines Patches werden anhand ihres Cr-Kanals analysiert und mit Referenzwerten typischer roter und grüner Kanalwerte verglichen:
\begin{equation*}
    \text{is\_color}(C_p, T_C, t_\text{red}, t_\text{green}) =
    \begin{cases}
        1 , & \text{wenn} ~\min\left( \vert~ t_\text{red} - C_p[0] ~\vert, \vert~ t_\text{green} - C_p[0]~\vert \right) < T_C, \\
        0, & \text{sonst}
    \end{cases}
\end{equation*}
\nomenclature{$t_\text{red} \in \mathbb{R}$}{Cr-Kanal-Referenzwert für rote Farbe.}
\nomenclature{$t_\text{green} \in \mathbb{R}$}{Cr-Kanal-Referenzwert für grüne Farbe.}
In dieser Gleichung stehen $t_\text{red}$ und $t_\text{green}$ für zu erwartende Referenzwerte roter und grüner Felder.

% -------------------------------------------------------------------------------------------------

\subsection{Klassifizierung von Surroundings}
\label{sec:surroundings_impl}

top/bottom, left/right: black/white/color

Kombination der Ecken -> Art der Surrounding (innen / außen von Ring)

Abgleich mit mittlerer Surrounding

\todo{Surroundings-Implementierung}

% !TEX root = ../main.tex

\section{Ergebnisse}
\label{sec:cv:ergebnisse}

\todo{Hier die Ergebnisse beschreiben.}

% -------------------------------------------------------------------------------------------------

\subsection{Metriken}
\label{sec:cv_metriken}

Für die Messung der Entzerrungsgenauigkeit des entwickelten Algorithmus werden zwei konsekutive Metriken verwendet. Die erste Metrik $\chi$ misst die Fähigkeit eines Systems, eine Homographie zur Entzerrung einer Dartscheibe zu ermitteln, unabhängig von ihrer Genauigkeit.

\begin{equation*}
    \chi(S, I) =
    \begin{cases}
        1, & \text{wenn System $S$ zu Bild $I$ eine Homographie ermitteln kann} \\
        0, & \text{ansonsten}
    \end{cases}
\end{equation*}

Die zweite Metrik $\Lambda(\widetilde{H}, \widehat{H})$ bestimmt die Genauigkeit der ermittelten Entzerrungs-Homographie $\widehat{H}$, gegeben einer Ziel-Homographie $\widetilde{H}$. Da ein trivialer Vergleich der Zahlenwerte der ermittelten Homographien wenig Aufschluss über die konkrete Genauigkeit der Entzerrung liefert, ist eine komplexere Metrik notwendig. Für die verwendete Metrik $\Lambda$ werden $N_\text{OP, max}=61$ unterschiedliche Orientierungspunkte verwendet. Diese befinden sich entlang der Feldlinien radial verteilt in den Ringen der äußeren Bulls, des äußeren Triple-Rings und des äußeren Double-Rings. Zusätzlich ist der Mittelpunkt als weiterer Orientierungspunkt mit aufgenommen. Die Positionen $P_{i \in [N_\text{OP}]}$ aller Orientierungspunkte im Zielbild sind durch die Definition der Entzerrung festgelegt. Diese Punkte werden durch die inverse Ziel-Homographie an ihre Ursprungspositionen $\widetilde{P}_i = \mathrm{inv}(\widetilde{H}) \times P_i$ transformiert und von dort durch die ermittelte Homographie zu den vorhergesagten Zielpositionen $\widehat{P}_i = \widehat{H} \times \widetilde{P}_i$ rücktransformiert. Der Wert der Metrik ist definiert durch:
\nomenclature{$\Lambda: (\left(\mathbb{R}^{3 \times 3}\right)^2 \mapsto \mathbb{R})$}{CV-Metrik zur Identifizierung der Ähnlichkeiten von Homographien}
\[ \Lambda(\widetilde{H}, \widehat{H}) = \frac{1}{N_\text{OP}} \sum_{i = 1}^{N_\text{OP}} \left\lVert P_i - \widehat{H} \times \mathrm{inv}(\widetilde{H}) \times P_i \right\rVert _2  \]
\nomenclature{$\widehat{H} \in \mathbb{R}^{3 \times 3}$}{Ermittelte Entzerrungstransformation}
\nomenclature{$N_\text{OP, max}$}{Maximale Anzahl der Orientierungspunkte in einem Bild}
\nomenclature{$P_{i \in [N_\text{OP}]} \in \mathbb{R}^2$}{Positionen der identifizierten Orientierungspunkte}
Durch diese Metrik ist eine Quantifizierung der Ähnlichkeit zweier Homographien zur Entzerrung einer Dartscheibe möglich. Zu sehen ist bei dieser Definition, dass $\Lambda(\widetilde{H}, \widehat{H}) = 0\,\text{px}$, wenn $\widetilde{H} = \widehat{H}$, da $\widehat{H} \times \mathrm{inv}(\widetilde{H}) = \text{Id}$, die Identitätsmatrix.
\nomenclature{$\text{Id} \in \mathbb{R}^{3 \times 3}$}{Identitäts-Transformation}

% -------------------------------------------------------------------------------------------------

\subsection{Verwendete Daten}
\label{sec:cv_ergebnisse_daten}

- Gen-Daten + DD-Daten
- keine negativen Sample, da lediglich Ergebnisse auf positiven Daten relevant sind.
- es geht darum, Dartscheiben zu entzerren, nicht darum, sie zu identifizieren
- dass Dartscheiben in den Bilden vorhanden sind, wird als Voraussetzung für die Verwendung des Systems angesehen

Zur Evaluierung des Algorithmus werden Daten benötigt. Die Daten dieser Auswertung stammen aus 5 unterschiedlichen Quellen und werden voneinander getrennt gehalten. Dies dient der Identifizierung von einerseits Verzerrungen auf Daten und andererseits Schwachstellen in dem getesteten System. Die Datenquellen sind in \autoref{tab:datenquellen} aufgelistet und stellen sich zusammen aus synthetischen Daten sowie Daten des DeepDarts-Systems.

Die Daten beinhalten lediglich positive Datensätze, indes in jedem Bild eine Dartscheibe vorhanden ist. Aufgabe der Systems ist nicht die Identifizierung von Dartscheiben, sondern die Entzerrung dieser, sodass eine Existenz einer Dartscheibe in den Bildern vorausgesetzt wird.

\begin{table}
    \centering
    \begin{tabular}{r||c|cc|cc}
        \multirow{2}{*}{Datenquelle} & \multirow{2}{*}{\begin{tabular}[c]{@{}c@{}}Generierte\\ Bilder\end{tabular}} & \multicolumn{2}{c|}{DeepDarts-D1} & \multicolumn{2}{c}{DeepDarts-D2}                        \\
                                     &                                                                              & Validierung                       & Test                             & Validierung & Test   \\ \hline
        Anzahl Bilder                & 2048                                                                         & 1000                              & 2000                             & 70          & 150    \\
        Automatische Annotation      & \cmark                                                                       & \xmark                            & \xmark                           & \xmark      & \xmark
    \end{tabular}
    \caption{Datenquellen für die Auswertung der Dartscheibenentzerrungen.}
    \label{tab:datenquellen}
\end{table}

% -------------------------------------------------------------------------------------------------

\subsection{Quantitative Auswertung}
\label{sec:cv_quantitative_auswertung}

Die quantitative Auswertung ist unterteilt in die Abschnitte Geschwindigkeit, Render-Ergebnisse, DeepDarts-Ergebnisse und Zusammenfassung. Die Aufteilung der Ergebnisse in Render-Daten und DeepDarts-Daten ergibt sich aus den Unterschieden der Daten. Während die DeepDarts-Daten derart vorverarbeitet sind, dass sie die Dartscheibe weitestgehend zentriert in Bildern fester Dimensionen zeigt, sind die Render-Daten wesentlich offener hinsichtlich der Darstellung der Dartscheiben. Diese Vorverarbeitung der DeepDarts-Daten soll zu keiner unvorhergesehenen Verzerrung der Daten führen.

Die Auswertungen sind jeweils unterteilt in das in dieser Thesis erarbeiteten System und das DeepDarts-System, um die Unterschiede der Genauigkeiten darzustellen. Da das DeepDarts-System ein Single-Shot-Neural-Network ist, in welchem die Normalisierung und die Lokalisierung der Dartpfeile nicht voneinander getrennt betrachtet werden können, wird die Geschwindigkeit der Normalisierung gleichgesetzt mit der Gesamtdauer der Vorhersage.

Ausgeführt wurde die Auswertung auf einer \textbf{XXX HIER CPU EINSETZEN XXX}. Es wurde trotz der Möglichkeit einer GPU-Ausführung des DeepDarts-Systems keine Grafikkarte verwendet, um die Vergleichbarkeit der Systeme unter der Verwendung der selben Hardware zu gewährleisten.

\todo{CPU-Modell von Rechner herausfinden un aufschreiben.}

\subsubsection{Geschwindigkeit der Vorhersagen} % -------------------------------------------------

\pgfplotstableread[col sep=comma]{
    system,       Render-Daten, d1-val, d1-test, d2-val, d2-test
    Thesis,       0.637,        0.545,  0.562,   0.525,  0.514
    DeepDarts-d1, 0.242,        0.132,  0.136,   0.124,  0.137
    DeepDarts-d2, 0.236,        0.132,  0.133,   0.119,  0.133
}\ExecutionTimes

Die Ausführungszeiten der jeweiligen Systeme sind in \autoref{fig:cv_dauer} dargestellt. Die Ausführungszeiten von DeepDarts liegen mit durchschnittlich $131\,\text{ms}$ auf den DeepDarts-Datensätzen und $239\,\text{ms}$ auf den gerenderten Daten weitaus unter den Zeiten des Systems dieser Thesis. Die Ausführungszeiten des Systems dieser Thesis liegen bei durchschnittlich $537\,\text{ms}$ für die DeepDarts-Daten und $637\,\text{ms}$ für die Render-Daten. Die Unterschiede der Inferenzzeiten der Datenquellen ergeben sich aus den Abmessungen der Bilder. Die DeepDarts-Daten sind bereits vorverarbeitet, sodass sie ein quadratisches Seitenverhältnis mit einer Auflösung von $800 \times 800\,\text{px}$ aufweisen. Die gerenderten Daten hingegen sind für diese Auswertung in keinerlei Weise vorverarbeitet und werden den Systemen in den originalen Auflösungen präsentiert. Die Seitenlängen von Bildern der Render-Daten betragen mindestens \textbf{XXX}, maximal \textbf{XXX} und die durchschnittliche Seitenlänge beträgt \textbf{XXX}.

\todo{Seitenlängen eintragen.}

\begin{figure}
    \centering
    \begin{tikzpicture}
        \begin{axis}[
                width=\textwidth, height=6cm,
                ybar,
                bar width=0.35cm,
                enlarge x limits=0.15,
                ylabel={Zeit (s/Sample)},
                symbolic x coords={Thesis,DeepDarts-d1,DeepDarts-d2},
                xtick=data,
            ]
            \addplot table[x=system,y=Render-Daten] {\ExecutionTimes};
            \addplot table[x=system,y=d1-val]       {\ExecutionTimes};
            \addplot table[x=system,y=d1-test]      {\ExecutionTimes};
            \addplot table[x=system,y=d2-val]       {\ExecutionTimes};
            \addplot table[x=system,y=d2-test]      {\ExecutionTimes};
            \legend{Render-Daten, d1-val, d1-test, d2-val, d2-test}
        \end{axis}
    \end{tikzpicture}
    \caption{Dauer der Normalisierung auf unterschiedlichen Datensätzen, gruppiert nach Systemen.}
    \label{fig:cv_dauer}
\end{figure}

Die tatsächlichen Ausführungszeiten der Systeme sind stark abhängig von der Infrastruktur, auf der die Systeme ausgeführt werden. Daher ist ihnen keine zu starke Bedeutung zuzusprechen. Die relativen Ausführungszeiten lassen sich jedoch miteinander vergleichen, um eine Einschätzung der Performance der Systeme zueinander zu erlangen. Die Inferenz des DeepDarts-Systems ist auf den DeepDarts-Daten un einen Faktor $4$ schneller und bei den gerenderten Daten um den Faktor $2,6$. Die Unterschiede liegen in den Arbeitsweisen der Systeme: DeepDarts verwendet ein neuronales Netz, dessen Ausführungszeit proportional zu den Eingabedaten skaliert, während die Bilddaten in dieser Thesis in einem Vorverarbeitungsschritt skaliert werden, um nahezu unabhängig von der Eingabegröße der Bilder zu sein. Die unterschiedlichen Ausführungszeiten zwischen DeepDarts-Daten und Render-Daten dieses Systems ergeben sich aus der minimalen Bildgröße dieses Vorverarbeitungsschritts, in welchem die Bilder zwischen $800$ und $1600\,\text{px}$ skaliert werden und damit über den Abmessungen der DeepDarts-Daten liegen.

\subsubsection{Findung einer Normalisierung} % ----------------------------------------------------

\begin{figure}[ht]
    \centering
    \begin{tikzpicture}
        \matrix (m) [
            matrix of nodes,
            nodes in empty cells,
            row sep=0.5cm,
            column sep=0.5cm,
            nodes={anchor=center}
        ]
        {
                         & Diese Thesis & DeepDarts-$d_1$ & DeepDarts-$d_2$ \\
            % -----------------------------------
            Render-Daten &
            \drawpie{97/my_green, 3/my_red}
                         &
            \drawpie{100/my_red, 0/my_green}
                         &
            \drawpie{98/my_red, 2/my_green}                                 \\

            % -----------------------------------
            $d_1$-val    &
            \drawpie{99/my_green, 1/my_red}
                         &
            \drawpie{100/my_green, 0/my_red}
                         &
            \drawpie{37/my_green, 63/my_red}                                \\


            % -----------------------------------
            $d_1$-test   &
            \drawpie{100/my_green, 0/my_red}
                         &
            \drawpie{100/my_green, 0/my_red}
                         &
            \drawpie{83/my_green, 17/my_red}                                \\

            % -----------------------------------
            $d_2$-val    &
            \drawpie{99/my_green, 1/my_red}
                         &
            \drawpie{100/my_red, 0/my_green}
                         &
            \drawpie{100/my_green, 0/my_red}                                \\

            % -----------------------------------
            $d_2$-test   &
            \drawpie{98/my_green, 2/my_red}
                         &
            \drawpie{100/my_red, 0/my_green}
                         &
            \drawpie{100/my_green, 0/my_red}                                \\
        };
    \end{tikzpicture}
    \caption{Auswertung der Findung von Normalisierungen auf Daten. Grüne Bereiche stehen für erfolgreiche Normalisierungen, rote Bereiche für fehlgeschlagene Normalisierungen.}
    \label{fig:cv_normalisierung}
\end{figure}

In diesem Teil der Auswertung wird die Fähigkeit der Systeme betrachtet, eine Normalisierung der Bilder durchzuführen. Eine erfolgreiche Normalisierung bezieht sich für diese Auswertung lediglich darauf, ob ausreichend Orientierungspunkte für eine Normalisierung identifiziert werden konnten. Das DeepDarts-System muss dafür in der Lage sein, drei Orientierungspunkte zu identifizieren, da das System einem fehlenden Orientierungspunkt durch Interpolation ergänzt. Für das System dieser Thesis beinhaltet diese Anforderung die Lokalisierung des Mittelpunkts und mindestens drei weiterer Orientierungspunkte. Die Wahl der Orientierungspunkte ist dabei bei dem DeepDarts-System auf vier vordefinierte Punkte festgelegt während das hier erarbeitete System 60 mögliche Punkte erkennen kann.

Die Ergebnisse dieser Auswertung sind in \autoref{fig:cv_normalisierung} in Form von Kuchendiagrammen dargestellt. Die Auswertung ist sowohl hinsichtlich der Systeme als auch hinsichtlich der Datensätze aufgeteilt. Der Algorithmus dieser Thesis ist auf allen Datensätzen in der Lage, in mindestens $97\%$ der Bilder eine Normalisierung zu ermitteln. Die Performance ist dabei weitestgehend unabhängig von dem Ursprung der Daten. Demgegenüber steht die Performance der DeepDarts-Systeme $d_1$ und $d_2$. Während $d_1$ zwar Auswertungen von $100\%$ auf den eigenen Validierungs- und Test-Daten erzielt, ist es nicht in der Lage, positive Ergebnisse auf anderen Daten zu erzielen. Die Auswertung von $d_2$ auf den eigenen Daten liegt ebenfalls bei $100\%$ und zumindest die Findung der Entzerrung der Test-Daten aus $d_1$ deckt sich mit $83.4\%$ mit den Beobachtungen aus dem eigenen Paper. Jedoch ist die Auswertung auf den Validierungs-Daten von $d_1$ mit $36.8\%$ nicht annähernd auf diesem Niveau und auf den gerenderten Daten konnten lediglich für $2.2\%$ der Daten normalisiert werden.

Diese Auswertung stärkt die Erkenntnis des Overfittings des DeepDarts-Systems und zeigt gleichzeitig die Fähigkeit dieses Systems, zumindest ausreichend Orientierungspunkte zu finden, um eine Normalisierung zu ermöglichen.

\todo{Diesen Satz evtl. in die Diskussion}

\subsubsection{Genauigkeit gefundener Normalisierungen} % -----------------------------------------

\pgfplotstableread[col sep=comma]{
    system,       Render-Daten, d1-val, d1-test, d2-val, d2-test
    Thesis,       17.234,       3.104,  3.148,   3.181,  3.690
    DeepDarts-d1, ,             0.289,  0.549,   ,       
    DeepDarts-d2, 1568.744,     1.499,  1.591,   0.828,  1.305
}\Similarities

\begin{figure}
    \centering
    \begin{tikzpicture}
        \begin{axis}[
                width=\textwidth,
                height=10cm,
                ymode=log,
                ybar,
                bar width=0.35cm,
                enlarge x limits=0.15,
                ylabel={Genauigkeit [px]},
                symbolic x coords={Thesis, DeepDarts-d1, DeepDarts-d2},
                xtick={Thesis,DeepDarts-d1,DeepDarts-d2},
                legend style={at={(0.5,0.98)}, anchor=north},
            ]
            \addplot table[x=system,y=Render-Daten]  {\Similarities};
            \addplot table[x=system,y=d1-val]        {\Similarities};
            \addplot table[x=system,y=d1-test]       {\Similarities};
            \addplot table[x=system,y=d2-val]        {\Similarities};
            \addplot table[x=system,y=d2-test]       {\Similarities};
            \legend{Render-Daten, d1-val, d1-test, d2-val, d2-test}
        \end{axis}
    \end{tikzpicture}
    \caption{Genauigkeiten der Normalisierungen auf unterschiedlichen Datensätzen, gruppiert nach Systemen. Sofern keine Normalisierung möglich war, existiert kein Balken. Aufgrund der Darstellung in logarithmischer Form ist die Grundlinie der Baken bei $y=1$. Balken für Werte kleiner als 1 werden daher nach unten ausgerichtet dargestellt.}
    \label{fig:cv_genauigkeit}
\end{figure}

Die Genauigkeit der Normalisierungen wird mit der in \autoref{sec:cv_metriken} eingeführten Metrik durchgeführt und die Ergebnisse sind in \autoref{fig:cv_genauigkeit} dargestellt. Es ist zu erkennen, dass signifikante Unterschiede zwischen den Systemen vorherrschen. DeepDarts' System $d_2$ erzielt auf seinen eigenen Daten ausgesprochen gute Ergebnisse mit durchschnittlich $0.419\,\text{px}$ Abweichung erzielt während es auf anderen Daten keine Ergebnisse erzielen kann. $d_2$ hingegen kann auf allen Datensätzen Ergebnisse erzielen, jedoch sind deutliche Unterschiede zwischen den Quellen der Datensätzen zu erkennen. Auf DeepDarts-Daten können sehr gute Ergebnisse mit durchschnittlich $1.545\,\text{px}$ Abweichung auf den $d_1$-Daten und $1.067\,\text{px}$ auf den $d_2$-Daten erzielt werden. Weit davon abweichend ist jedoch die Auswertung auf den gerenderten Daten, auf denen lediglich eine mittlere Abweichung von $1568.744\,\text{px}$ Abweichung möglich war.
Diese Beobachtung lässt darauf schließen, dass eine zuverlässige Normalisierung nicht mit diesem System möglich war, da die Bilder lediglich eine Größe von $800 \times 800\,\text{px}$ besaßen.
Das in dieser Thesis erarbeitete System war jedoch in der Lage, auf allen Datensätzen Ergebnisse zu erzielen, die sich etwa in gleichen Größenbereichen befinden.

\todo{Absatz beenden.}

\subsubsection{Zusammenfassung der Daten} % -------------------------------------------------------

- DD schneller
- MA besser

\todo{Daten zusammenfassen}

Die Systeme agieren auf unterschiedliche Weisen indes DeepDarts Ergebnisse durch ein neuronales Netz mit möglicher Beschleunigung durch GPU-Ausführung erlangt während das System dieser Arbeit single-threaded auf einer CPU ausgeführt wird.
DeepDarts verwendet ein mit PyTorch erstelltes und in TensorFlow verwendetes neuronales Netz, bei welchem die Ausführung als kompilierter Maschinencode umgesetzt ist. Dies ermöglicht die Parallelisierung der Ausführung für eine optimierte Laufzeit.
Demgegenüber steht die Implementierung des Systems dieser Thesis, in der zur Umsetzung hauptsächlich interpretierter Python-Code unter der Verwendung er Frameworks NumPy, Scikit-Learn und OpeCV verwendet wurden. Diese Frameworks sind ebenfalls teilweise unter Einbindung von kompiliertem Maschinencode verwendet worden, jedoch ist der Bottleneck der Ausführung durch die Geschwindigkeit der Implementierung von Logik in Python-Code limitiert.
Der direkte Vergleich der Systeme ist daher plattformabhängig und möglicherweise nicht belastbar. Er gibt jedoch einen groben Überblick über die Arbeitsweisen und Performances der unterschiedlichen Systeme.

