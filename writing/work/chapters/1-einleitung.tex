% !TeX root = ../main.tex

\chapter{Einleitung}
\label{cha:einleitung}

\todo{Einleitende Sätze der Einleitung.}

% -------------------------------------------------------------------------------------------------

\section{Projektübersicht}
\label{sec:projektuebersicht}

Darts ist ein beliebtes Spiel mit vielerlei Spielvariationen und geringer Einstiegsschwelle für neue Spieler. Es ist allseits bekannt und weit verbreitet. Während für den Heimgebrauch sowie für Gaststätten elektronische Dartscheiben mit eingespeicherten Spielvarianten und automatischem Scoring existieren, werden in professionellen Kreisen weiterhin analoge Dartscheiben ohne integrierte Mechanismen verwendet, die ein automatisches Scoring ermöglichen. Das Ermitteln der erzielten Punktzahlen kann im Steeldarts entweder manuell oder automatisch geschehen. Manuelles Errechnen der erzielten Punktzahlen erfordert Konzentration und Übung, um die finale Punktzahl korrekt und schnell zu berechnen:
\[ \text{Score} = \sum_{i=1}^{3} \text{mult}_i \cdot \text{Feld}_i \]
Automatisierte Techniken zur Bestimmung der Punktzahl unterscheiden sich in ihren Herangehensweisen. Die zuverlässigste und in professionellen Kreisen am weitesten verbreitete Herangehensweise ist der Einsatz von Multi-Camera-Systemen. In diesem werden mehrere kalibrierte Kameras um die Dartscheibe platziert und aufeinander abgestimmt, sodass eine akkurate Rekonstruktion der Dartscheibe möglich ist. Durch diese Rekonstruktion ist eine Bestimmung der Dartpfeilpositionen und folglich eine Bestimmung des Scorings ermöglicht. Diese Systeme sind gewerblich erhältlich, jedoch ist ihr Einsatz für den gelegentlichen Heimgebrauch nicht im Einklang mit ihren Preisen. Beispiele etablierter Systeme sind AutoDarts \cite{autodarts} und Scolia \cite{scoliadarts}.

In \autoref{sec:related_work} werden weitere Herangehensweisen des automatisierten Dart-Scorings aufgelistet, jedoch setzen diese Systeme allesamt spezielle Infrastruktur oder gesonderte Kalibrierung voraus. Ziel dieser Masterarbeit ist die Erarbeitung eines Systems, in welchem ein zuverlässiges Dart-Scoring ohne spezielle Hardware und ohne feine Kalibrierung ermöglicht wird. Dazu werden Techniken der herkömmlichen Computer Vision und neuronale Netze in einer Art und Weise miteinander kombiniert, die im Einklang mit der Ausführung auf mobilen Endgeräten ist.

Dieses System basiert auf Aufnahmen von Mobiltelefonen, in denen Dartscheiben abgebildet sind. In diesen Aufnahmen wird die Dartscheibe in einem ersten Verarbeitungsschritt algorithmisch identifiziert und es wird eine Entzerrung des Bildes durchgeführt, die die Dartscheibe normalisiert und ihre runde Grundform wiederherstellt. Diese normalisierten Bilddaten werden in einem zweiten Schritt durch ein neuronales Netz verarbeitet, welches die Spitzen der Dartpfeile sowie die Feldfarbe durch Klassifizierung identifiziert und durch eine Regression spezifisch lokalisiert. Durch die ermittelten Positionen in dem normalisierten Bild und die zusätzlichen Informationen der jeweiligen Feldfarben ist eine Zuordnung der getroffenen Felder und folglich eine Ermittlung der erzielten Punktzahl möglich.

Durch die Verwendung eines neuronalen Netzes besteht eine Notwendigkeit einer ausreichenden Anzahl an korrekt annotierten Trainingsdaten. Diese Daten werden durch ein ebenfalls in dieser Arbeit enthaltenes System der Datengenerierung bezogen, in welchem realistische Aufnahmen von Dartscheiben simuliert werden. Die Generierung der Trainingsdaten basiert auf einer Kombination aus prozeduraler Datenerstellung und Zufallsprozessen, durch die eine nahezu unendliche Anzahl unterschiedlicher Daten erstellt werden kann.

% - Was ist das Projekt?
% - - Ziel des Projekts (high level)
% - Was ist der Sinn dieses Projekts?
% - - Nur 1 Kamera zum Scoring -> Single-Camera-System
% - - Verweis: Related Work
% - Herangehensweise: Hybrides System aus CV und NN
% - - Lokalisierung + Normalisierung -> CV
% - - - herkömmliche CV, nicht CNNs / NN
% - - Identifizierung Dartpfeile -> NN
% - - - Was für ein Machine Learning-Problem? -> Klassifikation (sparse + herkömmlich) + Regression
% - - Benötigt (Trainings-) Daten
% - - - Werden synthetisch erstellt
% - - Vorverarbeitung der Bilder
% - - - Lokalisierung + Normalisierung durch CV
% - - Dart-Scoring durch Nachverarbeitungsschritte

% -------------------------------------------------------------------------------------------------

\section{DeepDarts}
\label{sec:deepdarts}

Der Anstoß dieses Projekts wurde durch ein Paper gegeben, in dem ein System zur Identifizierung von Dartpfeilen in Bildern mit anschließendem Scoring vorgestellt wurde \cite{deepdarts}. Das als DeepDarts bezeichnete System ist ebenfalls ein Single-Camera-System, welches auf der Verwendung eines neuronalen Netzes fußt, um ein Dart-Scoring umzusetzen.

DeepDarts verwendet ein YOLOv4-tiny-Netzwerk, welches auf einem Datensatz, bestehend aus etwa $16.000$ händisch annotierten Bildern von Dartscheiben, trainiert wurde \cite{deepdarts-data}. Das System wurde durch die Implementierung einer eigenen Loss-Funktion zur Beurteilung der Vorhersagen umgesetzt und konnte auf den gegebenen Daten teils sehr gute Ergebnisse erzielen.

Die Arbeitsweise von DeepDarts verfolgt einen Single-Shot-Approach, indem in einem Eingabebild bis zu 7 Punkte identifiziert werden, die sich aufteilen in 4 Orientierungspunkte und bis zu 3 Dartpfeilspitzen. Die Orientierungspunkte sind festgelegte Positionen auf der Dartscheibe, die eine Entzerrung dieser ermöglichen. Durch die relativen Positionen der Dartpfeilspitzen zu den Orientierungspunkten werden die getroffenen Felder anhand von nahezu standardisierten Werten der Dartscheibengeometrie abgeleitet.

Die Qualität der Vorhersagen der erzielten Punktzahl wurde durch eine eigene Metrik, dem \ac{pcs}, ausgewertet. Dieser setzt die Anzahl der korrekt vorhergesagten Punktzahlen in Verhältnis zur Anzahl aller präsentierter Daten:
\[ \text{PCS} = \frac{100}{N} \sum_{i=1}^{N} \delta \left(\left(\sum \hat{S}_i - \sum S_i\right) = 0\right)\% \]
In dieser Formel steht $N$ für die Anzahl der vorhergesagten Daten, $\sum S_i$ ist die erzielte Punktzahl und $\sum \hat{S}_i$ die vorhergesagte Punktzahl des Datensamples $i$. Die simple Aussage hinter dieser kompliziert ausgedrückten Formel ist, dass die Anzahl korrekt vorhergesagter Gesamtpunktzahlen ins Verhältnis zu der Gesamtzahl der vorhergesagten Daten gesetzt wird.

Das System wurde mit unterschiedlichen Aufteilungen der gegebenen Daten trainiert. Zusammenfassend konnte DeepDarts auf Validierungs- und Testdaten Auswertungen mit Werten von $\text{PCS} = 84,0\,\%$ bis $\text{PCS} = 94,7\,\%$ erzielen. Als zentrale Schwachstelle des Systems wurde das Identifizieren der Orientierungspunkte aufgelistet, welches für fehlerhafte Erkennungen sorgte.

Bei genauer Betrachtung und Inferenz des Systems auf eigenen Daten kristallisierte sich jedoch ein anderes Bild der tatsächlichen Performance heraus. Das System konnte auf Bildern außerhalb der IEEE-Daten wenig Erfolge verzeichnen, wodurch ein starkes Overfitting des Netzwerks naheliegt. Die zum Training verwendete Datenlage war aufgrund der verwendeten Aufnahmetechniken stark limitiert. Etwa $14.000$ der $16.000$ Trainingsdaten wurden frontal aufgenommen und zeigten die selbe Ausrichtung der selben Dartscheibe. Die restlichen $2.000$ Daten wurden mit einer zweiten Dartscheibe und zu Teilen aus unterschiedlichen Winkeln aufgenommen. Zudem wurde die Korrektheit der gelabelten Daten in Frage gestellt. Laut eigener Auswertung wurde für $1.200$ ausgewählte Daten eine Deckungsgleichheit der notierten und annotierten Punktzahlen von $97,6\,\%$ ermittelt; 29 Bilder wurden folglich nicht korrekt annotiert.

Durch diese Datenlage ist keine Abdeckung zu erwartender Aufnahmen in einem realen Einsatz des Systems dar, was bei der Inferenz auf eigenen Aufnahmen zu erkennen war. DeepDarts leidet unter massivem Overfitting und ist nicht in der Lage, zuverlässig auf unabhängigen Daten zu generalisieren. Diese Beobachtung wurde als zentrale Erkenntnis vermerkt und hat die Form der Arbeit stark geprägt.

Ziel dieser Masterarbeit ist es, die identifizierten Schwachpunkte sowie die eingesetzten Techniken von DeepDarts in einem neuen Ansatz zusammenzuführen und auf den gewonnenen Erkenntnissen aufzubauen. Der Aufbau dieser Arbeit ergibt sich aus den zentralen Fehlerquellen von DeepDarts.

% - Anstoß dieses Projekts
% - Paper für System zu Darts-Scoring in Single-Cam-System
% - Herangehensweise
% - - YOLO-Netzwerk
% - - ...
% - Ergebnisse
% - - ...
% - Schwachstellen
% - - stark limitierte Datenlage bzgl. Diversität
% - - eher Proof-of-Concept als einsetzbares System
% - - keine Generalisierbarkeit durch Overfitting
% -> Ziel dieser Arbeit: Erkenntnisse aus DeepDarts verwenden, um robusteres System aufzubauen

% -------------------------------------------------------------------------------------------------

\section{Einsatz synthetischer Datenerstellung}
\label{sec:einsatz_daten}

Eine wichtige Erkenntnis des DeepDarts-Systems ist der Mangel qualitativ hochwertiger Trainingsdaten. Die manuelle Aufnahme und Annotation von Daten ist sowohl zeitaufwändig wie fehleranfällig. Fehlerhafte Annotationen in den Daten werden von dem Netzwerk während des Trainings erlernt und beeinträchtigen die Qualität der Inferenz. Zudem zeichnet sich ein robustes Training durch eine möglichst uniforme Abdeckung aller zu erwartender Eingabedaten aus. In Hinsicht auf Bilder von Dartscheiben beinhaltet dies die Einbindung einer Vielzahl unterschiedlicher Dartscheiben sowie Umgebungsbedingungen. Eine reale Umsetzung dieser Anforderungen in Kombination mit einer ausreichenden Anzahl an Daten für ein geeignetes Training würde sehr viel Zeit in Anspruch nehmen, was weder dem Umfang noch dem Schwerpunkt dieser Arbeit entspricht.

Stattdessen wurde sich für die synthetische Erstellung von Trainingsdaten unter Verwendung von 3D-Modellierungssoftware entschieden. Auf diese Weise ist eine automatisierte Datenerstellung ermöglicht, die durch den Einsatz zeitgemäßer Technik prozedural und fotorealistisch erstellt werden kann. Durch einen anfänglichen Mehraufwand des Aufsetzen dieses Systems können anschließend beliebige Datenmengen erstellt werden, ohne dass manuelle Eingriffe notwendig sind. Zusätzlich sind alle relevanten Informationen der Szene zugänglich, sodass eine korrekte Annotation ermöglicht ist.

% - KI-Training basiert auf Daten
% - Korrektheit von Daten relevant
% - Generierung von Daten als Mittel zur Erstellung ausreichender Menge
% - viele Daten erstellen mit wenig Aufwand
% - Umfang der Daten klar definiert: Dartpfeile auf Dartscheibe verteilen
% - - schematische Beschreibung der Daten möglich

% -------------------------------------------------------------------------------------------------

\section{Einsatz herkömmlicher \acl{cv}}
\label{sec:einsatz_cv}

Die Erkennung der Dartscheibe geschieht in DeepDarts durch das selbe neuronale Netz, welches zugleich die Dartpfeile identifiziert. Das neuronale Netz ist darauf trainiert, spezifische Keypoints entlang der Außenseite der Dartscheibe zu identifizieren, anhand derer die Dartscheibe entzerrt wird. Während diese Herangehensweise für gute Ergebnisse in positiven Fällen gesorgt hat, ist sie an einfacher Verdeckung eben dieser spezifischen Positionen gescheitert. Diese Verdeckungen können sowohl durch externe Objekte wie einen Dartschrank als auch durch die Dartpfeile selbst entstehen. Sind die Punkte nicht eindeutig zu erkennen, werden sie nicht lokalisiert und die Entzerrung sowie alle nachfolgenden Identifizierungen schlagen fehl. Zusätzlich kann sich eine verschobene Erkennung der Orientierungspunkte auf die Genauigkeit der Vorhersage ausschlagen.

Zur Lösung dieses Problems wurde von \citeauthor{deepdarts} die Idee unterbreitet, Redundanz durch das Einbinden weiterer Orientierungspunkte zu schaffen. Dieser Ansatz wurde in dieser Masterarbeit verfolgt, jedoch nicht unter der Verwendung eines neuronalen Netzes. Ein zentrales Problem neuronaler Netze ist das Training, welches viele Ressourcen beansprucht und abhängig von den Trainingsdaten ist. Ihre Arbeitsweisen basieren auf den ihnen präsentierten Trainingsdaten und sind weder bekannt noch nachvollziehbar. Hintergründe fehlerhafter Erkennungen können daher nicht eingesehen werden und eine Adaption des Systems auf neue Gegebenheiten kann ausschließlich unter Verwendung einer ausreichenden Anzahl korrekt annotierter Daten ablaufen.

Aus diesen Gründen wurde sich für diese Thesis dazu entschieden, einen Schritt zurückzugehen und die Verwendung eines neuronalen Netzes für diese Aufgabe auszuschließen. Dartscheiben verbindet ein gemeinsamer Grundaufbau, der durch Richtlinien offizieller Regelwerke mit etwaigen Toleranzen festgelegt ist \cite{wdf-rules,pdc_rules}. Dieser Aufbau der Dartscheiben ist begünstigend für eine Verarbeitung mit herkömmlichen Techniken der \ac{cv}. Kontrastreiche Farbgebung und festgelegte, markante Geometrien sind ideale Merkmale, an denen \ac{cv}-Algorithmen ansetzen.

Der Ablauf der Entzerrung der Dartscheiben läuft in mehreren Schritten ab. Zuerst wird die Position einer Dartscheibe in einem Bild bestimmt. Diese unterliegt o.\,B.\,d.\,A. einer perspektivischen Verzerrung, die in den folgenden Schritten entzerrt wird. Dazu werden die Winkel der Felder radial um den Mittelpunkt bestimmt und ausgeglichen, sodass alle gleichwertig sind. Abschließend wird eine Vielzahl an Orientierungspunkten identifiziert, anhand derer eine perspektivische Entzerrung vorgenommen wird, die zu einer Normalisierung der Dartscheibe führt.

Durch diesen Algorithmus ist die Identifizierung sowie Entzerrung von Dartscheiben in Bildern beliebiger Größen möglich. Da dieser Algorithmus weitestgehend deterministisch agiert und die Arbeitsweise zu jeder Zeit transparent einsehbar ist, ist das Identifizieren von spezifischen Fehlerquellen und eine Adaption der Arbeitsweise mit weitaus weniger Aufwand verbunden als das Trainieren eines neuronalen Netzes. Zudem können neue Datenlagen anhand einer geringen Anzahl von Beispielbildern in den Algorithmus aufgenommen werden, beispielsweise die Integration neuer Farbschemata von Dartscheiben.

Der Umsetzung dieses Algorithmus ist \autoref{cha:cv} zugewiesen. In diesem werden Grundlagen der \ac{cv}, die Methodik und Hintergründe zu dem Algorithmus sowie Details spezifischer Implementierungen dargestellt. Die Ergebnisse dieses Algorithmus werden zuletzt ebenfalls dargestellt und erläutert.

% - ein Problem bei DD: Entzerrung
% - - Finden von Orientierungspunkten durch NN
% - - Punkte verdeckt = nicht erkennbar
% - Problempunkt NN:
% - - muss trainiert werden
% - - ist eine Blackbox
% - - Fehlschlagen kann nicht debuggt werden
% - Lösung dieser Arbeit: CV
% - - gemeinsamer Grundaufbau bei allen Dartscheiben vorhanden
% - - - keine Abweichungen durch unterschiedlichen Aufbau
% - - - Geometrie weitestgehend vereinheitlicht -> WDF Rules
% - - Dartscheiben sehr markante Farben und Formen
% - - algorithmische Erkennung dadurch möglich
% - Ablauf:
% - - Erkennung
% - - Entzerrung der Winkel
% - - Orientierungspunkte finden
% - - Entzerrung der restlichen Geometrie

% -------------------------------------------------------------------------------------------------

\section{Einsatz neuronaler Netze}
\label{sec:einsatz_nn}

Trotz zuvor aufgezählter Schwachpunkte neuronaler Netze sind sie algorithmischen Techniken in vielerlei Hinsicht überlegen. In dieser Arbeit werden neuronale Netze zur Lokalisierung von Dartpfeilspitzen in normalisierten Bildern genutzt. Das Ziel dieses Teils der Thesis ist die korrekte Lokalisierung von Dartpfeilen in Bildern zur Ermittlung der erzielten Punktzahl nach einer Dartsrunde.

Die von \citeauthor{deepdarts} geleistete Vorarbeit durch DeepDarts hat bereits die Fähigkeit neuronaler Netze gezeigt, diese Aufgabe zu bewältigen. Obwohl durch starkes Overfitting keine Generalisierbarkeit des Systems gezeigt werden konnte, ist die Möglichkeit der Bewältigung dieser Aufgabe ermittelt worden. In dieser Arbeit gilt es daher, auf den gewonnenen Erkenntnissen aufzubauen und ein System zu entwickeln, welches in der Lage ist, auf unbekannten Daten angewandt zu werden.

In dieser Arbeit wird daher der Ansatz verfolgt, ein neuronales Netz basierend auf der selben Modell-Familie zu verwenden, wie es in DeepDarts genutzt wurde. Seit der Veröffentlichung von DeepDarts \citeyear{deepdarts} wurden neue Architekturen der YOLO-Familie vorgestellt, deren Performance über der der verwendeten Architektur liegen. Zudem werden diese Architekturen als Grundgerüst genutzt, um durch Adaptionen ihres Aufbaus ein auf diese Aufgabe zugeschnittenes neuronales Netz zu etablieren.

Eine wesentliche Änderung ist eine Veränderung der Netzwerkausgaben. DeepDarts verwendet die herkömmlichen Ausgaben der Modellarchitektur, in der Existenzen, Positionen mit Bounding Boxen und Klassen ausgegeben werden. Die Bounding Boxen werden in DeepDarts lediglich im Training verwendet, jedoch nicht bei der Inferenz. Die verwendeten Klassen setzen sich zusammen aus je einer Klasse pro Orientierungspunkt und einer Klasse zur Annotation eines Dartpfeils. Für diese Arbeit wurden die Ausgaben derart angepasst, dass keine Bounding Boxes vorhergesagt werden und die vorhergesagten Klassen die Farben der getroffenen Feldern codieren.

Diese Adaptionen der Ausgaben sorgen dafür, dass keine irrelevanten Informationen in den Ausgaben enthalten sind und eine robustere Rückrechnung der Punktzahl ermöglicht ist. Wohingegen die Punktzahl bei DeepDarts durch relative Positionen zu den Orientierungspunkten ermittelt wird, wird in dieser Arbeit ein System verwendet, welches Positionen und Feldfarben auf einer normalisierten Dartscheibe kombiniert, um eine Punktzahl zu ermitteln. Der wesentliche Vorteil dieser Herangehensweise liegt in der Robustheit gegenüber Verschiebungen in der Normalisierung: Wird ein Dartpfeil an der Grenze zweier Felder erkannt, ist die Vorhersage trotz ungenauer Normalisierung möglich, da eine Ermittlung des Feldes durch die bestimmte Feldfarbe unterstützt wird.

% - Ziel: Erkennung von Dartpfeilspitzen in normalisierten Bildern
% - DD hat gezeigt, dass es möglich ist
% - - trotz Overfitting konnte Fähigkeit des Erlernens gezeigt werden
% - - Generalisierbarkeit noch nicht
% - verfolgter Ansatz
% - - Verwendung gleicher Modell-Familie wie DD
% - - aber neue Version
% - - und mit Adaptionen
% - neue Strukturierung der Netzwerk-Outputs
% - - DD: Existenz + Position (mit Bounding Box) + Typ (Orientierungspunkt 1,2,3,4 + Dartpfeil)
% - - - Rückrechnung der Punktzahlen durch relative Positionen zu Orientierungspunkten
% - - hier: Existenz + Position + getroffene Feldfarbe (schwarz, weiß, rot, grün, außerhalb)
% - - - Ermittlung der Punktzahl durch Position + Farbe
% - - Vorteile dieses Ansatzes:
% - - - keine irrelevanten Informationen durch Bounding Boxes in den Outputs
% - - - robustere Ermittlung der Punktzahl
% - - - - leichte Verschiebungen der Dartpfeile durch Normalisierung hat keinen Einfluss auf Edge-Cases

% -------------------------------------------------------------------------------------------------

\section{Forschungsfragen}
\label{sec:forschungsfragen}

In dieser Masterarbeit werden mehrere Systeme erarbeitet, die sich zu einem Gesamtkonzept zusammensetzen. Die zu behandelnden Forschungsfragen ergeben sich aus den jeweiligen Teilprojekten dieser Arbeit. Aus diesem Grund wird für jeden Teil dieser Thesis eine Forschungsfrage aufgestellt und das Zusammenspiel aller Bestandteile in einer abschließenden Forschungsfrage kombiniert.

\subsection*{1. Welche Qualität synthetischer, realistischer und variabler Daten können in einer automatisierten Pipeline zur Erstellung von Bildern von Dartscheiben mit korrekter Annotation erreicht werden?}

Diese Fragestellung bezieht sich inhaltlich auf die Ausarbeitung der synthetischen Datenerstellung, welche den ersten Teil dieser Arbeit ausmacht. Sie umfasst in erster Instanz die Umsetzbarkeit der automatisierten Erstellung von Bildern mit Dartscheiben. Anschließend ist diese Datenerstellung mit einer Randomisierung von Parametern verbunden, einschließlich Annotation der Daten hinsichtlich der Position der Dartpfeilspitzen in den Bildern. Zuletzt gilt es, diese Datenerstellung in eine automatisierte Pipeline einzubinden, mit der eine autonome Erstellung einer Vielzahl unterschiedlicher Daten ermöglicht wird.

Diese Daten unterliegen zusätzlich der Bedingung, ein möglichst realistisches Erscheinungsbild aufzuweisen und eine große Spanne unterschiedlicher Aussehen und Beleuchtungen abzudecken, sodass die Wahrscheinlichkeit der Datenverzerrung durch einseitige Darstellung minimiert wird. Die Qualität bezüglich des Grades der Realitätsnähe quantitativ einzuordnen ist nicht trivial und wird daher durch qualitative Vergleiche mit echten Daten ermittelt.

\subsection*{2. Zu welchem Grad lässt sich eine zuverlässige algorithmische Erkennung und Normalisierung von Dartscheiben in Bildern ohne den Einsatz neuronaler Netze umsetzen?}

Ziel dieser Forschungsfrage ist die Untersuchung des zweiten Teilprojekts dieser Arbeit. Die algorithmische Identifizierung und Normalisierung hinsichtlich der Dartscheibengeometrie ist in dieser Arbeit als Vorverarbeitungsschritt eingebunden und geschieht ohne den Einsatz von Systemen, deren Arbeitsweise auf Daten basierend automatisiert erlernt werden. Jegliche Arbeitsschritte und Parameter des Systems unterliegen der sorgfältigen Analyse zu erwartender Daten.

Zur Beantwortung dieser Forschungsfrage wird eine Analyse auf Daten unterschiedlicher Quellen durchgeführt, die Einblicke in die Arbeitsweise und Vielseitigkeit des erarbeiteten Systems liefern.

\subsection*{3. Wie zuverlässig ist eine Generalisierung eines durch \ac{ood}-Training mit synthetischen Daten trainiertes neuronales Netzwerk auf Daten realer Dartscheiben?}

Die im ersten Teil dieser Thesis erstellten Daten werden für das Training eines neuronalen Netzes eingesetzt. Durch die Verwendung synthetischer Daten als Datenlage sind Unterscheide zwischen Trainings- und Inferenzdaten zu erwarten. Der Grad des Unterschieds unterliegt der Qualität der Datenerstellung, kann jedoch als nicht unerheblich eingestuft werden. Diese Forschungsfrage untersucht die Fähigkeit eines neuronalen Netzes, zuverlässige Vorhersagen auf realen Daten trotz dieser Differenzen zu erbringen. Zur Beantwortung dieser Forschungsfrage wird die Inferenz auf unterschiedlichen Datensätzen protokolliert, die sowohl synthetische als auch echte Daten umfasst. Auf diese Weise wird die Fähigkeit der Generalisierung des Netzes auf synthetischen Daten im Vergleich mit echten Daten ins Verhältnis gesetzt, um einen relativen Unterschied aufzeigen zu können.

\subsection*{4. Ist das in dieser Thesis erarbeitete Gesamtsystem in der Lage, signifikante Verbesserungen hinsichtlich der Performance und Genauigkeit im Vergleich zu DeepDarts zu erzielen?}

Anstoß dieser Thesis ist die Arbeit von \citeauthor{deepdarts}, in der ein System mit dem Namen DeepDarts vorgestellt wurde \cite{deepdarts}. Dieses System konnte sehr gute Ergebnisse auf eigenen Daten erzielen und gilt damit als Maßstab für Genauigkeit und Geschwindigkeit. Die Konfigurationen und Gewichte, die für die Auswertung von DeepDarts verwendet wurden, sind in dieser Arbeit genutzt wurden, um Vergleichswerte zu setzen. Da dieses System als Erweiterung von DeepDarts vorgesehen ist, ist ein Übertreffen des Systems hinsichtlich unterschiedlicher Metriken das zentrale Ziel dieser Arbeit, auf das alle vorgestellten Systeme dieser Thesis ausgelegt sind.

% -------------------------------------------------------------------------------------------------

\section{Themenbezogene Arbeiten}
\label{sec:related_work}

Diese Arbeit weist mehrere Schwerpunkte auf, hinsichtlich derer verwandte Arbeiten betrachtet werden können. In den folgenden Unterabschnitten werden unterschiedliche Bereiche beleuchtet, in denen verwandte Arbeiten veröffentlicht wurden.

\subsection{Prozedurale Datenerstellung}

Das prozedurale Erstellen von Daten für eine vielfältige Auswahl unterschiedlicher Szenarien ist kein Themengebiet, dessen spezifischer Einsatz im Bereich des Trainings neuronaler Netze verankert ist. Im Fachgebiet der Spieleindustrie wird diese Technik verwendet, gesteuerten Zufall in das Spielerlebnis einfließen zu lassen \cite{proc_data_games_1,proc_data_games_2,proc_data_games_3}. Die Brücke zu dieser Arbeit wird dabei durch die zufällige Erstellung von Welten geschlagen, die analog zu der Randomisierung der Umgebung und dem Aussehen der Dartscheibe zu sehen sind, wie prozedurale Datenerstellung in dieser Arbeit verwendet wird. Diese Umsetzungen basieren auf dem gleichen Konzept, jedoch ist die Art der Umsetzung unterschiedlich gehandhabt.

% - Prozedurale Datenerstellung für Spiele:
% - - \cite{proc_data_games_1,proc_data_games_2,proc_data_games_3}
% - - Erstellung zufälliger Spielumgebungen auf Grundlage zufälliger Generation
% - - hier: zufällige Generierung von Darts-Runden statt Welten
% - - Konzept gleich, wird bereits genutzt
% - - PGD-G: Procedural Data Generation for Games

\subsection{Synthetische Datenerstellung für das Training neuronaler Netze}

Die Vorgehensweise synthetischer Datenerstellung zur Generierung von Trainingsdaten wurde bereits in unterschiedlichen Systemen eingesetzt \cite{synth_data}. Die Art der synthetisch erstellten Daten umfasst dabei eine große Spanne unterschiedlicher Szenarien. So wurde bereits die Effektivität der Nutzung von 3D-Modellen in der Datengenerierung erfolgreich eingesetzt \cite{synth_data_procedural,synth_data_cars_with_cam_aug,synth_data_pose_estimation}. Mit der Verwendung von 3D-Software zur Generierung von Trainingsdaten konnten ebenfalls positive Ergebnisse erzeugt werden \cite{synth_data_blender_defects,data_gen_importance,synth_data_importance_2}. Obwohl Unterschiede zwischen generierten Daten und echten Daten bestehen, konnte gezeigt werden, dass selbst durch reines \ac{ood}-Training Systeme trainiert werden konnten, die auf echte Daten generalisieren konnten \cite{ood_simulated_training}.
In Hinsicht auf das Darts-Scoring existieren bereits

% - - Generierung von Trainingsdaten für KI nicht unüblich: \cite{synth_data,synth_data_blender_defects,synth_data_cars_with_cam_aug,synth_data_importance_2,synth_data_pose_estimation,synth_data_procedural}

\subsection{Dart-Scoring}

Neben bereits etablierten Multi-Camera-Systemen wie Scolia und Autodarts \cite{,scoliadarts,autodarts} existieren weitere Herangehensweisen an diese Aufgabe. Scolia und Autodarts verwenden jeweils 3 Kameras, um eine 3D-Rekonstruktion der Dartscheibe und den auf ihr eingetroffenen Pfeilen zu ermöglichen. Diese Systeme sind jedoch kostspielig und nicht für Amateurnutzung vorgesehen. Um diese Hürden zu umgehen, wurde eine Arbeit um einen Algorithmus zur Nutzung günstiger Kameras für ein kostengünstiges, jedoch ebenfalls akkurates System mit fünf Kameras veröffentlicht \cite{dart_scoring_multicam}.

Die Idee des einfacher zugänglichen und erschwinglichen Dart-Scorings hat in Hobbyprojekten Anklang gefunden, sodass Systeme mit unterschiedlichen Herangehensweisen der Ermittlung eines Scorings entstanden sind. Diese Systeme reichen von der Verwendung von Mikrofonen zur akustischen Triangulation der Einstichstellen \cite{dart_scoring_microphone} über die Verwendung von fünf kostengünstigen Kameras in Kombination mit ausgeklügelter Kalibrierung \cite{dart_scoring_multicam}, stereoskopische Rekonstruktion der Dartscheibe mittels zweier Kameras \cite{darts_project_3,darts_project_4} bis hin zu Single-Camera-Systemen \cite{darts_project_1,darts_project_2}. Für alle diese Systeme ist jedoch mindestens einer der folgenden Schwachpunkte zutreffend: Komplexes oder sehr genaues Setup, Notwendigkeit besonderer Hardware oder mangelhafte Genauigkeit des Systems. Basieren die Systeme auf Differenzbildern, ist eine Verschiebung der Kamera verheerend; ebenso bei der Verwendung von Kameras, deren Positionen zueinander kalibriert werden müssen.

Der Schwachpunkt eines vorherigen Aufbaus oder der Kalibrierung ist durch ein System, wie es mit DeepDarts erhoben wurde, hinfällig, indem ein Scoring auf Grundlage eines beliebigen Kameramodells und ohne weitreichende Ansprüche an die benötigte Infrastruktur als Zielsetzung verfolgt wurde \cite{deepdarts}. Die Verwendung neuronaler Netze für Verarbeitung von Bilddaten ist dabei ein neuer Schritt der Einbindung aktueller Technologien in dieses Forschungsfeld. Dieser Ansatz wird mit dieser Arbeit weiter verfolgt mit der Zielsetzung, neue Technologien in dem Bereich des Dart-Scorings einzusetzen, um ein simples Dart-Scoring mit geringen Anforderungen an den Benutzer zu erzielen.

% - Dart-Scoring-Systeme
% - - GitHub-Projekte
% - - - \cite{darts_project_1,darts_project_2,darts_project_3,darts_proect_4}
% - - - Dart-scoring mittels Webcams + vorherigem Setup
% - - Multi-Cam system
% - - - 5 Kameras
% - - - \cite{dart_scoring_multicam}
% - - Mikrofon-System
% - - - akustische Triangulierung
% - - - \cite{dart_scoring_microphone}

% -------------------------------------------------------------------------------------------------

\section{Aufbau der Arbeit}
\label{sec:aufbau}

Diese Masterarbeit befasst sich mit den Themenbereichen der Datengenerierung, Bildnormalisierung und dem Training eines neuronalen Netzes. Das Zusammenspiel dieser Komponenten ist in \autoref{img:projektstruktur} veranschaulicht. Zunächst wird in \autoref{cha:daten} auf die synthetische Datengenerierung eingegangen. Anschließend wird in \autoref{cha:cv} ein Algorithmus vorgestellt, der eine Normalisierung von Bildern mit Dartscheiben ermöglicht. \autoref{cha:ki} rundet den thematischen Schwerpunkt dieser Masterarbeit mit der Konzeption und dem Training eines neuronalen Netzes zur Identifizierung und Lokalisierung von Dartpfeilspitzen in normalisierten Bildern ab. Diese Kapitel sind jeweils unterteilt in die Abschnitte Grundlagen, Methodik, Implementierung und Ergebnisse. Im Abschnitt \quotes{Grundlagen} werden Themen und Konzepte eingeführt, die für das Verständnis des Kapitels notwendig sind und nicht im erwarteten Wissensgebiet der Audienz liegen. In den Abschnitten \quotes{Methodik} werden die Konzepte und Herangehensweisen der Kapitel dargestellt. Auf Details zur Umsetzung dieser Methodiken werden in den jeweiligen Abschnitten \quotes{Implementierung} eingegangen. Den Abschluss der jeweiligen Kapitel bilden die Darstellung und Auswertung der erzielten Ergebnisse. Im Anschluss an die themenbezogenen Kapitel folgt eine zusammenfassende Diskussion der Ergebnisse in \autoref{cha:diskussion}. Danach wird in \autoref{cha:fazit} das Fazit dieser Masterarbeit gezogen, indem die Forschungsfragen erneut aufgegriffen werden und mit einem Ausblick über mögliche Erweiterungen und Verbesserungen des Systems abgeschlossen wird.

\begin{figure}
    \centering
    \includegraphics[width=0.8\textwidth]{imgs/ma_project_structure.pdf}
    \caption{Überblick über die Projektstruktur. (1) Datenerstellungs-Pipeline; bei der Datengenerierung werden Bilder, Masken und Annotationen erstellt und automatisch normalisiert. (2) Inferenz-Pipeline; beliebige Bilder von Dartscheiben werden algorithmisch normalisiert. (3) Dartpfeil-Erkennung und Scoring; die Erkennung geschieht durch ein neuronales Netz, das Scoring durch Nachverarbeitung der Outputs.}
    \label{img:projektstruktur}
\end{figure}

% - Aufbau der Arbeit beschreiben
% - Start: Datengenerierung
% - Danach: CV-Verarbeitung
% - - Entzerrung / Normalisierung
% - Danach: KI-Training
% - - Dartpfeil-Erkennung
% - Jeweils: Grundlagen, Methodik, Implementierung, Ergebnisse
% - Grund für Aufbau:
% - - Projekte weitestgehend voneinander abgekapselt
% - - Schnittstellen der Systeme klar definiert
% - - Komplexes Projekt, Betrachtung der einzelnen Systeme zur Übersicht
% - - Thematische Kapselung der Arbeit
% - Danach: Diskussion der jeweiligen Systeme
% - Zuletzt: Fazit der Arbeit + Ausblick
