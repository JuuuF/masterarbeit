% !TEX root = ../main.tex

\chapter{Einleitung}
\label{cha:einleitung}

\todo{Einleitende Sätze}

% -------------------------------------------------------------------------------------------------

\section{Projektübersicht}
\label{sec:projektuebersicht}

- Dart Scoring
- - Nutzen des Scorings
- - Scoring durch Single-Camera-System
- Verwendung von KI zur Vorhersage von Dartpfeil-Positionen
- - Was für ein Machine Learning-Problem?
- DeepDarts-System
- - Herangehensweise
- - - YOLO-Netzwerk
- - - ...
- - Ergebnisse
- - - ...
- - Schwachstellen
- - - stark limitierte Datenlage bzgl. Diversität
- - - eher Proof-of-Concept als einsetzbares System
- - - keine Generalisierbarkeit durch Overfitting

\todo{Projektübersicht}

% -------------------------------------------------------------------------------------------------

\section{Warum synthetische Datenerstellung}
\label{sec:warum_daten}

- KI-Training basiert auf Daten
- Korrektheit von Daten relevant
- Generierung von Daten als Mittel zur Erstellung ausreichender Menge
- viele Daten erstellen mit wenig Aufwand
- Umfang der Daten klar definiert: Dartpfeile auf Dartscheibe verteilen
- - schematische Beschreibung der Daten möglich
- - Generierung von Trainingsdaten für KI nicht unüblich: \cite{synth_data,synth_data_blender_defects,synth_data_cars_with_cam_aug,synth_data_importance_2,synth_data_pose_estimation,synth_data_procedural}

\todo{Warum synth. Daten?}

% -------------------------------------------------------------------------------------------------

\section{Related Work}
\label{sec:related_work}

- Prozedurale Datenerstellung für Spiele:
- - \cite{proc_data_games_1,proc_data_games_2,proc_data_games_3}
- - Erstellung zufälliger Spielumgebungen auf Grundlage zufälliger Generation
- - hier: zufällige Generierung von Darts-Runden statt Welten
- - Konzept gleich, wird bereits genutzt
- - PGD-G: Procedural Data Generation for Games

- Dart-Scoring-Systeme
- - GitHub-Projekte
- - - \cite{darts_project_1,darts_project_2,darts_project_3,darts_proect_4}
- - - Dart-scoring mittels Webcams + vorherigem Setup
- - Multi-Cam system
- - - 5 Kameras
- - - \cite{dart_scoring_multicam}
- - Mikrofon-System
- - - akustische Triangulierung
- - - \cite{dart_scoring_microphone}

\todo{Related Work}

% -------------------------------------------------------------------------------------------------

\section{Aufbau der Arbeit}
\label{sec:aufbau}

- Aufbau der Arbeit beschreiben
- Start: Datengenerierung
- Danach: CV-Verarbeitung
- - Entzerrung / Normalisierung
- Danach: KI-Training
- - Dartpfeil-Erkennung
- Jeweils: Grundlagen, Methodik, Implementierung, Ergebnisse
- Grund für Aufbau:
- - Projekte weitestgehend voneinander abgekapselt
- - Schnittstellen der Systeme klar definiert
- - Komplexes Projekt, Betrachtung der einzelnen Systeme zur Übersicht
- - Thematische Kapselung der Arbeit
- Danach: Diskussion der jeweiligen Systeme
- Zuletzt: Fazit der Arbeit + Future Work

\todo{Aufbau erklären}

% -------------------------------------------------------------------------------------------------

\autoref{img:projektstruktur}

\begin{figure}
    \centering
    \includegraphics[width=0.8\textwidth]{imgs/ma_project_structure.pdf}
    \caption{Überblick über die Projektstruktur. (1) Datenerstellungs-Pipeline Aus der Datengenerierung werden Bilder und Masken erstellt und automatisch normalisiert. (2) Inferenz-Pipeline. Beliebige Bilder werden mittels CV-Algorithmen normalisiert. (3) Dartpfeil-Erkennung und Scoring. Die Erkennung geschieht durch ein neuronales Netz und das Scoring verläuft durch Nachverarbeitung der Outputs.}
    \label{img:projektstruktur}
\end{figure}

% Datenquellen
% Die Datengrundlage für dieses Projekt setzt sich aus drei verschiedenen Quellen zusammen. Die erste Quelle sind die bereits annotierten Daten von \citeauthor{deepdarts} \cite{deepdarts-data}, die für das Training des DeepDarts-Systems verwendet wurden. Zur Aufnahme dieser Daten wurden zum einen ein fest montiertes und konstant ausgerichtetes iPhone verwendet, zum anderen eine auf einem Stativ aufgestellte DSLR-Kamera. Diese Daten zeigen zwei unterschiedliche Dartscheiben in wenig variierten Umgebungen und unterliegen daher einer Starken Einseitigkeit (Bias). Aus diesem Grund wurde sich dazu entschieden, für diese Arbeit weitere Daten aufzunehmen.
% Zusätzlich zu diesen realen Daten wurden weitere Daten manuell aufgenommen und annotiert. Dazu wurden unterschiedliche Orte aufgesucht, an denen sich Steeldarts-Scheiben befinden. 139 Aufnahmen wurden am 16. Dezember 2024 in Jess Bar in Kiel erhoben, 198 Aufnahmen wurden am 16. Januar 2025 im Strongbows Pub\footnote{\url{https://www.strongbowspub.de}} in Kiel erhoben und 59 Aufnahmen wurden privat erstellt. Für diese Aufnahmen wurden die Positionen der Dartpfeile und ihre Punktzahlen manuell eingetragen. Die dritte und relevanteste Datenquelle sind synthetisch generierte Daten. Für diese Thesis wurden 20.480 Trainingsdaten und 256 Validierungsdaten erstellt. 