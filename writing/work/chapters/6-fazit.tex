% !TeX root = ../main.tex

\chapter{Fazit}
\label{cha:fazit}

In diesem Kapitel wird ein Fazit zu den in dieser Masterarbeit erarbeiteten Systemen gezogen. Dazu werden die in der Einleitung aufgestellten Forschungsfragen erneut aufgegriffen und unter Einbezug der in den Ergebnissen und der Diskussion erarbeiteten Aspekte beantwortet. Anschließend werden die erzielten Ergebnisse und das Zusammenspiel der einzelnen Systeme zusammengefasst, um ein Fazit dieser Arbeit zu ziehen. Geschlossen wird mit einem Ausblick, in welchem spezifische Aspekte zur Erweiterung und Verbesserung der jeweiligen Systeme aufgezeigt werden. Durch diese kann zukünftige Arbeit an diesem Thema fortgeführt werden.

\section{Beantwortung der Forschungsfragen}
\label{sec:beantwortung_forschungsfragen}

Die Forschungsfragen dieser Arbeit wurden in \autoref{sec:forschungsfragen} eingeführt. Sie beziehen sich thematisch auf die Themenbereiche dieser Arbeit indes je eine Forschungsfrage für die Datenerstellung in \autoref{cha:daten}, die Normalisierung durch \ac{cv} in \autoref{cha:cv} und die Dartpfeil-Lokalisierung in \autoref{cha:ki} aufgestellt wurden. Zusätzlich wurde eine weitere Forschungsfrage hinsichtlich des Gesamtumfangs des Projekts gestellt, in welcher der Bezug zu dem Referenzsystem DeepDarts gezogen wird. Die Beantwortung dieser Forschungsfragen geschieht in den folgenden Unterabschnitten.

\subsection*{1. Welche Qualität synthetischer, realistischer und variabler Daten können in einer automatisierten Pipeline zur Erstellung von Bildern von Dartscheiben mit korrekter Annotation erreicht werden?}

Diese Forschungsfrage bezieht sich auf den ersten Themenbereich dieser Thesis, in welchem Daten durch Simulation automatisch generiert wurden. Die Beantwortung dieser Frage kann hinsichtlich unterschiedlicher Gesichtspunkte ausgelegt werden. Aufgrund einer fehlenden Metrik zur Quantisierung des Realismus der generierten Bilder wurde eine qualitative Analyse vollzogen. In dieser wurden die Ergebnisse der Datengenerierung kritisch betrachtet und es konnte festgestellt werden, dass die Aufnahmen keinen Fotorealismus darstellen. Bei der Betrachtung der Ergebnisse fällt die synthetische Natur der Bilder ins Auge und es ist in den meisten Fällen nicht von der Hand zu weisen, dass es sich bei den Bildern nicht um reale Aufnahmen handelt. Trotz dessen konnten realistische Gebrauchsspuren, Beleuchtungen und Aufnahmen simuliert werden, wie sie in Aufnahmen realer Dartscheiben zu finden sind. Durch Metadatenanalyse konnten realistische Spannen unterschiedlicher Parameter identifiziert und synthetisch simuliert werden. Zusammenfassend lässt sich die Qualität der erstellten Daten als annähernd realistisch einordnen.

Hinsichtlich der Korrektheit der Annotationen der Daten sind geringfügige Ungenauigkeiten in der automatischen Normalisierung der Bilder zur Generierung von Trainingsdaten für das Training neuronaler Netze vorzufinden. Diese Ungenauigkeiten befinden sich im Umfang von Abweichungen lediglich weniger Pixel. Gleiche Abweichungen gelten für die Lokalisierung von Dartpfeilen in Bildern. Die Magnitude dieser Fehler ist jedoch geringfügig und qualitativ gleichauf mit manueller Annotation, jedoch ohne fehlerhafte Annotation oder übersehene Datenpunkte.

Das Scoring der Dartpfeile ist durch den Zugriff auf die Positionen der Dartpfeile im 3D-Raum zuverlässig und fehlerfrei möglich. Die zur Verfügung stehenden Informationen der Dartpfeile im Raum ermöglichen eine exakte Lokalisierung der Dartpfeile auf der Dartscheibe. Auf diese Weise ist eine eindeutige Identifizierung der getroffenen Felder möglich, durch welche die erzielte Punktzahl korrekt abgeleitet werden kann.

\subsection*{2. Zu welchem Grad lässt sich eine zuverlässige algorithmische Erkennung und Normalisierung von Dartscheiben in Bildern ohne den Einsatz neuronaler Netze umsetzen?}

Mit der Unterteilung des Scorings in Normalisierung der Dartscheibe und Lokalisierung der Dartpfeile wird sich im Wesentlichen von der Herangehensweise von DeepDarts gelöst. Aufgrund der Nachteile in der Nutzung neuronaler Netze wurde sich für die Normalisierung für die Verwendung herkömmlicher \ac{cv} entschieden. Die Forschungsfrage dieses Themenbereichs der Arbeit thematisiert den Mehrwert dieser Aufteilung, indem die Zuverlässigkeit betrachtet wird. Die Auswertung der Normalisierung wurde mittels dreier Metriken vorgenommen, die die Dauer, den Erfolg und die Genauigkeit untersuchen.

Die Dauer der algorithmischen Normalisierung ist im Vergleich zu dem DeepDarts-Ansatz messbar größer. Die Ausführungszeit der Normalisierung beläuft sich auf etwa die doppelte Dauer der vollständigen DeepDarts-Inferenz. Absolut betrachtet ist die durchschnittlich gemessene Dauer von $\leq 0,\!5\,\text{s}$ jedoch kein schwerwiegender Kritikpunkt.

Die Fähigkeit der Normalisierung wurde mit $> 97\,\%$ gemessen und ist damit sehr zuverlässig. Einige Kritikpunkte der Ausführung wurden in der Diskussion bereits aufgefasst, anhand derer die Ausführung der Normalisierung verbessert werden kann. Die bereits erzielte Fähigkeit zur Identifizierung stellt jedoch eine sehr gute Grundlage dar.

Hinsichtlich der Genauigkeit der erzielten Normalisierungen wurde eine Metrik verwendet, in welcher die mittlere Verschiebung der Orientierungspunkte betrachtet wurde. Die Auswertungen beliefen sich in unterschiedlichen Datensätzen auf unterschiedliche Wertebereiche. Die geringste Genauigkeit wurde auf synthetischen Daten, die ebenfalls die größte Komplexität vorweisen, erzielt. Die gemessene mittlere Genauigkeit auf den gerenderten Daten beläuft sich auf $< 35\,\text{px}$, was $< 5\,\%$ der Bildbreite entspricht.

Zusammenfassend lässt sich zur Beantwortung dieser Forschungsfrage festhalten, dass die Kombination der Metriken das Bild einer sehr zuverlässigen Normalisierung ohne den Einsatz neuronaler Netze zeichnen. Die Erfolgsrate sowie die Genauigkeit des Algorithmus sind zuverlässig, einzig die Dauer der Ausführung ist als potenziell nicht zufriedenstellend festzuhalten.

\subsection*{3. Wie zuverlässig ist eine Generalisierung eines durch \ac{ood}-Training mit synthetischen Daten trainiertes neuronales Netzwerk auf Daten realer Dartscheiben?}

Das Training des neuronalen Netzes zur Lokalisierung der Dartpfeile in normalisierten Bildern verlief auf $98,\!5\,\%$ synthetischen Daten mit Salting von $1,\!5\,\%$ realer Daten. Damit ist kein reines \ac{ood}-Training vollzogen. Die Auswertung des neuronalen Netzes wurde auf Datensätzen unterschiedlicher Quellen vollzogen, in welchen die Effekte des \ac{ood}-Trainings dargestellt werden konnten.

Die Auswertung auf unterschiedlichen Metriken zur Betrachtung der Performance unterschiedlicher Bestandteile der Netzwerkarchitektur zeigt eine zwar messbare, jedoch geringfügige Präferenz synthetischer Daten gegenüber realen Daten. Die wesentliche Fehleranfälligkeit wurde in der Identifikation der Feldfarben ermittelt, was auf eine Diskrepanz simulierter und realer Farben hinweist. Trotz dieser Differenzen ist von keiner starken Verzerrung oder kritischer Einseitigkeit der Trainingsdaten auszugehen, welche nicht in realen Aufnahmen widergespiegelt ist.

Die Zuverlässigkeit des in dieser Arbeit trainierten neuronalen Netzes ist daher im Wesentlichen nicht durch den Einsatz von \ac{ood}-Training beeinflusst. Bei der Inferenz auf synthetischen und realen Daten können annähernd gleichwertige Auswertungen erzielt werden.

\subsection*{4. Ist das in dieser Thesis erarbeitete Gesamtsystem in der Lage, signifikante Verbesserungen hinsichtlich der Performance und Genauigkeit im Vergleich zu DeepDarts zu erzielen?}

Das Zusammenspiel der in dieser Thesis erarbeiteten Komponenten -- synthetische Datengenerierung, algorithmische Normalisierung und Lokalisierung von Dartpfeilen -- zielt auf eine Verbesserung des DeepDarts-Systems ab. Die Erkenntnisse von DeepDarts wurden genutzt, um die Strukturierung und den Aufbau dieser Arbeit zu bestimmen und die Ausarbeitung der Themenbereiche ist ausgerichtet auf die gezielte Steigerung der Performance unterschiedlicher Aspekte des Systems.

Die Auswertung dieses Systems zeigt eine klare Verbesserung hinsichtlich der Fähigkeit zur Generalisierbarkeit auf neue Daten auf. Während mit den Systemen von DeepDarts lediglich Erfolge auf eigenen Daten erzielt werden konnten, ist die erfolgreiche Inferenz auf dem System unbekannten Daten durch die strikte Trennung von Trainings-, Validierungs- und Testdaten gezeigt worden. Die Auswertungen dieses Systems zeigen eine signifikant bessere Performance hinsichtlich unterschiedlicher Metriken. Bei der Wahl von Daten und Metriken wurde sich zu Teilen auf die in DeepDarts verwendeten Vorgaben gestützt, um einen Vergleich ziehen zu können, der nicht durch vorteilhaft gestrickte Daten oder Metriken beeinflusst ist.

\section{Zusammenfassung des Systems}

In dieser Thesis ist eine neue Herangehensweise an ein bereits bestehendes System vorgestellt. Es wurde nicht nur eine automatisierte Erweiterung der Trainingsgrundlage eines neuronalen Netzes geschafft, sondern zusätzlich wurden neue Techniken zur Bewältigung der Aufgabe eingebunden und die Erkenntnisse aus dem bestehenden System wurden verwendet, um ein robustes System zu erschaffen.

Mit der synthetischen Datenerstellung wird das Erstellen beliebiger Datenmengen zum Trainieren eines neuronalen Netzes ermöglicht. Qualitativ betrachtet ist kein vollständiger Fotorealismus erzielt, jedoch ist die erzielte Qualität der Daten bei Weitem nicht unrealistisch. Der Realismus beläuft sich auf einen Grad, auf welchem das Trainieren eines neuronalen Netzes ohne wesentliche Verzerrungen möglich ist und eine Inferenz auf realen Daten mit geringer Diskrepanz zu synthetischen Daten ermöglicht. Durch gezieltes Angehen unterschiedlicher Kritikpunkte der Datenerstellung können die Unterschiede zwischen synthetischen und realen Daten verringert und ein erweitertes Training ermöglicht werden.

Im Vergleich zu DeepDarts konnte auf Grundlage der synthetischen Daten und der Verwendung ausgiebiger Augmentierung ein System trainiert werden, dessen Auswertungen nicht auf Overfitting der eigenen Daten schließen lässt. Dadurch ist die Generalisierbarkeit des Systems weitestgehend aufrechterhalten, sodass die Inferenz auf neuen Daten unter vorhersehbarer Genauigkeit abläuft.

Durch die algorithmische Normalisierung der Dartscheibe wurde ein System vorgestellt, welches in der Lage ist, beliebige Bilder von Dartscheiben zu normalisieren. Die algorithmische Natur dieser Herangehensweise schlägt sich in der Wartbarkeit und Interpretierbarkeit des Systems nieder. Während die Arbeitsweise von Ansätzen unter Einbindung neuronaler Netze nicht bekannt sind und Fehlerfälle durch intensive Auswertungen und Interpretierungen angegangen werden müssen, ist die Adaption eines Algorithmus gezielt möglich. Ursachen der fehlerhaften Verarbeitung können schnell ausfindig gemacht werden und es bedarf keine großen Datenmengen sowie Infrastruktur mit großer Rechenleistung, um das System auf neue Daten anzupassen.

Zusammenfassend kann von einem Erfolg bezüglich der gestellten Aufgabe gesprochen werden. Das erarbeitete System ist zu dem aktuellen Zeitpunkt nicht in geeignet für den realen Einsatz, beispielsweise in einer App. Die Fehlerquote des Systems liegt mit etwa $40\,\%$ deutlich zu hoch, jedoch ist eine solide Grundlage zur Erarbeitung eines leistungsstarken Systems gegeben.

\section{Ausblick}
\label{sec:ausblick}

Diese Arbeit stellt ein System vor, mit welchem ein automatisches Dart-Scoring durch die Verwendung herkömmlicher \ac{cv} sowie ein aktuelles neuronales Netz, welches durch automatisch und synthetisch erstellte Daten trainiert wurde. Das Potenzial dieser Systeme ist nicht vollkommen ausgeschöpft und kann auf unterschiedliche Arten weitergeführt werden. Dieser Abschnitt gibt einen Überblick über mögliche Ausgestaltungen und Verbesserungen dieser Systeme. Es wird begonnen mit dem Ausblick der Datenerstellung in \autoref{sec:ausblick_data}, gefolgt von dem Ausblick der Pipeline zur Normalisierung in \autoref{sec:ausblick_cv} und abschließend wird in \autoref{sec:ausblick_ki} ein Überblick über weitere mögliche Arbeit an der Dartpfeilerkennung gegeben.

% -------------------------------------------------------------------------------------------------

\vspace*{-0.15cm}
\subsection{Ausblick der Datenerstellung}
\label{sec:ausblick_data}

Die Datenerstellung ist ein komplexer Prozess, welcher ein Zusammenspiel vieler unterschiedlicher Komponenten ist. Die Möglichkeiten zur Erweiterung dieses Systems sind daher sehr vielseitig. Anstatt der Optimierung von Implementierungsdetails wird sich in diesem Ausblick auf die methodischen Aspekte zur Erweiterung der Datenerstellung fokussiert.

\vspace*{-0.15cm}
\paragraph{Integration von \ac{pbr}}

Ein wesentlicher Bestandteil der Datenerstellung ist die prozedurale Erstellung von Materialien. Diese ist aktuell nicht einheitlich über die Objekte eingebunden und bedarf der Integration von \ac{pbr}. In diesem werden unterschiedliche Texturen zur Steuerung verschiedener Parameter wie Glanz und Oberflächenbeschaffenheit strikt getrennt voneinander definiert, um realistische Texturen zu erstellen.

% - Implementierung von PBR in Datenerstellung

\vspace*{-0.15cm}
\paragraph{Anzahl der Objekte in der Szene}

Die Szene umfasst aktuell neben den Dartpfeilen fünf Objekte, die dynamisch ein- und ausgeblendet werden können, lediglich zwei dieser Objekte -- Lichtring und Dartschrank -- sind in den Aufnahmen zu sehen; die restlichen Objekte sind Lichtquellen. Durch Hinzufügen weiterer Objekte in die Szene, beispielsweise Wände und Dekoration, können mehr Bedingungen und Störfaktoren in die Daten aufgenommen werden, die in realen Aufnahmen zu erwarten sind. Die Aufnahmen können durch das gezielte Hinzufügen weiterer Objekte realistischer gestaltet werden.

% - Verbesserung der Datengenerierung, um realistischer zu werden und mehr Umgebungsbedingungen zu simulieren

\vspace*{-0.15cm}
\paragraph{Aussehen der Dartscheibe}

Das Aussehen der Dartscheibe ist konform der Regelwerke der \ac{wdf} und der \ac{pdc} gestaltet \cite{wdf-rules,pdc_rules}. Diese gibt Form und Farben der Dartscheibe mit etwaigen Toleranzen weitestgehend vor. Für die aktuelle Datenerstellung werden lediglich diese und ähnliche Dartscheiben als Vorlagen in Betracht gezogen, wodurch Möglichkeiten weiterer Dartscheiben ausgeschlossen sind. Durch Hinzufügen von Parametern zur Steuerung der Art der Dartscheibe können ebenfalls Dartscheiben mit anderen Farbgebungen und Geometrien in die Datenerstellung eingebunden werden, beispielsweise Dartscheiben mit blauen und roten Feldern statt schwarzer und weißer Grundfelder.

Ebenfalls ist eine Adaption der unmittelbaren Ausgestaltung der Dartscheibe durch Zahlenring und Texte denkbar. In der hier vorgestellten Dartscheibengenerierung werden Beschriftungen und Logos um die Dartfelder herum durch Textbausteine abstrahiert. Bei der Identifizierung von Dartpfeilen auf realen Dartscheiben hat sich jedoch gezeigt, dass diese Abstraktion von Logos durch Texte nicht ausreichend ist, um diese nicht als Dartpfeile zu klassifizieren. Eine weitergehende Ausgestaltung dieser Bereiche der Dartscheibe ist daher notwendig.

% - Datenerstellung auf weitere Farben und Formen der Dartscheibe erweitern
% - - z.B. blau-rote Felder
% - - ist aber meist nicht in Steeldarts gegeben, sondern eher in elektronischen Dartscheiben
% - - und bei elektronischen Dartscheiben ist dieses System ohnehin überflüssig

% -------------------------------------------------------------------------------------------------

\vspace*{-0.15cm}
\subsection{Ausblick der Normalisierung}
\label{sec:ausblick_cv}

Der in dieser Thesis vorgestellte Algorithmus zur Erkennung und Normalisierung von Dartscheiben in beliebigen Bildern basiert ausschließlich auf Techniken herkömmlicher \ac{cv}. Dadurch ist die Adaption des Algorithmus weitreichend möglich, sodass spezifische Aspekte gezielt angegangen und verbessert werden können. Im Verlauf der Entwicklung sowie der Auswertung sind verbesserungswürdige Bereiche des Algorithmus identifiziert worden, die in diesem Abschnitt betrachtet werden. Sie bieten eine Grundlage zur gezielten Erweiterung und Verbesserung der bestehenden Methodiken.

\paragraph{Verbesserung der Erkennung und Klassifizierung von Orientierungspunkten}

Die Erkennung von Orientierungspunkten geschieht in einem zweischrittigen Verfahren. Im ersten Schritt werden mögliche Kandidaten von Orientierungspunkten lokalisiert, welche in einem zweiten Schritt klassifiziert werden. Diese Klassifizierungen basieren auf der Einordnung von Farbwerten, die durch die Analyse aller identifizierter Kandidaten abgeleitet werden. Die Klassifizierung ist dadurch tendenziell fehleranfällig, sodass eine pessimistische Klassifizierung zur Minimierung von Fehlklassifizierungen stattfindet. Unter den Kandidaten der Orientierungspunkte befindet sich voraussichtlich eine ausreichende Anzahl korrekter Orientierungspunkte, welche großzügiges Aussortieren von Kandidaten ermöglichen.

Trotz erfolgreicher Normalisierungen geschieht an diesem Punkt ein Verlust relevanter Daten, die potenziell für robustere Ergebnisse des Algorithmus sorgen können. Durch eine Überarbeitung der Methodik zur Klassifizierung von Orientierungspunkten kann dieser Bereich des Algorithmus weiter ausgebaut werden. Zur Bewerkstelligung dieser Optimierung ist der Einsatz von \acp{cnn} denkbar. Die \acp{cnn} könnten die Aufgabe der Einordnung von Surroundings übernehmen, indem sie auf zuvor korrekt erkannten Surroundings trainiert werden. Die Aufgabe dieser \acp{cnn} befindet sich in einem Bereich der Komplexität, in welchem ein Netzwerk mit einer geringen Parameterzahl eingesetzt werden kann.

% - Verbesserung der Erkennung / Klassifizierung von Orientierungspunkten
% - - aktuell: pessimistische Klassifizierung, um Outlier zu minimieren, siehe \autoref{img:orientierung} (3)
% - - Problem: Einige Punkte werden nicht korrekt erkannt
% - - Lösung: Durch Überarbeitung der Klassifizierung genauer klassifizieren (ggf. mit kleinem CNN, das Surroundings / Farben klassifiziert)

Darüber hinaus besteht die Möglichkeit, die Funktionsweise Erkennung der Dartscheibe mit einer Ellipsenerkennung zu erweitern, mit der die Geometrie der Dartscheibe abgeleitet werden kann \cite{ellipse_detection_algorithm}. Die Verwendung von Ellipsenerkennung wurde bei der Implementierung dieser Arbeit zu Teilen implementiert, jedoch konnten keine zufriedenstellenden Ergebnisse erzielt werden. Trotz dessen liegt in der Verwendung dieser Technik viel Potenzial, welches in die Normalisierung von Dartscheiben einfließen kann.

% - Ellipsen-Erkennung in CV einbauen (\cite{ellipse_detection_algorithm})
% - - Möglichkeit zur Identifizierung von Ellipsen im Bild

\paragraph{Kompilierung des Algorithmus}

Zur Implementierung des Algorithmus, wie er in dieser Thesis vorgestellt und ausgewertet ist, wurde Python als Programmiersprache verwendet. Trotz der Verwendung der Bibliotheken NumPy und OpenCV, welche zu Teilen kompilierte Funktionen einsetzen, geschieht ein großer Anteil der Berechnungen durch interpretierten Quelltext. Interpretation von Quelltext ist weitaus ineffizienter als die Ausführung kompilierten Quelltexts.

Die Kompilierung des Algorithmus kann auf unterschiedliche Weisen umgesetzt werden. Bibliotheken wie Cython und Numba ermöglichen die Ausführung von Python-basiertem Quelltext bzw. Kompilierung von Teilen des Python-Quelltexts zur Optimierung von Ausführungszeiten \cite{cython,numba}. Durch diese Bibliotheken kann potenziell eine Optimierung der Laufzeit mit geringer Abänderung des vorhandenen Quelltexts erzielt werden. Ebenso ist eine Implementierung in einer kompilierten Programmiersprache wie C++ oder Rust denkbar, durch welche erhebliche Verbesserungen in der Ausführungszeit zu erwarten sind.

% - Kompilierung der CV-Pipeline
% - - entweder Cython \cite{cython} / Numba \cite{numba} oder Implementierung in kompilierter Sprache (C/C++/Rust)

\paragraph{Auswertungen auf unterschiedlichen Plattformen}

Die Auswertung hinsichtlich der Ausführungszeit dieses Systems beläuft sich auf ein bestimmtes System, auf welchem die Ausführungszeiten ermittelt wurden. Das verwendete System ist jedoch weitaus leistungsstärker als die für die vorgesehene Verwendung des Systems ausgelegten Systeme, namentlich Mobiltelefone. Eine Auswertung auf verschiedenen mobilen Geräten ermöglicht als fortführende Arbeit in der Hinsicht die Optimierung der Parameter des Algorithmus, um eine optimale Abstimmung von Geschwindigkeit und Genauigkeit zu ermitteln. Die in \autoref{sec:daten:ergebnisse} erzielten Resultate hinsichtlich der Ausführungszeiten von DeepDarts und dem in dieser Thesis erarbeiteten Algorithmus sind in diesem Aspekt lediglich relativ zueinander zu betrachten; eine Ableitung realer Ausführungszeiten ist nur bedingt möglich.

% - Vergleich mit DD auf unterschiedlichen Plattformen
% - - aktuell: eine Plattform
% - - möglich: Ausführung beider Systeme auf einem Mobiltelefon
% - - Hintergrund: System ist darauf ausgelegt, mobil genutzt zu werden

% -------------------------------------------------------------------------------------------------

\subsection{Ausblick der Dartpfeilerkennung}
\label{sec:ausblick_ki}

Die Lokalisierung der Dartpfeile beruht auf der Verwendung eines neuronalen Netzes, welches auf synthetischen Daten trainiert wurde. Hinsichtlich dieses Trainings und der Rahmenbedingungen werden in diesem Unterabschnitt mögliche zukünftige Aufgaben erläutert und Verbesserungsvorschläge hinsichtlich bestimmter Schwachstellen aufgelistet. Durch weitere Ausarbeitung dieser Themenbereiche wird eine Verbesserung des Systems als wahrscheinlich erachtet.

\paragraph{Netzwerktraining}

Das Training des neuronalen Netzes basiert auf einem \ac{ood}-Ansatz, in welchem synthetische Daten generiert wurden, um ein System zu trainieren, welches auf reale Daten angewandt wird. Zwischen synthetischen und realen Daten liegen jedoch nicht von der Hand zu weisende Diskrepanzen, die einem optimalen Training im Wege stehen. Durch die Aufnahme weiterer realer und zugleich variabler Daten kann die Datenlage realer Bilder für das Training vervielfacht werden, wodurch eine bessere Inferenz auf realen Daten zu erwarten ist.

Zusätzlich sind einige Rahmenbedingungen des Trainings verbesserungswürdig. Das Training verlief mit einer Batch Size von 32, welche für ein stabileres Training erhöht werden kann; aus technischen Gründen war dies im Umfang dieser Arbeit nicht möglich. Darüber hinaus wurde die Learning-Rate manuell angepasst und unterlag keinem Algorithmus oder festgelegtem Muster. Für weiteres Training ist in Ausschau zu setzen, dass ein Schedule zur Anpassung der Learning-Rate verwendet wird, der dem Verhalten der manuellen Learning-Rate folgt.

% - Training:
% - - neues Trainieren des Systems auf mehr realen Daten
% - - - Arbeit spezifisch hinsichtlich Aufnahme und Annotation neuer Daten
% - - Batch Size erhöhen
% - - Learning-Rate gescheit implementieren

\paragraph{Netzwerkarchitektur}

Ein wesentlicher Kritikpunkt des Netzwerktrainings dieser Thesis ist die verwendete Netzwerkarchitektur. Durch die eigene Implementierung und die vorgenommenen Adaptionen der YOLOv8-Architektur ist die Verwendung vortrainierter Gewichte nicht möglich. Es kann daher kein Transfer Learning durchgeführt werden und nicht auf bereits initialisierte und auf reale Bedingungen angepasste Gewichte zurückgegriffen werden. Durch diese Tatsache ist ein signifikanter Verlust der Performance zu erwarten. Durch ein vorheriges Trainieren des Backbones durch Unsupervised Learning auf realen Bildern können die Gewichte des Backbones an reale Gegebenheiten angepasst werden.

Zusätzlich können weitere Recherchen zur Verwendung unterschiedlicher Netzwerkarchitekturen vorgenommen werden, durch die Einblicke in die unterschiedlichen Performances erlangt werden können. Auf diese Weise kann ermittelt werden, welche Techniken in Netzwerkarchitekturen zielführend für diese Aufgabe sind und es kann ein Netzwerk konzipiert werden, welches optimal für die Erkennung von Dartpfeilspitzen in Bildern geeignet ist. Weiterhin kann zu der Untersuchung weiterer Netzwerkarchitekturen der Einfluss eigener Netzwerkbestandteile, wie dem Transition-Block, der in dieser Arbeit eingeführt wurde, genauer betrachtet und ausgewertet werden.

% - Netzwerkarchitektur:
% - - Transfer-Learning durch eigene Implementierung nicht gegeben
% - - - ggf. Unsupervised Learning durch Umwandlung des Backbones in Autoencoder
% - - weitere Netzwerkarchitekturen austesten
% - - Vorteile weiterer Adaptionen genauer untersuchen

\paragraph{Quantisierung des Netzwerks}

Das Quantisieren von Netzwerken ist eine Technik, bei der die Netzwerkgewichte von 32-bit Float-Werten zu Integern verschiedener Längen überführt werden. Dadurch kann eine effizientere und schnellere Inferenz des Netzwerks erzielt werden während lediglich geringfügige Einbußen in der Qualität der Ausgaben in Kauf genommen werden. Die Quantisierung von Netzwerken dient der Ausführung neuronaler Netze in Umgebungen, deren Ressourcen begrenzt sind. Insbesondere mit Blick auf ein potenzielles mobiles Deployment dieses Systems ist die Quantisierung des verwendeten neuronalen Netzes ein Thema, in das die Investition weiterer Ressourcen für sinnvoll erachtet wird.

% - Quantisierung der Netzwerke
% - - Verbesserung der Geschwindigkeit
% - - Verringerung der Ressourcennutzung (insbesondere hinsichtlich mobilem Deployment)

\paragraph{Erweiterte Herangehensweise für Vorhersagen}

Zuletzt kann die Art der Vorhersage des Netzwerks überarbeitet werden, um eine Anpassung an die spezifischen Gegebenheiten der verwendeten Dartscheibe vorzunehmen. Durch die Aufnahme eines Kalibrierungsbildes, in dem die Dartscheibe ohne Pfeile zu sehen ist, kann das Aussehen der Dartscheibe gespeichert werden. Bei der Vorhersage von Aufnahmen kann diese Referenzaufnahme der leeren Dartscheibe verwendet werden, um fehlerhafte Vorhersagen auf Grundlage des Erscheinungsbildes der Dartscheibe zu unterbinden. Beispielsweise sind besondere Abnutzungen oder Logos bzw. Schriftzüge um die Dartscheibe herum als häufige Fehlerquellen auf realen Daten identifiziert worden. Sofern das Aussehen der Dartscheibe ohne Pfeile bekannt ist, besteht die Möglichkeit, dieser Art fehlerhafter Vorhersagen gezielt entgegenzuwirken.

% - KI-Prediction auf Grundlage einer leeren Dartscheibe
% - - Kalibrierungs-Bild schließen und als Referenz nutzen
% - - Wenn bekannt ist, dass keine Dartpfeile auf Kalibrierungs-Bild vorhanden sind, ist die Wahrscheinlichkeit von Fehlklassifikationen des Hintergrundes geringer
