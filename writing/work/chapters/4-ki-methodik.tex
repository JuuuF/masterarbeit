% !TEX root = ../main.tex

\section{Methodik}
\label{sec:ki:methodik}

Methodik hier.

% -------------------------------------------------------------------------------------------------
\subsection{Wahl des Netzwerks}
\label{sec:netzwerk_wahl}

\todo{Netzwerkwahl beschreiben}

\subsubsection{Warum YOLOv8?}
\label{sec:warum_yolov8}

\todo{Warum denn nun?}

\subsubsection{Adaption der Netzwerkarchitektur}
\label{sec:yolo_adaption}

\paragraph{Multi-Scale-Output}

\paragraph{Bounding Boxes}

\paragraph{Dreiteilung des Outputs}

\begin{itemize}
    \item Existenz
    \item Position
    \item Klasse
    \item alles 3x pro Zelle
\end{itemize}

\todo{Adaption erklären}

\subsection{Datenaugmentierung}
\label{sec:daten_augmentierung}

\autoref{img:augmentierungs_pipeline}

\begin{figure}
    \centering
    \includegraphics[width=\textwidth]{imgs/ai/augmentation_pipeline.pdf}
    \caption{Augmentierungs-Pipeline}
    \label{img:augmentierungs_pipeline}
\end{figure}
