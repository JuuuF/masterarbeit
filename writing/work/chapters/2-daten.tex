% !TeX root = ../main.tex

\chapter{Simulation von Dartscheiben zur Generierung von Trainingsdaten}
\label{cha:daten}

Der erste Themenbereich dieser Masterarbeit ist die Erstellung von Daten. Diese Daten bestehen aus Bildern und Annotationen von Objekten in diesen Bildern. Bei den Bildern handelt es sich um Aufnahmen von Dartscheiben, in denen Dartpfeile stecken, deren Positionen und die mit diesen erzielten Punktzahlen die Annotationen ausmachen. In diesem Kapitel wird ein System vorgestellt, mit welchem die automatisierte Erstellung dieser Daten ermöglicht wird.

Das Aussehen dieser Bilder zielt darauf ab, Aufnahmen von Mobiltelefonen zu imitieren, wie sie im Einsatz in einer App zum automatischen Dart-Scoring anhand einzelner Bilder aufgenommen werden. Insbesondere bezieht diese Zielsetzung die charakteristischen Optiken von Kameras in Mobiltelefonen mit ein, beispielsweise erhöhte Nachverarbeitung durch Kontrasterhöhung oder Rauschen durch hohe ISO-Werte. Die Verteilung der Dartpfeile auf der Dartscheibe folgt typischen Mustern, die in Amateurspielen zu erwarten sind.

Umgesetzt wird die Datenerstellung durch das Modellieren einer 3D-Szene bestehend aus Dartscheibe, Dartpfeilen, verschiedenen Beleuchtungen und einer Kamera. Diese Objekte sind auf eine Weise konzipiert, dass die Erscheinungsbilder dieser Objekte prozedural gestaltet sind und eine beliebige Anzahl verschiedener Objekte von diesen abgeleitet werden kann. Dadurch wird eine große Variabilität der erstellten Daten ermöglicht.

Der Aufbau dieses Kapitels ist unterteilt in vier Unterkapitel. In dem ersten Unterkapitel werden Grundlagen hinsichtlich der Datenerstellung und die in ihr verwendeten Techniken vorgestellt. Diese bieten die Grundlage zum Verständnis der Methodik, welche im darauf folgenden Unterkapitel erläutert wird. Die Methodik umfasst die Konzepte, die zur Erstellung der Daten verwendet werden und auf denen sich die Implementierung stützt. Auf bestimmte Bereiche der Implementierung wird in dem dritten Unterkapitel eingegangen. Abschließend werden die Ergebnisse der synthetischen Datenerstellung im letzten Unterkapitel betrachtet und ausgewertet.

% !TEX root = ../main.tex

\section{Grundlagen}
\label{sec:daten:grundlagen}

Grundlagen hier.

% -------------------------------------------------------------------------------------------------

\subsection{Prozedurale Datenerstellung}  % DATENERSTELLUNG =======================================
\label{sec:prozedurale_datenerstellung}

Was ist es? Wozu ist es gut?

\todo{}

% -------------------------------------------------------------------------------------------------

\subsection{3D-Rendering}
\label{sec:3d_rendering}

Kurz Ray-Tracing anreißen.

\todo{}

% -------------------------------------------------------------------------------------------------

\subsection{Kameraparameter}
\label{sec:kameras}

Bei der Datenerstellung mittels 3D-Software und Ray-Tracing ist die Präsenz einer Kamera unabdingbar. Ursprüngliche Kameramodelle begonnen mit einer Pinhole-Kamera, die als Projektion eines 3D-Raumes in einen 2D-Raum vornimmt \cite{pinhole_camera}. Auf diesem Modell aufbauend wurden weitere Kameraparameter modelliert, bis die Simulation echter Kameras ermöglicht wurde. In aktuellen 3D-Softwares zum Rendern von Szenen sind eine Vielzahl an Kameraparametern implementiert und modifizierbar, sodass fotorealistische Aufnahmen simuliert werden können. Die Unterschiede zwischen der Verwendung einer simulierten und einer Echten Kamera sind daher für das ungeschulte Auge verschwindend gering und für das Erstellen von Trainingsdaten für ein neuronales Netz ideal geeignet. Die wichtigsten Parameter einer Kameraaufnahme werden in diesem Unterabschnitt grob erläutert, um ein oberflächliches Verständnis der Arbeitsweise einer Kamera zu erlangen.

\paragraph{Brennweite}

Die Brennweite einer Kamera -- bzw. eines Objektives -- bestimmt die Lichtbrechung bei der Aufnahme eines Bildes. Diese Lichtbrechung resultiert in einem unterschiedlich großen Bereich, der von der Kamera eingefangen wird. Optisch ist die Brennweite für den Zoom des Bildes zuständig. Eine Brennweite von 50mm ist eine typische Brennweite, die dem menschlichen Blickwinkel nahe kommt. Geringere Brennweiten sorgen für ein größeres Sichtfeld während größere Brennweiten mit einem größeren Zoom einhergehen. \cite{focal_lentgh}.

\paragraph{Öffnungswinkel}

\todo{}

\paragraph{Belichtungsdauer und Bewegungsunschärfe}

\todo{}

\paragraph{ISO und Rauschen}

Bei der Aufnahme von Bildern wird zwischen zwei Arten von Rauschen unterschieden: temporales und fixiertes Rauschen. Wohingegen fixiertes Rauschen zwischen Aufnahmen gleich bleibt, ändert sich temporales Rauschen zwischen Aufnahmen nichtdeterministisch. Die Ursprünge dieses Rauschens sind weitreichend von Imperfektionen in Objektiven zu physikalischen Gegebenheiten durch die Diskretisierung einer Szene auf dem Kamerasensor. Ein wesentlicher Grund für die Existenz von Rauschen ist die ISO. Der ISO-Wert gibt die Empfindlichkeit des Kamerasensors und damit die Lichtmenge an, die bei der Aufnahme mit einer Kamera auf den Sensor gelangt. Je höher der ISO-Wert ist, desto sensitiver ist der Kamerasensor für eintreffende Lichtstrahlen, wodurch hohe ISO-Werte übermäßiges Rauschen mit sich ziehen können \cite{camera_everything}. Dieses Rauschen wird insbesondere bei automatischer Einstellung der Kameraparameter in dunklen Umgebungen deutlich, wodurch dunkle Aufnahmen mit starkem Rauschen einhergehen. Diese Art von Bildern ist insbesondere bei Aufnahmen in Mobiltelefonen vermehrt zu finden, weshalb es im Kontext dieser Thesis besonders relevant ist.

\paragraph{Farbsäume}

Kameralinsen bestehen aus Glas mit einem Refraktionsindex, der von der Wellenlänge des eintreffenden Lichtes abhängt. Daraus resultiert eine unterschiedliche Lichtbrechung der jeweiligen Lichtwellen und es entstehen Farbsäume in dem aufgenommenen Bild. \cite{lens_distortion, camera_everything}. Bei Farbsäumen handelt es sich um die prismatische Auftrennung der Farbinformationen, die besonders an Kanten von Objekten und am Rand des aufgenommenen Bildes verstärkt auftreten. Dieser Effekt kann in einer Software auf zwei unterschiedliche Arten umgesetzt werden: Simulation oder Komposition. Bei der Simulation wird das eingefangene Licht durch rekonstruierte Linsen gebrochen und die Farbsäume werden direkt durch die Kamera aufgenommen. Dieser Schritt ist rechnerisch aufwendig und wird daher im Vergleich zur Komposition selten eingesetzt. Bei der Komposition wird die Aufnahme in der Nachverarbeitung derart abgeändert, dass der Effekt der Farbsäume nachgestellt wird. Da der Effekt in herkömmlichen Aufnahmen für das ungeschulte Auge schwer erkennbar ist, ist der Unterschied dieser Methoden für diese Anwendung verschwindend gering.

% -------------------------------------------------------------------------------------------------

\subsection{Binärbilder und Masken}
\label{sec:masken}

Was bringt die Erstellung von Masken? -> Informationsgewinn / -erhaltung / -extraktion -> mehr Informationen als in Render-Bild

\todo{}

% -------------------------------------------------------------------------------------------------

\subsection{Dartscheiben-Geometrie}  % DARTS ======================================================
\label{sec:dartscheiben_geometrie}

Die Geometrie und der Aufbaue einer Dartscheibe ist für die Erstellung der Daten zentral. Eine schematische Darstellung einer Dartscheibe ist in \autoref{img:dart_board} gegeben.

\paragraph{Die Dartscheibe}

Die Dartscheibe besteht grundlegend aus einer Scheibe mit 451mm Durchmesser. Sie besteht aus 20 einheitlich großen, radial angeordneten Feldern mit Zahlenwerden von 1 bis 20. Jedes Feld besitzt einen Double- und einen Triple-Ring mit einem Durchmesser von 8-10mm. Der Triple-Ring ist etwa 10cm vom Mittelpunkt entfernt; der Double-Ring etwa 16cm. Insgesamt ergibt sich ein Durchmesser von 34cm punkteerzielender Felder, der Bereich jenseits der 17cm Abstand des Mittelpunktes gibt keine Punkte \cite{wdf-rules}.

In der Mitte der Dartscheibe befinden sich die Felder Bull und Double Bull (oder Bull's Eye). Sie haben Durchmesser jeweils ca. 32mm und 12,7mm. Das Bull gibt 25 Punkte, das Double Bull 50.

Die Felder mit einfachen Punktzahlen sind abwechselnd schwarz und weiß gefärbt, die Mehrfach-Felder der schwarzen Felder sind rot und die der weißen Felder grün gefärbt. Das Bull ist grün und das Double Bull rot.

\paragraph{Rund um die Dartscheibe}

Die Punktzahlen der Dartfelder werden durch aus Drähten gefertigten Zahlen angezeigt. Diese befinden sich an einem Ring am Äußeren Ende der Dartscheibe und sind radial um die Dartscheibe angeordnet. Dieser Zahlenring ist nicht fest an der Dartscheibe montiert, sodass er nach Belieben rotiert werden kann, um eine ungleichmäßige Abnutzung der Dartscheibe auszugleichen.

\paragraph{Material}

Die Felder der Steeldarts-Dartscheibe werden aus Fasern hergestellt, typisch sind dabei Sisal-Fasern. Diese besitzen die Eigenschaft, dass ein eintreffender Dartpfeil nicht in sie eindringt, sondern die lediglich verdrängt, wodurch die bleibenden Schäden der Einstichlöcher gering gehalten werden.

\paragraph{Double und Triple}

Mit Double und Triple (auch Treble) werden die Ringe an Feldern bezeichnet, die Vielfache der Feldgrundzahl als Score erteilen. Der Triple-Ring befindet sich zwischen den Einzelfeldern der Dartscheibe und gewährt das Dreifache der Grundpunktzahl des Feldes. Der Double-Ring befindet sich auf der Außenseite der Dartscheibe und gewährt seinem Namen entsprechend das Doppelte der Feldzahl. Das Feld mit der größten Punktzahl ist die Triple-20 mit $3 \times 20 = 60$ Punkten.

\begin{figure}
    \centering
    \includegraphics[width=\textwidth]{imgs/rendering/dart_board.pdf}
    \caption{Dartscheibe}
    \label{img:dart_board}
\end{figure}


% -------------------------------------------------------------------------------------------------

\subsection{Dart-Terminologie}
\label{sec:dart_terminologie}

Im Darts gibt es eine Vielzahl an Begrifflichkeiten, von denen einige auch in dieser Thesis genutzt werden. Die grundlegenden Begriffe und ihre Bedeutungen werden in diesem Unterkapitel erläutert.

\paragraph{Tip, Barrel, Shaft, Flight}

Ein Dartpfeil besteht aus unterschiedlichen Bestandteilen. Die vier wesentlichen Bestandteile sind Tip, Barrel, Shaft und Flight\footnote{Abweichungen dieser Bezeichnungen sind möglich, sodass der Shaft auch häufig als Stem bezeichnet wird. Zur Vereinheitlichung wurden die genannten Begriffe verwendet.}, aufgezählt von der Vorderseite nach hinten \cite{wdf-rules,pdc_rules}. Die Tip ist die Spitze des Dartpfeils, die in die Dartscheibe eintrifft. Die Barrel ist der Teil des Dartpfeils, an dem er gegriffen wird und liegt direkt hinter der Tip. Auf die Barrel folgt der Shaft, der die Brücke zum Flight, dem Flügelende des Dartpfeils, schließt. Der Flight besteht aus vier Einzelflügeln, die in Abständen von $90\degree$ zueinander stehen und orthogonal zum Shaft verlaufen.

\paragraph{Spinne}

Als Spinne wird der Metallrahmen der Dartscheibe bezeichnet, der die Feldsegmente voneinander trennt. Die Ausprägung der Spinne ist unterschiedlich, jedoch ist ein tendenzieller Trend zu erkennen, dass neue Dartscheiben dünnere und unauffälligere Spinnen besitzen als alte Dartscheiben. Je unauffälliger die Spinne ist, desto geringer ist die Wahrscheinlichkeit eines Bounce-Outs, bei dem der Dartpfeil auf die Spinne trifft und nicht auf der Dartscheibe landet.

% -------------------------------------------------------------------------------------------------

\subsection{Steeldarts}
\label{sec:steeldarts}

\todo{Was Steeldarts?}

% -------------------------------------------------------------------------------------------------

\subsection{Material und Texturen}  % TEXTURIERUNG ================================================

Licht-Eigenschaften / Normal Maps / Shaders

\todo{}

% -------------------------------------------------------------------------------------------------

\subsection{Noise-Texturen}
\label{sec:noise}

Essenziell für prozedurale Datenerstellung ist die Verwendung von Noise-Texturen. Diese ermöglichen es, Kontrolle über die Variation der Daten zu behalten während der Zufall der Datenerstellung erhalten bleibt. Es gibt unterschiedliche Arten von Noise-Texturen, die für die Generierung zufälliger Texturen verwendet werden \cite{noise_generation}. In dieser Arbeit wurden vorgehend White Noise und Perlin Noise in der Datengenerierung genutzt, daher werden diese in den folgenden Unterabschnitten genauer erläutert.

\subsubsection{White Noise}

White Noise -- zu deutsch: weißes Rauschen -- ist eine Zufallsverteilung von Zahlenwerten, die nicht vorhersehbar ist und doch einem Muster folgt. Sie ist dadurch charakterisiert, dass jeder Ausgangswert mit der selben Wahrscheinlichkeit versehen ist \cite{white_noise}. Im 1-dimensionalen Beispiel von Audio entspricht weißes Rauschen der Charakteristik, dass jede Frequenz in einem Signal gleichermaßen vertreten ist. Hinsichtlich einer diskreten 2-dimensionalen Textur ist die Wahl jedes Intensitätswerts eines Pixels unabhängig voneinander gleichverteilt über das Intervall $[0, 1]$.

\subsubsection{Perlin Noise}

Perlin Noise bildet die Basis für gezielte Generierung von Strukturen prozeduraler Natur. 1982 von Ken Perlin für den Film Tron entwickelt, zielte Perlin Noise in Vergleich zu White Noise auf die Generierung von zusammenhängendem Rauschen ab, das zur Generierung natürlicher Strukturen verwendet werden kann \cite{perlin_noise_original,perlin_noise_extension}.

Zur Berechnung einer 2D-Textur mit Perlin Noise wird ein Bild in Regionen unterteilt. Jeder Ecke dieser Regionen wird ein Vektor zugeordnet, der in eine zufällige Richtung zeigt, die uniform aus dem Intervall $[0, 2\pi]$ gewählt wird, wie in \autoref{img:perlin_noise_generation} dargestellt. Für jeden Pixel $(x, y)$ im Bild wird die korrespondierende Region bestimmt und die Punktprodukte zwischen den Vektoren der Ecken der Region und den Verbindungsvektoren zu dem Punkt werden gebildet. Die Intensität des Pixels wird durch bilineare Interpolation dieser Werte auf Grundlage seiner Position in der Region bestimmt. Auf diese Weise entsteht ein zusammenhängendes Muster, das als Perlin Noise bezeichnet wird \cite{perlin_noise_original}.

Der Detailgrad von Perlin Noise wird durch Überlagerung mehrerer Schichten erzielt, die sich in der Größe ihrer Regionen unterscheiden; je kleiner die Regionen, desto größer der Detailgrad. Durch Variation der Gitterauflösung und Überlagerung mehrerer Schichten kann ein beliebiger Detailgrad erzielt werden, der in einem natürlichen Rauschen resultiert.

\begin{figure}
    \centering
    \includegraphics[width=0.8\textwidth]{imgs/rendering/grundlagen/perlin_noise.png}
    \caption{Generierung von Perlin Noise \cite{perlin_noise_img}. Zufällige Rotationsvektoren werden uniform über ein Bild verteilt, durch sie wird die Texturierung bestimmt.}
    \label{img:perlin_noise_generation}
\end{figure}

\subsection{Seeding}

Die algorithmische Generierung von Zufallszahlen ist nicht ohne Verzerrung möglich. Daher werden von Computern generierte Zufallszahlen auch als pseudo-zufällig bezeichnet. Die Generierung von Zufallszahlen beruht auf deterministischen Algorithmen, die üblicherweise einen Seed verwenden, um Zahlen zu generieren. Durch die Verwendung des selben Seeds ist das Ziehen von Zufallszahlen deterministisch wiederholbar, wodurch die Generierung von Daten reproduzierbar und nachvollziehbar ist.

\subsection{Thresholding und Maskierung}

Warum braucht man das? Wozu macht man das? -> Thresholding zur Kombination von Texturen miteinander / Layering mehrerer Texturen

In der Datengenerierung dieser Thesis wird Thresholding und Maskierung verwendet, um Texturen gezielt miteinander zu kombinieren. Im Bezug auf die Verwendung in der Datengenerierung wird  für Thresholding ein Schwellwert verwendet, der die Farbwerte einer Textur bestimmen. Ist die Pixelintensität oberhalb des Schwellwerts, wird dieser entweder verwendet, andernfalls nicht; das selbe ist umgekehrt möglich. Typische Einsätze von Thresholding in dieser Arbeit ist die Eingrenzung von Noise-Texturen auf die größten Pixelintensitäten oberhalb eines Schwellwerts.

Wird dieses globale Thresholding aller Pixel lokalisiert, sodass jedem Pixel ein dedizierter Threshold zugeordnet wird, spricht man von Maskierung. Eine Maske bestimmt die Grenzwerte, die zur Einblendung der modulierten Textur notwendig sind. Neben binärer Ein- und Ausblendung von Pixeln ermöglichen Masken durch Verwendung von Pixelwerten im Intervall $[0, 1]$ anteiliges Anzeigen der Quelltextur. Diese Arbeitsweise wird in dieser Thesis zur Erstellung einer realistischen Dartscheibe durch Überlagerung unterschiedlicher maskierter Texturen verwendet.

\todo{Dieser Text ist nicht gut, gerne überarbeiten. Mehr Zahlen und Definitionen. Und eine Quelle wäre wohl nicht verkehrt.}

% -------------------------------------------------------------------------------------------------

\subsection{Prozedurale Texturen}
\label{sec:was_prozedurale_texturen}

Prozedurale Texturen sind der Kern der Datenerstellung in dieser Thesis. Sie dienen als Blaupause zur Generierung zufälliger Texturen, indem sie ein generelles Erscheinungsbild einer Textur definieren. Dieses Erscheinungsbild ist aufgebaut aus Bestandteilen, die Seeding integrieren, um ihr Aussehen zu bestimmen. Die Änderung des verwendeten Seeds führt folglich zu einer Änderung der Textur. Der Grad der Änderung ist vorhersehbar und kann durch Grenzwerte begrenzt sein, die konkrete Änderung unterliegt jedoch weiterhin der Pseudo-Zufälligkeit. Unter Einbindung dieser Technik kann eine beliebige Anzahl unterschiedlicher Texturen generiert werden.

% !TEX root = ../main.tex

\section{Methodik}
\label{sec:daten:methodik}

In diesem Abschnitt wird die Methodik der automatisierten Datenerstellung thematisiert. Es werden die unterliegenden Konzepte erläutert und die Hintergründe dieser werden beschrieben. Bevor in die einzelnen Bereiche eingestiegen wird, ist es zunächst notwendig, einen Überblick über das Zusammenspiel der jeweiligen Komponenten zu gewinnen. In \autoref{img:rendering_pipeline} ist der Ablauf der Datenerstellung schematisch dargestellt.

Die Datenerstellung fußt auf einer 3D-Szene, welche in \autoref{sec:3d_szene} näher beschrieben wird. In der Szene befinden sich Objekte, die für die Gestaltung der Trainingsdaten verantwortlich sind. Hintergründe zu ihrem Aufbau und der konkreten Auswahl der Objekte werden in \autoref{sec:material_licht} gegeben. Diese Objekte der Szene werden durch ein Skript hinsichtlich ihrer Existenz, ihres Aussehens, ihren Eigenschaften und ihrer Positionierung algorithmisch randomisiert. Diese Randomisierung geschieht durch ein externes Skript, welches in \autoref{sec:scripting} näher betrachtet wird. Nachdem alle Objekte vorbereitet sind, folgt das Rendering der Szene, welches mit Imperfektionen wie Rauschen und Verzerrungen, wie es in Kameras aus Mobiltelefonen vorkommt, angereichert wird, wodurch ein weiterer Schritt des Realismus der erstellten Daten erzieht wird. Zusätzlich werden binäre Masken unterschiedlicher Objekte gerendert, welche zur Extraktion relevanter Informationen in den Daten relevant sind. Zuletzt werden durch Nachverarbeitungsschritte, welche in \autoref{sec:methodik_postprocessing} detailliert beschrieben werden, in den Daten enthaltene Metainformationen extrahiert und explizit gespeichert. Im Wesentlichen umfasst dies die Lokalisierung der Dartpfeilspitzen in den gerenderten Bildern und die Ermittlung der Normalisierung durch zuvor erwähnte Masken.

\begin{figure}
    \centering
    \includegraphics[width=\textwidth]{imgs/rendering/rendering_pipeline.pdf}
    \caption{Rendering-Pipeline}
    \label{img:rendering_pipeline}
\end{figure}

\subsection{3D-Szene} % ===========================================================================
\label{sec:3d_szene}

Das Fundament der Datengenerierung ist die Simulation realistischer Dartscheiben. Diese Simulation fußt auf der virtuellen 3D-Szene, deren Grundzüge in den folgenden Unterkapiteln beschrieben werden. Es werden die zentralen Objekte der Szene beschrieben, die Dreh- und Angelpunkt der Datenerstellung darstellen und die für die Varianz und Komplexität der erstellten Daten ausschlaggebend sind. An Objekten werden die Dartscheibe in \autoref{sec:dartscheibe}, die Dartpfeile in \autoref{sec:dartpfeile} und die Beleuchtungsmöglichkeiten in \autoref{sec:lichter} beschrieben, sowie die globalen Parameter der Szene in \autoref{sec:parameter}.

\subsubsection{Dartscheibe}
\label{sec:dartscheibe}

Die Dartscheibe ist jenes Objekt, welches statisch in der Szene vertreten ist und in jedem gerenderten Bild vorhanden ist. Damit ist ihr Aussehen zentral für die Qualität der Daten. Sie wurde gemäß der Richtlinien \quotes{Playing and Toutnament Rules} der \ac{wdf} erstellt \cite{wdf-rules}. Beschreibungen dieses Regelwerks wurden in dieser Thesis als Quelle für Maße und Toleranzen genutzt; Dartscheiben mit esoterischen Maßen und Feldfarben, die nicht konform mit den Regeln sind, wurden nicht explizit für diese Arbeit mit einbezogen. Zusätzlich zu diesen Regeln wurden unterschiedliche reale Dartscheiben als Referenzen genutzt, die reale Gebrauchsspuren aufweisen und anhand derer zu erwartende Beschaffenheiten eingefangen wurden.

\subsubsection{Dartpfeile}
\label{sec:dartpfeile}

Zusätzlich zentral für die Datenerstellung sind die Dartpfeile. Im Vergleich zur Dartscheibe ist das Aussehen der Dartpfeile nicht stark durch die Darts-Regulierungen vorgegeben \cite{wdf-rules,pdc_rules}. Lediglich die maximale Länge und das Gewicht sowie die grundlegende Zusammensetzung der Pfeile werden in den Regulierungen thematisiert. Dadurch ist die Spanne des Aussehens möglicher Dartpfeile sehr groß und muss dementsprechend behandelt werden, um systematische Fehler zu mindern.

\subsubsection{Beleuchtung und Lichtobjekte}
\label{sec:lichtobjekte}

Für die Beleuchtung der Szene werden globals und lokale Beleuchtungsmöglichkeiten verwendet. Globale Beleuchtung wird durch die Verwendung von Environment erzielt während lokale Beleuchtung durch spezielle Lichtobjekte umgesetzt wird. Diese Lichtobjekte stellen unterschiedliche Beleuchtungsmöglichkeiten dar, die bei der Recherche zum Aussehen und den Umgebungen von Dartscheiben beobachtet wurden. Diese sorgen für eine vielseitige Beleuchtung der Szene.

\subsubsection{Weitere Objekte}
\label{sec:weitere_objekte}

Eine weitere Beobachtung typischer Dartscheiben ist Anbringung von Dartscheiben in Dartschränken. Diese schützen die Wand vor an der Dartscheibe vorbei geworfenen Dartpfeilen und ermöglichen das Verschließen der Dartscheibe. Für die 3D-Szene wurde daher ein Dartschrank aus Holz modelliert, dessen Holzfarbe zufällig gesetzt wird. Die Existenz des Dartschranks ist verbunden mit der Abwesenheit eines Ringlichts, das in \autoref{sec:lichter} beschrieben wird, da sich diese Objekte überschneiden.

\subsubsection{Parametrisierung}
\label{sec:parameter}

Für Generierung von Zufallsvariablen stellt die Szene zwei Parameter zur Verfügung: Seed und Alter. Der Seed ist ein Wert in einem vorgegebenen Intervall, der zur deterministischen Generierung von Zufallsvariablen genutzt wird. Diese Zufallsvariablen werden in den Objekten genutzt, um Abnutzungen, Zusammensetzungen, Verschiebungen und Texturen zu beeinflussen (vgl. \autoref{sec:dartscheibe_parametrisierung}, \autoref{sec:dartpfeile_zusammensetzung}). Auf diese Weise ist ein deterministisches Erstellen von Szenen möglich. Der Seed wird ebenfalls zur Generierung des Alters-Parameters genutzt. Dieser gibt das Alter der Objekte in der Szene an und wird verwendet, um den Grad der Abnutzung in Dartscheibe und Dartpfeilen zu bestimmen.

% -------------------------------------------------------------------------------------------------

\subsection{Material und Licht}  % ================================================================
\label{sec:material_licht}

Verweis: Details in Implementierung.

Die Ausgestaltung der Objekte sowie die Arten der Beleuchtung sind essenziell für das Erstellen realistischer Daten und eine Abdeckung einer Vielzahl unterschiedlicher Szenarien. Für die Datenerstellung werden prozedurale Texturen verwendet, die in ihren Grundzügen reale Beobachtungen widerspiegeln. Die genaue Zusammensetzung und Ausgestaltung der Texturen wird in den folgenden Unterabschnitten genauer erläutert. Dazu werden insbesondere die Dartscheibe, die Dartpfeile und die Lichtquellen betrachtet.

\subsubsection{Material der Dartscheibe}

Die Dartscheibe besteht grundlegend aus den Darts-Feldern, der Spinne, dem Zahlenring und einer Beschriftung. Die Dartfelder sowie der Rahmen der Dartscheibe sind der Beschaffenheit von Sisal nachempfunden und werden mit unterschiedlichen Gebrauchsspuren versehen. Das Material von Sisal ist sehr diffus und wenig reflektiv. Die Oberfläche ist angeraut und weist Unebenheiten auf.

Die Gebrauchsspuren des Sisals wird in Form von Abnutzung durch Kratzer, Einstichlöcher und Staubansammlung sowie durch Risse im Material. Darüber hinaus ist zur Simulation von Alterung der Dartscheibe eine Verfärbung der Felder und Verstärkung der Abnutzungen eingebaut. Die Beschaffenheiten und Vorkommen dieser Abnutzungen sind realen Dartscheiben nachempfunden und zielen darauf ab, eine möglichst große Spanne unterschiedlicher Dartscheiben abzudecken.

Die Spinne der Dartscheibe ist als teilweise reflektives Metall modelliert. Die Spinne der Dartscheibe ist ebenfalls von einem Altersprozess betroffen, da beobachtet wurde, dass die Dicke und Präsenz der Spinne bei Dartscheiben zunehmenden Alters stärker ausgeprägt sind. Historisch ist dies dadurch begründet, dass der Herstellungsprozess von Dartscheiben im Laufe der Zeit fortschrittlicher wurde. Alte Spinnen

Da alte Dartscheiben aktuell weiterhin in Verwendung sind, ist eine Abdeckung ihrer Geometrien in der Datenerstellung von Relevanz. Die Spinne sowie der Zahlenring unterliegen darüber hinaus einer Wahrscheinlichkeit, von Rostbildung betroffen zu sein, und werden mit steigendem Alter der Dartscheibe verformt, sodass die nicht auf den zu erwartenden Positionen liegen. Diese Verformungen wurden ebenfalls auf realen Dartscheiben beobachtet und sind potenziell für Dartpfeile, die nahe der Spinne landen, relevant.

Die Beschriftung der Dartscheibe beinhaltet typischerweise den Herstellernamen der Dartscheibe sowie Symbole und Logos. Diese werden mit zufällig generierten Texten approximiert, die entlang des Randes der Dartscheibe verlaufen und zufällig platziert werden.

Konkrete Informationen zur Umsetzung der Texturierung der Dartscheibe und zum Aufbau des Materials sind in \autoref{sec:dartscheibe_parametrisierung} zu finden.

\subsubsection{Generierung der Dartpfeile}

Obligatorische Bestandteile von Dartpfeilen beinhalten Tip, Barrel, Shaft und Flight. Aus jeweiligen Pools von Objekten werden Dartpfeile zufällig zusammengestellt, um randomisierte Dartpfeile zu generieren.

Die Tips der Dartpfeile sind wenig variabel und unterscheiden sich hauptsächlich in Länge und Farbe. Trotz ihrer geringen Größe ist eine realistische Modellierung der Tips bedeutend für die Daten, da sie ausschlaggebend für die erzielte Punktzahl sind. Neben silbernen Tips sind ebenfalls schwarze oder auch bronzene Tips möglich, die in ihrer Reflektivität variieren.

Die Barrels sind im Vergleich zu den Tips wesentlich komplexer, sodass einige vordefinierte Barrels generiert wurden, die zufällig hinter eine Tip gefügt werden. Darüber hinaus wird die Länge der Barrels zufällig variiert, um weitere Variationen einzubinden. Die Beschaffenheit von Barrels realer Dartpfeile ist sehr unterschiedlich, sodass eine sehr große Variabilität ihrer Erscheinungsbilder möglich ist. Um weitere Variabilitäten einzubinden, wurden teilweise Materialien verwendet, deren Farbe zufällig gesetzt wird.

Auf die Barrel folgt der Shaft des Dartpfeils, der als Übergang zum Ende des Dartpfeils dient. Dieser wird ebenso wie die Barrel aus vorgefertigten Bauteilen ausgewählt, in seiner Länge modifiziert und teilweise mit zufälligen Farben versehen. Reale Dartpfeile folgen häufig einem kohärenten Farbschema während die Abstimmung von Barrels und Shafts in dieser Thesis zufällig ist. Hinsichtlich der Variabilität der Dartpfeile ist diese Herangehensweise jedoch präferiert.

Das Ende der Dartpfeile bilden die Flights. Diese sind die meist aus Plastik gefertigten Flügel des Dartpfeils und ihr Erscheinungsbild variiert von allen Bestandteilen am stärksten. Farben von Flights reichen von einzelnen Farben über Flaggen und Wappen bis hin zu abstrakten Bildern. Zusätzlich ist die Form von Flights nicht vorgegeben. Diese Gegebenheiten wurden in dieser Thesis durch Projektion eines zufälligen Bereichs eines Texturatlas\footnote{Der Texturatlas beinhaltet Länderflaggen sowie geometrische Formen und zufällige Farben. Teile des Atlas wurden durch die Bilderstellungs-KI von DeepAI \cite{deepai-image} generiert.} auf eine Grundformen für Flights realisiert. Die Grundformen sind ebenfalls vorgefertigt und orientieren sich an realen Formen für Flights. Weiterhin ist eine Verformung der Flights als Gebrauchsspur von Dartpfeilen identifiziert worden, die ebenfalls in die Generierung der Dartpfeile einbezogen ist. Die Materialien der Flights variieren in ihrer Reflektivität, sind jedoch sehr glatt und dem Erscheinungsbild von Plastik nachempfunden.

Die Umsetzung dieser Methodiken für die Datenerstellung wird in \autoref{sec:dartpfeile_zusammensetzung} der \nameref{sec:daten:implementierung} beschrieben.

\subsubsection{Lichtquellen}
\label{sec:lichter}

Hinsichtlich ber Beleuchtungsmöglichkeiten der Szene existieren unterschiedliche Objekte, die jeweils unterschiedliche Auswirkungen der Beleuchtung mit sich ziehen. Für die Datenerstellung wurden fünf unterschiedliche Arten der Beleuchtung modelliert.

\paragraph{Environment Maps}

Environment Maps, auch als HDRIs bezeichnet, sind $360\degree$-Scans von realen Umgebungen. Diese können bei dem Rendern von Szenen als Hintergrund genutzt werden, sodass die Farben zur Ausleuchtung der Szene dienen. Dadurch ist die Simulation realistischer Beleuchtungen möglich, ohne die jeweilige Szene nachzustellen. Die Intensität der Environment Maps bestimmt dabei die Ausprägung der Beleuchtung, sodass eine Intensität von $\nicefrac{1}{2}$ die Helligkeit der Environment Map reduziert, sodass eine Spanne unterschiedlicher Beleuchtungen unter der Verwendung der selben Environment Map möglich ist.

\paragraph{Kamerablitz}

Wenige Zentimeter neben der Kamera befindet sich ein Punktlicht, das als Kamerablitz fungiert. Es kann ein- und ausgeschaltet werden und sorgt unter dessen Verwendung für eine helle Ausleuchtung der Szene. Die Farbe des Lichtes ist kaltweiß und sorgt durch sein Positionierung für harte Kanten entlang der Kanten im Bild. Besonders stark ist dieser Effekt bei Dartpfeilen zu beobachten.

\paragraph{Spotlight}

Bei der Aufnahme realer Daten ist die Existenz von Spotlights aufgekommen. Bei Spotlights handelt es sich um ein oder mehrere Lichter, die auf die Dartscheibe gerichtet sind und den Feldbereich ausleuchten. Diese Art der Ausleuchtung sorgt ebenfalls für einen auffälligen Schattenwurf und wird in der 3D-Szene als Flächenlicht variierender Größe modelliert. Diese Art der Beleuchtung wurde im Strongbows Pub\footnote{\url{https://www.strongbowspub.de}} in Kiel beobachtet.

\paragraph{Ringlicht}

Ein typisches Accessoire für Dartscheiben sind Ringlichter. Diese bestehen aus einem Gestell, das an der Dartscheibe befestigt wird und an dem LEDs in einem Ring vor dieser angeordnet, um diese direkt zu beleuchten. Ringlichter sorgen für eine uniforme Ausleuchtung der Dartscheibe und wenig Schattenwurf der Pfeile. Modelliert sind Ringlichter nach dem Vorbild der Dartscheiben in Jess Bar in Kiel. Die LEDs sind bei dieser Art des Ringlichts an der Vorderkante eines zylindrischen Korpus angebracht, in dem die Dartscheibe befestigt ist. Dieser Korpus kann unterschiedliche Farben besitzen.

\paragraph{Deckenbeleuchtung}

Zusätzlich zu den bereits genannten Beleuchtungsmöglichkeiten existieren Deckenleuchten in der Szene, die für eine warmweiße Beleuchtung sorgen. Diese Lichter werden unter anderem als Rückfall-Beleuchtung verwendet, sofern -- mit Ausnahme der Environment Maps -- keine andere Beleuchtung aktiviert ist. Dadurch ist sichergestellt, dass keine unbeleuchtete Szene entsteht.

% -------------------------------------------------------------------------------------------------

\subsection{Scripting}  % =========================================================================
\label{sec:scripting}

Neben der 3D-Szene wird ein externes Skript genutzt, mit dem auf die Szene zugegriffen wird und durch das Einstellungen getätigt werden. Dieses Unterkapitel thematisiert den Hintergrund und die Arbeitsweise dieses Skriptes.

\subsubsection{Wozu externes Skript?}

Obwohl die 3D-Szene der zentrale Punkt der Datenerstellung ist, geschieht das Setzen spezifischer Parameter und das Rendern der Szene durch ein Skript. Der Hintergrund dessen ist die Flexibilität einer programmatischen Herangehensweise im Vergleich zum strikten Modellieren. Darüber hinaus ist die automatisierbare Arbeitsweise und der Verzicht auf eine grafische Nutzeroberfläche prädestiniert für die Erstellung einer großen Menge an Daten.

\subsubsection{Einfluss der Parametrisierung}

Für jedes generierte Sample ist der erste zentrale Schritt des Skriptes das Setzen des Seeds. Dieser beeinflusst das Aussehen der Objekte in der Szene, jedoch nicht ihre Positionierung oder Existenz. Der Seed beeinflusst lediglich das Aussehen und die Beschaffenheit der Dartscheibe sowie die Zusammensetzung der Dartpfeile.

Der Seed hält einen zufälligen Wert, jedoch ist diesem Wert keine konkrete Bedeutung zugeschrieben, da er zur Generierung von Zufallsvariablen genutzt wird, die wiederum in den Materialien und Geometrien der Objekte eingesetzt werden. Der tatsächliche Wert des Seeds wird einzig genutzt, um das Alter der Szene zu bestimmen. Kleine Werte werden als geringes Alter interpretiert, große Werte als hohes Alter. Das Alter schlägt sich in dem Anblick der Dartscheibe und der Dartpfeile nieder.

\subsubsection{Spezifisches Setzen von Parametern}

Der Seed und der Alters-Parameter bestimmen weitestgehend das Aussehen der Szene, jedoch existiert die Ausnahme der Texte um die Dartscheibe. Diese Texte werden durch das Skript in ihrem Inhalt, der Schriftart und in ihrer Positionierung manipuliert.

Darüber hinaus ist das Setzen der Dartpfeil-Positionen ein wichtiger Schritt des Skriptes. Diese werden anhand von Wahrscheinlichkeitsverteilungen auf der Dartscheibe verteilt und zufällig rotiert. Aus diesen Positionen kann die erzielte Punktzahl des simulierten Dartspiels abgeleitet werden.

Die Kamera wird von dem Skript in einem vordefinierten Raum positioniert und ihre internen sowie externen Parameter werden in definierten Intervallen randomisiert, zu Teilen auch basierend auf ihrer Position. Es werden unter anderem Brennweite, Fokuspunkt, Auflösung und das Seitenverhältnis gesetzt, um einer Datenverzerrung hinsichtlich spezifischer Kameraeinstellungen vorzubeugen. Würden alle Sample die selben Kameraparameter nutzen, besteht die Gefahr der Spezialisierung eines aus diesen Daten trainierten Systems auf diese Gegebenheiten. Dadurch besteht die Gefahr der fehlerhaften Inferenz auf Daten, die nicht diese exakten internen Kameraparameter vorweisen. Durch Randomisierung der Parameter wird diese Einschränkung der Generalisierungsfähigkeit umgangen.

\subsubsection{Statische und dynamische Objekte}

Die Präsenz und Absenz einiger Objekte in der Szene wird ebenfalls durch das Skript gesteuert. Somit ist eine Unterscheidung möglich zwischen statischen Objekten, die in jeder Szene vorhanden sind, und dynamischen Objekten, die nicht in jeder Szene vorhanden sind.

Statische Objekte der Szene sind die Dartscheibe, die Kamera und die Environment Map. Diese sind in jeder Zusammensetzung der Szene vorhanden, obgleich die Environment Map lediglich sehr fade erkennbar ist.

Entgegen der Annahme, die Dartpfeile seien ebenfalls statische Objekte, sind diese nicht in jeder Szene vorhanden. Da die variable Anzahl an Würfen abgebildet werden muss, ist ein zufälliges Ausblenden der Dartpfeile möglich. Dadurch besteht die Möglichkeit, dass alle Dartpfeile ausgeblendet werden und eine leere Dartscheibe abgebildet wird.

Darüber hinaus werden jegliche Lichtobjekte zu den dynamischen Objekten gezählt, da sie durch das Skript ein- und ausgeblendet werden können. Obwohl die Existenz mindestens einer Lichtquelle vorgegeben ist, ist keines der Lichtobjekte statisch in jeder Szene vorhanden. Diese Tatsache fußt ebenfalls auf der Unterbindung systematischer Fehler durch einseitige Modellierung der Szene.

Zuletzt ist ebenfalls der Dartschrank ein dynamisches Objekt, dessen Existenz von dem Skript gesteuert wird. Diese Existenz ist an die Abwesenheit gewisser Lichtobjekte geknüpft, sodass keine ungewollte Überschneidungen von Objekten in der Szene vorhanden sind.

\subsubsection{Rendern von Bildern}

Der Anstoß zum Rendern von Bildern aus der Szene geschieht ebenfalls durch das Skript. Nachdem die Szene vorbereitet wurde, wird ein Bild aus der Sicht der Kamera gerendert. In diesem Bild ist eine randomisierte Dartscheibe mit einem zufälligen Dartwurf abgebildet, die aus einem zufälligen Winkel in einer variierenden Beleuchtung eingefangen wurde. Zusätzlich zu diesem Bild werden weitere Masken von Objekten gerendert. Diese Masken werden als binäre Schwarzweißbilder exportiert und zeigen lediglich einzelne Objekte. Unter anderem werden die Dartpfeile, die Fläche der Dartscheibe, die Schnittpunkte der Dartpfeile und der Dartscheibe sowie Orientierungspunkte der Dartscheibe, wie sie im DeepDarts-System verwendet wurden, generiert. Durch diese Masken wird die Ableitung unterschiedlicher Informationen aus dem Bild genutzt.

Zusätzlich zum Speichern der Bilder werden Metainformationen zu der Szene gespeichert. Diese beinhalten u.\,a. Informationen zu Kameraparametern, Punktzahl, Existenz von Objekten und dem Kamerawinkel zur Dartscheibe. Diese Informationen dienen der Annotation der Daten sowie der Erhebung von Statistiken.

% -------------------------------------------------------------------------------------------------

\subsection{Nachverarbeitung und Fertigstellung der Daten}  % =====================================
\label{sec:methodik_postprocessing}

Nach dem Rendern von Bildern und Masken der Szene erfolgt eine Nachverarbeitung der Daten durch ein weiteres Skript. Durch dieses werden weitere Informationen von den gespeicherten Informationen angeleitet und für das Training eines datengestützten Systems vorbereitet.

\subsubsection{Normalisierung der Dartscheibe}

Für das Training des neuronalen Netzes zum Scoring von Dartsrunden wird in dieser Thesis von normalisierten Bildern ausgegangen. Da das Ziel der Datenerstellung eine Nachempfindung möglichst realistischer Daten ist, die in ihrem Umfang möglichst wenig beschränkt werden, besteht eine Diskrepanz zwischen gerenderten Bildern und Netzwerk-Inputs. In der Inferenz des in dieser Thesis entwickelten Systems wird diese durch die in \autoref{cha:cv} dargestellte Normalisierung geschlossen. Für den Schritt des Trainings wird diese Diskrepanz durch einen Nachverarbeitungsschritt in der Datenerstellung angegangen; die Ausgabe normalisierter Trainingsdaten ist daher Teil der Datenerstellung.

Die Normalisierung der Dartscheibe basiert auf Orientierungspunkten, deren Positionen relativ zur Dartscheibe bekannt sind. Diese Orientierungspunkte werden in den gerenderten Bildern durch binäre Masken ermittelt. Da die Positionen aller Punkte in den normalisierten Bildern bekannt sind, ist ein Mapping der Orientierungspunkte der gerenderten Bilder auf ihre Zielpositionen trivial zu ermitteln. Diese Herangehensweise hat sich bereits im DeepDarts-System bewährt und wird daher analog in dieser Thesis angewendet \cite{deepdarts}.

Dazu werden 4 Orientierungspunkte ermittelt, die in konstanten Abständen entlang der Außenkante der Dartfelder verteilt sind. Durch diese ist eine eindeutige Entzerrung der Dartscheibe möglich, indem eine Homographie gebildet wird. Dieser Ansatz wird ebenfalls in \autoref{cha:cv} verfolgt, um eine Entzerrung zu ermitteln. Genauere Hintergrundinformationen zu den Techniken werden dort ebenfalls erläutert.

\subsubsection{Identifizierung von Dartpfeil-Positionen in Bildern}

Beim Rendering der Bilder werden unter anderem Masken der Einstichstellen von Dartpfeilen in die Dartscheibe sowie Masken der Dartpfeile erstellt. Durch Überlagerung dieser Masken ist eine eindeutige Zuordnung von Dartpfeilen und Einstichstellen möglich, die eine Korrelation zwischen Positionen im Bild und erzielten Punktzahlen ermöglicht. Der Hintergrund der Notwendigkeit dieser Überlagerung liegt in der Speicherung der Daten: Die Punktzahlen werden anhand der Dartpfeil-Indizes sortiert während die Positionen als einzelne Maske exportiert wird. Durch die einzelnen Masken der Dartpfeile lässt sich ein korrektes Mapping herstellen.

Unter Verwendung der im vorherigen Unterabschnitt erläuterte Normalisierung der Dartscheibe lassen sich die Positionen der Dartpfeile im Ausgangsbild auf normalisierte Positionen überführen. Durch die Verwendung von Masken der Einstichstellen ist eine fehlerfreie Positionierung der Einstichstellen in den Bildern möglich, obgleich diese in den Bildern sichtbar sind oder von anderen Darts verdeckt sind. Dieser Punkt ist ein wesentlicher Vorteil automatisierter Datenerstellung gegenüber manueller Annotation, da in jedem Fall davon auszugehen ist, dass keine Ungenauigkeiten oder fehlerhafte Annotationen in den Daten enthalten sind.

\subsubsection{Statistiken über Sample erheben}

Zusätzlich zu den essenziellen Informationen zum Trainieren eines neuronalen Netzes werden Statistiken zu den generierten Daten abgeleitet und gespeichert. Diese dienen der Erhebung von Statistiken und ermöglichen einen potenziellen Einblick in die Stärken und Schwächen eines Systems. Inbegriffen in diesen Daten sind ein Maß zur Überdeckung der Dartpfeile zueinander. Darüber hinaus wird festgehalten, wie viele der Einstichlöcher durch andere Dartpfeile verdeckt sind und damit nicht eindeutig zu sehen sind.

Durch Approximation von Ellipsen und Linien auf Grundlage der Orientierungspunkte und Dartscheiben-Maske ist eine geometrische Beschreibung der Dartscheibe möglich. Die Verwendung dieser geometrischen Beschreibung der Dartscheibe un ihrer Ausrichtung im Bild wurde im Verlaufe der Thesis verworfen, jedoch existieren die notwendigen Daten weiterhin als Relikte in den Daten. Diese Daten können potenziell genutzt werden, um unterschiedliche Herangehensweisen zur Lokalisierung der Dartscheibe zu implementieren.

% -------------------------------------------------------------------------------------------------

% !TEX root = ../main.tex

\section{Implementierung}
\label{sec:daten:implementierung}

Implementierung hier.

\todo{Einleitende Sätze}

% -------------------------------------------------------------------------------------------------

\subsection{Parametrisierung der Dartscheibe}  % ==================================================
\label{sec:dartscheibe_parametrisierung}

\subsubsection{Einsatz von Noise-Texturen}

Das Material der Dartscheibe besteht aus unterschiedlichen Schichten, die miteinander kombiniert werden. Ein Grundbestandteil der meisten Schichten 

WIE wurde es umgesetzt? -> Maskierungen etc.

\todo{}

\subsection{Zusammensetzung der Dartpfeile}  % ====================================================
\label{sec:dartpfeile_zusammensetzung}

WIE sind sie zusammengesetzt? Nutzung von Geometry-Nodes

Die Umsetzung der Generierung von Dartpfeilen beruht auf der Nutzung von Geometry Nodes in Blender. Geometry Nodes bieten die Möglichkeit der deskriptiven Zusammensetzung von Objekten. Weiterhin ist die Einbindung von externen Parametern wie dem globalen Seed der Szene zur Steuerung von Zufallsvariablen durch sie ermöglicht. Aufgebaut werden die Dartpfeile aus einem Pool unterschiedlicher vordefinierter Objekte, die auf Grundlage des globalen Seeds zu einem zufälligen Dartpfeil zusammengesetzt werden. Die für die Generierung vordefinierten Objekte sind in \autoref{img:darts_parts} aufgelistet.

\begin{figure}
    \centering
    \includegraphics[width=\textwidth]{imgs/rendering/implementierung/darts.png}
    \caption{Bestandteile von Dartpfeilen. Von links nach rechts: Tips, Barrels, Shafts und Flights.}
    \label{img:darts_parts}
\end{figure}

\paragraph{Tips}

Der erste Schritt zur Generierung eines Dartpfeils ist die Wahl der Tip. Sie wird aus einem Pool von 4 Objekten gewählt und ihre Spitze wird zum Ursprung des finalen Dartpfeils. Ihre Textur der Tip wird ebenfalls auf Grundlage des Seeds zufällig gesetzt, sodass ein möglichst großes Spektrum unterschiedlicher Dartpfeilspitzen abgedeckt wird.

\paragraph{Barrels}

Auf die Tip folgt die Platzierung der Barrel. Die Barrel wird aus einem Pool von 7 Objekten ausgewählt, deren Ursprung jeweils derart positioniert ist, dass eine lückenlose Platzierung an der Tip ermöglicht ist. Die unterschiedlichen Barrel-Objekte verwenden verschiedene Methoden der Texturierung, sodass die Materialien einiger Barrel statisch vorgegeben sind, wohingegen andere Barrel ebenso wie die Tips zufällig texturiert werden. Die Spanne der unterschiedlichen Farben, die eine dynamische Barrel annehmen kann, ist im Gegensatz zu dem Farbspektrum der Tips größer, da alle RGB-Kanäle und die Oberflächenbeschaffenheit variabel sind. Darüber hinaus wird die Geometrie der Barrel bei ihrer Platzierung hinter der Tip um $\pm\,20\%$ sowohl in ihrer Länge als auch im Durchmesser variiert.

\paragraph{Shafts}

Im Anschluss an die Barrel wird der Shaft des Dartpfeils platziert. Dieser wird ebenfalls aus einem vordefinierten Pool von acht Objekten ausgewählt. Der Großteil dieser Objekte besitzt dynamische Texturen, die analog zu dynamischen Barrel-Textren agieren. Ebenfalls wird die Geometrie der Shafts um $\pm\,20\%$ in Länge und Durchmesser variiert.

\paragraph{Flights}

Die Flights sind die komplexesten Elemente der Dartpfeile, da sie sie größte Spanne an Erscheinungsbildern besitzen. Ihr Aussehen variiert nicht nur durch ihre Farben, sondern auch durch durch ihre Form. Flights setzen sich aus vier gleichen Flügeln zusammen, die entlang des Dartpfeils in einem Abstand von $90\degree$ platziert sind. Um die Variation der Formen einzufangen, wurden 15 unterschiedliche Formen für Flights modelliert, die sich an realen Formen von Flights orientieren. Ihre Textur wird aus einem Texturatlas mit einer Größe von $1920 \times 1920$ Pixeln gesampled. Dieser besteht aus 9 unterschiedlichen Grundtexturen, die aus Landesflaggen und abstrakten Formen unterschiedlicher Farbpaletten bestehen. Abhängig vom Altersfaktor wird die Verformung der Flights gesteuert, sodass neue Flights keine Deformierungen aufweisen, alte Flights jedoch stark deformiert werden, um Gebrauchsspuren zu simulieren.

\paragraph{\,\,}

Alle Dartpfeile einer Szene nutzen den selben Geometry Nodes, sodass lediglich gleiche Dartpfeile existieren. Eine Variation der Dartpfeile innerhalb einer Szene ist möglich, es wurde sich jedoch gegen diese Art der Umsetzung entschieden, da die Verwendung unterschiedlicher Dartpfeile für den selben Wurf unwahrscheinlich ist. Unterschiedliche zufällig generierte Dartpfeile sind in \autoref{img:dartpfeile} dargestellt. Hervorzuheben sind die unterschiedlichen Farben und Formen der Flights sowie die variierenden Bestandteile und ihre Texturierung.

\begin{figure}
    \centering
    \includegraphics[width=0.8\textwidth]{imgs/rendering/implementierung/darts_examples.png}
    \caption{Dartpfeile}
    \label{img:dartpfeile}
\end{figure}

\subsection{Generierung von Dartpfeil-Positionen}  % ==============================================
\label{sec:wie_dartpfeil_positionen}

\subsubsection{Positionierung}

Eine Uniforme Wahrscheinlichkeitsverteilung der Dartpfeilpositionen folgt weder Erwartungen realer Spiele noch wird es dem Anspruch dieser Arbeit gerecht. Zur realitätsnahen Simulation von Dartsrunden wurden daher reale Wahrscheinlichkeitsverteilungen analysiert und diese wurden in Form von Heatmaps in die Szene eingearbeitet.

Die für die Datengenerierung dieses Thesis genutzten Heatmaps sind in \autoref{img:heatmaps} dargestellt. Es wurden zwei unterschiedliche Heatmaps genutzt: Eine realistische Heatmap und eine Heatmap zur gezielten Erstellung von Multiplier-Feldern und ihren Umgebungen. Die generelle Heatmap orientiert sich an den für DeepDarts gefundenen Wahrscheinlichkeitsverteilungen \cite{deepdarts}, Verteilungen aus Online-Recherchen \cite{heatmap} und eigenen Beobachtungen. Die Wahrscheinlichkeitsverteilungen dieser Heatmaps beziehen die gesamte Dartscheibe ein, sodass Treffer außerhalb der Dartfelder ebenfalls möglich sind.

\begin{figure}
    \centering
    \includegraphics[width=0.8\textwidth]{imgs/rendering/methodik/heatmaps.pdf}
    \caption{Heatmaps für die Datenerstellung; (links) Generelle Heatmap; (rechts) Multiplier-Heatmap für Oversampling der Daten.}
    \label{img:heatmaps}
\end{figure}

Bei der Findung von Positionen der Dartpfeile geschieht der Platzierung auf Grundlage der aktiven Heatmap. Bereiche mit hohen Gewichten unterliegen einer höheren Wahrscheinlichkeit, als Position für einen Dartpfeil gewählt zu werden als Bereiche mit geringen Gewichten. Durch eine Adaption der Heatmap, können gezielt Positionen forciert werden. So wurde für die Datengenerierung dieser Arbeit eine weitere Heatmap erstellt, die sich auf die Multiplier-Felder und ihre Umgebungen fokussiert. Durch diese zweite Heatmap wird eine Erstellung von Daten ermöglicht, bei denen alle Dartpfeile entweder auf den Multiplier-Feldern liegen oder in ihrer Nähe. Treffer weit außerhalb der Dartscheibe sowie Treffer zentral in Einzelfeldern werden unter Verwendung jeder Heatmap nicht generiert.

Nach der Positionierung der Pfeile auf der Dartscheibe wird eine Nachverarbeitung vorgenommen, bei der Dartpfeile, die eine Überschneidung mit der Spinne aufweisen, von dieser entfernt werden. So wird sichergestellt, dass keine ambivalenten Dartpfeile existieren, die auf der Grenze zweier Felder eintreffen.

\subsubsection{Existenz}

Die Existenz von Dartpfeilen wird durch zwei Faktoren gesteuert. Vor der Positionierung eines Dartpfeils wird für jeden Pfeil entschieden, ob dieser existiert oder nicht. Die Wahrscheinlichkeit einer Ausblendung eines Dartpfeils liegt bei $\nicefrac{1}{3}$. Durch diese Zufallsentscheidung wird eine dynamische Anzahl an Dartpfeilen generiert.

Eine weitere Gegebenheit, unter der ein Dartpfeil ausgeblendet wird, ist die zu geringe Entfernung zu anderen Dartpfeilen. Liegt die Position eines Dartpfeils zu nahe an einem bereits platzierten Dartpfeil, wird dieser ausgeblendet. Es wurde sich gegen eine Adaption der Position entschieden, um zu starke Abweichung von Heatmaps und erneute Überschneidung der Spinne zu vermeiden; eine neue Positionierung des Dartpfeils wurde nicht eingesetzt, da die Möglichkeit besteht, dass die verwendete Heatmap nicht ausreichend Bereiche zur korrekten Platzierung aller Dartpfeile zur Verfügung stellt.

\subsubsection{Rotation}

Alle verbleibenden Dartpfeile auf der Dartscheibe werden im Anschluss rotiert. Die Rotation des Dartpfeils entlang der horizontalen Achse verläuft uniform im Intervall $[-5\degree, 35\degree]$. Diese Rotation bestimmt den Einschlagswinkel des Dartpfeils. Entlang der vertikalen Achse erfahren die Dartpfeile eine normalverteilte Rotation mit einer Standardabweichung von $\sigma = \frac{15\degree}{3}$ um $0\degree$ mit einem Clipping einer maximalen Rotation von $\pm\,15\degree$. Die Rotation entlang ihrer eigenen Achse ist uniform im Intervall $[0\degree, 360\degree]$.

\subsubsection{Scoring}

Nachdem die Dartpfeile Positioniert und rotiert sind wird das Scoring der Szene vorgenommen. An diesem Punkt sind die Position der Dartscheibe $p_\text{Dartscheibe} \in \mathbb{R}^3$ und die Positionen aller Dartpfeile $p_{Pfeil}, i \in \mathbb{R}^3$ bekannt. Durch ihre Winkel und Abstände lassen sich die Dartfelder identifizieren, in denen die Dartpfeile eingetroffen sind. Auf diese Weise lässt sich für jeden Dartpfeil ermitteln, in welchem Feld dieser eingetroffen ist und welche Punktzahl durch ihn erzielt wurde.

\subsection{Ermittlung von Kameraparametern}  % ===================================================
\label{sec:ermittlung_kameraparamater}

\subsubsection{Kameraraum}

Für die Positionierung der Kamera in der Szene existiert ein Objekt, das den Bereich umfasst, innerhalb dessen die Kamera platziert werden kann. Dieser kegelförmige Kameraraum ist ausgehend vom Mittelpunkt der Dartscheibe platziert und durch verschiedene Parameter definiert:

\begin{itemize}
    \item \textbf{Horizontaler Seitenwinkel $\phi_h$}: Öffnungswinkel des Kegels zu den Seiten der Dartscheibe
    \item \textbf{Vertikaler Winkel $\phi_v$}: Öffnungswinkel des Kegels in die Höhe
    \item \textbf{Kameraabstand $\left(d_\text{min}, d_\text{max}\right)$}: Minimaler und maximaler Abstand der Kamera von der Dartscheibe
    \item \textbf{Kamerahöhe $\left(y_\text{min}, y_\text{max}\right)$}: Minimale und maximale Höhe der Kamera im Raum\footnote{Es wird von einem realen Raum ausgegangen, in dem der Mittelpunkt der Dartscheibe auf einer Höhe von 2,07m angebracht ist.}
    \item \textbf{Maximaler Seitenabstand $dx_\text{max}$}: Maximaler seitlicher Abstand der Kamera zum Dartscheibenmittelpunkt
\end{itemize}

Die Generierung der Daten dieser Arbeit erfolgte mit den Parametern: $\phi_h = 110\degree$, $\phi_v = 60\degree$, $d_\text{min} = 60\,\text{cm}$, $d_\text{max} = 150\,\text{cm}$, $y_\text{min} = 160\,\text{cm}$, $y_\text{max} = 220\,\text{cm}$, $dx_\text{max} = 60\,\text{cm}$. Der verwendete Kameraraum ist in \autoref{img:camera_space} dargestellt.

\begin{figure}
    \centering
    \includegraphics[width=0.8\textwidth]{imgs/rendering/implementierung/camera_space.png}
    \caption{Visualisierung des für die Datenerstellung verwendeten Kamerabereich.}
    \label{img:camera_space}
\end{figure} Einstellungen verwendet

\subsubsection{Brennweite}

Das Setzen der internen Kameraparameter erfolgt nach der Positionierung der Kamera im Raum. So wird die Brennweite $l$ der Kamera in Abhängigkeit ihrer Distanz zur Dartscheibe gesetzt. Die Spanne der Brennweite reicht von 18mm bis 60mm. Die tatsächlichen Grenzwerte der Brennweiten in Abhängigkeit der Distanz werden wie folgt berechnet:

\begin{align*}
    l_\text{lower} & = l_\text{min} + \frac{d}{d_\text{max} - d_\text{min}} * \frac{l_\text{max} - l_\text{min}}{2}                          \\
    l_\text{upper} & = \frac{l_\text{max} - l_\text{min}}{2} + \frac{d}{d_\text{max} - d_\text{min}} * \frac{l_\text{max} - l_\text{min}}{2}
\end{align*}

Visualisiert sind diese Gleichungen in \autoref{fig:brennweiten}. Werte zwischen Ober- und Untergrenze werden zufällig uniform gewählt. Auf diese Weise wird eine Korrelation zwischen verwendeter Brennweite und Abstand zur Dartscheibe gewonnen unter Beibehaltung der Variabilität der Daten und Einfluss von Zufallswerten.

\begin{center}
    \begin{tikzpicture}
        \begin{axis}[
                ymin = 15,
                ymax = 65,
                domain = 60:150,
                samples=50,
                xlabel={Abstand $d$ der Kamera zur Dartscheibe [cm]},
                ylabel={Brennweite $l$ der Kamera [mm]},
                scaled x ticks = false,
                grid=both,
                extra x ticks={60,150},
                extra x tick labels={$d_\text{min}$, $d_\text{max}$},
                extra x tick style={ticklabel pos=top},
                extra y ticks={18,60},
                extra y tick labels={$l_\text{min}$, $l_\text{max}$},
                extra y tick style={ticklabel pos=right},
            ]
            \addplot[name path = upper, thick, densely dashed, data_primary]{0.23 * x + 25};
            \addplot[name path = lower, thick, densely dashed, data_primary]{0.23 * x + 4};
            \addplot[color=data_secondary, opacity=0.5] fill between[of = upper and lower];
        \end{axis}
    \end{tikzpicture}
    \captionof{figure}{Abhängigkeit der Brennweiten von dem Abstand zur Dartscheibe. Ober- und Untergrenzen sind durch Strichlinien angegeben, der farblich hervorgehobene Bereich stellt die Spanne möglicher Brennweiten für jeweilige Entfernungen dar.}
    \label{fig:brennweiten}
\end{center}

\subsubsection{Seitenverhältnis und Auflösung}

Unterschiedliche Seitenverhältnisse der Kameraaufnahmen sind ebenfalls in dieser Arbeit berücksichtigt. So wird aus unterschiedlichen Seitenverhältnissen, die von Kameras in Mobiltelefonen verwendet werden, ausgewählt. Mögliche Seitenverhältnisse sind $4:3$, $16:9$, $1:1$, $3:2$, $2:1$, $21:9$ und $5:4$. Die Ausrichtung der Kamera ist in $\nicefrac{2}{3}$ der Aufnahmen vertikal und in $\nicefrac{1}{3}$ der Aufnahmen horizontal. Die Auflösung der Kamera wird uniform im Intervall $[1000\text{px}, 4000\text{px}]$ gewählt, welches die Pixelzahl entlang der längeren Seite angibt.

\subsubsection{Fokuspunkt}

Der Fokuspunkt der Kamera ist in der Szene durch ein eigenes Objekt definiert, sodass die Kamera den Ursprung dieses Objektes fokussiert. Dieses Objekt wird im Umfeld um die Dartscheibe platziert. Ausgehend vom Dartscheibenmittelpunkt wird es normalverteilt mit den Standardabweichungen $\sigma_x = \sigma_z = \frac{r_\text{D}}{3}$ und $\sigma_y = 2\,\text{cm}$ platziert. $x$- und $z$-Positionen liegen dabei auf der Dartscheibe, die $y$-Achse verläuft parallel zur Normalen der Dartscheibe und $r_\text{D}$ ist der Gesamtdurchmesser der Dartscheibe. Der Kamerafokus ist damit grob auf den Mittelpunkt der Dartscheibe gerichtet, jedoch nicht deterministisch.

\subsubsection{Verwackelungen}

Zuletzt werden Verwackelte Kamerabilder simuliert. Mit einer Wahrscheinlichkeit von $10\%$ wird die Kamera während der Aufnahme bewegt, wodurch verschwommene Bilder aufgenommen werden. Die Kamera wird normalverteilt mit einer Standardabweichung von $\frac{2\,\text{cm}}{3}$ in $x$-, $y$- und $z$-Position verschoben. Durch diese Verschiebung entstehen Aufnahmen, die teilweise verschwommen sind.

\subsection{Render-Einstellungen}  % ==============================================================
\label{sec:render_einstellungen}

Zur Handhabung der Farbinformation bietet Blender eine Vielzahl unterschiedlicher Einstellungen. Die hohe Diversität unterschiedlicher gewünschter Erscheinungsbilder sorgt für viele Möglichkeiten zur Anpassung der Farbaufnahme der Kameras in den Szenen. Für einen Cartoon ist beispielsweise ein anderer Umgang mit Farben bei dem Rendern von Bildern erwünscht als für eine cinematische Szene. Die für diese Thesis verwendeten Farbräume wurden derart gewählt, dass Farben möglichst realistisch dargestellt werden. Dazu wurde als Darstellungsgerät und Sequencer sRGB mit einer AgX als Anzeigetransformation gewählt. Trotz häufiger Korrekturen von Handykameras hinsichtlich Kontrasterhöhung der Aufnahmen wurde sich für einen neutralen Basiskontrast entschieden. Eine Erhöhung des Kontrasts geschieht bei der Augmentierung der Trainingsdaten in \autoref{sec:augmentierung}.

\subsection{Berechnung von Entzerrung}  % =========================================================
\label{sec:berechnung_entzerrung}

Die Erstellung der Entzerrungshomographie geschieht auf Grundlage exportierter Masken von dem Rendering. Eine der exportierten Masken zeigt die Orientierungspunkte, wie sie im DeepDarts-System verwendet wurden \cite{deepdarts}. Die Orientierungspunkte liegen auf der Außenseite des Double-Rings zwischen den Feldern 5 und 20 (oben), 13 und 6 (rechts), 17 und 3 (unten) und 8 und 11 (links). Die Positionen dieser Punkte im entzerrten Bild sind bekannt, weshalb die Verschiebungen berechnet und die Homographie zur Entzerrung des Bildes abgeleitet werden kann. In der Szene befindet sich ein Objekt, welches aus vier einzelnen Punkten an den Positionen der Orientierungspunkte befindet. Dieses Objekt wird als Maske ausgehend von der finalen Kameraposition und mit den finalen Kameraparametern gerendert. Aus dieser Make lassen sich die Positionen durch Identifizierung von Mittelpunkten in Pixelclustern identifizieren und durch ihre Positionierung zueinander zu den jeweiligen Orientierungspunkten zuordnen.

Diese Art der Identifizierung der Dartscheibenorientierung ist jedoch nicht ideal und zieht Ungenauigkeiten mit sich. Genauer wird auf diese Ungenauigkeiten in der Diskussion in \autoref{sec:diskussion:daten} eingegangen.

% !TEX root = ../main.tex

\section{Ergebnisse}
\label{sec:daten:ergebnisse}

Die Resultate der Datenerstellung sind neben den gerenderten Bildern ebenfalls die normalisierten Bilder sowie die in den Bildern enthaltenen Positionen. Diese Informationen umfassen neben den Positionen der Dartpfeile und ihre erzielten Punktzahlen weitere Metainformationen, die das Bild ausmachen. In einem ersten Schritt werden exemplarische Daten aufgezeigt. Anschließend wird ein Überblick über die Rahmenbedingungen der Datenerstellung gegeben, bevor mit einer qualitativen Auswertung der gerenderten Bilder fortgefahren wird. Darauf folgen Einblicke in die Metainformationen der erstellten Daten. Danach wird auf die Korrektheit der Daten eingegangen und abschließend werden Ungenauigkeiten bei der Erstellung der Daten aufgezeigt.

Eine qualitative Auswertung der Bilddaten ist mangels aussagekräftiger Metriken nicht durchgeführt worden. Eine objektive Bewertung der Realitätsnähe von Bildern ist äußerst komplex und die Aussagekraft nicht eindeutig.

\subsection{Beispiel-Render}  % ===================================================================
\label{sec:render_beispiel}

In \autoref{img:render_examples} sind Bilder aus den für diese Arbeit erstellten Daten dargestellt. Die Auswahl der Bilder erfolgte durch subjektive Selektion exemplarischer Beispiele, die die Variation der Gesamtdaten einfangen. Eine Verzerrung der tatsächlichen Datenlage ist durch das Betrachten lediglich weniger Beispiele und die Art der Selektion nicht auszuschließen, jedoch wurde Acht gegeben, die Auswahl möglichst divers und repräsentativ für die Gesamtheit aller Daten zu halten.

\begin{figure}
    \centering
    \includegraphics[width=0.95\textwidth]{imgs/rendering/ergebnisse/example_renders.pdf}
    \caption{Gerenderte Bilder der Datenerstellung. Große Bilder an den Seiten sind Render-Outputs, kleine quadratische Bilder der mittleren Spalte sind dazugehörige normalisierte Bilder der Render.}
    \label{img:render_examples}
\end{figure}

Dargestellt sind 6 exemplarische Daten, die aus den für diese Thesis erstellten Trainingsdaten stammen. Diese Bilder zeigen die Spanne möglicher Bilder auf, die durch die Datenerstellung möglich sind. Die Variation der Kameraperspektiven ist in diesen Daten zu sehen, welche vollkommen unabhängig voneinander sind, im Vergleich zu den Trainingsdaten des DeepDarts-Systems. Ebenfalls ist eine Variation der Hintergründe und Beleuchtungen offensichtlich. Die Wahl der Environment-Maps, die den Hintergrund maßgeblich bestimmen, üben erheblichen Einfluss auf die Beleuchtung der Dartscheibe aus. Zusätzlich sind in zwei der dargestellten Bilder Ringlichter vorhanden, die darüber hinaus für eine Variation des Hintergrunds und der Beleuchtung sorgen. Die Dartscheiben der unterschiedlichen Bilder unterscheiden sich zusätzlich voneinander, wodurch eine große Spanne möglicher Dartscheiben simuliert werden kann. Hinsichtlich der Dartpfeile sind unterschiedliche Designs und Positionierungen in den Bildern vorhanden. Während in einigen Bildern alle Dartpfeile vorhanden sind, steckt in einem der dargestellten Bilder kein Dartpfeil in der Dartscheibe.

Unterschiedliche Seitenverhältnisse sind aus Gründen der Übersichtlichkeit bewusst nicht dargestellt worden, jedoch existiert eine uniforme Verteilung aller vordefinierter Seitenverhältnisse in den erstellten Daten.

Die entzerrten Bilder befinden sich vertikal entlang der Mitte der Abbildung. Diese weisen allesamt die selbe Art der Entzerrung auf sowie die selben Abmessungen. Auffällig ist die Verzerrung von Dartpfeilen durch die Normalisierung der Dartscheiben: Je geringer der Winkel\footnote{Ein großer Winkel steht für eine frontale Aufnahme der Dartscheibe während ein geringer Winkel eine stark seitliche Aufnahme der Dartscheibe zeigt.} zwischen Dartscheibe und Kamera, desto stärker ist die Verzerrung der Dartpfeile durch den Effekt der Normalisierung. Ebenfalls anzumerken ist das Abschneiden der Dartpfeilenden, sofern sich diese über die Ränder der Dartscheibe hinaus erstrecken.

\subsection{Rahmenbedingungen der Erstellung}  % ==================================================
\label{sec:render_info}

Erstellungszeit ~30s/Sample, Speichernutzung

\todo{}

\subsection{Qualitative Auswertung: Subjektiver Unterschied zwischen generierten und echten Bildern}  % =======
\label{sec:rendering_qualitativ}

Erklärung: Quantitative Auswertung nicht möglich (SSIM wäre eine Metrik, aber die zeigt nicht die Qualität der Render)

\begin{itemize}
    \item augenscheinlich kein Fotorealismus in den Rendern
    \item Unterscheidung zwischen echten und gerenderten Aufnahmen möglich
    \item ...aber nah dran
    \item Gründe dafür finden und aufzählen!
          \begin{itemize}
              \item Shader-Komplexität
              \item Scans von echten Dartscheiben / Erweiterung der prozeduralen Texturen
              \item PBR (Physically-based rendering)
          \end{itemize}
\end{itemize}

\todo{Tabelle textualisieren!}

\subsection{Quantitative Metadatenauswertung}
\label{sec:metadaten}

Eine quantitative Auswertung der Daten hinsichtlich ihres Erscheinungsbilds wurde aus bereits genannten Gründen nicht vollzogen. Jedoch existieren vielerlei Metadaten, die einer Auswertung unterzogen wurden. Dieser Unterabschnitt liefert einen Einblick in die Aufstellung der generierten Daten und die Resultate der in \autoref{sec:daten:methodik} beschriebenen Techniken. In \autoref{sec:kamera_ergebnisse} wird Einblick in ausgewählte intrinsische und extrinsische Parameter der Kamera gegeben, gefolgt von einer Übersicht über die Beleuchtungen und den Objekten, die um die Dartscheibe platziert werden können, sowie die Anzahl der Dartpfeile je Bild in \autoref{sec:beleuchtung_ergebnisse}. Zuletzt wird ein Blick auf die Verteilung der Dartpfeile über die Dartscheibe geworfen, indem die getroffenen Felder in \autoref{sec:felder_ergebnisse} betrachtet werden.

\subsubsection{Kamera-Auswertung}
\label{sec:kamera_ergebnisse}

Die erste quantitative Auswertung dreht sich um ausgewählte und aussagekräftige Kameraparameter. Die Ergebnisse sind in \autoref{img:kamera_meta} in Form von Violinengraphen dargestellt. Die Graphen sind lediglich in ihrer y-Achse beschriftet, da die x-Achse relative Häufigkeiten ausdrückt.

\begin{figure}
    \centering
    \begin{subfigure}[b]{0.475\textwidth}
        \centering
        \includegraphics[width=\textwidth]{imgs/rendering/ergebnisse/cam_angles.pdf}
        \caption{Winkel zur Dartscheibe [$\degree$].}
        \label{fig:cam_angle}
    \end{subfigure}
    \hfill
    \begin{subfigure}{0.475\textwidth}
        \centering
        \includegraphics[width=\textwidth]{imgs/rendering/ergebnisse/cam_dists.pdf}
        \caption{Distanzen zur Dartscheibe [m].}
        \label{fig:cam_dist}
    \end{subfigure}
    \vskip\baselineskip
    \begin{subfigure}{0.475\textwidth}
        \centering
        \includegraphics[width=\textwidth]{imgs/rendering/ergebnisse/cam_focal.pdf}
        \caption{Brennweiten der Kamera [mm].}
        \label{fig:cam_focal}
    \end{subfigure}
    \caption{Verteilungen der Kameraparameter in den Trainingsdaten. Breite der Violinengraphen geben relative Häufigkeiten an, Mittel- und Extremwerte sind dunkelblau gekennzeichnet. \autoref{fig:cam_angle} Kamerawinkel. \autoref{fig:cam_dist} Kameradistanzen. \autoref{fig:cam_focal} Kamerabrennweiten.}
    \label{img:kamera_meta}
\end{figure}

\paragraph{Winkel zur Dartscheibe}

In \autoref{fig:cam_angle} ist der Winkel der Kamera zur Dartscheibe dargestellt. Für die Berechnung der Winkel wurde die Normale der Dartscheibe, die wie die Dartpfeile denkrecht auf der Dartscheibe steht, betrachtet, invertiert und ihr Winkel zur z-Achse der Kamera berechnet. Resultierend ist ein Winkel von $0\degree$ eine Frontalaufnahme der Dartscheibe während ein Winkel von $90\degree$ eine Aufnahme von der Seite der Dartscheibe darstellt. Durch die Berechnung sind ausschließlich positive Werte der Winkel möglich.

Die häufigsten Winkel der Kamera befinden sich um $20\degree$ mit einem mittleren Winkel von $\sim27\degree$ und einem maximalen Winkel von $\sim70\degree$. Auffällig ist das verminderte Vorkommen frontaler Aufnahmen, die durch geringe Winkel charakterisiert sind.

\paragraph{Distanz zur Dartscheibe}

\todo{Distanz}

\paragraph{Brennweite der Kamera}

\todo{Brennweite}

\subsubsection{Beleuchtung und Objekte in der Szene}
\label{sec:beleuchtung_ergebnisse}

\begin{figure}
    \centering
    \begin{subfigure}[b]{0.475\textwidth}
        \centering
        \includegraphics[width=\textwidth]{imgs/rendering/ergebnisse/lights_bar_chart.pdf}
        \caption{Relative Auftrittswahrscheinlichkeiten der Beleuchtungs-Objekte und dem Darts-Schrank.}
        \label{fig:lights}
    \end{subfigure}
    \hfill
    \begin{subfigure}{0.475\textwidth}
        \centering
        \includegraphics[width=\textwidth]{imgs/rendering/ergebnisse/dart_counts.pdf}
        \caption{Relative Anzahl der Dartpfeile je gerendertem Bild.}
        \label{fig:dart_counts}
    \end{subfigure}
    \caption{Informationen zu Beleuchtung und Dartpfeilen. \autoref{fig:lights} zeigt Beleuchtungs-Objekte in der Szene. \autoref{fig:dart_counts} zeigt die Anzahlen der Dartpfeile je Bild.}
    \label{img:light_dart_meta}
\end{figure}

\subsubsection{Getroffene Felder}
\label{sec:felder_ergebnisse}

\begin{figure}
    \centering
    \includegraphics[width=\textwidth]{imgs/rendering/ergebnisse/dartboard_stacked_final.pdf}
    \caption{Relative Verteilung der Dartpfeile je Feld. Links: Felder 1-20; rechts: Bull und Outs.}
    \label{img:dart_verteilung}
\end{figure}

\todo{Grafiken einfügen und beschreiben}

\subsection{Korrekte Annotation der Daten}  % =====================================================
\label{sec:korrekte_annotation}  % Danke, Bruder!

Bei der Erstellung von Daten werden die Positionen der Dartpfeile im 3D-Raum festgelegt und liegen für die Bestimmung der von den Dartpfeilen getroffenen Felder vor. Da Position sowie Transformation der Dartscheibe statisch sind, ist eine Rückrechnung der getroffenen Dartfelder sowie die korrekte Errechnung der erzielten Punktzahl fehlerfrei möglich. Dies siegelt sich ebenfalls in den Daten wider: Die Dartpfeile sind allesamt korrekt annotiert und es existieren durch die in \autoref{sec:dartpfeil_positionierung} erläuterten Methodiken des Umgangs mit ambivalenten Dartpfeilen keine Positionen, die auf mehrere Weisen gedeutet werden können. Anzumerken ist an dieser Stelle jedoch, dass eine Kollision der Dartpfeile mit der Spinne durch ihre Transformation als Resultat der Abnutzungssimulation nicht ausgeschlossen werden kann. Dieser Umstand tritt selten ein, jedoch wurden vereinzelte Daten mit dieser Anomalie identifiziert. Die Auftrittswahrscheinlichkeit dieses Umstands ist jedoch sehr gering und die Korrektheit der Annotation ist dadurch nicht beeinflusst\footnote{Aufgrund der Komplexität einer algorithmischen Identifizierung dieser Kollisionen wurde keine quantitative Auswertung über diesen Umstand vollzogen. Jedoch kann ein struktureller Fehler der Annotation nach manueller Betrachtung einer Vielzahl erstellter Daten ausgeschlossen werden.}.

\subsection{Ungenauigkeiten der Datenerstellung}  % ===============================================
\label{sec:daten_ungenauigkeiten}

Durch die Verwendung von 3D-Modellierung sind alle unterliegenden Informationen der Datenerstellung vorhanden und können zur korrekten Annotation der Daten verwendet werden. Trotz der Informationen über Kameraposition und -parameter sowie Dartpfeilpositionen und Nachverarbeitung geschieht die Bestimmung der Dartscheibenorientierung sowie die Lokalisierung von Dartpfeilpositionen im exportierten Bild nicht durch Berechnungen, sondern durch Nachverarbeitungsschritte. Diese gehen mit einem gewissen Grad Ungenauigkeit einher und sind resultierend nicht $100\%$ akkurat.

Alle Informationen hinsichtlich in- und extrinsischer Kameraparameter, Objektpositionen sowie Nachverarbeitungsschritten liegen während der Erstellung der Daten vor, jedoch ist die Verwendung dieser zur Rückrechnung der Positionen im gerenderten Bild sehr komplex. Stattdessen werden Positionen durch Überschneidungen von Objekten und Rendering dieser als Binärbilder mit den selben Exportparametern exportiert. Durch Nachverarbeitungsschritte wie Clustering werden die dargestellten Positionen approximiert. Dieser Prozess unterliegt einem gewissen Grad der Ungenauigkeit, da mit diskretisierten Werten und Approximationen gearbeitet wird. Das Resultat dieser Erstellung ist eine minimale Variation der Orientierungspunkte hinsichtlich der Ausrichtung der Dartscheibe. Diese Abweichungen befinden sich in der Größenordnung weniger Pixel, jedoch ist diese Ungenauigkeit anzumerken. Eine Beeinträchtigung der Trainingserfolge durch diese Ungenauigkeiten wird jedoch durch die Anwendung von Augmentierung überschattet (vgl. \autoref{sec:daten_augmentierung}).

