% !TEX root = ../main.tex

\section{Ergebnisse}
\label{sec:daten:ergebnisse}

Ergebnisse hier.

\subsection{Beispiel-Render}  % ===================================================================
\label{sec:render_beispiel}

Variationen aufzeigen

\todo{}


\subsection{Rahmenbedingungen der Erstellung}  % ==================================================
\label{sec:render_info}

Erstellungszeit ~30s/Sample, Speichernutzung

\todo{}

\subsection{Qualitative Auswertung: Subjektiver Unterschied zwischen generierten und echten Bildern}  % =======
\label{sec:rendering_qualitativ}

\begin{itemize}
    \item augenscheinlich kein Fotorealismus in den Rendern
    \item Unterscheidung zwischen echten und gerenderten Aufnahmen möglich
    \item ...aber nah dran
    \item Gründe dafür finden und aufzählen!
          \begin{itemize}
              \item Shader-Komplexität
              \item Scans von echten Dartscheiben / Erweiterung der prozeduralen Texturen
              \item PBR (Physically-based rendering)
          \end{itemize}
\end{itemize}

\todo{}

\subsection{Korrekte Annotation der Daten}  % =====================================================
\label{sec:korrekte_annotation}  % Danke, Bruder!

\todo{}

\subsection{Ungenauigkeiten der Datenerstellung}  % ===============================================
\label{sec:daten_ungenauigkeiten}

Grundlage: Masken bei Orientierung -> keine exakten Punkte

- Kameraperspektive + Geometrie-Differenzen

- optimale Entzerrung durch Verzerrungen ggf. auch gar nicht möglich

-> \quotes{korrekte} Daten evtl. nicht 100\% korrekt

\todo{}
