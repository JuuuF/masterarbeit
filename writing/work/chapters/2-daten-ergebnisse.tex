% !TeX root = ../main.tex

\section{Ergebnisse}
\label{sec:daten:ergebnisse}

Die Resultate der Datenerstellung sind neben den gerenderten Bildern ebenfalls die normalisierten Bilder sowie die in den Bildern enthaltenen Positionen. Diese Informationen umfassen neben den Positionen der Dartpfeile und ihre erzielten Punktzahlen weitere Metainformationen, die das Bild ausmachen. In einem ersten Schritt werden exemplarische Daten aufgezeigt. Anschließend wird ein Überblick über die Rahmenbedingungen der Datenerstellung gegeben, bevor mit einer qualitativen Auswertung der gerenderten Bilder fortgefahren wird. Darauf folgen Einblicke in die Metainformationen der erstellten Daten. Danach wird auf die Korrektheit der Daten eingegangen und abschließend werden Ungenauigkeiten bei der Erstellung der Daten aufgezeigt.

Eine qualitative Auswertung der Bilddaten ist mangels aussagekräftiger Metriken nicht durchgeführt worden. Eine objektive Bewertung der Realitätsnähe von Bildern ist äußerst komplex und die Aussagekraft nicht eindeutig.

% =================================================================================================

\subsection{Beispiel-Render}  % ===================================================================
\label{sec:render_beispiel}

In \autoref{img:render_examples} sind Bilder aus den für diese Arbeit erstellten Daten dargestellt. Die Auswahl der Bilder erfolgte durch subjektive Selektion exemplarischer Beispiele, die die Variation der Gesamtdaten einfangen. Eine Verzerrung der tatsächlichen Datenlage ist durch das Betrachten lediglich weniger Beispiele und die Art der Selektion nicht auszuschließen, jedoch wurde Acht gegeben, die Auswahl möglichst divers und repräsentativ für die Gesamtheit aller Daten zu halten.

\begin{figure}
    \centering
    \includegraphics[width=0.95\textwidth]{imgs/rendering/ergebnisse/example_renders.pdf}
    \caption{Gerenderte Bilder der Datenerstellung. Große Bilder an den Seiten sind Render-Outputs, kleine quadratische Bilder der mittleren Spalte sind dazugehörige normalisierte Bilder der Render.}
    \label{img:render_examples}
\end{figure}

Dargestellt sind 6 exemplarische Daten, die aus den für diese Thesis erstellten Trainingsdaten stammen. Diese Bilder zeigen die Spanne möglicher Bilder auf, die durch die Datenerstellung möglich sind. Die Variation der Kameraperspektiven ist in diesen Daten zu sehen, welche vollkommen unabhängig voneinander sind, im Vergleich zu den Trainingsdaten des DeepDarts-Systems. Ebenfalls ist eine Variation der Hintergründe und Beleuchtungen offensichtlich. Die Wahl der Environment Maps, die den Hintergrund maßgeblich bestimmen, üben erheblichen Einfluss auf die Beleuchtung der Dartscheibe aus. Zusätzlich sind in zwei der dargestellten Bilder Ringlichter vorhanden, die darüber hinaus für eine Variation des Hintergrunds und der Beleuchtung sorgen. Die Dartscheiben der unterschiedlichen Bilder unterscheiden sich zusätzlich voneinander, wodurch eine große Spanne möglicher Dartscheiben simuliert werden kann. Hinsichtlich der Dartpfeile sind unterschiedliche Designs und Positionierungen in den Bildern vorhanden. Während in einigen Bildern alle Dartpfeile vorhanden sind, steckt in einem der dargestellten Bilder kein Dartpfeil in der Dartscheibe.

Unterschiedliche Seitenverhältnisse sind aus Gründen der Übersichtlichkeit bewusst nicht dargestellt worden, jedoch existiert eine uniforme Verteilung aller vordefinierter Seitenverhältnisse in den erstellten Daten.

Die entzerrten Bilder befinden sich vertikal entlang der Mitte der Abbildung. Diese weisen allesamt die selbe Art der Entzerrung auf sowie die selben Abmessungen. Auffällig ist die Verzerrung von Dartpfeilen durch die Normalisierung der Dartscheiben: Je geringer der Winkel\footnote{Ein großer Winkel steht für eine frontale Aufnahme der Dartscheibe während ein geringer Winkel eine stark seitliche Aufnahme der Dartscheibe zeigt.} zwischen Dartscheibe und Kamera, desto stärker ist die Verzerrung der Dartpfeile durch den Effekt der Normalisierung. Ebenfalls anzumerken ist das Abschneiden der Shafts und Tips, sofern sich diese über die Ränder der Dartscheibe hinaus erstrecken.

% =================================================================================================

\subsection{Rahmenbedingungen der Erstellung}  % ==================================================
\label{sec:render_info}

Die Datenerstellung wurde parallel auf mehreren Rechnern mit unterschiedlichen Grafikkarten ausgeführt. Als Mittelwert aller Daten wurde eine Erstellungszeit von 30 Sekunden je Sample ermittelt. Die Erstellung eines Samples beinhaltet das Einlesen der Szene, Setzen von Objekten und Parametern, Rendern des Output-Bildes sowie den Binärmasken und die Nachverarbeitung der Daten zur Anreicherung der Metadaten und Normalisierung für das Training neuronaler Netze. Insgesamt wurden $24.576$ Trainingsdaten, 256 Validierungsdaten und $2.048$ Testdaten erstellt. Die Speichernutzung je Sample beträgt durchschnittlich 8\,MB, wobei die Datenmenge durch Entfernen von Masken, die nach der Erstellung der Daten nicht mehr verwendet werden, weiter reduziert werden kann. Insgesamt wurden für diese Arbeit Daten mit einem Umfang von etwa 216\,GB erstellt. Das Blender-Projekt beläuft sich auf eine Größe von $220\,\text{MB}$, zu denen weitere $1,8\,\text{GB}$ durch 208 Environment Maps hinzukommen, die extern dynamisch in die Blender-Szene eingebunden werden.

\subsection[Qualitative Auswertung]{Qualitative Auswertung: Subjektiver Unterschied zwischen generierten und echten Bildern}  % =======
\label{sec:rendering_qualitativ}

Da eine quantitative Auswertung im Bezug auf das Erscheinungsbild der gerenderten Bilder schwer umsetzbar ist, wird stattdessen eine qualitative Betrachtung der Daten durchgeführt. Dabei wurde auf Merkmale geachtet, die eine Unterscheidung echter und gerenderter Bilder ermöglichen. Die in \autoref{img:render_examples} dargestellten Render von Dartscheiben können als Referenz erstellter Bilder herangezogen werden, indem sie unter Vorbehalt von Verzerrungen einen Querschnitt möglicher generierter Daten darstellen.

Augenscheinlich ist den meisten Bildern eindeutig anzuerkennen, dass diese synthetisch erstellt wurden und keine realen Dartscheiben abbilden. Ein zentraler Punkt dieser Beobachtung ist die Umgebung der Dartscheibe. Es wurde sich bewusst gegen die Einbindung eines statischen Hintergrundes entschieden, um die unmittelbare Umgebung der Dartscheibe variabel zu halten. Diese Design-Entscheidung schlägt sich hingegen darin nieder, dass die Dartscheiben in Situationen dargestellt sind, die in realen Aufnahmen nicht existieren. Die Dartscheiben schweben in unüblichen Szenarien im Raum ohne sichtliche Befestigung. Für den menschlichen Betrachter ist dieser Umstand sehr auffällig und sorgt für eine direkte Identifikation als gerendertes Bild. Für den Einsatzbereich des Trainings und der Entzerrung ist dieser Aspekt jedoch nicht von Relevanz. Die Semantik eines möglicherweise fehlenden thematischen Kontexts um die Dartscheibe liegt nicht im Interessenbereich der algorithmischen Normalisierung sowie des Trainings eines neuronalen Netzes. Für diese Systeme ist lediglich die Existenz variierender Informationen relevant, die strukturelle Ähnlichkeiten zu realen Daten aufweist. In Bildern, in denen unmittelbare Hintergründe der Dartscheibe beispielsweise durch Einblendung eines Dartschranks gegeben sind, wirken für den Betrachter erheblich realistischer. Beispiele von Bildern mit Hintergrundobjekten sind in \autoref{img:render_examples} in der rechten Spalte dargestellt. Insbesondere das unterste Bild wirkt durch die Existenz des Dartschranks weitaus realistischer als Bilder der linken Spalte.

Insgesamt ist jedoch trotz der erwähnten Schwachpunkte nach subjektivem Empfinden eine realistische Darstellung von Dartscheiben gelungen. Die Dartscheiben weisen realistische Gebrauchsspuren wie Risse, Verfärbungen, Einstichlöcher und Verformungen plastischer Elemente auf, die sich durch prozedurale Generierung in jeder Dartscheibe unterscheiden. Ebenfalls ist eine realistische Simulation möglicher Kameraparameter und -positionen abgedeckt, die eine zu erwartende Spanne von Parametern in echten Aufnahmen abdecken. Die Verteilung der Dartpfeile sowie ihre Orientierung bei Eintreffen auf der Dartscheibe wirken ebenfalls realistisch und kohärent zueinander.

% =================================================================================================

\subsection{Quantitative Metadatenauswertung} % ===================================================
\label{sec:metadaten}

Eine quantitative Auswertung der Daten hinsichtlich ihres Erscheinungsbilds wurde aus bereits genannten Gründen nicht vollzogen. Jedoch existieren vielerlei Metadaten, die einer Auswertung unterzogen wurden. Dieser Unterabschnitt liefert einen Einblick in die Aufstellung der generierten Daten und die Resultate der in \autoref{sec:daten:methodik} beschriebenen Techniken. In \autoref{sec:kamera_ergebnisse} wird Einblick in ausgewählte intrinsische und extrinsische Parameter der Kamera gegeben, gefolgt von einer Übersicht über die Beleuchtungen und den Objekten, die um die Dartscheibe platziert werden können, sowie die Anzahl der Dartpfeile je Bild in \autoref{sec:beleuchtung_ergebnisse}. Zuletzt wird ein Blick auf die Verteilung der Dartpfeile über die Dartscheibe geworfen, indem die getroffenen Felder in \autoref{sec:felder_ergebnisse} betrachtet werden.

\subsubsection{Kamera-Auswertung}
\label{sec:kamera_ergebnisse}

Die erste quantitative Auswertung dreht sich um ausgewählte und aussagekräftige Kameraparameter. Die Ergebnisse sind in \autoref{img:kamera_meta} in Form von Violinengraphen dargestellt. Die Graphen sind lediglich in ihrer y-Achse beschriftet, da die x-Achse relative Häufigkeiten ausdrückt.

\begin{figure}
    \centering
    \begin{subfigure}[b]{0.475\textwidth}
        \centering
        \includegraphics[width=\textwidth]{imgs/rendering/ergebnisse/cam_angles.pdf}
        \caption{Winkel zur Dartscheibe [$\degree$].}
        \label{fig:cam_angle}
    \end{subfigure}
    \hfill
    \begin{subfigure}{0.475\textwidth}
        \centering
        \includegraphics[width=\textwidth]{imgs/rendering/ergebnisse/cam_dists.pdf}
        \caption{Distanzen zur Dartscheibe [m].}
        \label{fig:cam_dist}
    \end{subfigure}
    \par
    \begin{subfigure}{0.475\textwidth}
        \centering
        \includegraphics[width=\textwidth]{imgs/rendering/ergebnisse/cam_focal.pdf}
        \caption{Brennweiten der Kamera [mm].}
        \label{fig:cam_focal}
    \end{subfigure}
    \caption{Verteilungen der Kameraparameter in den Trainingsdaten. Breite der Violinengraphen geben relative Häufigkeiten an, Mittel- und Extremwerte sind dunkelblau gekennzeichnet. \autoref{fig:cam_angle} Kamerawinkel. \autoref{fig:cam_dist} Kameradistanzen. \autoref{fig:cam_focal} Kamerabrennweiten.}
    \label{img:kamera_meta}
\end{figure}

\paragraph{Winkel zur Dartscheibe}

In \autoref{fig:cam_angle} ist der Winkel der Kamera zur Dartscheibe dargestellt. Für die Berechnung der Winkel wurde die Normale der Dartscheibe, die wie die Dartpfeile senkrecht auf der Dartscheibe steht, betrachtet, invertiert und ihr Winkel zur z-Achse der Kamera berechnet. Resultierend ist ein Winkel von $0\degree$ eine Frontalaufnahme der Dartscheibe, während ein Winkel von $90\degree$ eine Aufnahme von der Seite der Dartscheibe darstellt. Durch die Berechnung sind ausschließlich positive Werte der Winkel möglich.

Die häufigsten Winkel der Kamera befinden sich um $20\degree$ mit einem mittleren Winkel von $\approx27\degree$ und einem maximalen Winkel von $\approx70\degree$. Auffällig ist das verminderte Vorkommen frontaler Aufnahmen, die durch geringe Winkel charakterisiert sind.

\paragraph{Distanz zur Dartscheibe}

Bei der Erstellung der Daten wird die Kamera zufällig in einem vordefinierten Kamerabereich platziert. Die resultierenden Distanzen der Kamera in den Daten sind in \autoref{fig:cam_dist} dargestellt. Es ist zu erkennen, dass ein geringes Aufkommen kurzer Distanzen unter $50\,\text{cm}$ zu verzeichnen ist, da der Bereich, in dem sich die Kamera für derartige Distanzen befinden muss, verhältnismäßig klein ist. Die mittlere Distanz der Kamera beträgt $\approx1\,\text{m}$ und die maximale Distanz $\approx160\,\text{cm}$. Diese Verteilungen spiegeln nach subjektivem Empfinden Verteilungen wider, die in realen Szenarien zu erwarten sind. Geringere Aufnahmen sowie Aufnahmen mit größerer Entfernung als die verzeichneten wurden bereits bei der Parametrisierung des Kamerabereichs als unwahrscheinlich eingestuft und kategorisch ausgeschlossen, da die Dartscheibe in den Fällen unter Umständen nicht in einem Maße zu identifizieren ist, in welchem eine sinngemäße Verarbeitung der Daten möglich ist.

\paragraph{Brennweite der Kamera}

Die Brennweiten der Kamera richten sich nach der in \autoref{fig:brennweiten} dargestellten Verteilung auf Grundlage der Kameradistanzen zur Dartscheibe. Die resultierenden Brennweiten in \autoref{fig:cam_focal} sind im Einklang mit einer zu erwartenden Verteilung indes sie approximativ die Breite der möglichen Distanzen je Brennweite darstellen, welche bei einer uniform verteilten Kameradistanz zu erwarten wäre. Ein Zuschnüren der geringwertigen Brennweiten durch die Unterrepräsentation geringer Distanzen, wie im vorherigen Abschnitt identifiziert, ist zu erkennen. Die minimale Brennweite in den Daten beträgt $\approx10\,\text{mm}$, die maximale Brennweite $\approx60\,\text{mm}$. Die mittlere Brennweite der Daten beträgt $\approx38\,\text{mm}$ und ist damit wertgleich zu dem oberen Brennweiten-Grenzwert minimaler Abstände sowie zu der unteren Grenze maximaler Abstände zur Dartscheibe.

\subsubsection{Beleuchtung und Objekte in der Szene}
\label{sec:beleuchtung_ergebnisse}

\begin{figure}
    \centering
    \begin{subfigure}[b]{0.475\textwidth}
        \centering
        \includegraphics[width=\textwidth]{imgs/rendering/ergebnisse/lights_bar_chart.pdf}
        \caption{Relative Auftrittswahrscheinlichkeiten der Beleuchtungs-Objekte und dem Darts-Schrank.}
        \label{fig:lights}
    \end{subfigure}
    \hfill
    \begin{subfigure}{0.475\textwidth}
        \centering
        \includegraphics[width=\textwidth]{imgs/rendering/ergebnisse/dart_counts.pdf}
        \caption{Relative Anzahl der Dartpfeile je gerendertem Bild.}
        \label{fig:dart_counts}
    \end{subfigure}
    \caption{Informationen zu Beleuchtung und Dartpfeilen. \autoref{fig:lights} zeigt Beleuchtungs-Objekte in der Szene. \autoref{fig:dart_counts} zeigt die Anzahlen der Dartpfeile je Bild.}
    \label{img:light_dart_meta}
\end{figure}

In der Szene werden Objekte zufällig ein- und ausgeblendet, basierend auf unterschiedlichen Wahrscheinlichkeitsverteilung und Umgebungsbedingungen. Diese in den Daten aufgetretenen dynamischen Objekte sind in \autoref{fig:lights} dargestellt. Hinsichtlich der Beleuchtungsobjekte ist das Deckenlicht mit $\approx67\,\%$ am häufigsten vertreten. Diese Beobachtung beruht darauf, dass die Deckenleuchten als Rückfallbeleuchtung verwendet werden, sofern keine andere Lichtquelle vorhanden ist. Der Kamerablitz und das Spotlight sind in $\approx40\,\%$ der Daten vorhanden. Ringlicht und Dartschrank sind Objekte, die sich gegenseitig ausschließen und daher nicht zusammen auftreten können. Ihre Existenzen sind daher miteinander gekoppelt. Dass diese wesentlich geringer vertreten sind als andere dynamische Objekte ist ebenfalls durch ihre Wahrscheinlichkeitsverteilungen bedingt. Ihre Existenz sorgt für wenig diverse Umgebungen um die Dartscheiben und wurde daher gering gehalten. Mit $\approx15\,\%$ Vorkommen des Ringlichts und $\approx5\,\%$ Vorkommen des Dartschranks sind ihre Existenzen von allen dynamischen Objekten am seltensten.

Neben den dynamischen Objekten wurde ebenfalls die Anzahl an Dartpfeilen in \autoref{fig:dart_counts} ausgewertet. Die Existenz jedes Dartpfeils wird während der Platzierung der Dartpfeile bestimmt und unterliegt ebenfalls einer vordefinierten Wahrscheinlichkeit. In $\approx69\,\%$ der Daten sind alle 3 Dartpfeile vorhanden, in $\approx28\,\%$ sind 2 Dartpfeile auf der Dartscheibe verteilt und in $\approx3\,\%$ aller Trainingsdaten ist lediglich 1 Dartpfeil vorhanden. Es wurden keine Bilder ohne Dartpfeile erstellt. Die Existenz von Dartpfeilen geht nicht automatisch mit einer Punktzahl $>0$ einher, da Pfeile sowohl auf der Dartscheibe als auch außerhalb platziert werden können. Die Existenz bezieht sich lediglich auf das Vorhandensein von Dartpfeilen in den gerenderten Bildern.

\subsubsection{Getroffene Felder}
\label{sec:felder_ergebnisse}

\begin{figure}
    \centering
    \includegraphics[width=\textwidth]{imgs/rendering/ergebnisse/dartboard_stacked_final.pdf}
    \caption{Relative Verteilung der Dartpfeile je Feld. Links: Felder 1-20; rechts: Bull und Outs.}
    \label{img:dart_verteilung}
\end{figure}

Zur Platzierung der Dartpfeile wurde eine Heatmap verwendet, die zu erwartenden realistischen Verteilungen der Dartpfeile auf der Dartscheibe nachempfunden ist. Zusätzlich wurden spezifisch Daten erstellt, in denen die Pfeile ausschließlich auf die Bereiche der Double- und Triple-Felder sowie dem Bull fokussiert sind. In \autoref{img:dart_verteilung} ist eine Auswertung der getroffenen Felder aller Pfeile in den Trainingsdaten visualisiert. Die Grafik teilt sich auf in die Felder 1-20 (links) und Bull und Fehlwürfe (rechts). Die Verteilung der Felder 1-20 entsprechen einer zu erwartenden Verteilung, in der Felder geringer Werte seltener getroffen wurden als Felder mit hohen Zahlenwerten. Die Felder 20, 14 und 19 wurden am häufigsten getroffen, was typische Spielweisen widerspiegelt\footnote{Das Bespielen der 14 zieht bei Verfehlen höhere Zahlenwerte mit sich als das Verfehlen von 19 oder 20. Daher präferieren einige Spieler die 14 über 19 oder 20.}.

Das Bull ist in den Daten verhältnismäßig gering vertreten in statistisch einem in 200 Bildern. Übermäßig häufig hingegen ist das Verfehlen der Dartscheibe vertreten mit etwa jedem vierten Dartpfeil. Der Bereich des Bulls und der Außenfläche der Dartscheibe sind jedoch analog sehr klein bzw. sehr groß im Verhältnis zu regulären Feldern. Daher sind diese Verteilungen hinsichtlich der Feldflächen nicht ungewöhnlich.

% =================================================================================================

\subsection{Korrekte Annotation der Daten}  % =====================================================
\label{sec:korrekte_annotation}  % Danke, Bruder!

Bei der Erstellung von Daten werden die Positionen der Dartpfeile im 3D-Raum festgelegt und liegen für die Bestimmung der von den Dartpfeilen getroffenen Felder vor. Da Position sowie Transformation der Dartscheibe statisch sind, ist eine Rückrechnung der getroffenen Dartfelder sowie die korrekte Errechnung der erzielten Punktzahl fehlerfrei möglich. Dies spiegelt sich ebenfalls in den Daten wider: Die Dartpfeile sind allesamt korrekt annotiert und es existieren durch die in \autoref{sec:dartpfeil_positionierung} erläuterten Methodiken des Umgangs mit ambivalenten Dartpfeilen keine Positionen, die auf mehrere Weisen gedeutet werden können. Anzumerken ist an dieser Stelle jedoch, dass eine Kollision der Dartpfeile mit der Spinne durch ihre Transformation als Resultat der Abnutzungssimulation nicht ausgeschlossen werden kann. Dieser Umstand tritt selten ein, jedoch wurden vereinzelte Daten mit dieser Anomalie identifiziert. Die Auftrittswahrscheinlichkeit dieses Umstands ist jedoch sehr gering und die Korrektheit der Annotation ist dadurch nicht beeinflusst\footnote{Aufgrund der Komplexität einer algorithmischen Identifizierung dieser Kollisionen wurde keine quantitative Auswertung über diesen Umstand vollzogen. Jedoch kann ein struktureller Fehler der Annotation nach manueller Betrachtung einer Vielzahl erstellter Daten ausgeschlossen werden.}.

% =================================================================================================

\subsection{Ungenauigkeiten der Datenerstellung}  % ===============================================
\label{sec:daten_ungenauigkeiten}

Durch die Verwendung von 3D-Modellierung sind alle unterliegenden Informationen der Datenerstellung vorhanden und können zur korrekten Annotation der Daten verwendet werden. Trotz der Informationen über Kameraposition und -parameter sowie Dartpfeilpositionen und Nachverarbeitung geschieht die Bestimmung der Dartscheibenorientierung sowie die Lokalisierung von Dartpfeilpositionen im exportierten Bild nicht durch Berechnungen, sondern durch Nachverarbeitungsschritte. Diese gehen mit einem gewissen Grad Ungenauigkeit einher und sind resultierend nicht $100\,\%$ akkurat.

Alle Informationen hinsichtlich in- und extrinsischer Kameraparameter, Objektpositionen sowie Nachverarbeitungsschritten liegen während der Erstellung der Daten vor, jedoch ist die Verwendung dieser zur Rückrechnung der Positionen im gerenderten Bild sehr komplex. Stattdessen werden Positionen durch Überschneidungen von Objekten und Rendering dieser als Binärbilder mit den selben Exportparametern exportiert. Durch Nachverarbeitungsschritte wie Clustering werden die dargestellten Positionen approximiert. Dieser Prozess unterliegt einem gewissen Grad der Ungenauigkeit, da mit diskretisierten Werten und Approximationen gearbeitet wird. Das Resultat dieser Erstellung ist eine minimale Variation der Orientierungspunkte hinsichtlich der Ausrichtung der Dartscheibe. Diese Abweichungen befinden sich in der Größenordnung weniger Pixel, jedoch ist diese Ungenauigkeit anzumerken. Eine Beeinträchtigung der Trainingserfolge durch diese Ungenauigkeiten wird jedoch durch die Anwendung von Augmentierung überschattet (vgl. \autoref{sec:daten_augmentierung}).
