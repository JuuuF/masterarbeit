% !TEX root = ../main.tex

\section{Ergebnisse}
\label{sec:daten:ergebnisse}

Ergebnisse hier.

\subsection{Beispiel-Render}  % ===================================================================
\label{sec:render_beispiel}

Bevor zu der Auswertung der Ergebnisse der Datengenerierung übergegangen wird, wird ein Überblick über die erstellten Daten gegeben. In diesem Unterkapitel werden daher die Ergebnisse dargestellt und ihre Eigenschaften werden aufgezeigt.

In \autoref{img:render_examples} sind Bilder aus den für diese Arbeit erstellten Daten dargestellt. Die Auswahl der Bilder erfolgte durch subjektive Selektion exemplarischer Beispiele, die die Variation der Gesamtdaten einfangen. Eine Verzerrung der tatsächlichen Datenlage ist durch das Betrachten lediglich weniger Beispiele und die Art der Selektion nicht auszuschließen, jedoch wurde Acht gegeben, die Auswahl möglichst divers zu halten.

\begin{figure}
    \centering
    \includegraphics[width=0.95\textwidth]{imgs/rendering/ergebnisse/example_renders.pdf}
    \caption{Gerenderte Bilder der Datenerstellung. Große Bilder an den Seiten sind Render-Outputs, kleine quadratische Bilder der mittleren Spalte sind dazugehörige normalisierte Bilder der Render.}
    \label{img:render_examples}
\end{figure}

Die Variationen der Bilder

\todo{entzerrte Bilder datrstellen}


\subsection{Rahmenbedingungen der Erstellung}  % ==================================================
\label{sec:render_info}

Erstellungszeit ~30s/Sample, Speichernutzung

\todo{}

\subsection{Qualitative Auswertung: Subjektiver Unterschied zwischen generierten und echten Bildern}  % =======
\label{sec:rendering_qualitativ}

Erklärung: Quantitative Auswertung nicht möglich (SSIM wäre eine Metrik, aber die zeigt nicht die Qualität der Render)

\begin{itemize}
    \item augenscheinlich kein Fotorealismus in den Rendern
    \item Unterscheidung zwischen echten und gerenderten Aufnahmen möglich
    \item ...aber nah dran
    \item Gründe dafür finden und aufzählen!
          \begin{itemize}
              \item Shader-Komplexität
              \item Scans von echten Dartscheiben / Erweiterung der prozeduralen Texturen
              \item PBR (Physically-based rendering)
          \end{itemize}
\end{itemize}

\todo{}

\subsection{Korrekte Annotation der Daten}  % =====================================================
\label{sec:korrekte_annotation}  % Danke, Bruder!

\todo{}

\subsection{Ungenauigkeiten der Datenerstellung}  % ===============================================
\label{sec:daten_ungenauigkeiten}

Grundlage: Masken bei Orientierung -> keine exakten Punkte

- Kameraperspektive + Geometrie-Differenzen

- optimale Entzerrung durch Verzerrungen ggf. auch gar nicht möglich

-> \quotes{korrekte} Daten evtl. nicht 100\% korrekt

\todo{}
