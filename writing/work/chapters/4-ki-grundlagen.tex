% !TEX root = ../main.tex

\section{Grundlagen}
\label{sec:ki:grundlagen}

Grundlagen hier.

% -------------------------------------------------------------------------------------------------
\subsection{Neuronale Netze}
\label{sec:neuronale_letze}

Kernbaustein dieser Sektion des Projekts sind neuronale Netze, häufig auch als \ac{KI} bezeichnet. In diesem Unterabschnitt wird auf die grundlegende Arbeitsweise neuronaler Netze und ihre Einsatzbereiche eingegangen.

\subsubsection{Multi-Layer Perzeptron}
\label{sec:multi_layer_perceptron}

Der Grundbaustein aller neuronaler Netze ist das Perzeptron. Ein Perzeptron besitzt eine Liste an Inputs, deren gewichtete Summe durch eine Aktivierungsfunktion zu einem Output-Wert transformiert wird. Dieser Aufbau ist den Neuronen in Gehirnen nachempfunden und ist intern einem Gewicht für jeden Input und einem Bias-Term, der zu den gewichteten Gewichten addiert wird, parametrisiert. Diese Parameter sind für den Output des Perzeptrons ausschlaggebend. Mathematisch ist ein Perzeptron definiert durch:

\[ o = f \left(\sum_{i=0}^{N} w_i x_i + b \right) \]

Durch Kombination mehrerer Perzeptrons ist die 

\todo{Quellen finden}
\todo{Neuronale Netze weiter beschreiben}

\subsubsection{Convolutional Neural Networks}
\label{sec:cnn}

\todo{CNNs beschreiben}

\subsubsection{Klassifizierung und Regression}
\label{sec:klassifizierung_regression}

\todo{Klassifizierung und Regression beschreiben}

\subsubsection{Training und Backpropagation}
\label{sec:training_backpropagation}

\todo{Training und Backpropagation beschreiben}

\subsubsection{Loss-Funktionen}
\label{sec:loss_funktionen}

\todo{Loss-Funktionen beschreiben}

% -------------------------------------------------------------------------------------------------
\subsection{Terminologie}
\label{sec:terminologie}

\paragraph{Trainingsdaten}
\paragraph{Validierungsdaten}
\paragraph{Testdaten}
\paragraph{Out-of-distribution-Training}
\paragraph{Overfitting}
\paragraph{Underfitting}

% -------------------------------------------------------------------------------------------------
\subsection{Augmentierung}
\label{sec:augmentierung}

\todo{Augmentierung beschreiben}

% -------------------------------------------------------------------------------------------------
\subsection{YOLOv8}
\label{sec:yolov8}

\todo{YOLOv8 beschreiben}

% -------------------------------------------------------------------------------------------------
\subsection{Oversampling / Klassenungleichgewicht}
\label{sec:oversampling}

\todo{Oversampling beschreiben}

