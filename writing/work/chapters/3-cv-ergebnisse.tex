% !TeX root = ../main.tex

\section{Ergebnisse}
\label{sec:cv:ergebnisse}

In diesem Abschnitt werden die Ergebnisse der algorithmischen Normalisierung von Dartscheiben aufgezeigt. Dazu werden die verwendeten Metriken in \autoref{sec:cv_metriken} vorgestellt. Anschließend werden die Datenquellen in \autoref{sec:cv_ergebnisse_daten} aufgezeigt, anhand derer die Auswertung stattfindet. In \autoref{sec:cv_quantitative_auswertung} wird eine quantitative Auswertung vorgenommen. Es werden die Ergebnisse, die mit dem in dieser Arbeit vorgestellten Algorithmus erzielt werden konnten, hinsichtlich der aufgezeigten Metriken sowie weiteren Merkmalen analysiert und mit den durch DeepDarts auf denselben Daten erzielten Ergebnissen verglichen.

% -------------------------------------------------------------------------------------------------

\vfill
\subsection{Metriken}
\label{sec:cv_metriken}

Für die Messung der Entzerrungsgenauigkeit des entwickelten Algorithmus werden zwei Metriken verwendet. Die erste Metrik $\chi$ misst die Fähigkeit eines Systems, eine Homographie zur Entzerrung einer Dartscheibe zu ermitteln, unabhängig von ihrer Genauigkeit:

\begin{equation*}
    \chi(S, I) =
    \begin{cases}
        1, & \text{wenn System $S$ zu Bild $I$ eine Homographie ermitteln kann} \\
        0  & \text{sonst}
    \end{cases}
\end{equation*}

Die zweite Metrik $\Lambda(\widetilde{H}, \widehat{H})$ bestimmt die Genauigkeit der ermittelten Entzerrungs-Homographie $\widehat{H}$, gegeben einer Ziel-Homographie $\widetilde{H}$. Da ein trivialer Vergleich der Zahlenwerte der ermittelten Homographien wenig Aufschluss über die konkrete Genauigkeit der Entzerrung liefert, ist eine komplexere Metrik notwendig. Für die verwendete Metrik $\Lambda$ werden $N_\text{OP, max}=61$ unterschiedliche Orientierungspunkte verwendet. Diese befinden sich entlang der Feldlinien radial verteilt in den Ringen der äußeren Bulls, des äußeren Triple-Rings und des äußeren Double-Rings. Zusätzlich ist der Mittelpunkt als weiterer Orientierungspunkt mit aufgenommen. Die Positionen $P_{i \in [N_\text{OP, max}]}$ aller Orientierungspunkte im Zielbild sind durch die Definition der Entzerrung festgelegt. Diese Punkte werden durch die inverse Ziel-Homographie an ihre Ursprungspositionen $\widetilde{P}_i = \mathrm{inv}(\widetilde{H}) \times P_i$ transformiert und von dort durch die ermittelte Homographie zu den vorhergesagten Zielpositionen $\widehat{P}_i = \widehat{H} \times \widetilde{P}_i$ rücktransformiert. Der Wert der Metrik ist definiert durch:
\nomenclature{$\Lambda: (\left(\mathbb{R}^{3 \times 3}\right)^2 \mapsto \mathbb{R})$}{\ac{cv}-Metrik zur Identifizierung der Ähnlichkeiten von Homographien}
\[ \Lambda(\widetilde{H}, \widehat{H}) = \frac{1}{N_\text{OP, max}} \sum_{i = 1}^{N_\text{OP, max}} \left\lVert P_i - \widehat{H} \times \mathrm{inv}(\widetilde{H}) \times P_i \right\rVert _2  \]
\nomenclature{$\widehat{H} \in \mathbb{R}^{3 \times 3}$}{Ermittelte Entzerrungstransformation}
\nomenclature{$N_\text{OP, max}$}{Maximale Anzahl der Orientierungspunkte in einem Bild}
\nomenclature{$P_{i \in [N_\text{OP}]} \in \mathbb{R}^2$}{Positionen der identifizierten Orientierungspunkte}
Durch diese Metrik ist eine Quantifizierung der Ähnlichkeit zweier Homographien zur Entzerrung einer Dartscheibe möglich. Zu sehen ist bei dieser Definition, dass $\Lambda(\widetilde{H}, \widehat{H}) = 0\,\text{px}$, wenn $\widetilde{H} = \widehat{H}$, da $\widehat{H} \times \mathrm{inv}(\widetilde{H}) = \text{Id}$, die Identitätsmatrix, ist.
\nomenclature{$\text{Id} \in \mathbb{R}^{3 \times 3}$}{Identitäts-Transformation}

% -------------------------------------------------------------------------------------------------

\vfill
\subsection{Verwendete Daten}
\label{sec:cv_ergebnisse_daten}

Die Daten dieser Auswertung stammen aus fünf unterschiedlichen Quellen und werden strikt voneinander getrennt gehalten. Dies dient der Identifizierung von einerseits Verzerrungen auf Daten und andererseits Schwachstellen in dem getesteten System. Die Datenquellen sind in \autoref{tab:datenquellen_cv} aufgelistet und stellen sich zusammen aus synthetischen Daten sowie Daten des DeepDarts-Systems.

Die Daten beinhalten lediglich positive Datensätze, in welchen ausschließlich Bilder vorhanden sind, die eine Dartscheibe abbilden. Da die Aufgabe des Systems nicht die Identifizierung von Dartscheiben ist, sondern die Entzerrung dieser, kann die Existenz von Dartscheiben in den Bildern vorausgesetzt werden. Die Ausführung des Algorithmus auf Bildern ohne Dartscheibe ist nicht von Relevanz.

\begin{table}
    \centering
    \begin{tabular}{r||c|cc|cc}
        \multirow{2}{*}{Datenquelle} & \multirow{2}{*}{\begin{tabular}[c]{@{}c@{}}Gerenderte\\ Bilder\end{tabular}} & \multicolumn{2}{c|}{DeepDarts-$d_1$} & \multicolumn{2}{c}{DeepDarts-$d_2$}                        \\
                                     &                                                                              & Validierung                          & Test                                & Validierung & Test   \\ \hline
        Anzahl Bilder                & 2.048                                                                        & 1.000                                & 2.000                               & 70          & 150    \\
        Automatische Annotation      & \cmark                                                                       & \xmark                               & \xmark                              & \xmark      & \xmark
    \end{tabular}
    \caption{Datenquellen für die Auswertung der Dartscheibenentzerrungen.}
    \label{tab:datenquellen_cv}
\end{table}

% -------------------------------------------------------------------------------------------------

\vfill
\newpage
\subsection{Quantitative Auswertung}
\label{sec:cv_quantitative_auswertung}

Die quantitative Auswertung ist unterteilt in die Auswertung der Geschwindigkeit, der zuvor beschriebenen Metriken und einer zusammengefassten Auswertung. Die Aufteilung der Ergebnisse in Render-Daten und DeepDarts-Daten ergibt sich aus den Unterschieden der Daten. Während die DeepDarts-Daten derart vorverarbeitet sind, dass sie die Dartscheibe weitestgehend zentriert in Bildern fester Dimensionen zeigt, sind die synthetischen Daten wesentlich variabler hinsichtlich der Darstellung der Dartscheiben.

Die Auswertungen sind jeweils unterteilt in das in dieser Thesis erarbeitete System und die DeepDarts-Systeme DeepDarts-$d_1$ und DeepDarts-$d_2$, um die Unterschiede der Genauigkeiten darzustellen. Da die DeepDarts-Systeme Single-Shot-Neural-Networks sind, in welchem die Normalisierung und die Lokalisierung der Dartpfeile nicht voneinander getrennt betrachtet werden können, wird die Geschwindigkeit der Normalisierung gleichgesetzt mit der Gesamtdauer der Vorhersage.

Ausgeführt wurde die Auswertung auf einer AMD Ryzen 3 3100 CPU. Es wurde trotz der Möglichkeit einer GPU-Ausführung der DeepDarts-Systeme keine Grafikkarte verwendet, um die Vergleichbarkeit der Systeme unter der Verwendung derselben Hardware zu gewährleisten. Aufgrund des Nichtdeterminismus des Algorithmus durch die Verwendung von RANSAC wurden fünf Durchläufe der Auswertung vorgenommen, deren Mittelwerte zur Evaluation verwendet wird.

\subsubsection{Geschwindigkeit der Vorhersagen} % -------------------------------------------------

\pgfplotstableread[col sep=comma]{
    system,       Render-Daten, d1-val, d1-test, d2-val, d2-test
    Thesis,       0.433,        0.341,  0.358,   0.321,  0.310
    DeepDarts-d1, 0.242,        0.132,  0.136,   0.124,  0.137
    DeepDarts-d2, 0.236,        0.132,  0.133,   0.119,  0.133
}\ExecutionTimes

Die Ausführungszeiten der jeweiligen Systeme sind in \autoref{fig:cv_dauer} dargestellt. Die Ausführungszeiten von DeepDarts liegen mit durchschnittlich $131\,\text{ms}$ auf den DeepDarts-Datensätzen und $239\,\text{ms}$ auf den gerenderten Daten weitaus unter den Ausführungszeiten des Systems dieser Thesis. Diese liegen bei durchschnittlich $333\,\text{ms}$ für die DeepDarts-Daten und $433\,\text{ms}$ für die Render-Daten. Die Ausführungszeiten des in dieser Thesis erarbeiteten Algorithmus belaufen sich im Mittel etwa auf die doppelte Dauer der DeepDarts-Systeme, jedoch variiert dieser Faktor stark basierend auf der Datenquelle.

Unterschiede der Inferenzzeiten der Datenquellen ergeben sich aus den Abmessungen der Bilder. Die DeepDarts-Daten sind bereits vorverarbeitet, sodass sie ein quadratisches Seitenverhältnis mit einer Auflösung von $800 \times 800\,\text{px}$ aufweisen. Die gerenderten Daten hingegen sind für diese Auswertung nicht vorverarbeitet und werden den Systemen in den originalen Auflösungen präsentiert, sodass die Seitenlängen von Bildern der Render-Daten betragen mindestens $434\,\text{px}$, maximal $3.998\,\text{px}$ und die durchschnittliche Seitenlänge beträgt $2.147\,\text{px}$. Verteilungen der Seitenverhältnisse sind in \autoref{img:cv_render_seiten} dargestellt.

\begin{figure}
    \centering
    \begin{tikzpicture}
        \begin{axis}[
                width=0.8\textwidth,
                height=6cm,
                ybar,
                bar width=0.4cm,
                enlarge x limits=0.25,
                ylabel={Zeit [s/Sample]},
                symbolic x coords={Thesis,DeepDarts-d1,DeepDarts-d2},
                xtick={Thesis,DeepDarts-d1,DeepDarts-d2},
                legend style={at={(1.02, 1.00)}, anchor=north west},
                every axis plot/.append style={
                        single ybar legend,
                    },
            ]
            \addplot+[draw=black, fill=bar_1]      table[x=system,y=Render-Daten] {\ExecutionTimes};
            \addplot+[draw=black, fill=bar_3]      table[x=system,y=d1-val]       {\ExecutionTimes};
            \addplot+[draw=black, fill=bar_3!60]   table[x=system,y=d1-test]      {\ExecutionTimes};
            \addplot+[draw=black, fill=bar_4]      table[x=system,y=d2-val]       {\ExecutionTimes};
            \addplot+[draw=black, fill=bar_4!60]   table[x=system,y=d2-test]      {\ExecutionTimes};
            \legend{Render-Daten, $d_1$-val, $d_1$-test, $d_2$-val, $d_2$-test}
        \end{axis}
    \end{tikzpicture}
    \caption{Dauer der Normalisierung auf unterschiedlichen Datensätzen, gruppiert nach Systemen.}
    \label{fig:cv_dauer}
\end{figure}

\begin{figure}
    \centering
    \includegraphics[width=0.8\textwidth]{imgs/cv/ergebnisse/bilder_seiten.pdf}
    \caption{Verteilung der Seitenverhältnisse der gerenderten Bilder.}
    \label{img:cv_render_seiten}
\end{figure}

Tatsächliche Ausführungszeiten der Systeme sind stark abhängig von der Infrastruktur, auf der die Systeme ausgeführt werden. Daher ist ihnen keine zu starke Bedeutung zuzusprechen. Die relativen Ausführungszeiten lassen sich jedoch miteinander vergleichen, um eine Einschätzung der Performance der Systeme zueinander zu erlangen. Die Inferenz des DeepDarts-Systems ist auf den DeepDarts-Daten um einen Faktor $4$ schneller und bei den gerenderten Daten um den Faktor $2,\!6$. Diese Unterschiede liegen in den Arbeitsweisen der Systeme: DeepDarts verwendet ein neuronales Netz, dessen Ausführungszeit proportional zu den Eingabedaten skaliert, während die Bilddaten in dieser Thesis in einem Vorverarbeitungsschritt skaliert werden, um nahezu unabhängig von der Eingabegröße der Bilder zu sein. Die unterschiedlichen Ausführungszeiten zwischen DeepDarts-Daten und Render-Daten dieses Systems ergeben sich aus der minimalen Bildgröße dieses Vorverarbeitungsschritts, in welchem die Bilder zwischen $800\,\text{px}$ und $1.600\,\text{px}$ skaliert werden und damit über den Abmessungen der DeepDarts-Daten liegen.

\subsubsection{Findung einer Normalisierung} % ----------------------------------------------------
\label{sec:findung_normalisierung}

\pgfplotstableread[col sep=comma]{
    system,         Render-Daten,   d1-val,     d1-test,    d2-val,     d2-test
    Thesis,         97.17,          100,        99.8,       100,        99.33
    DeepDarts-d1,   0,              100,        100,        0,          0
    DeepDarts-d2,   1.03,           36.8,       90.85,      100,        100
}\CVNorm

\begin{figure}
    \centering
    \begin{tikzpicture}
        \begin{axis}[
                width=0.8\textwidth,
                height=6cm,
                ybar,
                ymin=1,
                ymax=110,
                bar width=0.4cm,
                enlarge x limits=0.25,
                ylabel={Normalisierungen [\%]},
                symbolic x coords={Thesis, DeepDarts-d1, DeepDarts-d2},
                xtick={Thesis,DeepDarts-d1,DeepDarts-d2},
                legend style={at={(1.02, 1.00)}, anchor=north west},
                every axis plot/.append style={
                        single ybar legend,
                    },
            ]
            \addplot+[draw=black, fill=bar_1]      table[x=system,y=Render-Daten]  {\CVNorm};
            \addplot+[draw=black, fill=bar_3]      table[x=system,y=d1-val]        {\CVNorm};
            \addplot+[draw=black, fill=bar_3!60]   table[x=system,y=d1-test]       {\CVNorm};
            \addplot+[draw=black, fill=bar_4]      table[x=system,y=d2-val]        {\CVNorm};
            \addplot+[draw=black, fill=bar_4!60]   table[x=system,y=d2-test]       {\CVNorm};
            \legend{Render-Daten, $d_1$-val, $d_1$-test, $d_2$-val, $d_2$-test}
        \end{axis}
    \end{tikzpicture}
    \caption{Auswertung der Fähigkeit unterschiedlicher Systeme, Normalisierungen für Daten zu finden.}
    \label{fig:cv_normalisierung}
\end{figure}

In diesem Teil der Auswertung wird die Fähigkeit der Systeme betrachtet, eine Normalisierung der Bilder durchzuführen. Eine erfolgreiche Normalisierung bezieht sich für diese Auswertung lediglich darauf, ob ausreichend Orientierungspunkte für eine Normalisierung identifiziert werden konnten. Das DeepDarts-System muss dafür in der Lage sein, drei Orientierungspunkte zu identifizieren, da das System einem fehlenden Orientierungspunkt durch Interpolation ergänzt. Für das System dieser Thesis beinhaltet diese Anforderung die Lokalisierung des Mittelpunkts und mindestens drei weiterer Orientierungspunkte. Die Wahl der Orientierungspunkte ist dabei bei dem DeepDarts-System auf vier vordefinierte Punkte festgelegt, während das in dieser Thesis erarbeitete System 60 mögliche Punkte erkennen kann.

Erzielte Ergebnisse dieser Auswertung sind in \autoref{fig:cv_normalisierung} dargestellt. Die Auswertung ist sowohl hinsichtlich der Systeme als auch hinsichtlich der Datensätze aufgeteilt. Der Algorithmus dieser Thesis ist auf allen Datensätzen in der Lage, in mindestens $97\,\%$ der Bilder eine Normalisierung zu ermitteln. Die Performance ist dabei weitestgehend unabhängig von dem Ursprung der Daten. Dem gegenüber steht die Performance der DeepDarts-Systeme. Während DeepDarts-$d_1$ Auswertungen von $100\,\%$ auf den eigenen Validierungs- und Testdaten erzielt, ist es nicht in der Lage, positive Ergebnisse auf anderen Daten zu erzielen. Die Auswertung von DeepDarts-$d_2$ auf den eigenen Daten liegt ebenfalls bei $100\,\%$. Die Auswertung auf den Validierungs- und Testdaten von DeepDarts-$d_1$ liegt bei $37\,\%$ und $91\,\%$. Auf den gerenderten Daten werden lediglich für $2\,\%$ der Daten Normalisierungen identifiziert.

Diese Auswertung stärkt die Erkenntnis des Overfittings von DeepDarts und zeigt zugleich die Fähigkeit dieses Systems, ausreichend Orientierungspunkte zu finden, um eine Normalisierung zu ermöglichen.

\subsubsection{Genauigkeit gefundener Normalisierungen} % -----------------------------------------
\label{sec:genauigkeit_normalisierung}

\pgfplotstableread[col sep=comma]{
    system,       Render-Daten, d1-val, d1-test, d2-val, d2-test
    Thesis,       32.465,       2.879,  2.963,   3.080,  3.284
    DeepDarts-d1, ,             0.289,  0.549,   ,
    DeepDarts-d2, 1568.744,     1.499,  1.591,   0.828,  1.305
}\Similarities

\begin{figure}
    \centering
    \begin{tikzpicture}
        \begin{axis}[
                width=0.8\textwidth,
                height=6cm,
                ymode=log,
                ybar,
                bar width=0.4cm,
                enlarge x limits=0.25,
                ylabel={Genauigkeit [px]},
                ymin=0.1,
                log origin=infty,
                symbolic x coords={Thesis, DeepDarts-d1, DeepDarts-d2},
                xtick={Thesis,DeepDarts-d1,DeepDarts-d2},
                legend style={at={(1.02,1.00)}, anchor=north west},
                every axis plot/.append style={
                        single ybar legend,
                    },
            ]
            \addplot+[draw=black, fill=bar_1]      table[x=system,y=Render-Daten]  {\Similarities};
            \addplot+[draw=black, fill=bar_3]      table[x=system,y=d1-val]        {\Similarities};
            \addplot+[draw=black, fill=bar_3!60]   table[x=system,y=d1-test]       {\Similarities};
            \addplot+[draw=black, fill=bar_4]      table[x=system,y=d2-val]        {\Similarities};
            \addplot+[draw=black, fill=bar_4!60]   table[x=system,y=d2-test]       {\Similarities};
            \legend{Render-Daten, $d_1$-val, $d_1$-test, $d_2$-val, $d_2$-test}
        \end{axis}
    \end{tikzpicture}
    \caption{Genauigkeiten der Normalisierungen auf unterschiedlichen Datensätzen, gruppiert nach Systemen. Sofern keine Normalisierung möglich ist, existiert kein Balken.}
    \label{fig:cv_genauigkeit}
\end{figure}

Die Genauigkeit der Normalisierungen wird mit $\mu_\text{xst}$ durchgeführt und die Ergebnisse sind in \autoref{fig:cv_genauigkeit} dargestellt. Es ist zu erkennen, dass signifikante Unterschiede zwischen den Systemen vorherrschen. DeepDarts-$d_1$ erzielt auf den ihm zugewiesenen Validierungs- und Testdaten gute Ergebnisse mit durchschnittlich $0,\!4\,\text{px}$ Abweichung während es auf anderen Daten keine Ergebnisse erzielen kann. DeepDarts-$d_2$ hingegen kann auf allen Datensätzen Ergebnisse erzielen, jedoch sind deutliche Unterschiede zwischen den Quellen den Datensätzen zu erkennen. Auf DeepDarts-Daten können sehr gute Ergebnisse mit durchschnittlich $1,\!5\,\text{px}$ Abweichung auf den $d_1$-Daten und $1,\!1\,\text{px}$ auf den $d_2$-Daten erzielt werden. Auf den gerenderten Daten kann lediglich eine mittlere Abweichung von $1.568,\!7\,\text{px}$ erzielt werden. Diese Beobachtung lässt darauf schließen, dass keine zuverlässige Normalisierung mit diesem System möglich ist, da die Bilder lediglich eine Größe von $800 \times 800\,\text{px}$ besaßen.

Das in dieser Thesis erarbeitete System ist in der Lage, auf allen Datensätzen Normalisierungen zu finden. Darüber hinaus bewegen sich die mittleren Verschiebungen über die Datensätze in etwa ähnlichen Wertebereichen: Die Render-Daten konnten mit einer mittleren Verschiebung von $17,\!2\,\text{px}$ normalisiert werden und die DeepDarts-Daten mit $3,\!3\,\text{px}$. Die höhere Genauigkeit auf den DeepDarts-Datensätzen stammt von der Vorverarbeitung der Daten, sodass diese eine feste Größe besitzen. Da Skalierungen mit einem Informationsverlust einhergehen, steht die Anwendung dieser im Zusammenhang mit größeren Abweichungen der Auswertung.

\subsubsection{Zusammenfassung der Auswertung} % --------------------------------------------------

Die dargestellten Auswertungen zeichnen ein deutliches Bild der Arbeitsweisen und Genauigkeiten der unterschiedlichen Systeme. Während die DeepDarts-Systeme sehr gute Auswertungen auf den ihnen zugeschriebenen Daten erzielen, sind sie nicht in der Lage, auf ihnen unbekannte Daten zu generalisieren und ähnliche Ergebnisse zu erzielen. Das bereits in \autoref{sec:deepdarts} erwähnte Overfitting wird durch diese Auswertung bereits verdeutlicht.

Die Inferenzzeit von DeepDarts ist geringer als die des in dieser Thesis erarbeiteten Systems. Hintergründe dafür können in den Implementierungen der jeweiligen Systeme gefunden werden. Während in dieser Thesis ein Algorithmus vorgestellt ist, dessen Implementierung in Python erfolgt, verwendet DeepDarts ein neuronales Netz, welches nahezu vollständig kompiliert ist und keinerlei Kontrollfluss wie Verzweigungen und Schleifen verwendet. Dadurch profitiert DeepDarts von starker Parallelität und effizienter Ausführung. Dieser Unterschied ist bei der Interpretation der Ergebnisse nicht außer Acht zu lassen.

Mit der Auswertung der Fähigkeit, Normalisierungen auf Daten zu finden, in Kombination mit der Genauigkeit dieser gefundenen Normalisierungen, ist hingegen ein wesentlicher Unterschied der Systeme erkennbar. Die DeepDarts-Systeme sind nicht in der Lage, Bilder sinngemäß zu normalisieren, welche nicht aus den für das Training verwendeten Daten stammen.
