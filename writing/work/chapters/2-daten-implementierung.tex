% !TEX root = ../main.tex

\section{Implementierung}
\label{sec:daten:implementierung}

Implementierung hier.

% -------------------------------------------------------------------------------------------------

\subsection{Parametrisierung der Dartscheibe}  % ==================================================
\label{sec:dartscheibe_parametrisierung}

\subsubsection{Einsatz von Noise-Texturen}

WIE wurde es umgesetzt? -> Maskierungen etc.

\todo{}

\subsection{Zusammensetzung der Dartpfeile}  % ====================================================
\label{sec:dartpfeile_zusammensetzung}

WIE sind sie zusammengesetzt? Nutzung von Geometry-Nodes

\todo{}

\subsection{Render-Einstellungen}  % ==============================================================
\label{sec:render_einstellungen}

Farbexport realistisch, nicht cinematisch oder animatronisch (?) -> wie bei Handykamera

\todo{}

\subsection{Generierung von Dartpfeil-Positionen}  % ==============================================
\label{sec:wie_dartpfeil_positionen}

\subsubsection{Heatmaps}
\label{sec:heatmaps}

\todo{}

\begin{figure}
    \centering
    \includegraphics[width=0.8\textwidth]{imgs/rendering/methodik/heatmaps.pdf}
    \caption{Heatmaps für die Datenerstellung; (links) Generelle Heatmap für \nicefrac{2}{3} der Daten; (rechts) Multiplier-Heatmap für Oversampling auf \nicefrac{1}{3} der Daten.}
    \label{img:heatmaps}
\end{figure}

\subsubsection{Scoring}

\todo{}

\subsection{Ermittlung von Kameraparametern}  % ===================================================
\label{sec:ermittlung_kameraparamater}

Brennweite abh. von Abstand,
Auflösung abh. von Seitenverhältnis,
Fokuspunkt auf Dartscheibe

\todo{}

\subsection{Berechnung von Entzerrung}  % =========================================================
\label{sec:berechnung_entzerrung}

Orientierungspunkt-Masken -> Homographie

\todo{}
