% !TEX root = ./main.tex

% -----------------------------------------------
% PACKAGES
\usepackage[utf8]{inputenc}
\usepackage[german]{babel}
\usepackage{csquotes}
\usepackage{amsmath}
\usepackage{array}
\usepackage{amssymb}
\usepackage{nicefrac}
\usepackage{graphicx}
\usepackage{hyperref}
\usepackage{multirow}
\usepackage{subcaption}
\usepackage{nameref}
\usepackage{svg}
\usepackage[backend=biber,sorting=none]{biblatex}
\usepackage{titling}
\usepackage{microtype}
\usepackage{listings}

\addbibresource{bibliography/bibliography.bib}
\addbibresource{bibliography/foundations.bib}
\addbibresource{bibliography/images.bib}
\addbibresource{bibliography/related_work.bib}
\addbibresource{bibliography/ausblick.bib}

% -----------------------------------------------
% Document setup

% Inhaltsverzeichnis
% \usepackage[tocflat]{tocstyle}
% \usetocstyle{standard}

% Tiefe der Nummerierung der Kapitel
\setcounter{secnumdepth}{3}
\setcounter{tocdepth}{2}

% Kapitelbeginn
% \newcommand{\pagediff}{\clearpage}
\newcommand{\pagediff}{\cleardoublepage}

% Zeilenabstand
% \renewcommand{\baselinestretch}{1.5}

% Seitenabstand
\usepackage[a4paper]{geometry}

% Seitennummern
\usepackage{fancyhdr}
\pagestyle{fancy}
\fancyhf{}

\fancyhead[LE,LO]{\small\nouppercase{\leftmark}} % Shows the chapter title
\fancyhead[RE,RO]{\small\nouppercase{\rightmark}} % Shows the section title

\fancyfoot[LE,RO]{\thepage}

\fancypagestyle{plain}{
  \fancyhf{}
  \fancyfoot[LE,RO]{\thepage}
}
\renewcommand{\headrulewidth}{0pt}

% Worttrennungen
\hyphenpenalty=1000
\tolerance=5000

% Hurenkinder und Schusterjungen unterbinden
\usepackage[all]{nowidow}
\clubpenalty = 10000
\widowpenalty = 10000
\displaywidowpenalty = 10000

% Dark Mode
\usepackage[dvipsnames]{xcolor}
\usepackage{pagecolor}

\newcommand{\darkmode}{
  \pagecolor{black} % Set background to black
  \color{white} % Set text color to white
  \hypersetup{
    colorlinks=true,
    linkcolor=cyan,  % Adjust hyperlink colors for dark mode
    citecolor=lightgray,
    urlcolor=yellow
  }
}
% \darkmode

% Anführungszeichen
\newcommand{\quotes}[1]{\glqq{#1}\grqq{}}

% Grad °
\newcommand{\degree}{^{\circ}}
\DeclareMathOperator{\arctan2}{arctan2}
\DeclareMathOperator{\atan}{arctan}

% Check und Haken
\usepackage{pifont}
\newcommand{\cmark}{\ding{51}}%
\newcommand{\xmark}{\ding{55}}%

% Abstände in Enumerations und Item lists
\usepackage{enumitem}
\setlist[itemize]{itemsep=0pt, parsep=2pt}

% Plots
\usepackage{pgfplots}
\usepackage{caption}
\usepackage{pgf-pie}
\captionsetup[figure]{hypcap=false}
\pgfplotsset{compat=1.18}

% Tables
\usepackage{booktabs}

% Bar Chart Utils
\usepackage{calc}
\usepackage{ifthen}

% Tikz
\usepackage{tikz}
\usetikzlibrary{matrix}
\usetikzlibrary{positioning, shapes.geometric, arrows, fit}
\usepgfplotslibrary{groupplots}
\usepgfplotslibrary{fillbetween}


\tikzstyle{io} = [rectangle, rounded corners, minimum width=3cm, minimum height=1cm,text centered, draw=black, fill=red!30]
\tikzstyle{prep} = [rectangle, minimum width=3cm, minimum height=1cm, text centered, draw=black, fill=green!20]
\tikzstyle{edges} = [rectangle, minimum width=3cm, minimum height=1cm, text centered, draw=black, fill=blue!20]
\tikzstyle{lines} = [rectangle, minimum width=3cm, minimum height=1cm, text centered, draw=black, fill=orange!20]
\tikzstyle{orientation} = [rectangle, minimum width=3cm, minimum height=1cm, text centered, draw=black, fill=blue!20]
\tikzstyle{arrow} = [thick,->,>=stealth]

\definecolor{data_primary}{HTML}{76608A}
\definecolor{data_secondary}{HTML}{AA95BD}
\definecolor{cv_primary}{HTML}{C91B00}
\definecolor{cv_secondary}{HTML}{CC4747}
\definecolor{ai_primary}{HTML}{FF8C00}
\definecolor{ai_secondary}{HTML}{FFB459}

\definecolor{my_red}{RGB}{255,120,120}
\definecolor{my_green}{RGB}{120,255,120}

\definecolor{bar_1}{HTML}{800080}
\definecolor{bar_2}{HTML}{1fb8ed}
\definecolor{bar_3}{HTML}{d7191c}
\definecolor{bar_4}{HTML}{59d800}

% Pie Chart Utils
\newcommand{\slice}[4];
}

\newcounter{a}
\newcounter{b}
\newcommand{\drawpie}[1]{%
  \begin{tikzpicture}[baseline]
    \setcounter{a}{0}
    \setcounter{b}{0}
    \foreach \p/\c in {#1}{%
        \setcounter{a}{\value{b}}%
        \addtocounter{b}{\p}%
        \slice{\thea/100*360}{\theb/100*360}{\p}{\c}%
      }
  \end{tikzpicture}%
}

% TODO-Notes -------------------------------------------------------------
\newcounter{todo}
\newcommand\todo[1]{
  \refstepcounter{todo}
  {\reversemarginpar
    \marginpar{
      \color{red}{#1}
    }
  }
  \addcontentsline{tod}{subsection}{
    \thetodo~#1
  }
}

\makeatletter
\newcommand\todoname{todo}
\newcommand\listtodoname{List of todos}
\newcommand\listoftodos{%
  \section*{\listtodoname}\@starttoc{tod}}
\makeatother

\AtBeginDocument{%
  %   \renewcommand*{\subsectionautorefname}{\sectionautorefname}%
  \renewcommand*{\subsubsectionautorefname}{\subsectionautorefname}%
  %   \renewcommand*{\paragraphautorefname}{\subsubsectionautorefname}%
  \renewcommand*{\subparagraphautorefname}{\paragraphautorefname}%
}

% Vectors
\newcommand{\threedvec}[3]{
  \begin{bmatrix}
    #1 \\
    #2 \\
    #3
  \end{bmatrix}
}

% -----------------------------------------------
% Acronyms

% \usepackage{tocloft}
% \usepackage[nottoc,notlot,notlof]{tocbibind}
\usepackage{acro}

% \SetupAcroTemplate[list]{lof}{
%   \setlength\cftfignumwidth{6em}  % <— change “6em” to whatever you need
% }

% \acsetup{
  % trailing/activate = dash,
  % pages/display = first,
  % pages/fill = \acrodotfill
% }


\DeclareAcronym{wdf}{
  short = WDF ,
  long = World Dart Federation ,
}

\DeclareAcronym{cv}{
  short=CV ,
  long=Computer Vision ,
}

\DeclareAcronym{ki}{
  short=KI ,
  long=Künstliche Intelligenz ,
}

\DeclareAcronym{pcs}{
  short=PCS ,
  long=Percent Correct Score ,
}

\DeclareAcronym{ood}{
  short=OOD ,
  long=out-of-distribution ,
}

\DeclareAcronym{pbr}{
  short=PBR ,
  long=physically-based rendering ,
}

\DeclareAcronym{pan}{
  short=PANet ,
  long=Path Aggregation Network ,
}

\DeclareAcronym{nms}{
  short=NMS ,
  long=Non-Maximum-Suppression ,
}

% -----------------------------------------------
% Nomenklatur

\usepackage[german]{nomencl}
\setlength{\nomlabelwidth}{3cm}
\setlength{\nomitemsep}{-0.5\parsep}
\makenomenclature
