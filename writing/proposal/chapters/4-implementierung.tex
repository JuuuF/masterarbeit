\section{Implementierung}
\label{sec:implementierung}

\begin{enumerate}
    \item Datenerstellung
    \begin{itemize}
        \item Blender, siehe \autoref{img:blender}, \autoref{img:render}
        \item Modell von Dartscheibe, Raum, Lichtern und mehreren Arten von Pfeilen

        \item Zugriff auf Szene mittels "bpy"
        \begin{itemize}
            \item Python-Framework zur Interaktion mit Blender
            \item \glqq API\grqq\, für Blender-Szenen
            \item Zugriff auf alle Objekte und Parameter per Python-Skript
        \end{itemize}

        \item Modelle durch Parameter einstellbar
        \begin{itemize}
            \raggedright
            \item Farbe und Beschaffenheit von Texturen
            \item Lichtverhältnisse (Helligkeit, Farbe etc.) (Blitzlicht, HDRI, Deckenlampen, ...)
            \item Abnutzung des Boards
            \item Aussehen der Dartpfeile
            \item Umgebung um das Board / Raumausstattung (Dartboard-Schrank, Bilder, weitere Dartscheiben im Bild, ...)
        \end{itemize}

        \item Parameter zufällig oder gewichtet setzen
        \begin{itemize}
            \item Kameraposition z.B. zufällig in definiertem Raum
            \item Pfeilpositionen z.B. anhand von Heatmaps gewichtet positionieren
        \end{itemize}

        \item Korrektheit der Daten gesichert
        \begin{itemize}
            \item 3D-Positionen von Dartpfeilen und Scheibe bekannt $\rightarrow$ Feldwerte können analytisch bestimmt werden
            \item Statistik nicht nur über Feldwert, sondern exakte Positionen möglich
        \end{itemize}

        \item Rendering auf GPU-Server, da bpy keine GUI benötigt
    \end{itemize}

    \item Erkennung der Dartscheibe
    \begin{itemize}
        \raggedright
        \item Große Flächen, hoher Kontrast
        \item bestehend aus Primitiven: Kreise, Linien, Dreiecke
        \item Verzerrung: Kreise $\rightarrow$ Ellipsen
        \item insgesamt gute Voraussetzungen, um mittels herkömmlicher CV-Methoden erkannt zu werden $\rightarrow$ OpenCV
        \item Ggf. simples CNN zur Identifizierung von Dartscheiben: Daten ebenfalls durch 3D-Rendering erstellbar
        \item Fisheye-Effekte könnten problematisch sein
    \end{itemize}

    \item KI-Training:
    \begin{itemize}
        \item Training mit TensorFlow auf GPU-Server
        \item Datenaufteilung:
        \begin{itemize}
            \item Trainingsdaten: Synthetische Daten (Masse, korrekt annotiert)
            \item Validierungsdaten: Reale Daten, eigene
            \item Testdaten: Baseline-Daten
            \item Argumentierung: Wir sind auf deren Daten besser als sie selbst, ohne auf den Daten trainiert zu haben
        \end{itemize}

        \item Datenaugmentierung
        \begin{itemize}
            \item Pixel-Augmentierung: Noise, Helligkeit, Kontrast, ...
            \item Transformations-Augmentierung: Rotation, Skalierung, Flips, ...
        \end{itemize}

        \item Ggf. Hyperparameter-Tuning durch Random Grid / Bayesian Search
        \item Ggf. k-Fold Cross-Validation
    \end{itemize}

\end{enumerate}

\begin{figure}
    \centering
    \includegraphics[width=0.8\linewidth]{imgs/blender.png}
    \caption{Blender-Projekt}
    \label{img:blender}
\end{figure}

\begin{figure}
    \centering
    \includegraphics[width=0.5\linewidth]{imgs/render.png}
    \caption{Render}
    \label{img:render}
\end{figure}
