\section{Methodik}
\label{sec:methodik}

\begin{itemize}
    \item Datensammlung
    \begin{itemize}
        \raggedright
        \item Durch 3D-Software mittels Ray-Tracing und fotorealistischer Texturen
        \item Aufnahme realer Steeldarts-Runden:
        \begin{itemize}
            \item Live-Protokollierung
            \item Aufnahme durch unterschiedliche Kameras
            \item Daten sammeln in realem Einsatz
        \end{itemize}
        \item Identifizierung von typischen Bildaufnahme-Parametern durch Metadaten-Analyse und Online-Recherche
    \end{itemize}

    \item Dartscheiben-Identifizierung
    \begin{itemize}
        \raggedright
        \item Aufbau auf Erkennungsmechanismen von Schachbrettmustern etc.
        \item Farbraum-Transformationen, Thresholding, Filterung, Blob-/Keypoint-Detektion
        \item Hough-Transformationen, Ellipsen-Erkennung
        \item Ggf. CNN zur Objekterkennung
        \begin{itemize}
            \item Transfer-Learning durch bereits bestehende Modelle (ResNet, YOLO, FOMO, etc.)
        \end{itemize}
    \end{itemize}

    \item KI-Training / Dartpfeil-Identifikation
    \begin{itemize}
        \item Transfer Learning von Obkekterkennungs-CNN
        \item Eigene Methodiken zur Lenkung des Trainings
        \item Nutzung von ML- und CV-Frameworks
    \end{itemize}

    \item System-Statistiken sammeln
    \begin{itemize}
        \item Wie gut werden bestimmte Felder erkannt?
        \item Metriken aus Paper übernehmen (Vergleichbarkeit)
        \item Statistiken zu:
        \begin{itemize}
            \item Feld-Art (Polar-Radius auf Dartscheibe)
            \item Feldwert (Polar-Winkel auf Dartscheibe)
            \item Pfeilabstand zueinander
            \item Aufnahmewinkel der Kamera
        \end{itemize}
    \end{itemize}
\end{itemize}
