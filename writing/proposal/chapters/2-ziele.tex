\section{Ziele}
\label{sec:ziele}

Diese Masterarbeit stellt eine Erweiterung und Verbesserung eines bereits bestehenden Systems dar, indem spezielle Techniken zum Trainieren eines KI-basierten Systems eingesetzt werden. Der Kern der Zielstellung dieser Arbeit ist analog zu dem Referenzpaper die Etablierung eines Single-Camera-Systems für das Scorekeeping von Steeldarts, wobei der Fokus jedoch darauf gesetzt ist, mit synthetischen Daten eine möglichst solide Datengrundlage zu schaffen. Dieses System ist darauf ausgelegt, potenziell auf mobilen Endgeräten eingesetzt zu werden, wodurch in Anspruch zu nehmende Ressourcen begrenzt sind. Diese Voraussetzung ist kein zentraler Aspekt dieser Arbeit, sollte aber bei der Wahl der eingesetzten Techniken stets im Hinterkopf behalten werden.

Das zentrale Ziel dieser Arbeit besteht darin, die Schwachstellen des Systems von McNally et al. systematisch zu identifizieren und gezielt zu adressieren.

\subsection{Schwachstellen im Paper}
\label{sub:schwachstellen}

Im Referenzpaper wurden bereits einige Schwachstellen des Systems erwähnt. Neben diesen explizit erwähnten Problemen wurden weitere Problembereiche identifiziert, die in diesem Unterkapitel aufgeführt und erläutert werden.

\subsubsection{Datengrundlage}
\label{sec:ziele:schwachstellen:datengrundlage}

Der Ansatz von McNally et al. basiert auf neuronalen Netzen, die auf einer begrenzten und unsicheren Datengrundlage trainiert wurden. Die verwendeten Bilddaten stammen aus zwei Quellen: Ein Großteil der Daten wurde mit einem fix montierten iPhone aufgenommen, während der Rest mit einer DSLR-Kamera aus unterschiedlichen Perspektiven aufgenommen wurde \cite{deepdarts-data}. Beide Quellen sind nicht repräsentativ für den vorgesehenen Einsatz des Systems, in dem mit Handyaufnahmen aus stark variierenden Perspektiven in unterschiedlichen Bedingungen umgegangen werden muss. Des Weiteren wurden diese Daten manuell per Hand annotiert, was mit Ungenauigkeiten einhergeht; diese Ungenauigkeiten wurden in dem Paper von den Autoren anerkannt. Die Validität des Modells ist dadurch beeinträchtigt.

\subsubsection{Entzerrung der Bilder}
\label{sec:ziele:schwachstellen:entzerrung}

Im Gegensatz zu einer trivialen Herangehensweise geht das Referenzpaper davon aus, dass Bilder von Dartscheiben nicht perfekt sind und mit einer perspektivischen Verzerrung einhergehen. Um verlässliche Vorhersagen bezüglich der Dartpfeile treffen zu können, muss die Dartscheibe entzerrt werden. Der vorgeschlagene Ansatz zur Entzerrung der Bilder besteht darin, vier Fixpunkte der Dartscheibe zu identifizieren und anhand dieser eine Homografie abzuleiten, die die Dartscheibe von einer Ellipse in einen perfekt ausgerichteten Kreis transformiert. Diese Herangehensweise setzt jedoch voraus, dass alle vier Fixpunkte im Bild erkennbar sind. Sollte einer der Punkte beispielsweise durch einen Dartpfeil verdeckt oder aufgrund der Abnutzung der Dartscheibe nicht mehr zu erkennen sein, ist eine korrekte Entzerrung auf diese Weise nicht möglich. Dieser Schwachpunkt wurde ebenfalls im Paper anerkannt und als verbesserungswürdig hervorgehoben.

\subsubsection{Generelle Verbesserung der Vorhersagen}
\label{sec:ziele:schwachstellen:verbesserung}

Laut eigener Metrik hat das vorgestellte System eine Genauigkeit von 84\%. Diese Genauigkeit ist nach subjektiver Einschätzung ausreichend, um einen Proof of Concept zu validieren, jedoch nicht für den realen Einsatz dieses Systems in einem Live-Umfeld. Teile dieser Ungenauigkeit sind auf bereits genannte Probleme zurückzuführen. Darüber hinaus sind seit der Veröffentlichung des Papers neue und fortgeschrittene Technologien im Bereich der neuronalen Netze und der Computer Vision entwickelt worden, die Möglichkeiten zur Verbesserung des Systems bieten.

% Ebenfalls aufnehmen: Mehrere Trainings statt nur einem?

\subsection{Problemlösung}
\label{sec:ziele:problemlösung}

Die zentrale Zielstellung dieser Masterarbeit ist das systematische Angehen der genannten Probleme. Dazu liegt der Fokus dieser Arbeit darauf, ein genaueres und robusteres System zu konzipieren, das durch synthetisch erstellte Daten trainiert wird. Auf diese Weise können Verzerrungen der Daten aufgrund festgelegter Umgebungsbedingungen, Begrenzung auf wenige Kameraparameter oder fehlerhaft annotierter Daten gemindert werden. Darüber hinaus werden reale Daten aufgenommen, die im späteren Einsatz in einer App anfallen könnten. Dabei wird ein Protokoll über getroffene Felder geführt, um die Wahrscheinlichkeit fehlerhafter Erkennungen zu minimieren.

Es stehen für diese Masterarbeit die folgenden Forschungsfragen zur Beantwortung aus:

\begin{enumerate}
    \item Inwiefern ist es möglich, synthetische Daten automatisiert mit einer 3D-Software zu erstellen, die die Variationen echter Handyaufnahmen realistisch genug abbilden, sodass eine KI auf diesen Bildern mit zufriedenstellender Genauigkeit trainiert werden kann?
    \item Ist es im Umfang dieser Arbeit möglich, ein Konzept zur Entzerrung der Dartscheibe zu entwickeln, das Ansätze der Computer Vision nutzt und robuster agiert als das vorgestellte Referenzsystem?
    \item Mit welcher Genauigkeit können einzelne Pfeile auf den Bildern von einer KI identifiziert werden?
    \item Mit welcher Genauigkeit können mehrere Pfeile gleichzeitig auf den Bildern von einer KI identifiziert werden?
    \item Wie sehr beeinträchtigen Herausforderungen wie Verdeckung der Pfeilspitzen, unscharfe Aufnahmen oder eng beieinander steckende Pfeilspitzen die Genauigkeit des Systems?
\end{enumerate}

Um das Referenzsystem mit dem System dieser Masterarbeit angemessen vergleichen zu können, ist es von Bedeutung, dass das im Paper verwendete Modell lauffähig gemacht wird und die Daten des Trainings und der Evaluation verwendet werden können. Die relevanten Daten sind öffentlich verfügbar und können eingesehen und heruntergeladen werden \cite{deepdarts-data}. Sobald das System funktionsfähig ist, können zusätzlich eigene Statistiken erhoben werden, um die Performance der verschiedenen Ansätze zu vergleichen, wobei eine Verbesserung der Resultate angestrebt wird.

Um zukünftigen Forschungsarbeiten eine vergleichbare Datengrundlage bereitzustellen, werden die im Rahmen dieser Arbeit generierten Daten öffentlich zugänglich gemacht. Durch die Bereitstellung der Datensätze wird weiteren Projekten ermöglicht, auf diese Datenbasis zuzugreifen und eigene Systeme unter Verwendung der vorhandenen Daten in Analogie zu den in dieser Arbeit entwickelten Ansätzen zu evaluieren und mit bestehenden Systemen zu vergleichen.

\iffalse

\begin{itemize}
    \item Verbesserung des Systems, wie es im Paper genutzt wurde \cite{deepdarts}
    \item Ziel des Papers weiterhin verfolgen: Einsatz auf mobilen Endgeräten $\rightarrow$ begrenzte Ressourcen

    \item Schwachstellen aus Paper gezielt angehen
    \begin{itemize}
        \item Verzerrte, unsichere Datengrundlage \cite{deepdarts-data}
        \begin{itemize}
            \item Originaldaten aus 2 Quellen: iPhone (an Decke befestigt), DSLR (wenig Variation, idealisierte Daten)
            \item Händisch durch Sichtung der Bilder annotiert, als fehlerbehaftet eingestuft
        \end{itemize}
        \item Fehleranfällige Erkennung der Dartscheibe
        \begin{itemize}
            \item Identifizierung von 4 Fixpunkten
            \item Punkt nicht erkannt: Problem.
        \end{itemize}
        \item Insgesamt Verbesserung des Systems möglich: 84\% Test-Score
    \end{itemize}

    \item Problemlösung
    \begin{itemize}
        \item Nutzung von 3D-Software, um eine ausreichende Menge korrekt annotierter und variabler Daten zu erstellen
        \item Erstellung korrekt annotierter, realer Daten
        \item Einsetzen von CV-Methodiken, um Dartscheibe robust zu erkennen
        \item Nutzung aktueller KI-Methodiken zur Erkennung von Dartpfeilen
    \end{itemize}

    \item Zur korrekten Erkennung der Probleme: Paper-Modell zum Laufen bringen
    \item Erstellung von aussagekräftigen Statistiken zur Identifizierung der Performance
    \item Vergleich der Ansätze, idealerweise Verbesserung der Resultate

\end{itemize}

\fi
