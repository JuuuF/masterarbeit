\section{Evaluation}
\label{sec:evaluation}

Diese Arbeit stellt eine Erweiterung eines bestehenden Systems dar und muss daher im Kontext der ursprünglichen Ergebnisse evaluiert werden. In dem Referenzpaper wurden bereits Metriken zur Messung der Genauigkeit und Leistungsfähigkeit des Systems etabliert, die für die Evaluation dieser Arbeit übernommen werden, um eine konsistente Vergleichbarkeit der Systeme zu gewährleisten. Zusätzlich besteht die Möglichkeit, eigene, falsifizierbare Metriken zu entwickeln, um spezifische Aspekte der Systemleistung weiter zu untersuchen. Die Bewertung des Systems wird nicht ausschließlich anhand eigener Daten erfolgen, sondern insbesondere durch den Einsatz der in Referenzpaper von McNally et al. verwendeten Datensätze, um die Vergleichbarkeit der Ergebnisse zu gewährleisten.

Hinsichtlich der Systemperformance werden umfangreiche Statistiken gesammelt, die durch die synthetische Datenerstellung vielseitig ausfallen. Eine denkbare Auswertung betrifft die Genauigkeit der Vorhersagen in spezifischen Bereichen der Dartscheibe. Zudem kann untersucht werden, in welchen Szenarien die Methodik von McNally et al. Ungenauigkeiten aufweist und wie sich das in dieser Arbeit entwickelte System in diesen Fällen verhält. Im Gegenzug lassen sich auch Schwachstellen des hier präsentierten Ansatzes identifizieren und vergleichen. Weitere statistische Auswertungen könnten sich auf die Trefferart (einfach, doppelt oder dreifach), die getroffenen Felder oder die Genauigkeit der Vorhersagen in Abhängigkeit vom Abstand der Dartpfeile zueinander beziehen.

Da das System für einen möglichen Einsatz in mobilen Endgeräten vorgesehen ist, bietet sich zudem ein Effizienzvergleich der Systeme an. Hierbei könnte die Erkennungsdauer auf identischer Hardware als Maßstab herangezogen werden, um die Laufzeitoptimierung und Eignung für mobile Anwendungen zu bewerten.

\iffalse
\begin{itemize}
    \item Metriken aus Paper übernehmen
    \item Ggf. Erweiterung um eigene Metriken (müssen falsifizierbar sein)
    \item Auswertung auf eigenen Daten, sowie den Daten des Papers
    \item Statistiken über System-Performance
    \begin{itemize}
        \item Wie gut werden bestimmte Felder erkannt?
        \item Wo scheitert deren System, wie handhabt unser System diese Fälle?
        \item Wo scheitert das neue System? (Fisheye, Verdeckung etc.)
        \item Statistiken zu:
        \begin{itemize}
            \item Feld-Art (Polar-Radius auf Dartscheibe)
            \item Feldwert (Polar-Winkel auf Dartscheibe)
            \item Pfeilabstand zueinander
        \end{itemize}
        \item Metriken aus Paper übernehmen (Vergleichbarkeit)
        \item Vergleich der Effizienz des Systems aus Paper mit dem neuen System
    \end{itemize}
\end{itemize}
\fi
