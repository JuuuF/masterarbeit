
\section{Zeitplanung}
\label{sec:zeitplanung}

In diesem Abschnitt wird die geplante zeitliche Strukturierung der Masterarbeit dargestellt. Die verschiedenen Aufgabenbereiche wie Systemaufbau, Datengenerierung, KI-Training, Evaluation und das Schreiben der Abschlussarbeit wurden in einem Gantt-Chart\footnote{Paketdokumenation: \url{https://ftp.rrze.uni-erlangen.de/ctan/graphics/pgf/contrib/pgfgantt/pgfgantt-doc.pdf}} organisiert, um einen Überblick über die Projektfortschritte zu gewährleisten. Dieses Chart stellt die voraussichtlichen Zeiträume dar, in denen die jeweiligen Aufgaben bearbeitet werden sollen.

Die Struktur des Gantt-Charts erlaubt die einfache Visualisierung der parallel laufenden Arbeitsschritte. Es zeigt, wie Datengenerierung, System-Setup, Implementierung, Evaluation und Schreiben der Arbeit ineinandergreifen. Diese Abbildung erleichtert die kontinuierliche Anpassung des Zeitplans, falls unvorhergesehene Herausforderungen auftreten. Der Schreibprozess findet parallel zur Umsetzung des Projektes statt und ist so geplant, dass eine Phase direkt nach ihrer Vollendung mit aufgenommen werden kann. Zudem wurde ausreichend Zeit für etwaige Verzögerungen eingeplant.

Die detaillierte Planung erlaubt es, wichtige Meilensteine und Deadlines zu definieren und gibt einen klaren Überblick über die zeitliche Abfolge der Schritte. Der Fortschritt wird im Laufe der Arbeit anhand dieser Grafik gemessen und eingeordnet, um sicherzustellen, dass das Projekt im vorgesehenen Rahmen abgeschlossen wird.

\definecolor{colA}{HTML}{ff595e}
\definecolor{colB}{HTML}{ffca3a}
\definecolor{colC}{HTML}{8ac926}
\definecolor{colD}{HTML}{1982c4}
\definecolor{colE}{HTML}{6a4c93}

\centerline{\small
\begin{ganttchart}[
    vgrid={*{1}{draw=gray, line width=0.2pt}, *{3}{dotted}},
    y unit chart=0.6cm,
    bar label font=\footnotesize,
    group label font=\bfseries\footnotesize,
]{0}{25}
    % Kopfzeile
    \gantttitle{}{1}
    \gantttitle{Monat 1}{4}
    \gantttitle{Monat 2}{4}
    \gantttitle{Monat 3}{4}
    \gantttitle{Monat 4}{4}
    \gantttitle{Monat 5}{4}
    \gantttitle{Monat 6}{4}
    \gantttitle{}{1}\\

    % ------------------------------------
    % Datenaufnahme
    \ganttgroup[group/.append style={fill=colA!75!black, draw=black, line width=0.75pt}]{Daten}{1}{6}\\

    \ganttbar[bar/.append style={fill=colA}]{Reale Daten}{1}{2}\\
    \ganttbar[bar/.append style={fill=colA}]{Paper-Datenaufbereitung}{2}{3}\\ 
    \ganttbar[bar/.append style={fill=colA}]{Synthetische Daten}{2}{6}\\

    % ------------------------------------
    % System-Setup
    \ganttgroup[group/.append style={fill=colB!75!black, draw=black, line width=0.75pt}]{System-Setup}{1}{7}\\
    \ganttbar[bar/.append style={fill=colB}]{Deep-Darts-System}{1}{2}\\
    \ganttbar[bar/.append style={fill=colB}]{GPU-Server}{2}{3}\\
    \ganttbar[bar/.append style={fill=colB}]{Trainingsumgebung}{4}{7}\\

    % ------------------------------------
    % Implementierung
    \ganttgroup[group/.append style={fill=colC!75!black, draw=black, line width=0.75pt}]{Implementierung}{4}{16}\\
    \ganttbar[bar/.append style={fill=colC}]{CV-Logik}{4}{6}\\
    \ganttbar[bar/.append style={fill=colC}]{KI-Konzeption}{5}{9}\\
    \ganttbar[bar/.append style={fill=colC}]{KI-Umsetzung}{8}{12}\\
    \ganttbar[bar/.append style={fill=colC}]{KI-Training}{10}{16}\\

    % ------------------------------------
    % Evaluation
    \ganttgroup[group/.append style={fill=colD!75!black, draw=black, line width=0.75pt}]{Evaluation}{6}{17}\\
    \ganttbar[bar/.append style={fill=colD}]{System-Vergleich}{6}{7}
    \ganttbar[bar/.append style={fill=colD}]{}{14}{17}\\
    \ganttbar[bar/.append style={fill=colD}]{Metrikfindung}{7}{8}\\
    \ganttbar[bar/.append style={fill=colD}]{System-Schwachstellen}{12}{16}\\
    \ganttbar[bar/.append style={fill=colD}]{Statistikerhebung}{14}{16}\\

    % ------------------------------------
    % Schreiben
    \ganttgroup[group/.append style={fill=colE!75!black, draw=black, line width=0.75pt}]{Schreiben}{4}{22}\\
    \ganttbar[bar/.append style={fill=colE}]{Einleitung}{19}{21}\\
    \ganttbar[bar/.append style={fill=colE}]{Grundlagen}{4}{6}
    \ganttbar[bar/.append style={fill=colE}]{}{21}{22}\\
    \ganttbar[bar/.append style={fill=colE}]{Methodik}{6}{10}
    \ganttbar[bar/.append style={fill=colE}]{}{21}{22}\\
    \ganttbar[bar/.append style={fill=colE}]{Implementierung}{9}{13}
    \ganttbar[bar/.append style={fill=colE}]{}{21}{22}\\
    \ganttbar[bar/.append style={fill=colE}]{Ergebnisse}{13}{16}
    \ganttbar[bar/.append style={fill=colE}]{}{21}{22}\\
    \ganttbar[bar/.append style={fill=colE}]{Diskussion}{14}{17}
    \ganttbar[bar/.append style={fill=colE}]{}{21}{22}\\
    \ganttbar[bar/.append style={fill=colE}]{Fazit}{16}{19}
    \ganttbar[bar/.append style={fill=colE}]{}{21}{22}\\
    \ganttbar[bar/.append style={fill=colE}]{Präsentation}{17}{19}\\

    % ------------------------------------
    % Milestones
    % \ganttmilestone[inline]{Daten fertig}{12}\\

    % ------------------------------------
    % Bars
    \ganttvrule{Präsentation}{19}
    \ganttvrule{\phantom{X}\raisebox{-3ex}{\centering Soft Deadline}}{22}
    \ganttvrule{Deadline}{24}

\end{ganttchart}
} % centerline
